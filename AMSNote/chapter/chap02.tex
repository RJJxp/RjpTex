\section{招聘总结}
实习几周后结束, 渐渐感觉到自己能写点什么. 

\subsection{令人头疼的候选人}
一开始做猎头招聘, 懵懵懂懂收集简历, 表格记录, 电话面试, 很快摸清套路. 

收集简历是痛苦的. 客户公司屈指可数的简历投递意味着, 要花更多的精力时间主动搜索简历. 叉车工候选人受普遍教育程度较低, 简历上信息寥寥无几, 更有甚者简历上只有电话姓名年龄. 他们好似倔强的小孩子, 努力用底层中年人的仅剩尊严抵抗信息化的洪流, 最终被时代抛弃. 打过去的电话往往无人接听, 半个小时后再打, 依然无人接听, 三四次后, 只好将他们添加到次日的名单. 

不同招聘网站可能存在同一份简历, 好记性和表格记录的优势便显现出来了. 更重要的是, 记录完一份简历, 一定要在招聘网站上给相应简历做标记, 对我来说, 不仅仅是方便明天工作不要重复记录, 更重要是, 等我实习结束, 下一位同事在招聘网站看到我的标记, 便知道哪些候选人已经沟通过, 这样可以极大提高整体招聘效率. 

电话面试, 开口先介绍自己的客户公司, 告知候选人获取其简历渠道, 现在是否还在找工作? 有无叉车证? 这时总会有候选人打断我, ``你们招嘉兴(太仓)叉车工, 为啥电话是上海的?'' 我解释一番三方招聘, 如果候选人年龄较大(四十五岁左右), 他们的语气会变的将信将疑, 认为我是电信诈骗; 如果候选人较为年轻(三十五岁左右)则不会这样打断我, 他们在全国各地跑, 见多识广, 沟通效率极高, 三言两语, 谈笑间完成一场简单的电话面试. 而尴尬的是, 在我说 ``不包吃住''后, 多半候选人都惊讶又带些恼怒, 用反问的语气重复, ``不包吃住?'' 即刻间电话面试草草结束.

如果他们刚入叉车行, 对行情不是很了解, 之后很容易能进行到具体的薪资绩效, 地址的介绍. 一开始我单纯地念仓库地址, 因为潜意识中我认为叉车工师傅都是本地人, 他们应该比我更熟悉当地. 在打了几天电话后发现, 候选人中一部分是流动人口, 从一个城市到另一个城市, 生活随仓库走. 我和他们对于仓库地址一样陌生. 从这一点也能理解为何叉车工师傅对待遇中一定要求包吃住. 因此, 对各个仓库的大致区位的掌握显得十分重要. 比如太仓仓库在太仓西南, 靠近昆山, 那么太仓东部的浮桥, 渡桥, 浏河等地的候选人, 只能祈祷他可以开车通勤. 有时询问候选人地址, 我会边通话边搜索, 然后告知他们和仓库的通勤难度. 如果实在困难, 只好加微信, 发送具体仓库地址. 

让我诧异的是, 有一小部分人止步于叉车证这样的硬性要求. 后面才意识到, 国内物流企业有时候不需要叉车证也能上岗, 尽管候选人们看到JD上严格要求叉车证, 他们理所当然或是抱着侥幸心理, 投递了我的客户公司. 可惜客户公司是规矩严格的外企. 

\subsection{被算法统治}
简历主要从51job, 智联招聘, 菁英和人才网进行收集, 相比之下, 人才网的简历更靠谱一点, 拨出的电话很少没人接. 这么多招聘网站, 它们都做了一件让我不理解的事情, 候选人并未投递, 招聘网站自动帮其投递. 在 ``候选人'' 惊讶的表示``我啥时候投过?'' 或 ``我没有投过啊?'' 我深感歉意. 客户公司付了钱, 却没有收到简历, 为了不让我的客户公司尴尬, 招聘网站自动投递简历, 还能多捞点钱, 就让我这个小猎头时不时 ``愧疚'' 一下下, ``尴尬''一会会. 

之后Miya认为Boss直聘或许更适合我们的客户公司的RPO项目. 相比之前传统招聘网站, Boss直聘更专业同时操作也更麻烦. 其网站设置只能开一个窗口, 所有网页内容压到一个javascript的脚本中, 从技术角度来讲, 这是很好的反爬机制, 提高数据安全性, 同时也避免了爬虫给服务器其带来的巨大压力. 账号和手机绑定, 网页端只能登陆一个账号, 而新开账号不仅要走AMS的流程, 还要走客户公司的繁琐流程, 这对于我这样的共用Miya账号的实习生来说, 登陆账号十分痛苦. Boss直聘主打候选人隐私保护的旗号, 虽在互联网时代, 并无隐私可言, 其做法是, 猎头账号所有图片都看不到, 与候选人聊天审核很严格, 禁止发送其他联系方式, 多次违规永封账号. 相比传统招聘网站, 轻松购买廉价的积分就能获取候选人联系方式, Boss这种做法极大的增加猎头方的沟通成本. 为了保护候选人在互联网时代根本不存在的隐私, 牺牲的便是猎头的权力. 

最让我感到悲哀的则是招聘信息都被算法垄断. Boss直聘以 ``打招呼'' 的方式直接跟候选人沟通, 根据账号等级, 每日有 ``打招呼'' 限制. 我的账号一天一百个. 一开始的列表中, 往往是叉车工, 刷新几次后的列表中, 待沟通列表里的候选人并非目标工种, 大多数是和叉车不沾边的装配工或是电工, 维修工等等.从每天简历来推测, Boss是有大量高质量的简历, 但算法决定, 每天它只会放出来一点点, 它以好简历搭配着大量低质量简历的混合套餐, 徒增猎头工作量. 这样导致了一个后果, 我每天一百个消息的时候, 一上午我能找到两到三个可选候选人; 而当我每天四百个消息, 我只能找到三到四个合格候选人. 工作量增加了四倍, 而成果并不是成比例的线性增长, 随着简历池的枯萎, 高质量简历的收集犹如大海捞针. 

不过现实中, 在我给一位过往经历全是电工的候选人发消息后, 他立马回复可以面试, 毫不犹豫. 我以为他在耍我, 还在犹豫. 之前也有这样的经历, 我Boss发完了岗位信息, 候选人一条 ``待遇这样, 还要招人, 是在搞备战吗?'' 搞得我哭笑不得. 没想到的是, 这位电工候选人去了后, 火速入职. 这样小小的惊喜让实习增光添彩, 这大概是上天对我的眷顾吧. 

可纵观这一流程, 客户公司花钱买了服务, 或许招聘网站提供了所谓的相应套餐, 但最终的信息权还是牢牢掌握在招聘网站手中. 大数据时代, 便是用户被算法统治的时代.

\subsection{万恶之源:客户公司}
后期在和候选人沟通的过程中, 我常常反客为主, 追问候选人拒绝的缘由, 并继续深挖, ``既然如此, 那您呆过的公司待遇一般如何呢?'' 在和多个健谈的候选人沟通后, 我惊觉客户公司的RPO项目出力不讨好, 这和我在招聘过程中的直觉相吻合, 得出以下结论:

客户公司招聘条件苛刻, 属于外企水土不服的典型. 其薪资是国内同行的下限, 虽然按照劳动法缴纳五险一金, 但其不包吃住, 夜班, 包括绩效方式, 任何一个说出口, 在叉车一行呆过的老师傅都会严词拒绝, 避而远之. 高位叉车的工作内容, 有经验的老师傅拒绝的理由一般是脖子疼, 视力不好; 上夜班很多四十多岁的师傅无法接受, 尤其有家室的, 一丝都不能容忍; 同行按时间绩效, 客户公司按照数量绩效, 这会使工作压力骤增, 很累很累的刷数量, 可能薪资还没有同行计时的高. 别人个人绩效, 自己干的多便拿的多, 客户公司团队绩效, 你做的再认真, 团队有人偷懒, 你也是白做; 不包吃住不仅理论上拒绝了流动人口, 对于不谙世事的叉车工, 他们人生地不熟, 租房, 怎么租哪里租, 如果租金过高岂不是白来一趟? 

这样的条件也很快确定了叉车工候选人的画像. 不包吃住意味着住的离仓库近; 低薪资留不住年轻人, 能招到的便只有上了岁数的叉车工; 而夜班又把中年身体不适的候选人筛选掉. 与众不同的绩效方式只能坑蒙拐骗初入叉车的门外汉. 能招到的便只有傻乎乎的叉车新人或是家住附近却愿意夜班的中年人. 但这样的人会上网吗? 在招聘网站的算法统治下, 又能找到几份这样的简历呢? 如果按照此画像筛选, 恐怕一天打的电话连三个都不到, 更别提真正愿意来客户公司上班的候选人. 客户公司人力部门要求每天提交电话面试记录, 如果某天只有零星的几个, 想必我的上司Miya也不会好过. 

于是一场冠冕堂皇的新装秀开始了. 我知道这个人打过去可能拒绝, 但还是要装模做样的打, 按部就班询问每个关键问题, 候选人也意料之中拒绝, 迅速记录结果, 内心毫无波澜, 挂了电话下一位. 凑够了每天的数量, 也能松一口气. 

\subsection{对于指挥链的简单思考}
沟通技巧欠缺, 一股脑把岗位的优劣一并发出去, 往往被拒绝. 适当的避重就轻就好. 回顾我的实习, 有一种发自内心深处的无奈, 尽管我十分喜欢我的实习和周围的同事, 和她们共事很幸福, 具体工作内容也并不繁琐, 至少远低于我的不可接受阈值. 那为何还会无力?

涉密信息-涉密信息-涉密信息-涉密信息-涉密信息-涉密信息-涉密信息-涉密信息-涉密信息-涉密信息-涉密信息.涉密信息-涉密信息-涉密信息-涉密信息-涉密信息-涉密信息-涉密信息-涉密信息. 但短短一个多月我已经意识到, 客户公司的RPO项目只是一个花了钱的甲方对乙方蛮横无礼的欺凌罢了, 至少在我看来, 作为乙方的我们反馈渠道很有限. 我和Miya不可能直接跟甲方就招聘难度进行直接沟通, 就算Miya和甲方直接沟通, 也不会有什么实质性结果. 只能通过我们汇报我们的上级, 我们的上级汇报她的领导, 一步步上报, 等待AMS跟客户公司的管理层打交道反应情况.

事实上如果我们把此种情况汇报至上级, 上级其实也知道客户公司无理的招聘条件, 他们会怎么想? 他们只会告诉我, 虽然客户公司条件苛刻, 但架不住给的钱多, 你只要好好抗住压力, 当做我们和客户公司的缓冲带即可, 一切都会过去. 如果你实在受不了, 要么涨薪继续做, 要么你辞职换人. 涉密信息-涉密信息-涉密信息-涉密信息-涉密信息-涉密信息-涉密信息-涉密信息-涉密信息-涉密信息-涉密信息.涉密信息-涉密信息-涉密信息-涉.

涉密信息-涉密信息-涉密信息-涉密信息-涉密信息-涉密信息-涉密信息-涉密信息-涉密信息-涉密信息-涉密信息.涉密信息-涉密信息-涉密信息-涉密信息-涉密信息-涉密信息-涉密信息-涉密信息-涉密信息-涉密信息-涉密信息-涉密信息-涉密信息-涉密信息-涉密信息-涉密信息-涉密信息-涉密信息-涉密信息-涉密信息-涉密信息-涉密信息.涉密信息-涉密信息-涉密信息-涉密信息-涉密信息-涉密信息-涉密信息-涉密信息-涉密信息-涉密信息-涉密信息-涉密信.

Don't hate the player. Hate the game.

\begin{flushright}
    \bfseries

    任家平

    二零二一年九月十二日
\end{flushright}