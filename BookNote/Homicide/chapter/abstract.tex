\renewcommand\abstractname{\Large\textbf{序}}
\begin{abstract}
    二零二一年十月二十六日从淘宝购得此书, 翻看豆瓣记录, 双十一读完. 从购书至今已快一年, 中途重读了``局外人''并完成了读书笔记, 而``凶年''的读书笔记一直停滞, 其中缘由记不清楚. 本打算国庆假期完成笔记, 父母到来打乱了计划. 放浪形骸几天后, 国庆假期一转眼就要结束, 我才下定决心在公司电脑上开始整理. 估计十月份每天晚上有事可做了. 

    之前雄心壮志整理 \textit{The Wire} 的台词, 中道崩殂.

    或许我没有想清楚, 读书笔记和台词整理的区别. 我真正紧急的需求是什么呢? 在整理凶年的过程中, 我需要细细思索一番. 

    作者 \textit{David Simon} 记者出身, 本书多线叙事却毫无违和感的写作技巧值得借鉴. 指挥链, 数字游戏, 审讯, 政治红球, 种族歧视, 凶案组趣事等等等, 每一部分看似杂乱无章, 联系较弱, 实则层层拨开缭绕在美国法律制度和警局官僚体系周身的迷雾. 
\end{abstract}

\addcontentsline{toc}{section}{序}
\pagenumbering{roman}