\section{第一部分}

\paragraph*{page~020}
\begin{quotation}
    \itshape
    你还要对另外一些人动用街头智慧: 他们有的人认识死者, 有的租房给死者, 有的是死者的老板, 有的和死者上过床, 打过架, 吸过毒. 你问你自己, 他们是在说谎吗? 他们当然在说谎. 说谎是人的天性. 他们说得谎比他们平日说的多吗? 很有可能. 那么, 他们说的那些半真半假的话和你从犯罪现场所了解的相匹配吗? 还是完全就是扯淡? 你应该先对哪位大吼? 你对哪位大吼的音量应该最大? 如果你威胁起诉谋杀从犯的话, 他们中的哪一个会吃这套? 你又要对哪个威胁说他要不做证人就要不做嫌疑犯? 你还要为哪个提供台阶下------也就是所谓的``出口''------好吧, 这个可怜的杂种本就该死, 任何在他那圈子里的人都有可能杀了他, 他们杀他只是因为他挑衅了他们, 他们本不想这么做, 但结果擦枪走火了, 抑或仅仅是正当防卫.
\end{quotation}

\paragraph*{page~022}
\begin{quotation}
    \itshape
    接着, 一位助理州检察官会打通你的电话. 他基本不会有什么好脾气, 因为他是个想尽一切办法让自己的定罪率维持在水平线之上, 并期望日后能在一个优秀的刑法事务所找个职位的人.
\end{quotation}

\paragraph*{page~023}
\begin{quotation}
    \itshape
    切勿以为你修得了学位, 受过了特殊训练或读过很多专业书目便可以成为一位合格的凶案组警探, 因为当你无法读懂街头时, 世上的所有理论都是无用的.
\end{quotation}

\paragraph*{page~025}
\begin{quotation}
    \itshape
    巴尔的摩有一条不成文的规定------如果某事看上去像坨屎, 闻起来像坨屎, 吃起来还是像坨屎的话, 那么就让凶案组吞下去吧. 警局总部的食物链要求它这么做. 
\end{quotation}

\paragraph*{page~026}
\begin{quotation}
    \itshape
    当然了, 警察局长如果还想活下去的话, 那他首先要想尽方法满足市长的需求: 只有当警局不给他造成尴尬或丑闻的情况下, 他才有可能连任市长宝座; 其次, 局长得以市长觉得最合适的方式服务于他; 最后才是为了公共利益和犯罪分子搏斗. 这三件事的重要性是依次递减的. 
\end{quotation}

\paragraph*{page~028}
\begin{quotation}
    \itshape
    如果一个凶杀组警探想要生存下去, 那么他必须学会像吉普赛人阅读茶叶一样读懂这个官僚指挥系统. 当上级有问题时, 他不可以无言以对. 当他们希望把某人送上绞刑架时, 他会准备一份恰合其意的报告, 让他们以为他是每天抱着警局的指挥原则睡觉的. 如果他们只需要一个草率的答案以了事, 那么他也会学着让自己消失.
\end{quotation}

\paragraph*{page~051}
\begin{quotation}
    \itshape
    达达里奥说话温柔, 善于思辨, 他相信警局不应是个泛军事化的组织, 而这种观点鲜有人支持. 在警局里, 很多领导都会在第一时间威胁他的下属尽快破案, 然后监督他们每个举动, 指导他们的每次调查. 达达里奥一直认为这种做法是错误的, 并很早就学会抑制自己的这一冲动了. 在分局里, 这样的行为也屡见不鲜. 这通常是因为有个新官刚上任, 他觉得避免让下属把自己看扁的最佳方式是做一个小肚鸡肠的独裁者. 每个分局都有这样的轮值警督和警司------一个警探迟到十分钟打卡, 一个巡逻警半夜四点在警车里睡觉, 这都不是些大事情, 可一旦被他们发现, 这些下属就会被命令填写95表格. 这样的领导会有两种结局: 不是节节高升, 就是底下的破案好手们都纷纷离去, 落得个人走楼空.
\end{quotation}

\paragraph*{page~051}
\begin{quotation}
    \itshape
    当然, 警衔的高低固然重要, 可在凶案组, 如果一位警督每时每刻都要行使他的神圣权力, 对下属指手画脚的话, 那他最终所得的便是一整群不肯和他交心的警司与过于小心谨慎的警探. 最坏的情况是, 到那个时候, 这些人已经丧失了本能判断, 再也不愿意干活了.
\end{quotation}

\paragraph*{page~059}
\begin{quotation}
    \itshape
    ``水往低处流, 屎往低处滚. 可不是吗, 长官.'' 
\end{quotation}

\paragraph*{page~091}
\begin{quotation}
    \itshape
    直到有一天, 金凯德被叫到了副局长的办公室, 后者安抚他说他的做法完全符合指导准则, 他完全有权力这么做. 完全正确. 可如果金凯德选择在法庭上直面警长的指控的话, 虽然他很有可能会被证明是无辜的, 但他的仕途亦将随之被毁------他会从凶案组调离, 前往某个费城南部郊区很近的区域当巡逻警. 副局长给了他另外一个选择------停薪留职五天, 然后回来继续做他的警探. 金凯德屈服了, 警局运作的动力可不源自逻辑. 
\end{quotation}

\paragraph*{page~099}
\begin{quotation}
    \itshape
    每个现场都各不相同. 你会花而十分钟勘查一起发生在街上的枪击案, 你也会花十二个小时勘查一起发生在二层排屋的二人刺杀案. 无论你负责的是哪起案件, 你都得懂得平衡的哲学, 你得知道你必须做什么, 也得知道你能做什么. 
\end{quotation}

\paragraph*{page~121}
\begin{quotation}
    \itshape
    理论上说, 只有在警局其他部门干的出色的警官才有资格加入凶案组, 如果他还在警局六楼的其他调查组工作过那就好了. 事实上, 满足这个条件的警官有很多, 但他们是否能最终入选确是由其他因素决定的. 在过去的十年里, 黑人警探总比白人更受欢迎; 而如果他和某位副总警监或副局长相熟, 并是由后者一路提拔上来的话, 那对此事也颇有助益. 
\end{quotation}

\paragraph*{page~134}
\begin{quotation}
    \itshape
    沃尔登从来不知道开枪有什么好玩的, 而那付之阙如的一发子弹却永远在他内心留下了不可磨灭的印迹. 对他来说, 盾牌才是警察威慑力的体现, 而一位警察的优良和他的射术鲜有关系, 而是要看他在街头有多少震慑力. 
\end{quotation}

\paragraph*{page~135}
\begin{quotation}
    \itshape
    在那个时代, 巴尔的摩的警察会自豪地说, 他们才是这座城市中组织最大, 作风最强硬, 配置最高端的``黑帮''.
\end{quotation}

\paragraph*{page~135}
\begin{quotation}
    \itshape
    事实上, 在那个时代, 大多数和警察开枪相关的案件都带有种族歧视的色彩. 对于世世代代都生活在巴尔的摩中心城区的黑人而言, 他们早已明了, 那帮号称是城市正义天使的警察只不过是一场瘟疫. 他们人生的困难有四个源头------贫穷, 无知, 绝望, 警察. 巴尔的摩的黑人打小就知道, 他们最做不得的两件事------和警察争吵以及逃匿警察的追捕------一旦这二者之一发生了, 他们至少会挨一顿痛打, 最坏的情况则是被警察击毙. 即便是黑人社区中最有权势的人物也要让着警察两三分; 在六十年代之前, 警局和警察基本上就是负面词汇. 
\end{quotation}

\paragraph*{page~140}
\begin{quotation}
    \itshape
    这里的问题是警局不再愿意牺牲它本身的利益了, 它反而开始挑战那条关于警察工作的真理------这是一个被体制内化的观念, 即无论在何种情况下, 一个好警察开了枪, 那他的动机一定是合理的. 
\end{quotation}

\paragraph*{page~141}
\begin{quotation}
    \itshape
    警局知道这是一个谎言, 但容忍着它, 因为一旦谎言破灭, 维系警察使用武力权力的神话就将告终. 这也是个公众渴求的谎言, 因为一旦谎言破灭, 正义和邪恶之间的界限就将模糊, 而这又是多么可怕. 
\end{quotation}

\paragraph*{page~142}
\begin{quotation}
    \itshape
    对于那些无论是黑皮肤还是白皮肤的街头警察而言, ``少年''麦克吉之案告诉他们, 他们已经孤立无援了, 体制不再为他们提供庇护. 为了保存自己的权威, 警局已经开始自我净化, 它清洗了那些使用暴力和信仰暴力的人, 也打击了那些在面对突发事件时作出错误决定的警察. 如果开枪是合理的, 那么警局仍会保护你, 虽然即便最正义的抉择也得不到公众的支持. 在这个时代, 一旦这样的事情发生, 有人必然会在电视镜头前说是警察杀害了那个人. 如果这个人可开可不开, 警局也仍有可能保护你, 前提是你知道怎样写一手好报告. 而如果这全然是个错误的决定, 那么, 对不起了, 警局会不假思索地放弃你. 
\end{quotation}

\paragraph*{page~183}
\begin{quotation}
    \itshape
    这就是你的选择. 你为什么要当警察? 在此之后, 她向他抱怨道. 麦克拉尼无法给出自己的理由; 他知道, 他没有权利和她争辩. 他已经三十二岁了, 他有自己的家庭; 他是个大学毕业生, 但大多数大学毕业生赚的钱比他多一倍. 而他呢? 他却被人像刍狗一样击中, 差点亡命街头. 的确, 连麦克拉尼自己都会承认, 事实很简单也很残酷------警察就是一份吃力不讨好的活. 但他不会因为中了枪就改变自己的想法, 警察所代表的一切早已超越了他的生命.  
\end{quotation}

\paragraph*{page~184}
\begin{quotation}
    \itshape
    巴尔的摩警局有个不成文的传统: 如果有人因公受伤, 那么等他恢复之后, 他可以选择任何他胜任的职位. 
\end{quotation}

\paragraph*{page~225}
\begin{quotation}
    \itshape
    年长的亲属正在把死者的公寓洗劫一空. 家属们没有时间悲伤, 死者的母亲决定在今晚必须把遗物都分干净, 不能让那些盗贼得逞. 
\end{quotation}

\paragraph*{page~228}
\begin{quotation}
    \itshape
    他之所以如此在意破案率, 倒不是因为虚荣心作祟; 在他看来, 破案率是衡量工作的核心标准. 贾尔维富有行动力, 做事麻利, 追求完美, 他不但喜欢破案, 也难得地没有把破案当作纯粹的工作; 如果某起案件还未告破, 或者某起案件的证据仍然不足, 都会让他感到十分不安. 这让他成了警局古老道德的活化石. 要知道, 先于他们一两代的警察都是抱着 ``不成功便成仁''的态度破案的, 而后, 所有的巴尔的摩公务员的箴言便成了``这不归我管''. 现如今, 他们的口头禅更为明哲保身------``万物自有其法''.
\end{quotation}

\paragraph*{page~229}
\begin{quotation}
    \itshape
    那个时候, 警察声誉扫地, 很少有人会说做警察是自己的梦想, 可贾尔维却不同. 当然, 他自有现实的考量. 他认为警察这份工作很有趣, 但他十分明了, 即便是经济最衰败的时代, 警察也是个铁饭碗. 然而, 等到快要毕业的时候, 他蓦然发现, 即便连警察都不是铁饭碗了. 
\end{quotation}

\paragraph*{page~239}
\begin{quotation}
    \itshape
    当然, 两组轮值人马的破案率差距有其原因, 但是等到这组数据到了上级那儿时, 他们的结论就只有一个------斯坦顿的警探们更懂得如何破案, 而达达里奥的人则不懂. 事实上, 在达达里奥那一组人手所处理的案件里, 五分之三是涉毒案; 而斯坦顿破获的十起案件中则有七起是家庭暴力或其他争执导致的凶杀, 其破案难易差距可想而知------但上级是不会听你的解辫的. 
\end{quotation}

\paragraph*{page~239}
\begin{quotation}
    \itshape
    所有这些事实都需要解释. 可像至尊法典一般的 ``板儿''却只会显示破案率. 对数据的膜拜已是所有现代警局的通行观念. 只有当数据漂亮时, 警监才能成为警长, 警长才能成为总警监, 总警监才能成为副局长; 当数据糟糕时, 所有这些领导的晋升之路都会像一条污水管道一般堵住. 
\end{quotation}

\paragraph*{page~240}
\begin{quotation}
    \itshape
    如果有人胆敢指出这个数据不真实, 那么我可以告诉你, 没有一个数据是真实的. 只要在警局的计划及研究部待上一星期, 你便会明了, 盗窃组的破案率并不意味着的确有那么多罪犯被逮捕, 犯罪率的上升也不意味着真有那么多人犯罪, 事实上这或许和警局想要申请的预算有关. 凶杀案的破案率同样可以做手脚------只要它们不违背FBI有关罪犯上报的规矩就行. 
\end{quotation}

\paragraph*{page~245}
\begin{quotation}
    \itshape
    达达里奥已经勤勤恳恳地干了八年警督了, 可在警局上层看来, 这并不重要. 对他们来讲, 唯一重要的是近期发生的红球事件. 警局官僚系统信仰的是一种务实的政治哲学------你最近为我做了什么? 
\end{quotation}

\paragraph*{page~245}
\begin{quotation}
    \itshape
    如果破案率漂亮而红球案件亦悉数告破的话, 那么, 上层是不会管达达里奥是怎样管理自己的团队的. 你说你会给手下的警探和警司很大的自主权, 放手让他们自己做判断------很明显, 一个好领导就是要把信心和责任心灌输给下属. 你说你会把权力下放给警司, 让他们训练和管理警探------很明显, 一个好领导懂得科学地分配权力. 你说你的加班费超过了预算的百分之九十------没问题, 想吃蛋饼也不得先打破几个鸡蛋吗? 你说你的法庭出席费也超额了-------好吧, 这至少证明更多杀人犯被起诉了. 只要破案率漂亮, 一切都好说. 然而, 一旦破案率直线下降, 再棒的警督也是一坨屎------他没能力指导和管理下属, 他放权过多, 他控制不了成本.  
\end{quotation}

\paragraph*{page~247}
\begin{quotation}
    \itshape
    法勒泰齐指出, 如果那一天真的到来, 那也是凶案组自作自受; ``我们每年都给他们看高于平均的破案率, 所以他们每年都觉得天下太平.''

    ``这话不假.''诺兰说.

    ``所以啊, ''法勒泰齐继续说, ``当我们回过头来问他们要更多的人手, 更好的警车和无线电, 更强的训练或者随便什么东西时, 他们就会看着往年的破案率说, `去他妈的, 去年他们没有这些玩意破案率不也好好吗?' ''

    ``这就叫做`自作孽不可活', 现在报应要来了.'' 诺兰说. 
\end{quotation}

\paragraph*{page~253}
\begin{quotation}
    \itshape
    谁都知道, 警察和凶手之间除了对立之外别无其他关系可言. 可这个窗户的幻象却会让凶手以为他们站在同一战线上. 当然, 这是个谎言, 这是个欺骗, 这是种操控. 这是警探在扮演某个莫须有的角色, 然后全方位地控制你的思考. 你得明白, 所谓审讯室就是一个舞台, 你和警探就是演员, 你以为你找到了你们之间的共通点, 可那只是你的幻想. 在这个被警察操控的炼狱中, 有罪的人会在不经意之间, 鲜有忏悔地坦白自己的罪.
\end{quotation}

\paragraph*{page~258}
\begin{quotation}
    \itshape
    这门艺术的精华便是控制. 嫌疑人总是坐在离门最远的地方, 这是为了控制; 审讯室房间只有用钥匙才能打开, 而钥匙则在警探手里, 这也是为了控制. 每当嫌疑人提出请求或警探主动向他提出``要不要来一支烟'', ``要不要喝水'', ``要不要喝咖啡''或者``要不要上厕所''时, 警探都是在提醒他------他已经被控制了. 
\end{quotation}

\paragraph*{page~259}
\begin{quotation}
    \itshape
    控制. 为了保持控制, 你必须一个劲地说. 你翻来倒去地说个不停, 直到你觉得安全为止. 因为一旦嫌疑人察觉到他也可以控制审讯的走向时, 他就会要求律师, 而你就完蛋了. 
\end{quotation}

\paragraph*{page~260}
\begin{quotation}
    \itshape
    这也需要角色扮演, 只有经验老道的警探才能演好这出戏. 如果你的嫌疑人或证人脾气爆烈, 你就用更加暴烈的脾气压制他. 如果他显得有点害怕, 你就给他安慰. 如果他看上去很弱, 你则要强势. 如果他看上去孤助无缘, 你就给他开个玩笑, 然后再给他一瓶苏打水. 如果他信心满满, 那你要比他更有信心, 你要告诉他, 他铁定被定罪, 而你只是想知道一些无关紧要的细节. 如果他很傲慢, 如果他不想配合, 那你要威胁他, 吓他, 让他知道, 除了你, 没人能救他于水火之中了.  
\end{quotation}

\paragraph*{page~262}
\begin{quotation}
    \itshape
    在审讯室里, 撒谎是要有限度的, 它基于警探目前所得的事实------也基于嫌疑人本人的智商高低------无论警探低估了自己的嫌疑人, 还是他吹嘘了自己对现场信息的了解, 他都会失去好不容易才建立起来的信任感. 
\end{quotation}

\paragraph*{page~315}
\begin{quotation}
    \itshape
    沃尔登与生俱来的权威感超越了他本职工作所限定的范围. 这个人可是个天生的警探啊. 
\end{quotation}

\paragraph*{page~318}
\begin{quotation}
    \itshape
    不过, 沃尔登并不仅仅是为了别人才不退休的. 他能清晰地听见自己内心的呼唤------再没有比凶案组更适合你的工作了, 你天生就是干警探的料, 而你还有足够的时间来享受这份工作. 说实在的, 这其实都是沃尔登的自我暗示. 
\end{quotation}

\paragraph*{page~326}
\begin{quotation}
    \itshape
    警监向达达里奥坦诚道, 人事变动的压力来自警长. 让达达里奥颇感失望的是, 警监并没有表示对他的支持, 反而也对创新低的破案率抱怨了一通. 达达里奥仿佛听见了警监内心想问, 却又碍于情面没敢问的问题: ``如果你不是问题, 那谁又是呢?'' 
\end{quotation}

\paragraph*{page~327}
\begin{quotation}
    \itshape
    达达里奥缺少高层的朋友. 于是, 他只有两个选择: 吞下这颗苦果, 接受调任, 去另一个小组工作; 或者, 再撑一段时间, 希望破案率在段时间内上爬, 且至少有一起红球案件告破. 如果他执意留任, 他的上级会继续向他施加压力让他离开, 但他知道, 一切不可能来得那么快, 让一个警督调任的程序本身就够烦, 够花时间了. 上级先要准备好充分的理由, 而后还有很多程序文件需要填写交接. 当然, 他必然是这场战役的失败者, 可警局的上层也不会好受------警监和警长也都明白这一点. 
\end{quotation}

\paragraph*{page~327}
\begin{quotation}
    \itshape
    与此同时, 达达里奥也明白, 自己强硬留任的决定会伤害到他底下的人------如果破案率持续走低的话, 他无法再像以前那样保护他们了. 他们得在上层面前做出一副模样来: 每个警探都得兢兢业业, 一丝不苟, 而达达里奥则要表现出足够的威慑力让上层以为一切尽在他的掌控之中. 加班费没那么好拿了, 那些负责更少案件的警探则需要迎头赶上. 每个警探都得时刻警惕自己的工作, 写好每一个案件的每一份报告与卷宗, 千万别给上级落下口实. 达达里奥知道, 这是顺应警局官僚主义的做法, 警探花越多时间写报告, 留给他实际破案的时间也就越少. 可是, 这便是权力的游戏, 而此时此刻, 达达里奥和他的手下不得不玩一场这样的游戏. 
\end{quotation}

\paragraph*{page~329}
\begin{quotation}
    \itshape
    达达里奥像模样地玩着这个游戏. 他给那些正在朝百分之五十冲刺的警探发了警告信------并抄送给了警监和警长, 然后把他们调到了白班. 令他欣慰的是, 他的警司和警探们都很配合. 所有人都知道这完全是无理取闹, 但也理解达达里奥今日之无奈处境. 如果他们想造反的话, 其实很简单------他们只要联手还在午夜值班的警探, 让他们懈怠工作, 并将越积越多的未破案件怪罪在这一荒唐政策头上, 警局上级也只会束手无策. 毕竟, 谋杀案是这个世界上最无法预测的事情. 
\end{quotation}

\paragraph*{page~329}
\begin{quotation}
    \itshape
    诺兰很拿他的警司徽章当一回事, 他也喜欢在这样一个准军事化的组织里工作. 他比凶案组的大多数人都乐于遵守警局的官僚运作体系------按级别区分高低贵贱, 对体制毫无保留的忠诚, 且不越级办事. 这倒不意味着诺兰是个难以相处的警司; 他对手下的保护比凶案组其他警司有过之而无不及, 他的警探从来都可以安心办案, 知道没人可以越过诺兰来搞他. 
\end{quotation}

\paragraph*{page~332}
\begin{quotation}
    \itshape
    诺兰历经磨难, 堪称少数没有被警局官僚体制抹杀的幸存者. 因此, 他也格外在意, 对他的警衔和地位感到骄傲. 他指挥起人总是煞有介事, 每当朗兹曼, 麦克拉尼或者达达里奥带着过分嬉笑的成分命运自己的下属时, 他总是会对他们感到失望. 每次凶案组开警司以上级别会议时, 他总是会提出管理和运营团队的新方案------有些值得采纳, 有些则一无是处, 但这些方案基本都是关乎程序的问题. 他的建议从来不会得到严肃的对待, 会议也总是开不长: 朗兹曼会嘲笑他肯定是脑子有问题, 让他赶紧去抽几根大麻醒醒神; 接着, 麦克拉尼会说起一个和他提议毫不相关的笑话; 最后, 让诺兰倍感挫折的是, 达达里奥竟然也不待见他, 就此宣告会议结束. 这三位警司属于不同的世界: 朗兹曼和麦克拉尼喜欢就事论事, 应该一辈子都要栽在案件这苦活里了; 而诺兰却是块管理人才的料.
\end{quotation}

\paragraph*{page~332}
\begin{quotation}
    \itshape
    不过, 诺兰手下的警探不是束手束脚, 一丁点自由都没有. 诺兰在意的是文件, 管理及人事问题, 但从本质上说, 凶案组的责任便是破案------诺兰可不会在这个问题上为难任何一位警探. 他的手下都按照自己的速度和风格办案, 而诺兰从来不会插手. 
\end{quotation}

\paragraph*{page~341}
\begin{quotation}
    \itshape
    那个瘦成一条电线杆样的毒贩正靠在警车边上, 一顶黑色贝雷帽盖住了他的前额. 他穿着高帮乔丹牌气垫球鞋, 约达西牌牛仔裤和耐克牌的T恤------贫民区孤魂野鬼的标准装扮. 
\end{quotation}

\paragraph*{page~352}
\begin{quotation}
    \itshape
    如果艾杰尔顿也已经成为警监和达达里奥对峙中的关键筹码. 如果艾杰尔顿能破案, 那么, 他便证明达达里奥这位轮值警督有能力管好自己的下属; 而如果他不能的话, 他便会沦为警监和行政警督指责达达里奥疏于管理的证据. 
\end{quotation}

\paragraph*{page~354}
\begin{quotation}
    \itshape
    领导就是领导. 自己人之间可以无所顾忌地放开了说话, 可向上级打小报告就是另一回事了.
\end{quotation}

\paragraph*{page~358}
\begin{quotation}
    \itshape
    法庭里座无虚席. 很多西区的制服警都来了. 然而, 西区警局局长并没有出席, 我们也看不见巡逻警主管或副局长的身影------每个出席的普通警察都苦涩地意识到了这一点. 一旦某位警察因公受了伤, 他便失去了上级对他的庇护; 高层会来医院看望你, 也肯定会出席你的葬礼, 可是, 他们对你的记忆总是短暂得有些残酷. 席间的警察没有一个警衔是超过警司一级的.  
\end{quotation}

\paragraph*{page~393}
\begin{quotation}
    \itshape
    最近, 沃尔登根本不同别人交流, 他独自酝酿着这场愤怒, 仿佛随时都要爆发. 就在这早班交班的时刻, 根本没人敢靠近他宽慰一句. 不过, 实话实说, 对于沃尔登的愤怒, 其他人又有什么话好讲呢? 这是一个有着自己的尊严, 按照自己的行事原则工作了一辈子的警探, 可现在, 他那份宝贵的尊严成了政治家们讨价还价的筹码. 在警局工作二十五年里, 对体制的忠诚早已变成了他的血肉, 他生活的方式, 可现在, 突然之间, 这个体制却背叛了他. 当这样的情况出现时, 宽慰还有用吗?
\end{quotation}

\paragraph*{page~395}
\begin{quotation}
    \itshape
    沃尔登是这么考虑的: 无论如何, 总得有人要来负责此案, 而一旦年轻探员接手这样的案子, 他的一辈子就可能要栽在这上面了. 要牺牲, 就牺牲老头子吧. 最初负责此案的斯丹顿分队里的那两个警探早已聪明地溜之大吉. 贾尔维也庆幸自己及时摆脱了这个烫手山芋. 可我沃尔登却不怕. 在接手此案后, 沃尔登这样对他的同事说. 不过, 在外人看来, 他听上去更像是要说服他自己.
\end{quotation}

\paragraph*{page~398}
\begin{quotation}
    \itshape
    沃尔登和詹姆斯仔细研究了这位记者的报道, 从报道所披露的信息来判断, 两人都觉得泄密者是警局内部的人. 他们猜测合情合理: 在警局高层, 并不是每个人都是这位议员的政治同盟, 而对绑架事件的暴光显然能让议员脸上蒙灰. 当然, 一旦机密泄露, 每个政客都会演起信息透明, 铁面无私的好戏来.  
\end{quotation}

\paragraph*{page~399}
\begin{quotation}
    \itshape
    可如今, 这一切都无关紧要了. 如今, 领导们啥都不想要了, 他们只要警探们调查拉里杨议员的私生活和他那场莫须有的绑架事件. 警局会派几位最优秀的探员证明议员是在说谎, 三个绑匪开着一辆面包车绑架这件事完全没有发生过. 然后, 议员会因谎报案情而遭起诉------虽说这只是一个蝇头小罪------并接受裁决. 而裁决的结果人人心知肚明: 检察官和警局根本不会赢. 这只是一场战略部署, 只是一场为了安抚民众而上演的好戏. 无论沃尔登说什么------他代表举步维艰的凶案组所说的每一句话------都不再重要了. 对于整个警局而言, 那完全是可以牺牲的筹码. 
\end{quotation}

\paragraph*{page~400}
\begin{quotation}
    \itshape
    话已至此, 达达里奥完全猜不透警长的心思了, 于是, 他闭上了嘴. 警长这么说是在赐予他免死令牌吗?------只要牺牲沃尔登一个人就够了, 他们两人都可以明哲保身------如果那样的话, 达达里奥希望警长也明白了沉默的含义: 他会坚决站在沃尔登这一边, 生死同归. 而如果警长只是随口那么一说而并不是话里有话, 那么, 不予回应也是最好的选择.  
\end{quotation}

\paragraph*{page~401}
\begin{quotation}
    \itshape
    达达里奥无法得出结论, 但他和朗兹曼都同意, 如果上级真的要牺牲沃尔登, 那么, 他们也将对警长宣战, 誓死不休. 在警局跌打滚爬这么多年, 达达里奥早已对官僚体系毫无底线的下作行径心知肚明, 但让沃尔登来做替罪羊这个主意仍让他不寒而栗. 沃尔登是凶案组最优秀的警探之一, 怎么可以一遇到危机就不容商量地牺牲掉呢? 

    尽管达达里奥用沉默的方式拒绝了警长牺牲唐纳德沃尔登的主意, 他的所作所为很快就在整个轮值队伍里传开了. 警探们都说警督是个好人, 不枉费他们替他卖命这么多年. 

    虽说在此之前, 达达里奥曾因破案率过低而向上级屈服过, 但这算不上对他底下人的背叛. 如果换种眼光看, 他这么做, 其实也是一种迂回之计, 可以暂时缓解上级施压, 让底下人不受干扰地继续工作. 而现在, 那个在本年早些时间让达达里奥备受质疑的破案率再一次站在了他这一边. 夏天是凶案的高峰期, 可是, 他的轮值队伍竟然保持着百分之七十的高破案率. 这位警督不再受质疑, 反而再次得到了上级的器重. 破案率成了达达里奥手中的牌. 
\end{quotation}

\paragraph*{page~404}
\begin{quotation}
    \itshape
    在过去的六个月里, 沃尔登啃着一堆难以下咽的硬骨头, 可现在, 巴尔的摩却愿意为了一个政客的闹剧向他支付没有上限的加班费. 
\end{quotation}

\paragraph*{page~404}
\begin{quotation}
    \itshape
    沃尔登的话. 当他还在西北分局工作时, 没人敢不听他的话; 当他还在逃犯缉捕组工作时, 他的话一言九鼎; 当他还在刑事调查部盗窃组工作时, 如果你发现正在和沃尔登一起勘查现场而他对你说了什么话时, 你完全可以把他的话当作事实接受下来. 可是, 他现在身在凶案组------这个出尔反尔, 背信弃义的地方------他再次得到了教训: 在这里, 老板们说的话那才算话. 

    ``无论怎样, '' 他对窗外吐了口烟圈, 对詹姆斯说, ``他们至少改变不了你的就职日期.''
\end{quotation}



