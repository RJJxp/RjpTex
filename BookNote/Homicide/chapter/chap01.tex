\section{第一部分}

\paragraph*{page~001}
\begin{quotation}
    \itshape
    我所谓的伟大不是走红运的政治家或是立战功的军人的伟大; 这种人显赫一时, 与其说是他们本身的特质倒不如说沾了他们地位的光, 一旦事过境迁, 他们的伟大也就黯然失色了. 人们常常发现一位离了职的首相当年只不过是个大言不惭的演说家; 一个卸甲归田的将军无非是个平淡无味的市井英雄. 但是查理斯$\cdot$思特里克兰德的伟大却是真正的伟大. 
\end{quotation}

\paragraph*{page~002}
\begin{quotation}
    \itshape
    崇拜者对他的赞颂同贬低者对他的诋毁固然都可能出于偏颇和任性, 但有一点是不容置疑的, 那就是他具有天才.

    而那位克里特岛画家的作品却有一种肉欲和悲剧性的美, 仿佛作为永恒的牺牲似地把自己灵魂的秘密呈现出来.
\end{quotation}

\paragraph*{page~005}
\begin{quotation}
    \itshape
    制造神话是人类的天性. 对那些出类拔萃的人物, 如果他们生活中有什么令人感到诧异或者迷惑不解的事件, 人们就会如饥似渴地抓住不放, 编造出种种神话, 而且深信不疑, 近乎狂热. 这可以说是浪漫主义对平凡暗淡的生活的一种抗议. 
\end{quotation}

\paragraph*{page~008}
\begin{quotation}
    \itshape
    但他的目光敏锐, 一眼就望穿了隐含在一些天真无邪的行为下的可鄙的动机. 他既是一个艺术研究者, 又是一个心理--病理学家. 他对一个人的潜意识了如指掌. 没有哪个探索心灵秘密的人能够像他那样透过普通事物看到更深邃的意义. 探索心灵秘密的人能够看到不好用语言表达出来的东西, 心理病理学家却看到了根本不能表达的事物.
\end{quotation}

\paragraph*{page~011}
\begin{quotation}
    \itshape
    我不记得是谁曾建议过, 为了使灵魂宁静, 一个人每天要做两件他不喜欢的事. 说这句话的是个聪明人, 我也一直在一丝不苟地按照这条格言行事: 因为我每天早上都起床, 每天也都上床睡觉.
\end{quotation}

\paragraph*{page~012}
\begin{quotation}
    \itshape
    战争来了, 战争也带来了新的生活态度. 年轻人求助于我们老一代人过去不了解的一些神祗, 已经看得出继我们之后而来的人要向哪个方向活动了. 年轻的一代意识到自己的力量, 吵吵嚷嚷, 早已经不再叩击门扉了. 他们已经闯进房子里来, 坐到我们的宝座上, 空中早已充满了他们喧闹的喊叫声. 老一代的人有的也模仿年轻人的滑稽动作, 努力叫自己相信他们的日子还没有过去; 这些人同那些最活跃的年轻人比赛喉咙, 但是他们发出的呐喊听起来却那么空洞, 他们有如一些可怜的浪荡女人, 虽年华已过, 却仍然希望靠涂脂抹粉, 靠轻狂浮荡来恢复青春的幻影. 聪明一点儿的则摆出一副端庄文雅的姿态. 他们的莞尔一笑中流露着一种宽容的讥诮. 他们记起了自己当初也曾经把一代高踞宝座的人践踏在脚下, 也正是这样大喊大叫, 傲慢不逊; 他们预见这些高举火把的勇士们有朝一日也同要让位于他人. 谁说的话也不能算最后拍板. 说这些豪言壮语的人可能还觉得他们在说一些前任未曾道过的真理, 但是实际上连他们说话的腔调前任也已经用过一百次, 而且丝毫没有变化. 钟摆摆过来又荡过去, 这一旅程反复循环. 
\end{quotation}

\paragraph*{page~013}
\begin{quotation}
    \itshape
    但是像济慈同华兹华斯写的颂歌, 柯勒律治的一两首诗, 雪莱的更多的几首, 确实发现了前任未曾探索过的广阔精神领域. 克莱布先生已经陈腐过时了, 但是克莱布先生还是孜孜不倦地继续写他的押韵对句诗. 我也断断续续读了一些我们这一时代的年轻人的诗作, 他们当中可能有一位更炽情的济慈或者梗一尘不染的雪莱, 而且已经发表了世界将长久记忆的诗章, 这我说不定. 我赞赏他们的优美词句--尽管他们还年轻, 却已才华横溢, 因此如果仅仅说他们很有希望, 就显得荒唐可笑了---, 我惊叹他们精巧的文体; 但是虽然他们用词丰富, 却没有告诉我们什么新鲜东西. 在我看来, 他们知道的太多, 感觉过于肤浅; 对于他们拍我肩膀的那股亲热劲儿同闯进我怀抱时的那种感情, 我实在受不了. 我觉得他们的热情似乎没有血色, 他们的梦想也有些平淡. 我不喜欢他们. 我已经是过时的老古董了.
\end{quotation}

\paragraph*{page~015}
\begin{quotation}
    \itshape
    我想在过去的日子里我们都羞于使自己的感情外露, 因为怕人嘲笑, 所以都约束着自己不给人以傲慢自大的形象. 我并不认为当时风雅放浪的诗人作家执身如何端肃, 但我却不记得那时候文艺界有今天这么多风流韵事. 我们对自己一些荒诞不经的行为遮上一层保持体面的缄默, 并不认为这是虚伪. 我们讲话讲究含蓄, 并不总是口无遮拦, 说什么都直言不讳.
\end{quotation}

\paragraph*{page~017}
\begin{quotation}
    \itshape
    艺术家较之其他行业的人有一个有利的地方, 他们不仅可以讥笑朋友的性格和仪表, 而且可以嘲弄他们的著作. 他们的评论恰到好处, 话语滔滔不绝, 我实在望尘莫及. 在那个时代谈话仍然被看作是一种需要下功夫陶冶的艺术, 一句巧妙的对答比锅子底下噼啪爆响的荆棘更受人赏识, 格言警句当时还不是痴笨的人利用来冒充聪敏的工具, 风雅人物的闲谈中随便使用几句会使得谈话妙趣横生. 
\end{quotation}

\paragraph*{page~025}
\begin{quotation}
    \itshape
    思特里克兰德太太是很会同情人的. 同情体贴本是一种很难得的本领, 但是却常常被那些知道自己有这种本领的人滥用了. 他们一看到自己的朋友有什么不幸就恶狠狠地扑到人们身上, 把自己地全部才能施展出来, 这就未免太可怕了. 同情心应该像一口油井一样喷薄而出; 惯爱表同情的人让它纵情奔放, 反而使那些受难者非常困窘. 
\end{quotation}

\paragraph*{page~027}
\begin{quotation}
    \itshape
    她笑了, 她的笑容很甜, 脸上微微泛起一层红晕; 像她这样年纪的女人竟这么容易脸红, 是很少有的. 也许她最迷人的地方就在于她的纯真.

    她用这个词一点儿也没有贬抑的意思, 相反地, 倒是怀着一股深情, 好像由她自己说出他最大的缺点就可以保护他不受她朋友们的挖苦似的.
\end{quotation}

\paragraph*{page~030}
\begin{quotation}
    \itshape
    文明社会这样消磨自己的心智, 把短促的生命浪费在无聊的应酬上实在令人莫解.
\end{quotation}

\paragraph*{page~032}
\begin{quotation}
    \itshape
    很清楚, 他一点儿也没有社交的本领, 但这也不一定人人都要有的. 他甚至没有什么奇行怪癖, 使他免于平凡庸俗之嫌. 他只不过是个一个忠厚老实, 索然无味的普通人. 一个人可以钦佩他的为人, 却不愿意同他待在一起, 他是一个毫不引人注意的人. 他可能是一个令人起敬的社会成员, 一个诚实的经纪人, 一个恪尽职责的丈夫和父亲, 但是在他身上你没有任何必要浪费时间.
\end{quotation}

\paragraph*{page~034}
\begin{quotation}
    \itshape
    如果纯粹从善于辞令这一角度衡量一个人智慧, 也许查理斯$\cdot$思特里克兰德不算聪明, 但是在他自己的那个环境里, 他的智慧还是绰绰有余的, 这不仅是事业成功的敲门砖, 而且是生活幸福的保障.

    这一定是世间无数夫妻的故事. 这种生活模式给人以安详亲切之感. 它使人想到一条平静的小河, 蜿蜒流过绿茸茸的牧场, 与郁郁的树荫交相掩映, 直到最后泻入烟波浩瀚的大海中. 
\end{quotation}

\paragraph*{page~048}
\begin{quotation}
    \itshape
    尽管她悲痛的感情是真实的, 却没忘记使自己的衣着合乎她脑子里的礼规叫她扮演的角色. 
\end{quotation}

\paragraph*{page~056}
\begin{quotation}
    \itshape
    现在我的眼睛已经看不到思特里克兰德太太一副痛楚不堪的样子, 好像能够更冷静地考虑这件事了. 我在思特里克兰德太太的举动里发现一些矛盾, 感到疑惑不解. 她非常不幸, 但是为了激起我的同情心, 她也会把她的不幸表演给我看. 她显然准备要大哭一场, 因为她预备好大量的手帕; 她这种深思远虑虽然使我佩服, 可是如今回想起来, 她的眼泪的感人力量却不免减低了. 我看不透她要自己丈夫回来是因为爱他呢, 还是因为怕别人议论是非; 我怀疑使她肠断心伤的失恋之痛是否也掺杂着虚荣心受到损害的悲伤; 这种疑心也使我很惶恐. 我那时还不了解人性多么矛盾, 我不知道真挚中含有多少做作, 高尚中蕴藏着多少卑鄙, 或者, 即使在邪恶里也找得到美德. 
\end{quotation}

\paragraph*{page~065}
\begin{quotation}
    \itshape
    ``别的都不要说了, 你总不能一个铜板也不留就把你女人甩了啊!''

    ``为什么不能?''

    ``她怎么活下去呢?''

    ``我已经养活她十七年了. 为什么她不能换换样, 自己养活自己呢?''

    ``她养活不了.''

    ``她不妨试一试.''

    不论对哪方面讲, 这都是一件极端严肃的事, 可是他的答话却带着那么一种幸灾乐祸, 厚颜无耻的劲儿; 为了不笑出声来, 我拼命咬住嘴唇. 我一再提醒自己他的行为是可恶的. 我终于激动起自己的义愤来. 
\end{quotation}

\paragraph*{page~074}
\begin{quotation}
    \itshape
    他表白自己似乎非常困难, 倒好像言语不是他的心灵能运用自如的工具似的. 你必须通过他的那些早被人们用得腐烂不堪的词句, 那些粗俗的俚语, 那些既模糊又不完全的手势才能猜测他的灵魂的意图. 但是虽然他说不出什么有意义的话来, 他的性格中却有一种东西使你觉得他这人一点也不乏味. 或许这是由于他非常真挚.
\end{quotation}

\paragraph*{page~078}
\begin{quotation}
    \itshape
    有一些人通过激变, 有如愤怒的激流把石块一下子冲击成齑粉; 另一些人则由于日积月累, 好像不断的水滴, 迟早要把石块磨穿. 思特里克兰德有着盲信者的直截了当和使徒的狂热不羁. 
\end{quotation}

\paragraph*{page~079}
\begin{quotation}
    \itshape
    思特里克兰德咯咯地笑起来. 他似乎一点也没有灰心丧气. 别人的意见对他是毫无影响的. 在我同他打交道的时候, 正是这一点使我狼狈不堪. 有人也说他不在乎别人对他的看法, 但这多半是自欺欺人. 一般说来, 他们能够自行其是是因为相信别人都看不出他们的怪异想法; 最甚者也是因为有几个近邻知交表示支持, 才敢违背大多数人的意见行事. 如果一个人违反传统实际上是他这一阶层人的常规, 那他在世人面前做出违反传统的事倒也不困难. 相反地, 他还会为此洋洋自得. 他既可以标榜自己的勇气又不致冒什么风险. 但是我总觉得事事要邀获别人的批准, 或许是文明人类最根深蒂固的一种天性. 一个标新立异的女人一旦冒犯了礼规, 招致了唇枪舌剑的物议, 在没有谁会像她那样飞快地跑去寻找尊严体面的庇护了. 那些告诉我他们毫不在乎别人对他们看法的人, 我是绝不相信的. 这只不过是一种无知的虚张声势. 他们意思是: 他们相信别人根本不会发现自己的微疵小瑕, 因此更不怕别人对这些小过失加以谴责了. 

    但是这里却有一位真正不计较别人如何看待他的人, 因而传统礼规对他一点也奈何不得. 
\end{quotation}

\paragraph*{page~080}
\begin{quotation}
    \itshape
    ``有一句格言你显然并不相信: 凡人立身行事, 务使每一行为堪为万人楷模.''

    ``我从来没听说过, 但这是胡说八道.''

    ``你不知道, 这是康德说的.''

    ``随便是谁说的, 反正是胡说八道.''
\end{quotation}

\paragraph*{page~080}
\begin{quotation}
    \itshape
    我把良心看作是一个人心灵中的卫兵, 社会为要存在下去制订出的一套礼规全靠它来监督执行. 良心是我们每人心头的岗哨, 它在那里执勤站岗, 监视着我们别做出违法的事情来. 它是安插在自我的中心堡垒中的暗探. 因为人们过于看重别人对他的意见, 过于害怕舆论对他的指责, 结果自己把敌人引进大门里来; 于是它就在那里监视着, 高度警觉地卫护着它主人的利益, 一个人只要有半分离开大溜儿地想法, 就马上受到它严厉苛责. 它逼迫着每一个人把社会利益置于个人至上. 它是把个人拘系于整体地一条牢固的链条. 人们说服自己, 相信某种利益大于个人利益, 甘心为它效劳, 结果沦为这个主子的奴隶. 他把他高举到荣誉的宝座上. 最后, 正如宫廷里的弄臣赞颂皇帝按在他肩头的御杖一样, 他也为自己有着敏感的良心而异常骄傲.

    亲爱的朋友, 我就希望你能够叫她看清这一点. 可惜女人都是没有脑子的. 
\end{quotation}

\paragraph*{page~089}
\begin{quotation}
    \itshape
    思特里克兰德太太不耐烦地耸了耸肩膀. 我觉得我对她有些失望. 当时我还同今天不一样, 总认为人的性格是单纯统一的; 当我发现这样一个温柔可爱的女性报复心居然这么重的时候, 我感到很丧气. 那时我还没认识到一个人的性格是极其复杂的. 今天我已经认识到这一点: 卑鄙与伟大, 恶毒与善良, 仇恨与热爱是可以互不排斥地并存在同一颗心里的. 
\end{quotation}

\paragraph*{page~094}
\begin{quotation}
    \itshape
    天天做的事几乎一摸一样, 使我感到厌烦得要命. 我的朋友过着老一套的生活, 平淡无奇, 再也引不起我的好奇心了. 有时候我们见了面, 不待他们开口, 我就知道他们要说什么话了. 就连他们的桃色事件也都是枯燥乏味的老一套. 我们这些人就像从终点站到终点站往返行驶的有轨电车, 连乘客的数目也能估计个八九不离十. 生活被安排得太有秩序了. 
\end{quotation}

\paragraph*{page~095}
\begin{quotation}
    \itshape
    但是尽管如此, 她却认为自己谋生糊口有失身份, 总有些抬不起头来. 同别人谈话的时候, 她忘不了向对方表白自己的高贵出身, 动不动就提到她认识得一些人物, 叫你知道她得社会地位一点没有降低. 对自己经营打字行业的胆略和见识她不好意思多谈, 但是一说起第二晚上要在一位家住南肯星顿的皇家法律顾问那里吃晚饭, 却总是眉飞色舞. 

    思特里克兰德太太这种孤芳自赏的态度叫我心里有点儿发凉. 

    但是我知道她答应做这件事并不是出于仁慈的心肠. 有人说灾难不幸可以使人性高贵, 这句话并不对; 叫人做出高尚行动的有时候反而是幸福得意, 灾难不幸在大多数情况下只能使人们变得心胸狭小, 报复心更强. 
\end{quotation}

\paragraph*{page~99-100}
\begin{quotation}
    \itshape
    他很大方; 那些手头拮据的人一方面嘲笑他那么天真地轻信他们变造的不幸故事, 一方面厚颜无耻地伸手向他借钱. 

    他的生活好像是按照那种充满打闹的滑稽剧的格式写的一出悲剧. 

    他谈话时那种又急切又热情, 双手挥舞的神情总是跃然纸上. 
\end{quotation}



