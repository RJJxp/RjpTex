\section{第二部分}

\paragraph*{page~103}
\begin{quotation}
    \itshape
    她并不是他那种相思倾慕的幻觉中的令人神驰目眩的美女, 但是却另有一种端庄秀丽的风姿. 
\end{quotation}

\paragraph*{page~109}
\begin{quotation}
    \itshape
    为什么你认为美---世界上最宝贵的财富---会同沙滩上的石头一样, 一个漫不经心的过路人随随便便地就能够捡起来? 美是一种美妙, 奇异的东西, 艺术家只有通过灵魂的痛苦折磨才能从宇宙的混沌中塑造出来. 在美被创造出以后, 它也不是为了叫每个人都能认出来的. 想要认识它, 一个人必须重复艺术家经历过的一番冒险. 他唱给你的是一个美的旋律, 要是想在自己心里重新听一遍就必须有知识, 有敏锐的感觉和想象力. 
\end{quotation}

\paragraph*{page~116}
\begin{quotation}
    \itshape
    他主动谈起来他的生活来. 但是由于他太无口才, 对他自己这一段时间的经历讲得支离破碎, 许多空白都需要我用自己的想象去填补. 对于这样一个我深感兴趣的人只能了解个大概, 这真是一件吊人胃口的事, 简直像读了一部残缺不全的稿本. 
\end{quotation}

\paragraph*{page~118}
\begin{quotation}
    \itshape
    从他的谈话里我了解到, 他在绘画上遇到的困难很大, 因为他不愿意接受别人指点, 不得不浪费许多事件摸索一些技巧上的问题, 其实这些问题过去的画家早已逐一解决了. 他在追求一种我不太清楚的东西, 或许连他自己也知道得并不清楚.

    我有一种感觉, 他好像把自己得强烈个性全部倾注在一张画布上, 在奋力创造自己心灵所见到的景象时, 他把周围的一切事物全都忘记了. 
\end{quotation}

\paragraph*{page~122}
\begin{quotation}
    \itshape
    我猜想你是这样的一种情况. 一连几个月你脑子里一直不想这件事, 你甚至可以使自己相信, 你同这件事已经彻底绝缘了. 你为自己获得了自由而高兴, 你觉得终于成为自己灵魂的主人了. 你好像昂首于星斗中漫步. 但是突然间, 你忍不住了. 你发觉你的双脚从来就没有从污泥里拔出过. 你现在索性全身躺在烂泥塘里翻滚. 

    我现在要告诉你一件看来一定是很奇怪的事: 等到那件事过去以后, 你会感到自己出奇的洁净. 你有一种灵魂把肉体甩脱掉的感觉, 一种脱离形体的感觉. 你好像一伸手就能触摸到美, 倒仿佛`美'是一件抚摸到的实体一样. 你好像同飒飒的微风, 同绽露嫩叶的树木, 同波光变幻的流水息息相通. 你觉得你自己就是上帝. 你能够给我解释这是怎么回事吗?
\end{quotation}

\paragraph*{page~131}
\begin{quotation}
    \itshape
    他惯会寻找这位不幸的荷兰人的痛处, 技巧的高超实在令我钦佩. 他这次用的不是讥刺的细剑, 而是谩骂的大棒. 他的攻击来得非常突然. 施特略夫被打得个措手不及, 完全失掉防卫能力. 像一只受了惊的小羊, 没有目的地东跑西窜, 张皇失措, 晕头转向. 最后, 眼泪扑簌簌地从他眼睛里滚出来. 这件事是最糟糕的地方在于, 尽管你非常恼怒思特里克兰德, 尽管你感到这出戏很可怕, 你还是禁不住要笑出来. 有一些人很不幸, 即使他们流露的是最真挚地感情也令人感到滑稽可笑, 戴尔克$\cdot$施特略夫正是这样的一个人. 
\end{quotation}

\paragraph*{page~171}
\begin{quotation}
    \itshape
    勃朗什$\cdot$施特略夫的行为还是容易解释的, 我认为她做出那种事来只不过是屈服于肉体的诱惑. 她对自己地丈夫从来就没有什么感情, 过去我认为她爱施特略夫, 实际上只是男人的爱抚和生活的安适在女人身上引起的自然反应. 大多数女人都把这种反应当做爱情了. 这是一种对任何人都可能产生的被动的感情, 正像藤蔓可以攀附在随便哪颗株树上一样. 因为这种感情可以叫一个女孩子嫁给任何一个需要她的男人, 相信日久天长便会对这个人产生爱情, 所以世俗的见解便断定了它的力量. 但是说到底, 这种感情是什么呢? 它只不过是对有保障的生活的满足, 对拥有家资的骄傲, 对有人需要自己的沾沾自喜, 和对建立起自己家庭的洋洋得意而已; 女人门秉性善良, 喜爱虚荣, 因此便认为这种感情极富于精神价值. 但是在冲动的热情面前, 这种感情是毫无防卫能力的. 
\end{quotation}

\paragraph*{page~174}
\begin{quotation}
    \itshape
    在爱这种感情中主要成分是温柔, 但思特里克兰德却不论对自己还是对别人都不懂得温柔. 爱情中需要有一种软弱无力的感觉, 要有体贴爱护的要求, 有帮助别人, 取悦别人的热情---如果不是无私, 起码是巧妙地掩盖起来的自私; 爱情包含着某种程度的腼腆怯懦. 而这些性格特点都不是我在思特里克兰德身上所能找到的. 爱情要占领一个人莫大的精力, 它要一个人离开自己的生活专门去做一个爱人. 即使头脑最清晰的人, 从道理上他可能知道, 在实际中却不会承认爱情有一天会走到尽头. 爱情赋予他明知是虚幻的事物以实质形体, 他明知道这一切不过是镜花水月, 爱它却远远超过喜爱真实. 它使一个比原来的自我更丰富了一些, 同时又使他比原来的自我更狭小了一些. 他不再是一个人, 他成了追求某一个他不了解的目的的一件事物, 一个工具. 爱情从来免不了多愁善感, 而思特里克兰德确是我认识的人中最不易犯这种病症的人. 我不相信他在任何时候会害那种爱情的通病---如痴如醉, 神魂颠倒; 他从来不能忍受外界加给他的任何桎梏. 如果有任何事情妨碍了他那无人能理解的热望, 这种热望无时或止地刺激着他, 叫他奔向一个他也不清楚的目标, 我相信他会毫不犹豫把它从心头上连根拔去, 即使忍受莫大痛苦, 弄得遍体鳞伤, 鲜血淋漓也在所不惜. 
\end{quotation}

\paragraph*{page~177}
\begin{quotation}
    \itshape
    他的眼睛流露着痛苦而迷惘的神色, 他的痛苦让人看着心酸, 而他的迷惘又有些滑稽. 
\end{quotation}

\paragraph*{page~179}
\begin{quotation}
    \itshape
    有时候一个人地外貌同他的灵魂这么不相称, 这实在是一种痛苦不堪的事. 施特略夫就是这样: 他心里有罗密欧的热情, 却生就一副托比$\cdot$培尔契爵士的形体. 他的秉性仁慈, 慷慨, 却不断闹出笑话来; 他对美的东西从心眼里喜爱, 但自己却只能创造出平庸的东西; 他的感情非常细腻, 但举止却很粗俗. 他在处理别人的事务时很有手腕, 但自己的事却弄得一团糟. 
\end{quotation}

\paragraph*{page~181}
\begin{quotation}
    \itshape
    我不太愿意摆出一副义愤填膺的架势来, 这里面总有某种自鸣得意的成分.
\end{quotation}

\paragraph*{page~205}
\begin{quotation}
    \itshape
    施特略夫现在遍体鳞伤, 他的思想又让他回去寻找慈母的温情慰抚. 多少年来他忍受的揶揄嘲笑现在好像已经把他压倒, 勃朗什对他的背叛给他带来最后一次打击, 使他失去了以笑脸承受讥嘲的韧性. 他不能再同那些嘲笑他的人一起放声大笑了. 
\end{quotation}

\paragraph*{page~206}
\begin{quotation}
    \itshape
    世界是无情的, 残酷的. 我们生到人间没有人知道为了什么, 我们死后没有人知道到何处去. 我们必须自甘卑屈. 我们必须看到清冷寂寥的美妙. 在生活中我们一定不要出风头, 露头角, 惹起命运对我们的注目. 让我们去寻求那些淳朴, 敦厚的人的爱情吧. 他们的愚昧远比我们的知识更可贵. 让我们保持沉默吧, 满足于自己小小的天地, 像他们一样平易温顺吧. 这就是生活的智慧. 
\end{quotation}

\paragraph*{page~207}
\begin{quotation}
    \itshape
    现在你已经了解了艺术会给人们带来些什么. 你还愿意改变你的生活吗? 你肯放弃艺术给予你的所有那些快感吗?
\end{quotation}

\paragraph*{page~214}
\begin{quotation}
    \itshape
    这幅画之所以能显示出这样强烈, 这样独特的个性, 并不只是因为它那极为大胆的简单的线条, 不只是因为它的处理方法, 尽管那肉体被画得带有一种强烈的, 几乎可以说是奇妙的欲情, 也不只是因为它给人的实体感, 使你几乎奇异地感觉到那肉体的重量, 而且还因为它有一种纯精神的性质, 一种使你感到不安, 感到新奇得精神, 把你的幻想引向前所未经得路途, 把你带到一个朦胧空虚得境界, 那里为探索新奇的神秘只有永恒的星辰在照耀, 你感到自己的灵魂一无牵挂, 正经历着各种恐怖和冒险. 
\end{quotation}

\paragraph*{page~215}
\begin{quotation}
    \itshape
    但是有一件事我还是清楚的: 人们动不动就谈美, 实际上对这个词并不理解; 这个词已经使用得太滥, 失去了原有的力量; 因为成千上万的琐屑事物都分享了 ``美'' 的称号, 这个词已经被剥夺它崇高的含义了. 一件衣服, 一只狗, 一篇布道词, 什么东西人们都用 ``美'' 来形容, 当他们面对面地遇到真正的美时, 反而认不出它来了. 他们用以遮饰自己毫无价值的思想的虚假夸大使他们的感受力变得迟钝不堪. 

    一旦他见到真正美的事物, 他变得恐惧万分. 
\end{quotation}

\paragraph*{page~223}
\begin{quotation}
    \itshape
    说不定作家在创作恶棍时实际上是在满足他内心深处的一种天性, 因为在文明社会中, 风俗礼仪迫使这种天性隐匿到潜意识的最隐秘的底层下; 给予他虚构的人物以血肉之躯, 也就是使他那一部分无法表露的自我有了生命. 他得到的满足是一种自由解放的快感.

    作家更关心的是了解人性, 而不是判断人性. 
\end{quotation}


\paragraph*{page~226}
\begin{quotation}
    \itshape
    她的安详沉默就像笼罩着暴风雨侵袭后的岛屿上的凄清宁静. 她有时显出了快活的笑脸也是绝望中的强颜欢笑. 

    女人可以原谅男人对她的伤害, 但是永远不能原谅他对她做出的牺牲. 
\end{quotation}

\paragraph*{page~230}
\begin{quotation}
    \itshape
    你还记得我的妻子吗? 我发觉勃朗什一点一点地施展起我妻子那些小把戏来. 她以无限的耐心准备把我网罗住, 困住我的手脚. 她要把我拉到她那个水平上; 她对我这个人一点也不关心, 唯一想的是叫我依附于她. 为了我, 世界上什么事情他都愿意做, 只有一件事除外: 不来打扰我.
\end{quotation}

\paragraph*{page~231}
\begin{quotation}
    \itshape
    你没有勇气坦白承认你正真的思想. 生命并没有什么价值, 勃朗什$\cdot$施特略夫自杀并不是因为我抛弃了她, 而是因为她太傻, 因为她精神不健全. 
\end{quotation}

\paragraph*{page~232}
\begin{quotation}
    \itshape
    我恨不得一下子刺穿了他那副冷漠的甲胄. 但是我也知道, 归根结底, 他的话也不无道理. 虽然我们没有明确意识到, 说不定我们还是非常重视别人看重不看重我们的意见, 我们在别人身上是否有影响力的; 如果我们对一个人的看法受到他的重视, 我们就沾沾自喜, 如果他对这种意见丝毫也不理会, 我们就讨厌他. 我想这就是自尊心中最厉害的创伤. 
\end{quotation}

\paragraph*{page~233}
\begin{quotation}
    \itshape
    我凝望着他. 他一动不动地站在我面前, 眼睛里闪着讥嘲地笑容. 但是尽管他脸上是这种神情, 一瞬间我好像还是看到一个受折磨的, 炽热的灵魂正在追逐某种远非血肉之躯所能想象的伟大的东西. 我瞥见的是对某种无法描述的事物的热烈追求. 我凝视着站在我面前的这个人, 衣服褴褛, 生着一个大鼻子和炯炯发光的眼睛, 红火的胡须, 蓬乱的头发. 我有一个奇怪的感觉, 这一切只不过是个外壳, 我真正看到的是一个脱离了躯体的灵魂. 
\end{quotation}

\paragraph*{page~235}
\begin{quotation}
    \itshape
    作品最能泄露一个人的真实思想和感情. 在交际应酬中, 一个人之让你看到他希望别人接受他的一些表面现象, 你只能借助他无意中做出的一些小动作, 借助不知不觉中掠过他脸上的一些表情对他作出正确的了解. 有些时候, 人们把一副假面装得逼真, 时间久了, 他们真的会变成他们装扮得这样一个人了. 但是在他写的书, 画的画里面, 他却毫无防范地把自己显露出来. 如果他作势唬人, 那只能暴露出他的空虚. 他那些涂了油漆冒充铁板地木条还会看出来只不过是木条. 假充具有独特地个性无法掩盖平凡庸俗地性格. 对于一个人信笔一挥的作品也完全可以泄露他灵魂深处的隐私.
\end{quotation}

\paragraph*{page~238}
\begin{quotation}
    \itshape
    只有一件事我觉得我是清楚的---也许连这件事也是我的幻想---, 那就是, 他正竭尽全力想挣脱掉某种束缚着他的力量. 但是这究竟是怎样一种力量, 他又将如何寻求解脱, 我一直弄不清楚. 我们每个人生在世界上都是孤独的. 每个人都被囚禁在一座铁塔里, 只能靠一些符号同别人传达自己的思想; 而这些符号并没有共同的价值, 因此它们的意义是模糊的, 不确定的. 我们非常可怜地想把自己心中的财富传送给别人, 但是他们却没有接受这种财富的能力. 
\end{quotation}

\paragraph*{page~239}
\begin{quotation}
    \itshape
    他的这些画给我的最后一个印象是他为了表现某一精神境界所作的惊人的努力. 我认为, 要想解释他的作品为什么这样使我这样惶恐莫解, 也必须从这一角度去寻找答案. 对于思特里克兰德, 色彩和形式显然具有一种独特的意义. 他几乎无法忍受地感受到必须把自己的某种感受传达给别人; 这是他进行创作的唯一意图. 只要他觉得能够接近他追寻的事物, 采用简单的线条也好, 画得歪七扭八也好, 他一点也不在乎. 他根本不考虑真实情况, 因为他要在一堆互不相关的偶然的现象下寻找他自己感到意义重大的事物. 他好像已经抓到了宇宙的灵魂, 一定要把它表现出来不可. 尽管这些画使我困惑, 混乱, 我却不能不被它们特有的热情所触动.

    我想你失掉勇气了. 你肉体的软弱感染了你的灵魂. 我不知道是怎样一种无限思慕之情把你攫在手里, 逼着你走上一条危险的, 孤独的道路, 你一直在寻找一个地方, 希望到达那里就可以使自己从那折磨你的精灵手中解放出来. 我觉得你很像一个终生跋涉的香客, 不停地寻找一座可能根本不存在的神庙. 我不知道你寻求的是什么不可思议的涅槃. 
\end{quotation}

