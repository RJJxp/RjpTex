\section{第二部分}

\paragraph*{page~428}
\begin{quotation}
    \itshape
    ``你觉得是起谋杀吗?'' 沃尔登面无表情地问道, 一边从口袋里摸出了雪茄, 点上了. 

    ``如果它有助于提高我们的破案率, 那它就是一起谋杀.''贾尔维同样面无表情地回答道. 
\end{quotation}

\paragraph*{page~442}
\begin{quotation}
    \itshape
    同事们和艾杰尔顿开着玩笑. 这些玩笑是温和的, 甚至会在不经意之间流露出大家对他的喜爱. 要知道, 这可是凶案组啊, 他们每个都是损人专家. 之前, 基尼斯康坦丁被查出得了糖尿病, 于是同事们在咖啡室的白板上做了个``实名调查''. 白板的一边写着``会对康斯坦丁之死觉得遗憾的人'', 另一边则写着``他死不死我都无所谓的人''. 后一栏的签名比前一栏长很多, 他们伪造了钱尔斯警司, 斯坦顿警督, 圣母特蕾莎和芭芭拉康斯坦丁等人的签名, 而前一栏里则有基尼他自己, 还有工会代表. 
\end{quotation}

\paragraph*{page~483}
\begin{quotation}
    \itshape
    三个月前, 斯特里克街上曾发生过一起相似的案件, 麦克埃利斯特也是那起案件的主责警探. 案情完全一模一样: 嫌疑人在宾夕法尼亚大道上挑了个妓女, 把她带到了空旷的停车场. 他脱下裤子, 让妓女给他口交------这活花费了他二十块钱. 然后, 一个西区便衣警察走了过来, 企图逮捕他; 嫌疑人惊慌了, 做了一些威胁警察的动作; 最终, 警察开枪了; 嫌疑人被送进了急救室------他不但没有享受到口活, 而且应该对自己老婆提供的合法性行为备加怀念. 
\end{quotation}

\paragraph*{page~490}
\begin{quotation}
    \itshape
    而罗杰诺兰则不是这么想的. 在他看来, 艾杰尔顿是个优秀的警探, 他比凶案组的大多数同事都努力, 而且他受过伤, 并且已经回归了, 还能要求他怎样呢? 的确, 哈里经常会开小差, 可那又怎么样呢? 他只不过弄错了休假日期而已嘛. 我们得怎么惩罚他呢? 让他写检讨, 检讨一下为什么这么喜欢打电子游戏? 要不让他停薪留职先放个假? 这么做又有什么好处呢? 这样的处罚手段在巡逻警那里行不通, 在凶案组更是行不通. 杰朗兹曼就经常迟到. 有一次, 有个领导实在受不了了, 让朗兹曼写检讨. 于是, 朗兹曼写道: ``我迟到是因为我发现有一辆德国潜艇堵住了我家的车门.'' 这就是凶案组, 而诺兰也不会为了让一个手下好过就去为难另一个手下. 
\end{quotation}

\paragraph*{page~495}
\begin{quotation}
    \itshape
    法医和警探之间的矛盾还源于双方教育背景的不同. 警探都觉得, 每个刚从学校毕业的法医都是标准的书呆子, 他们根本不了解现实世界是怎样运作的. 他们就像警探带在身上的崭新手枪皮套, 需要开开合合好几次才会变得好用. 而在法医看来, 大多数警探都是被美化了的巡逻警, 但这改变不了他们的本质, 他们没受过系统的训练, 更不懂如何从科学分析的角度看待一起案子. 他们都是经验主义者, 而他们的经验越少, 办起案就越像是个业余侦探. 
\end{quotation}

\paragraph*{page~501}
\begin{quotation}
    \itshape
    电视剧和流行文化中充斥着错误的观念, 这其中以对子弹致命程度的描述为甚. 在好莱坞制作里, 一颗廉价手枪里射出的子弹就足以让受害者倒地.可是, 子弹专家会告诉你, 没有什么子弹会让人倒地, 除非它大得像砲弹一样. 无论子弹有多重, 无论它是什么形状, 无论它的飞行速度多快, 也无论发射它的枪有多大, 子弹根本无法让人一击倒地. 这一误导性描述完全有悖于物理学: 假设子弹真能让被击中的人倒地, 那么这也意味着, 当凶手开枪时, 他自己也会被巨大的反作用力冲倒到地. 事实上, 这世上没有什么枪能做到这一点. 
\end{quotation}

\paragraph*{page~502}
\begin{quotation}
    \itshape
    电锯的声音, 被撬开的头颅, 被扯下来的头皮------``死亡面前人人平等'' 这句话终于在尸检室里找到了终极的印证, 每位死者的脸皮都仿佛只是一层橡皮胶, 被扭曲, 被折叠, 被覆盖, 好像我们每个人都只不过是戴着万圣节面具来到今世走一遭的游客, 当面具被剥离之后, 我们每个人都一样. 
\end{quotation}

\paragraph*{page~509}
\begin{quotation}
    \itshape
    巴尔的摩有两种亚文化: 一种是贫民窟文化, 一种是乡巴佬文化. 在巴尔的摩警察看来, 这两种亚文化同样可笑, 同样令人唾弃. 这一事实证明了, 巴尔的摩警察对不同阶级的歧视歧视更甚于对不同种族的歧视. 凶案组里的办公室政治很好的说明了这一点. 白人警探完全能和黑人警探无间合作. 贝提娜希尔瓦不会因为她是个黑人女警而遭歧视, 艾迪布朗, 哈里艾杰尔顿和罗杰诺兰也会受到白人同事的尊重. 如果你是个穷人, 在加上你是个黑人, 并且你的名字曾在巴尔的摩犯罪数据库里登记过, 那么, 你就是个黑鬼, 就是个黑佬------如果那个警察更粗野点的话------就是个脑残黑. 可是, 如果你做做旁边桌的艾迪布朗, 如果你是州检察官办公室的格雷格加斯金斯, 如果你是法院的克里夫戈尔迪, 或者任何按时纳税的合法公民, 那么, 你就是个黑人. 
\end{quotation}

\paragraph*{page~547}
\begin{quotation}
    \itshape
    正是这种挫败感和后悔导致了噩梦的频发. 我们可以几乎肯定地说, 睡不好觉是所有优秀警探的共通特征. 
\end{quotation}

\paragraph*{page~552}
\begin{quotation}
    \itshape
    沃尔登想, 这才是警察真正应该干的事嘛. 警察就应该混迹于街头. 没有油腔滑调的政治家, 没有背信弃义的上司, 没有被死尸吓的屁滚尿流的菜鸟. 在街头, 你只会遇到撒谎的, 狡猾的罪犯, 可沃尔登并不会抱怨. 这才是他们该干的活. 这才是他该干的活. 
\end{quotation}

\newpage
\paragraph*{page~554}
\begin{quotation}
    \itshape
    说实话, ``审判''这词并不准确. 这更像是一场戏, 一场由检察官和警探------虽然他们无一真心想追求本案的真相------主导的, 演给公众看的戏. 州检察官办公室的蒂姆多利亲自出马, 暗度陈仓, 不失颜面地故意输掉了官司. 他详细描述了这位议员谎报案情的细节, 却又没有传唤议员助理作为证人出席, 放弃了追究议员谎报案情动机的机会, 也保护了议员的私生活不被公众所知.

    沃尔登倒是能理解和接受州检察官的宽容和大度, 他无法接受的是他们竟然还要公然把宽容和大度表演给大众看; 州检察官办公室和警局都急于表现他们对公正不舍不弃的追求, 他们必须为大家演上一出戏: 起诉拉里杨, 审判拉里杨, 然后得出结论, 拉里杨只是因为太傻才谎报了案情, 然后再宣布无罪释放------这才是沃尔登出离愤怒的原因. 可是, 沃尔登别无选择. 他没法把自己的情绪发泄出来. 当被传唤出庭作证时, 他依然就范了. 议员的律师问起他和议员之间的那次关键对话, 沃尔登想都没想便承认了事实, 戳中了检方最大的漏洞.  
\end{quotation}

\paragraph*{page~555}
\begin{quotation}
    \itshape
    当拉里杨走出法庭时, 他友好地向沃尔登伸出了手. ``谢谢你的诚实.'' 议员对沃尔登说.

    沃尔登惊讶地看着他回答道: ``我为什么要撒谎呢? ''

    议员貌似是在感谢沃尔登, 可在他听来确是极大的侮辱. 毕竟, 一个警探为什么要撒谎呢? 为什么要做伪证呢? 他又有什么必要仅仅为了赢下这么一个官司而牺牲自己的人格, 更别提自己的工作和退休金呢? 难道仅仅是为了剥下政治家虚伪的假面吗? 抑或是为了赢得拉里杨政敌的青睐?

    每一个警察都是犬儒主义者, 沃尔登也不例外. 但他并非圣人. 破不了的案子和公然的背叛------沃尔登本年度的两大主题------仍然折磨着他. 他并没有表现出来, 可你仍然可以感觉到他内心的愤怒以及他对警局怯懦的办公室政治的无声抗议. 负面情绪并没有爆发, 而在他体内慢慢滋生, 并加剧着他那日渐严重的高血压. 
\end{quotation}

\paragraph*{page~556}
\begin{quotation}
    \itshape
    麦克拉尼相信, 这个世界上, 只有一种药才能救沃尔登, 那就是真正的警察工作. 
\end{quotation}

\paragraph*{page~575}
\begin{quotation}
    \itshape
    只可惜, 这一切都是幻想. 如果真是那样的话, 为什么每当警探走入法院一楼大厅的金属检测仪接受副警长搜身时, 他们总是如此垂头丧气, 并将警徽如此无奈地放在一边的盒子里? 如果真是那样的话, 为什么当警探走向电梯时, 他们的步伐总是如此沉重, 且完全忽视法院建筑的恢宏之美? 如果真是那样的话, 他们又怎会如此大逆不道地把香烟屁股扔在地板上然后踩灭它, 继而敲响检察官办公室的门, 仿佛即将投身炼狱之中万劫不复? 如果真是那样的话, 为什么当警探带着他们的案子------那些花费了他们所有心血的案子------来到正义裁决的终点时, 他们一个个都仿佛是来引咎辞职的? 
\end{quotation}

\paragraph*{page~576}
\begin{quotation}
    \itshape
    可是, 除此之外, 任何警探------任何了解其工作属性的警探------走进这座法律的殿堂时, 都会有不详的预感. 正义终究被伸张? 不, 不. 有经验的老探员从来都不会被法院的宏伟庄严所欺骗, 他们所信仰的是凶案组办案手册的第九条规律:
    
    9A. 对陪审团而言, 任何疑惑都是确凿的.

    9B. 案子越确凿, 陪审团就越差

    除此之外, 还有一条:

    9C. 世上鲜有善者, 要集齐十二位善者, 无疑等同于奇迹.

    于是, 每个警探都有心理准备------强烈的怀疑主义, 这才是他们惯有的心态. 对美国法律体制抱有充分信心的警探就像个敞开双臂, 活生生接受被打的拳击手. 
\end{quotation}

\paragraph*{page~577}
\begin{quotation}
    \itshape
    在这个国度, 法律的基石------这真是一块精致而又尊贵的石头啊------乃是一下论断: 直到十二位陪审团成员抑制认定其有罪之前, 被告便是无罪的. 为了不错杀一个无辜者, 我们宁愿让一百个罪人逃脱法网. 好吧, 就这条标准而言, 巴尔的摩的司法体制的确是在天衣无缝地运行着.
\end{quotation}

\paragraph*{page~582}
\begin{quotation}
    \itshape
    陪审团全是巴尔的摩的居民------你从艾什博顿和车里山挑几个黑人, 从海兰德城和汉弥尔顿地区挑几个白人------你总能从这十二个人里面找到几个有脑子, 有判断力的人. 他们中的有些人可能高中毕业, 有一两个还上过大学. 大多数都是工人阶级, 有几个是专业人士. 巴尔的摩是个蓝领城市. 它位于美国东海岸的铁锈地带, 在美国的钢铁业和造船业开始走下坡路后, 巴尔的摩便再没有复苏过. 它的城市人口就业率很低, 也是全美受教育率最低的城市. 二十多年以来, 那些依法纳税的公民一直都在逃离本城. 现如今, 大多数中产阶级和上层阶级------无论他们是白人还是黑人------都已经搬到了郊区, 这些人正是县法庭的陪审团的组成人员. 
\end{quotation}

\paragraph*{page~586}
\begin{quotation}
    \itshape
    至少三分之二的谋杀案都得接受辩诉交易------这是警探们不得不接受的事实. 虽然局外人都认为 ``辩诉交易''代表着双方的苟且妥协, 体制内的人却明白他存在的必要性. 如果没有辩诉交易, 司法体系便会负担过重而至崩溃.
\end{quotation}

\paragraph*{page~589}
\begin{quotation}
    \itshape
    老探员在出庭之前便对此案的优势和弱点了然于胸; 他们能预测律师的质疑, 也会提前准备自己的答案. 这倒不是说受到质疑的警探会撒谎, 而是说他们会巧妙地避重就轻, 尽可能少地暴露本案的弱点. 
\end{quotation}

\paragraph*{page~591}
\begin{quotation}
    \itshape
    在现实世界的街头执法中, 合理依据永远是一个笑话, 一个无法真正实现的体制漏洞. 

    站在证人席上的凶案组警探并不怕被问起合理依据, 他们最怕辩护律师问他们被告是否在被逼供的情况下才做出了供词, 而在他做出这一供词前是否想请律师. 打心底说, 越是优秀的警探就越明白审讯的本质------所有供词都是在逼供下做出的, 它们只是程度的区别, 而没有质的区别. 
\end{quotation}

\paragraph*{page~636}
\begin{quotation}
    \itshape
    说实话, 凶案组警探不过是数以千万计的美国中年男子里的一小撮, 他们不会比自己的同类更高尚, 也不会比自己的同类更猥琐. 只不过, 警探的工作便是窥探他人的隐私, 所以他们也就不再在意自己的隐私. 当处理谋杀案成为你的家常便饭时, 你自己所犯下的那些小罪小恶又有什么不可告人呢? 任何人都会喝多了把车给撞坏了, 可当凶案组警探对分队同事讲起此事时, 他既不会虚张声势地不承认自己喝醉了, 也不会因撞坏了公家车而倍感愧疚. 任何人都会在酒吧喝酒时相中某个女郎把她带走, 可凶案组警探还会搞笑地为同事们绘声绘色地描述之后在旅馆里发生的事情. 任何人都会对妻子撒谎, 可凶案组警探会堂而皇之地坐在咖啡室里, 当着大家的面冲着电话大吼: 我还要加班, 如果你不相信的话, 那就去死吧. 然后, 他终于说服了她. 他狠狠地挂下电话, 向衣架走去.

    ``我去马其特酒吧喝一杯.'' 他对其余五位警探------他们都强忍着不笑出声来------说, ``如果她再打电话来, 就告诉她我破案去了. ''
\end{quotation}

\paragraph*{page~695}
\begin{quotation}
    \itshape
    事实上, 警局一年多之后才把荣誉勋章授予卡西迪. 在麦克拉尼看来, 警局应该在卡西迪出院后就把这事办了. 因公殉职的警察享有隆重的葬礼仪式------彩色护旗队, 二十一次鸣枪致意, 警察局长把叠好的国旗送给遗孀. 然而, 警局却不知道拿因公受伤的警察怎么办; 上级领导不知道改说什么, 更别提打破官僚习气为伤者争取些利益了.
\end{quotation}

\paragraph*{page~701}
\begin{quotation}
    \itshape
    他当然可以揍他一顿. 他可以把他揍到血肉模糊, 也不会有人来阻止他. 他们甚至会帮他: 制服警会和他统一口径做好书面说明, 其他警探会堵住门口不让别人看见, 甚至还会加入. 要是有长官过来询问, 他就把这个婊子养的对迈克尔肖做的好事全说出来, 把这个孩子躺在不锈钢担架上的样子告诉他, 他应该会明白的. 

    难道有人说打这个人是个错误行为吗? 难道如此简单而又迅速的报复却不是正义的吗? 警察的尊严. 所谓警察的尊严是你不打那些戴着手铐的人或者那些无力反抗的人, 你不会为了招供而打人, 你也不打那些情有可原的人. 暴力执法? 去他妈的暴力执法吧. 警察的工作永远是暴力的; 事实上, 越优秀的警察就越暴力. 
\end{quotation}

\paragraph*{page~714}
\begin{quotation}
    \itshape
    在这个办公室里, 即便是真诚想表达节日祝福的装点也不禁变了味. ``世界和平'', ``善待他人''这样的话在这里毫无意义. 在这个为了纪念人类拯救者出生而举世共庆的日子里, 办公室里的人类依然没有得到拯救, 依然沉沦在枪杀案, 利器杀人案和吸毒过量致死案中. 
\end{quotation}

