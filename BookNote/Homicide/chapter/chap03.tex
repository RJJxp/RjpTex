\section{第三部分}

\paragraph*{page~244}
\begin{quotation}
    \itshape
    或许一个人如果想体会到生活中的浪漫情调就必须在某种程度上是一个演员; 而想要跳出自身之外, 则必须能够对自己的行动抱着一种既超然物外又沉浸于其中的兴趣. 
\end{quotation}

\paragraph*{page~245}
\begin{quotation}
    \itshape
    当勃朗什发现思特里克兰德除了偶尔迸发出一阵热情以外, 总是离她远远的, 心里一定非常痛苦; 而我猜想, 即使在那些短暂的时刻, 她也知道得很清楚, 思特里克兰德不过只把她当做自己取乐的工具, 而不把她当人看待. 他始终是一个陌生人, 她用一切可怜的手段拼命想把他牢系在自己身边. 她示图用舒适的生活网罗主他, 殊不知他对安逸的环境丝毫也不介意. 她费劲心机给他弄合他口味的东西吃, 却看不到他吃什么东西都无所谓. 她害怕叫他独自一个人待着, 总是不断地对他表示关心, 照护, 当他的热情酣睡得时候, 就想尽各种办法唤醒它, 因为这样她至少还可以有一种把他把持在手的假象. 
\end{quotation}

\paragraph*{page~247}
\begin{quotation}
    \itshape
    尽管也有少数男人把爱情当作世界上的头等大事, 但这些人常常是一些索然无味的人; 即便对爱情感到无限兴趣的女人, 对这类男人也不太看得起. 
\end{quotation}

\paragraph*{page~247}
\begin{quotation}
    \itshape
    性的饥渴在思特里克兰德身上占的地位很小, 很不重要, 或毋宁说, 叫他感到很嫌恶. 他的灵魂追求的是另外一种东西. 他的感情非常强烈, 有时候贪欲会把它抓住, 逼得他纵情狂欢一阵, 但是对这种剥夺了他宁静自持的本能他是非常厌恶的. 我想他甚至讨厌他在淫逸放纵中那必不可少的伴侣; 在他重新控制住自己之后, 看到那个他发泄情欲的女人, 他甚至会不寒而栗. 他的思想这时会平静地漂浮在九天之上, 他对那个女人感到又嫌恶又可怕, 也许那感觉就像一只翩翩飞舞于花丛中的蝴蝶, 见到它胜利地蜕身出来的肮脏的蛹壳一样. 
\end{quotation}

\paragraph*{page~248}
\begin{quotation}
    \itshape
    我们大多数人受不住各种诱惑, 总要对世俗人情做出一些让步; 你却无法赞扬思特里克兰德抵拒得住这些诱惑, 因为对他来说, 这种诱惑是根本不存在的. 他的脑子里从来没有想到要做任何妥协. 
\end{quotation}

\paragraph*{page~249}
\begin{quotation}
    \itshape
    思特里克兰德不善讲话, 他根本不会把自己想要说的用精辟的言辞讲出来, 给听的人留下较深的印象. 他说话没风趣. 如果说我多少还成功地记录下他的一些话语, 从中可以看出他的某些幽默感, 这种幽默也主要表现为冷嘲热讽. 他辩驳别人话的时候非常粗野, 有时候由于直言不讳, 会叫你发笑; 但是这些话之所以让你觉得滑稽, 只是因为他的话说的不多. 
\end{quotation}

\paragraph*{page~251}
\begin{quotation}
    \itshape
    但是这些材料后来都遗失了, 留下来的只是一种情感的回忆. 
\end{quotation}

\paragraph*{page~255}
\begin{quotation}
    \itshape
    塔西提也存在着某些凄凉, 可怖的东西. 但这种印象并没有长久留在你的脑中, 这只能使你更加敏锐地感到当前生活的快乐. 这就像一群兴高采烈的人在听一个小丑打诨, 正在捧腹大笑时, 会在小丑的眼睛里看到凄凉的眼神一样; 小丑的嘴唇在微笑, 他的笑话越来越滑稽, 因为他逗人发笑的时候他更加感到自己无法忍受的孤独. 
\end{quotation}

\paragraph*{page~264}
\begin{quotation}
    \itshape
    我真希望我能画出几幅绚丽多彩的图画, 把尼科尔斯船长的生动叙述在我想象中唤起的一幅幅画面也让读者看到. 他叙述他们两人在这个海港的下层生活中的种种冒险完全可以写成一本极有趣的书, 从他们遇到的形形色色的人物身上, 一个研究民俗学的人也可以找到足够的材料编纂一本有关流浪汉的大辞典. 但是在这本书里我却只能用不多几段文字描写他们这一段生活. 我从他的谈话得到的印象是: 马赛的生活既紧张又粗野, 丰富多彩, 鲜明生动. 相形之下, 我所了解的马赛---人群杂沓, 阳光灿烂, 到处是舒适的旅馆和挤满了有钱人的餐馆---简直变得平淡无奇, 索然寡味了. 
\end{quotation}

\paragraph*{page~283}
\begin{quotation}
    \itshape
    我还听说, 有的人到这里来, 准备在哪个公司干上一年, 他们对这个地方骂不绝口, 离开的时候, 发誓赌咒, 宁肯上吊也绝不再回来. 可是半年以后, 你有看见他们登上这块陆地; 他们会告诉你说, 在别的任何地方他们也无法生活下去. 
\end{quotation}

\paragraph*{page~285}
\begin{quotation}
    \itshape
    我认为有些人诞生在某一个地方可以说未得其所. 机缘把他们随便抛到一个环境中, 而他们却一直思念着一处他们自己也不知道坐落在何处的家乡. 在出生的地方他们好像是过客; 从孩提时代就非常熟悉的浓荫郁郁的小巷, 同小伙伴游戏其中的人烟稠密的街衢, 对他们来说都不过是旅途中的一个宿站. 这种人在自己亲友中可能终生落落寡合, 在他们唯一熟悉的环境里也始终孑身独处. 也许正是在本乡本土的这种陌生感才逼着他们远游异乡, 寻找一处永恒定居的寓所. 说不定在他们内心深处仍然隐伏着多少世代前祖先的习性和癖好, 叫这些彷徨者再回到他们祖先在远古就已离开的土地. 
\end{quotation}

\paragraph*{page~290}
\begin{quotation}
    \itshape
    个性? 在我看来, 一个人因为看到另外一种生活方式更有重大的意义, 只经过半小时的考虑就甘愿抛弃一生的事业前途, 这才需要很强的个性呢. 贸然走出这一步, 以后永不后悔, 那需要的个性就更多了.
\end{quotation}

\paragraph*{page~291}
\begin{quotation}
    \itshape
    我很怀疑, 阿伯拉罕是否真的糟蹋了自己. 做自己最想做的事, 生活在自己喜爱的环境里, 淡泊宁静, 与世无争, 这难道是糟蹋自己吗? 与此相反, 做一个著名的外科医生, 年薪万镑, 娶一位美丽的妻子, 就是成功吗? 我想, 这一切都取决于一个人如何看待生活的意义, 取决于他认为对社会应尽什么义务, 对自己有什么要求. 但是我还是没有说什么; 我有什么资格同一位爵士争辩呢? 
\end{quotation}

\paragraph*{page~310}
\begin{quotation}
    \itshape
    使思特里克兰德着了迷的是一种创作欲, 他热切地想创造出美来. 这种激情叫他一刻也不能宁静. 逼着他东奔西走, 他好象是一个终生跋涉的朝圣者, 永远思慕着一块圣地. 盘踞在他心头的魔鬼对他毫无怜悯之情. 世上有些人渴望寻获真理, 他们的要求非常强烈, 为了达到这个目的, 就是叫他们把生活的基础全部打翻, 也在所不惜. 思特里克兰德就是这样一个人; 只不过他追求的是美, 而不是真理. 对于像他这样的人, 我从心眼里感到怜悯. 
\end{quotation}

\paragraph*{page~334}
\begin{quotation}
    \itshape
    那些颜色都是我熟悉的颜色, 可是又有所不同; 它们都具有自己独特的重要性. 而那些赤身裸体的男男女女, 他们即都是尘寰的, 是他们揉捏而成的尘土, 又都是神灵. 人的最原始的天性赤裸裸地呈现在你眼前, 你看到的时候不由得感到恐惧, 因为你看到的是你自己. 
\end{quotation}

\paragraph*{page~336}
\begin{quotation}
    \itshape
    他用一对失明的眼睛望着自己的作品, 也许他看到的比他一生中看到的还要多.
\end{quotation}

\paragraph*{page~344}
\begin{quotation}
    \itshape
    ``啊, 当然了,'' 她大大咧咧地说, ``当年我开那家打字所主要也是为了觉得好玩, 没有其他什么原因. 后来我的两个孩子都劝我把它出让给别人. 他们认为太损耗我的精神了.''

    我发现思特里克兰德太太已经忘记了她曾不得不自食其力这一段不光彩的历史. 同任何一个正派女人一样, 她真实地相信只有依靠别人养活自己才是规矩地行为. 
\end{quotation}

\paragraph*{page~346-347}
\begin{quotation}
    \itshape
    这以后我把我听到的查理斯$\cdot$思特里克兰德在塔西提的情形给他们讲了一遍. 我认为没有必要提到爱塔和她生的孩子, 但是其余的事我都如实说了. 在我谈完他惨死的情况以后我就没有再往下说了. 有一两分钟大家都没有说话. 后来罗伯特$\cdot$思特里克兰德划了根火柴, 点着了一支纸烟.

    ``上帝的磨盘转动很慢, 但是却磨得很细, ''罗伯特说, 颇有些道貌岸然的样子.

    思特里克兰德太太和朵纳尔德逊太太满腹虔诚地低下头来. 我一点也不怀疑, 这母女两人所以表现的这么虔诚是因为他们都认为罗伯特刚才是从<<圣经>>上引证了一句话.

    ...

    <<圣经>>上的一句话也到了我的唇边, 但是我却控制着自己, 没有说出来, 因为我知道牧师不喜欢俗人侵犯他们的领域, 他们认为这样是有渎神明的. 我的亨利叔叔在威特斯台柏尔教区做了二十七年牧师, 遇到这种机会就说: ``魔鬼要干坏事总可以引证<<圣经>>. 他一直忘不了一个先令就可以买十三只大牡蛎的日子.'' 
\end{quotation}