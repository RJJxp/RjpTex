\section{第三部分-后记}

\paragraph*{page~768}
\begin{quotation}
    \itshape
    我恐惧, 是因为瓦伦斯一案是全书的线索, 它之未破便意味着这本书没有高潮, 结尾也必将是开放的, 虚无的, 有瑕疵的. 
\end{quotation}

\paragraph*{page~770}
\begin{quotation}
    \itshape
    那年夏天, 当尸体随着热浪来临越垒越高, 我突然觉得, 自己是站在一个生产死亡的车间里. 这是一条死亡的流水线. 在这个衰败的美国老工业区, 什么都已经停止生长了, 唯独死亡还在生生不息, 唯独``心碎''还在大批量地被生产着. 我告诉自己, 也许, 真正超越现实的是我们生活本身吧. 
\end{quotation}

\paragraph*{page~775}
\begin{quotation}
    \itshape
    太阳报拥有悠久的历史, 却也被传统所束缚, 它就像一个老妇, 优雅而又举步维艰. 
\end{quotation}

\paragraph*{page~778}
\begin{quotation}
    \itshape
    更加可悲的是, 和任何企业单位一样, 一旦专家离开这里之后, 他们就再也不会回来了. 

    在我身处的世界里, 同样的事情也在上演. 太阳报中最优秀的记者都辞职了, 他们去了纽约时报, 华盛顿邮报和其他报业------和警探一样, 他们是被体制对他们的傲慢态度赶走的.

    当这个费城来得管理层离开这份报纸时, 他们留下了十二年内三次获得普利策奖的``丰功伟绩''------可是, 在他们来到这里之前的十二年里, 太阳报的晨报和晚报一样获得过三次普利策奖. 
\end{quotation}

\paragraph*{page~780}
\begin{quotation}
    \itshape
    火线的基调更加愤世嫉俗, 也更具政治挑战性. 
\end{quotation}

\paragraph*{page~781}
\begin{quotation}
    \itshape
    这些警探生活着, 工作着, 却并不抱什么幻想. 有一天深夜, 当我在修改本书的第三稿还是第四稿时, 我突然意识到, 我是在替他们发声, 替他们说出最真实的感言.

    难道还有比这更有意义的事吗? 明星读者的趋之若鹜, 其他同行的嫉妒, 还有那些颁给本书的奖项, 所有这些都不再有所谓了. 十五年前, 当我面对电脑写作此书时, 警探们的想法便是唯一的判断标准. 如果他们读完之后觉得我句句诚实, 那么, 我就不会对自己这一将他们个人隐私透露于纸上的行为而感到后悔. 

    这倒不是说我所写的都是褒赞他们或将他们美化的东西. 我想, 读者们也已发现, 在本书中, 他们中有些人是种族主义者, 有些人则对种族主义表现木讷; 有些人歧视女性; 有些人则有恐同症; 而他们的笑话时常源自他人的贫穷和苦难. 可是, 无论他们的政治偏见如何, 只要地上躺着一具尸体------无论这具尸体的皮肤是黑的, 棕色的, 还是(相当罕见的)白的------他们都会一致对待. 在我们这个毫无优雅可言的时代, 职业操守便是一种优雅, 它足以令人原谅此人的其他方面的小罪小恶. 我相信, 我的读者会原谅他们, 正如我原谅了他们一样. 我希望, 当你们读完这本厚达六百多页的书后, 这些警探的率真诚实将不再让他们难堪, 而是变成了他们的优点. 
\end{quotation}