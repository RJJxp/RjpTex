\section*{引子}
\addcontentsline{toc}{section}{引子}
\rjpWxReaderNote{以上人祭宗教及角斗产业的消亡,都源于外来文化的干预。罗马人后来皈依了基督教,传统的阿兹特克宗教被西班牙殖民者的天主教所取代,殷商则与之不同:周灭商后,人祭被周人消除,但周人并未开创一种新的宗教,而是采用世俗的人文主义立场,与极端宗教行为保持距离,不允许其干预现实生活,所谓“敬鬼神而远之”。这奠定了后世中国的文化基础。}

\rjpWxReaderNote{故而,《易经》的内容多是文王的翦商谋略,也正因此,这部分内容最为隐晦。}

\rjpWxReaderNote{当早期人类社会有了一定程度的复杂化,开始形成王权和统治阶层,但统治体系尚未完全成形和稳固时,统治者需要借用一种强大的机制来维持其权力,这就是人祭宗教和战俘献祭行为产生的基础}


% new chapter mark, for the convience of reading
\section{新石器时代的社会升级}
\rjpWxReaderNote{以上两千年历程,是新石器中晚期到文明(青铜)时代前夜的变化大趋势:从村落到部落再到早期国家。通俗一点说,就是从村级到乡级、县级的递增升级。}



% new chapter mark, for the convience of reading
\section{大禹治水真相:稻与龙}
\rjpWxReaderNote{我们还是能发现“大禹治水”的最初内核:一场龙山末期部分古人改造湿地、开发平原的活动。}

\rjpWxReaderNote{这其实是新石器晚期以来几乎全人类共同的事业。比如,古罗马城是在公元前6世纪王政时期的排干沼泽工程中初步建成的,甚至直到工业时代初期,巴黎的凡尔赛宫,乃至整座圣彼得堡市,也都是排干沼泽后营建出来的。}

\rjpWxReaderNote{由此亦可见,绿松石龙很可能代表的是夏—二里头人的图腾。}



% new chapter mark, for the convience of reading
\section{二里头:青铜铸造王权}
\rjpWxReaderNote{在二里头之前的一千多年里,从江南到华北,已经出现若干辉煌古国—石家河、良渚、南佐、陶寺、石峁、清凉寺……它们一度建立大型的城邑,距离“文明”和王朝似乎只有一步之遥,然而经过短暂的繁荣,又都自然解体,复归简单无为的部落时代。那么,二里头是如何走出昙花一现的旧循环的?因为他们有了新的统治技术——青铜。}

\rjpWxReaderNote{但到早期国家出现之后,特别是二里头这种青铜王朝,都城的贫富悬殊则已经非常剧烈,一面是各种宏大建筑和兴旺发达的手工产业,一面是大量赤贫者抛尸街头,各种残酷现象也最为集中。}

\rjpWxReaderNote{铜器成本较高,是上层社会的奢侈品,对二里头普通民众来说,使用最多的还是石器和骨器。石器可能是在洛河中采集砾石敲打制造的。到后来的商朝、西周以至春秋,最基本的农具还是石器。石制农具和工具被完全取代,要到冶铁技术已经普及的战国。}

\rjpWxReaderNote{青铜器从未完全淘汰石器,它更多体现的是上层人生活的改变,就像古人发明文字后,社会上的多数人仍是文盲。社会的发展水平往往是被占人口少数的精英阶层代表的。}

\rjpWxReaderNote{学术界以往对“文明”的界定比较严格,其中有三个关键要素:城市、冶金技术和文字。}

\rjpWxReaderNote{总体来说,商代符合严格的文明标准,争议不大。但商代之前的夏—二里头,缺少文字要素}

\rjpWxReaderNote{在二里头之前,大型城邑(古国)已经有过若干座,狭义文明标准的第一项要素已经有了,但不够稳定。二里头的创新是第二项,也就是青铜冶铸技术,有了它,大型城邑(古国)才能维持和发展下去,让第一项要素真正确立,并继续发展出第三项——文字。}

\rjpWxReaderNote{对比之下,石器时代的古国王权,并没有凌驾于民众阶层之上的武装优势,统治者可以用玉礼器表现自己的高贵奢华,但玉兵器的战斗力并不能超越石兵器。}

\rjpWxReaderNote{二里头以往的一千多年里,从长江中游、江浙到华北,众多古国兴起又解体。到三期时,二里头也进入了古国盛极而衰的节点:统治者豪奢营建,底层人群极度贫困,劳役无休,对立情绪终将引爆。恰在此时,成熟的青铜技术让二里头得以续命,社会上层继续维持其统治。}

\rjpWxReaderNote{在宫殿区尚未修筑宫墙时,二里头的手工业作坊区已经建起了围墙。当时属于二里头二期阶段,铸铜作坊规模还不算大,也没有发现青铜兵器,那为何要把作坊区的安危放在王宫之前?可能的答案是,作坊区的围墙是手工业族群自己兴建的,他们需要承担自己的防务,并且有这种资源和实力。实际上,青铜冶铸者的宗教风俗也有别于宫殿区,这也是他们族群整体自治的表现。}

\rjpWxReaderNote{冶铸人群承认宫廷王权的权威,但自主管理族群。}

\rjpWxReaderNote{夏朝还不是后世人观念中的大一统政权,内部族邦林立,}



% new chapter mark, for the convience of reading
\section{异族占领二里头}
\rjpWxReaderNote{这很可能是食人者房屋旁边的垃圾坑}

\rjpWxReaderNote{这也使人产生了这样一个联想:也许,铸铜族群是灭亡夏朝—二里头的“第五纵队”,他们不甘心宫殿区族群垄断权力,坐享统治收益,便联合外来者商族,一起征服了宫殿区族群。铸铜族群和商人的协作,换来了二里头古城半个世纪的寿命}



% new chapter mark, for the convience of reading
\section{商族来源之谜}
\rjpWxReaderNote{夏都二里头被外来者占领后,增加了一些外来样式的陶器,但难以解释的是,这些陶器并没有统一的风格,如前文所述,有的属于河北和河南两省交界处的下七垣文化,有的属于山东地区的岳石文化。之后,商人新建了两座城邑——郑州商城和偃师商城,但其陶器也分为好多种风格,多样程度甚至超过被占领后的二里头古城。}

\rjpWxReaderNote{上古时代,即便是同一种陶器文化内部,通常也存在着众多部落,彼此互不统属,甚至不共戴天。而来自不同陶器文化的人群居然共同参与了灭夏和建商,这委实让人难以理解。}

\rjpWxReaderNote{上古时代,常有女子未婚生育的神话,据说这是母系时代“知其母,不知其父”的特征。周族史诗也是如此,他们的女性始祖姜嫄在荒野踩到巨人脚印而怀孕,生下弃(后稷),从而繁衍出周族。}

\rjpWxReaderNote{杜甫诗云:“人生不相见,动如参与商。”商星都是黎明时在东方出现,参星总是黄昏时在西方出现,永远一东一西,所以人生分离难聚也被称为“参商”。}

\rjpWxReaderNote{春秋的贵族还说,宋国是辰星之族的故地,所谓:“宋,大辰之虚也。”(《左传·昭公十七年》)宋国的都城在商丘,而宋人是商人后裔,可见,从王朝兴起之前到灭亡之后,商丘一直和商人有缘。}

\rjpWxReaderNote{这也说明,《易经》卦爻辞中的商代史事并不完全可信,周文王可能会基于西土周人的环境错误地理解商人历史。}

\rjpWxReaderNote{商族人很可能就是在经营贸易的过程中发现夏朝有机可乘,与下七垣、岳石文化中的一些族群建立起紧密联系,逐渐形成了同盟势力。}

\rjpWxReaderNote{因夏都的王族和铸铜族群的矛盾日渐激化,二里头铸铜人应该是在危急之中联络了商族,于是,商汤带领东方同盟各族大举西征,攻占了夏朝。但在管理王朝和青铜技术方面,商族和它的东方盟友都缺乏经验,用了半个世纪左右才完整吸收了夏朝的遗产,并融合各原有文化,形成了新的、更广泛意义上的商族。}



% new chapter mark, for the convience of reading
\section{早商:仓城奇观}
\rjpWxReaderNote{ 不过,如果回到开国之王商汤的时代,形势还没有这么乐观。}

\rjpWxReaderNote{从考古提供的信息看,商族人灭亡夏朝之后,一度保留了二里头古城,并同时开始兴建两个中心城市:一座是在二里头以东8公里处的偃师商城;另一座是在二里头以东100公里处的郑州商城——堪称早商的东西两都。}

\rjpWxReaderNote{夏都二里头固然繁华了数百年,但内部一直是二元分立结构:宫殿区和铸铜作坊区长期保持共治,而且彼此的矛盾还是夏朝灭亡的直接根源。征服二里头后,商人对这两种人群加以分化、拉拢,并拆解到二里头和偃师两座古城中,最终把他们同化入自己的王朝之内。}

\rjpWxReaderNote{而且,商王家族擅长用生意人的思维来管理新王朝,扩张目标主要指向矿产资源丰富的地区,如盐矿和铜锡矿产地。对商朝的远程扩张来说,贸易是和征战同样重要的手段。}

\rjpWxReaderNote{青铜技术曾长期被圈禁在二里头,而商人则更现实,各商人部族均掌握了冶铸铜技术,而且不仅是在偃师和郑州两座核心商城,每一支向远方征伐的商人队伍也都有铸铜技师。他们携带着铸造铜镞、铜戈的石范,一旦发现小规模铜矿,便可以就地生产兵器。}

\rjpWxReaderNote{青铜技术曾长期被圈禁在二里头,而商人则更现实,各商人部族均掌握了冶铸铜技术,而且不仅是在偃师和郑州两座核心商城,每一支向远方征伐的商人队伍也都有铸铜技师。他们携带着铸造铜镞、铜戈的石范,一旦发现小规模铜矿,便可以就地生产兵器。}

\rjpWxReaderNote{从垣曲古城继续向西北,翻越中条山脉,进入开阔的运城盆地,又出现了一座夯土小城,这便是夏县东下冯商城。}



% new chapter mark, for the convience of reading
\section{人祭繁荣与宗教改革运动}
\rjpWxReaderNote{在中国的城市中,很少有郑州这样的巧合:它完整地覆压在了3500年前的商代城址之上。}

\rjpWxReaderNote{从夏都二里头和偃师商城宫廷区的祭祀行为看,商朝和夏朝存在非常明显的继承性。偃师商人学习了夏朝的宫殿区建设和祭祀方式:祭祀区集中在宫城北侧,以猪为主要祭品;开始时以幼年猪为主,国力强盛后升级为成年猪。在这方面,夏商两代的历程如出一辙。}

\rjpWxReaderNote{比起二里头—夏朝,这是一个新变化:人祭是商朝的国家宗教,也是商族人的全民宗教。}

\rjpWxReaderNote{但就在早商与中商之交,即商朝开国200年左右时,某位商王可能曾试图改革人祭宗教,用埋葬器物献祭代替杀人和杀牲。这场革新运动的另一个表现是,王宫区锯制头盖骨的工作场戛然而止,大量即将完工的成品被投入壕沟埋葬,似乎商朝上层一夜之间皈依了“不杀人”的新宗教。}

\rjpWxReaderNote{结合前文对新石器末期到中商这上千年人祭历程的梳理,本书认为可以从两方面来理解:理论层面,王的大量献祭(意味着他获得神的福佑)是王权融合神权的标志;现实层面,战争让本国族的民众团结起来一致对外,从而更巩固了王的权力。}



% new chapter mark, for the convience of reading
\section{武德沦丧南土:盘龙城}
\rjpWxReaderNote{从盘龙城土著的立场看,他们之所以接受这些外来者的统治,除了青铜兵器的威慑力,更重要的应该还是青铜产业带来的利益。}

\rjpWxReaderNote{但商朝腹地旋即发生大动荡,郑州商城和偃师商城被毁弃,继起的小双桥朝廷再次回归人祭宗教。这意味着商王朝内战的胜出者是盘龙城的对立面——盘龙城和王朝腹地的联系就此彻底断绝。}



% new chapter mark, for the convience of reading
\section{3300年前的军营: 台西}



% new chapter mark, for the convience of reading
\section{殷都王室的人祭}
\rjpWxReaderNote{后世的迁都往往只是换一座都城,对全国的影响一般不太大;但在国族一体的上古时代,国都是统治者族群最集中的地方,除了那些散布在远方的零星据点,整个统治族群,或者说国家和王朝,都要整体搬走,不仅是王宫,还有所有的商族部落和家支,以及他们的牲畜、家奴。所以,迁都动议充满争论,多数商人并不愿搬迁}

\rjpWxReaderNote{盘庚从哪里搬迁,目的地又是哪里?《史记·殷本纪》说是从黄河北迁到了河南:“帝盘庚之时,殷已都河北,盘庚渡河南,复居成汤之故居,乃五迁,无定处。”而事实正好相反,根据现代考古,实是从郑州小双桥迁到安阳殷墟(殷都)。}

\rjpWxReaderNote{腰坑殉狗}

\rjpWxReaderNote{王陵墓道中摆放着比较完整的人头,以示对刚去世的先王的尊重。这是孝子的人之常情。}