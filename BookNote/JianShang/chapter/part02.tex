\section{商人的思维与国家}
\rjpWxReaderNote{商代甲骨文和后世的汉字一脉相承,从未中断,这自然会给现代人释读甲骨文带来天然的便利,但也会有误导,容易让现代人以为商朝的文化和政体很容易理解。其实,它和西周之后的华夏文明很不一样,和战国之后的中国更是判若云泥。}

\rjpWxReaderNote{现代考古学也只是揭开了真实商代的一个小角,不仅如此,如何解读考古发现甚至复原真正的商文化,却是更加困难的工作。}

\rjpWxReaderNote{后来的周人史诗把他们的灭商事业称为“翦商”,也是取其宏大之意。}

\rjpWxReaderNote{如“教”字,甲骨文写作:右边是攵,手拿一根棍子;左上是“爻”,一种用摆放草棍计数的方式;下面是“子”,也就是幼儿。顾名思义,用棍子督促孩子学习算数,就是教。爻字可能让人联想到八卦,不过早期的爻还没有八卦占算之意,只是单纯的算数,但更晚的八卦的确是从草棍算数发展来的。}

\rjpWxReaderNote{当然,甲骨文里也有女人的形象。“女”,甲骨文写作,像一个跪坐姿势的女子,以驯服的造型和较大的胸部为特征。手抓一名女子,则是“妥”,甲骨文写作,一种用女子献祭的方式。“母”,甲骨文写作,在女字的胸前加两点,突出其哺乳的特征。}

\rjpWxReaderNote{由于社会以族为基本单位,没有完备的政府体系,也就没有赋税和兵役制度。殷商王室和朝廷的开支不是靠在王朝境内普遍征税,而是由王室自己的产业承担。}

\rjpWxReaderNote{所谓:“国之大事,在祀与戎。”这正是家族分封制而非官僚帝制时代的规则。}

\rjpWxReaderNote{各宗族参战,既是对王的义务,也是抢掠战利品和人口的机会。}

\rjpWxReaderNote{《史记》记载,周文王曾被商纣王任命为“西伯”。这个称呼是有所依据的,只不过在周灭商后,后人已不了解“伯”在商文化中的意义了。}

\rjpWxReaderNote{商朝不可能被武力摧毁,却可能会因异族熏染而堕落。}

\rjpWxReaderNote{早商不仅疆域过大,殖民城邑前出太远,而且王权也过于发达,其标志就是郑州和偃师商城庞大的城池与仓储体系。这就需要职业化官僚团队,而“职业”则意味着脱离原生态的宗族生活,只对雇主,也就是商王负责,从而丧失传统文化特质。}

\rjpWxReaderNote{唯命是从的官僚体系是难以起到纠正作用的——它只能充当王的工具。}

\rjpWxReaderNote{但武丁意识到了这种危险,转而放弃洹北商城的营建,让各商人族邑自行其是,自由发展,以维持商族旧有的小共同体社会结构和传统部族生活方式}

\rjpWxReaderNote{ 或者说,武丁的主要政策都源自对早商朝廷崩溃的反思}

\rjpWxReaderNote{武丁的扩张战争总是伴随着大规模的人祭典礼。那些异族俘虏本可以作为劳作的奴隶,但人祭宗教是商王朝的精神支柱和商族获得诸神眷顾的根源,所以,舍弃一点现实的物质利益,也要取悦诸神,维护商文化的兴旺。}



% new chapter mark, for the convience of reading
\section{王后的社交圈}
\rjpWxReaderNote{妇好和M18墓主都是女性高级贵族,可能生前都喜欢孩子,所以后人会给她们献祭一些打扮得漂漂亮亮的儿童。}



% new chapter mark, for the convience of reading
\section{大学与王子}
\rjpWxReaderNote{故而,这些人牲应当是大学生们练习射箭和搏杀的陪练}



% new chapter mark, for the convience of reading
\section{西土拉锯战:老牛坡}
\rjpWxReaderNote{这是上古时代的常态:并非所有的人类社群都会自动进化成更大的共同体和国家;事实是,多数会一直停滞无为,直到被强大的古国或王朝吞并,被强行裹挟进人类的“发展”大潮中。}

\rjpWxReaderNote{再到几代人之后,商朝授予周文王的封号则是“周方伯”。[插图]伯是异族酋长,商朝是不会给异族头领“侯”的称号的。武丁曾占卜一位“妇周”的病情会不会延续。[插图]“妇某”的称呼专用于商族血统的后妃,比如著名的妇好。倘若是异族女子,哪怕成为商王宠妃,也不会享有这种称呼,比如末代商纣王宠爱的妲己,她来自“己”姓的苏国,而非“子”姓的商族,所以不能称“妇妲”}

\rjpWxReaderNote{其实,在比周文王早二百年的商王武丁时代,甲骨卜辞中就已经出现了崇侯虎。}



% new chapter mark, for the convience of reading
\section{周族的起源史诗与考古}
\rjpWxReaderNote{周族出自羌人}

\rjpWxReaderNote{脱离姜姓有邰氏部族的生活圈后,后稷的后人为自己选择了一个新的族姓——姬,以表示他们和姜姓群体的血缘关系已经足够遥远,可以通婚了。这就是后来建立周王朝的姬周族。}

\rjpWxReaderNote{碾子坡遗址的非正常死亡和被随意或恶意抛掷的尸骨极少,占比非常低,甚至远低于仰韶半坡文化时期的典型遗址。可以说,这里的生活非常和平。}

\rjpWxReaderNote{那姬周族人拿什么交换呢?商人城邑统治着周边土著居民,应该不太会缺粮食,所以姬周最适合用来贸易的商品是牲畜,尤其是马和牛。}

\rjpWxReaderNote{周是小族群,生长在羌人为主的大环境里;自命姬姓,以显示自己和羌人不同;}

\rjpWxReaderNote{总结一下,“周地”只有一个,就是周原地区;但名为“周”的人群,则有三个:一,姜姓的周族。这是台玺和叔均的后人形成的族群,可能从夏代起就一直住在周原,到商王武丁时被剿灭。这里说的夏只是时间概念,夏朝并不能统治关中。二,武丁王分封的商人周侯之国,存在时间很短。三,后稷和不窋的后人形成的族群。他们从夏代就离开周原,迁入山林过戎狄的生活去了,但到商朝末期,又迁回周原(叔均后人曾经的生息之地),成为后世熟知的姬周族,并且灭商建立了周朝。}



% new chapter mark, for the convience of reading
\section{成为商朝爪牙:去周原}
\rjpWxReaderNote{不过,后世周人并不愿提及此次迁徙的商朝因素,需要我们从文献里抽丝剥茧进行还原。}

\rjpWxReaderNote{这里记载的三千辆马车,实在过于夸张,因为从碾子坡的考古看,豳地时期的周人还没有马车。}

\rjpWxReaderNote{子思讲的这个版本,虽然有后人添加的道德色彩,但仍显露了周人早期部落时代的特点:族长没有绝对专断的权力,部落民众有较大自主权,他们可以决定是否迁徙。}

\rjpWxReaderNote{但史书所说的戎狄袭击豳地,在考古中则找不到迹象。碾子坡聚落一直在延续,墓葬随葬品还有增加,每座西周墓一般都随葬有几件陶器,说明在亶父带部分族人迁走之后,豳地并没有发生过外来征服和剧变,甚至居民的生活水平还在持续提高。}

\rjpWxReaderNote{既然豳地—碾子坡并没有什么外来威胁,为什么亶父和族人还要迁徙?其实,这是武乙王西部大扩张的副产品:商朝希望招募一个仆从部族,让他们定居到周原,充当商朝的附庸和马前卒。这才是姬周族来到周原定居的根本原因,因而也是周人灭商后不愿再提起的黑历史。}

\rjpWxReaderNote{在后世周人的史诗里,亶父被尊为“大王”(太王,古老的王),他迁居岐山之阳的周原,也被描述成周人“翦商”事业的开端:后稷之孙,实维大王。居岐之阳,实始翦商。(《诗经·鲁颂·閟宫》)但在亶父的时代,周族还完全没有挑战商朝的可能性,也不可能有称王的非分之想,这应该都是周朝建立之后对历史的改造。}

\rjpWxReaderNote{武乙王恩准姬周族迁居到周原是有条件的,立足安居之后,周族人需要承担相应的义务,这便是替商朝捕猎人牲,以供商王献祭。甲骨文中用于献祭的羌人,是周人的同宗、近邻和联姻盟友。因此,为商朝捕猎羌人(周人文献里的姜姓戎人)并不符合周人的传统伦理。这可能是泰伯、仲雍与父亲决裂的根源,他们希望躲开这件可怕的事。而幼弟季历则和父亲站在一起。毕竟,只有依附强大的商朝,周族才有发展的机会。}

\rjpWxReaderNote{不过在亶父和季历时代,这大概是周族能攀附的离商朝最近的婚事。在《诗经·大雅·大明》中,周人向西土各部族宣称,这位新夫人是从殷商王朝嫁过来的,暗示她是来自商王家族的公主}

\rjpWxReaderNote{在商人看来,刚从豳地—碾子坡迁出来的周族,近乎生番;而挚国,则更接近中原文化圈,国君家族应当比较商化,可能会使用商人的文字和官方语音,如此,新娘大任给季历和周族带来的影响是深远的,特别是她生了一个叫周昌的儿子,也就是后来的文王。}

\rjpWxReaderNote{事实上,文丁王和季历可能都是死于商朝内斗。下一位商王是文丁的儿子帝乙(第二十九王),他一上台就废除了商朝传统的祭祀方式,改用了一套被现代研究者称为“周祭”的制度。}

\rjpWxReaderNote{但祖甲死后,旧宗教迅速回潮,直到末代二王帝乙和帝辛(纣王)时期,新派的“周祭”才算正式确立下来。新派宗教甚至不仅称先王为“帝”,也称在世之王为帝,所以商朝末代两王的称号分别是帝乙和帝辛(纣王)。按照商人的传统宗教,这肯定触犯了天界上帝的独尊地位,几乎是大逆不道的僭越。}

\rjpWxReaderNote{帝乙初年重启革新,新旧两派争的就不仅是仪式,也是权力分配。老派宗教祭祀的各种自然神,可以包含一些非商族起源的神灵,这为商王拉拢异族提供了操作空间。新派却是一个更加保守的王族小群体,排斥一切没有商王族血统之人,因而季历这种当红的蛮族酋长自然下场堪忧。}



% new chapter mark, for the convience of reading
\section{周文王地窖里的秘密}
\rjpWxReaderNote{考古学者多认为,这座凤雏村甲组基址是周族人的宗庙,依据的是后来《周礼》中“藏龟于庙”的说法。但在周昌时代,周族还没有这种严格的礼制,甚至西周中期的垃圾坑里也还是会发现占卜后的甲骨,所以《周礼》的说法并不符合先周和西周的实际。}

\rjpWxReaderNote{周昌对此还有一种异乎寻常的兴趣和探索精神,他不仅学习商人的甲骨占卜,还改造了易卦预测技术,创作了《易经》文本。}

\rjpWxReaderNote{周昌在去世前才把都城搬到了丰京(今西安市西郊)}



% new chapter mark, for the convience of reading
\section{《易经》里的猎俘与献俘}
\rjpWxReaderNote{商与周的这种关系,从古公亶父晚年开始,历经季历和周昌两代人,甚至可能持续到灭商之前的周武王初年。同期的商朝,则经历武乙、文丁、帝乙(小乙)和帝辛(商纣)四代商王,跨度超过五十年。}

\rjpWxReaderNote{在史书和文献里,周人的这段历史被抹去了,几乎没有留下任何痕迹。和这段历史一起被遗忘的,是商朝的鬼神血祭文化。自周朝建立,人们的记忆里便再也没有了那个血腥、恐怖而漫长的年代,“历史”成为一连串古代圣王哺育和教化群氓的温情往事。但即便如此,仍有些蛛丝马迹被保留了下来,这便是文王周昌创作的《易经》。周昌一直生活在暗黑的商代,没能等到商朝灭亡便已死去,但他在《易经》里给后人留下了很多珍贵的记录,其中就包括商人的血祭仪式和周族充当人牲捕猎者的经验。}

\rjpWxReaderNote{自亶父迁居周原,周人一直为商朝捕猎羌人,所以周昌在研究《易经》占算方法时,很关注预测捕俘的结果。}

\rjpWxReaderNote{《易经》里为何会有这么多周人生活的真实记录?这便涉及周昌创作《易经》的目的:研究各种事物背后的因果联系,最终建立一套翦商的理论和操作方法。}

\rjpWxReaderNote{大壮卦为何要用公羊代表羌人,还写得这么隐晦?本书认为,这可能是因为周人和羌人有古老的同宗亲缘,对周族来说,替商朝捕猎羌人在道义上是一种耻辱。}

\rjpWxReaderNote{而且,在《易经》中,周昌记录捕羌用的都是“孚”字——这个字不带族群含义,应该也有不触及周人隐痛之意。}

\rjpWxReaderNote{当然,频频外出捕猎俘虏,并不意味着周人已经是西土最强大的部族,可以高枕无忧了。因为结怨太多,周族人也会遭到其他部落的报复,导致他们时刻生活在惊惧的警戒之中。这在《易经》中也有反映。}

\rjpWxReaderNote{这条关于箕子的卜辞,学者一般解释为:周武王灭商后,箕子来到关中投降周朝时,周武王占卜应如何接待。但这种解释未必成立。}

\rjpWxReaderNote{《礼记》是东周时人编写的,当时的人已经不太知道商人的人祭行为,所以才会以为商人和周人一样都只用家畜献祭。还原到商代的真实场景,这显然包含人牲的叫喊。}



% new chapter mark, for the convience of reading
\section{羑里牢狱记忆}
\rjpWxReaderNote{商朝有自己的王族后裔“多子族”,任何一位族长都比周昌地位高,他没有可能进入商朝的正式权力核心。}

\rjpWxReaderNote{现代史家一般不采用《殷本纪》的说法,认为是周族的强大引起了纣王的警觉。这种解释离现代常识更近一些。不过,周昌被捕时,周族还没开始大肆扩张,商人在老牛坡的崇国完全有实力管控或消灭它。}

\rjpWxReaderNote{《易经》的很多内容都和周昌的囚禁生活有关。对他来说,这是最为惶惧的一段日子,而监禁中的闲暇,则促其潜心研究六十四卦占算之术。}

\rjpWxReaderNote{《易经》的坎卦是关于牢狱生活的}

\rjpWxReaderNote{和坎卦类似,噬嗑卦也是关于牢狱生活的记录。}

\rjpWxReaderNote{《易经》的困卦也是关于牢狱生活的。}

\rjpWxReaderNote{羑里囚禁中的周昌虽沉迷于《易经》推演,但其中并没有让他获释的秘诀,最终还是要靠他的臣僚和家人的努力。}

\rjpWxReaderNote{后世史书虽把纣王描绘成一个荒唐彻底、残暴无比的末代之君,但也强调了他的过人之处。}

\rjpWxReaderNote{看来,有一种可能是,伯邑考到达殷都后,先是和苏忿生家族建立了联系,并通过苏妲己见到了纣王,最终使父亲获释。如前文所述,剥卦六五爻曾说:“贯鱼以宫人宠,无不利。”这里说的宫内宦官贯鱼到牢狱探访文王可能就是苏妲己授意的。在后世的演义文学中,苏妲己是可怕的狐狸精,一心谋害周昌,而真实的历史很可能是,苏妲己才是让周昌获释出狱的关键因素。}



% new chapter mark, for the convience of reading
\section{翦商与《易经》的世界观}
\rjpWxReaderNote{概而言之,《易经》是由六十四卦的卦名、卦象、卦辞和爻辞组成的。}

\rjpWxReaderNote{最早从《易经》的卦爻辞中探寻历史的学者,是顾颉刚。}

\rjpWxReaderNote{而用甲骨文和商代考古知识研究《易经》的,高亨先生可谓开先河者,他的著作《周易古经今注》就是只讨论文王的《易经》,而不涉及东周时人写的《易传》,以避免让后世的误解逆行侵入商代历史。}

\rjpWxReaderNote{《易经》卦爻辞中,除了和商代的捕俘及人祭有关的内容,还包含很多周族人的活动。这应该也是周文王比较关注的内容,否则他不会一一记载下来。}

\rjpWxReaderNote{诸神的决定是因,表现到人间就是果,甲骨占卜是读取这种因果关系的工具。至于人类有时候占卜错了,那也是误读了神的旨意,错在人而不在神。}

\rjpWxReaderNote{但六十四卦则与此不同,它的原理更复杂。它认为,世间的一切并不都是由神直接决定的,而是各种事物会发生互相影响并形成一种因果发展的链条,其对应的就是卦里六个爻的阴阳顺序}

\rjpWxReaderNote{一切事物都是无常和可变的,六爻的不同组合对应着不同的卦象,哪怕只变换一个爻,也会变成另一种卦象,这就是“易”,变易无常。}

\rjpWxReaderNote{六十四卦分为三十二对,其中只有四对是这种情况:同位之爻,阴阳相反。}

\rjpWxReaderNote{总结一下,在《易经》中,每卦六个爻都是自下而上的顺序,而六十四卦组对的原则总体上是六爻“颠倒成对”;只有八个上下对称和无法颠倒的卦,才按“相反成对”的原则组成四对。}

\rjpWxReaderNote{综合前述,本书认为,它是文王发现的世间规律,或者说,一种被称为“易”的思维方式:世间的一切都不是永恒和持续不变的,它们都可以有另一种相反的存在形式,一切也都可以颠倒重来一遍。否,颠倒重来就是泰;损,颠倒重来就是益……一切事件的发展过程,都可以“倒放”一遍,从终点回到起点。这意味着,一切皆有可能。}

\rjpWxReaderNote{在六十四卦中,肯定有某个卦象代表的是商朝的崛起;而与它对立的卦,则代表了商朝的灭亡。因此,只要把这些卦找出来,研究各爻的原理,也即每个爻可能代表什么具体事件或条件(文王会因此代入不同的事物进行推演),就有可能找到灭亡商朝的密码。}

\rjpWxReaderNote{《易经》并非文王专门编写的算命教材,而更像是他自己的练习簿,所以内容驳杂,有大量的私人琐事。从萌生翦商之念始,文王就反复将其代入和推算,并随时验证、修订和增补,希望总结出一套最精确的占算方法,而最终目的,当然就是在消灭商朝的战争里运用这套预测技术。}

\rjpWxReaderNote{和《易经》的其他卦不同,乾、坤两卦在六条爻辞后,还分别有一条“用九”和“用六”,这可能是因为文王对这两卦格外重视,所以各增加了一句总结。}