\section{殷都民间的人祭}
\rjpWxReaderNote{多数规格稍高的墓都已经被盗空,剩余有殉人的墓四座。}

\rjpWxReaderNote{大司空村的发掘报告给我们展示的,不只是商代人祭的血腥,屠剥人牲固然是献祭者给诸神和自己加工食物的过程,但献祭者似乎也喜欢观赏人牲被剁去肢体后的挣扎、绝望和抗争。献祭是一种公共的仪式和典礼,从这种血腥展示中获得满足感的,应该不只是使用刀斧的操作者,更还有大司空村从贵族到平民的广大看客。}

\rjpWxReaderNote{总之,这些商人聚落的总体规律是,伴随着聚落规模的扩大,人祭和随意杀戮现象同步增长,并在殷墟末期达到顶峰。}

\rjpWxReaderNote{当谈论商文化的血腥和残暴时,我们应当知道,那时也有过戚家庄东的“箙”氏和“爰”氏这样的聚落与部族。}



% new chapter mark, for the convience of reading
\section{纣王的东南战争}
\rjpWxReaderNote{商人跟西土的羌及周族群泾渭分明,但和东夷族群的关系,以及殷都对东夷地区的控制情况,依然还有很多未知的领域。从甲骨卜辞可见,纣王很关注东夷地区,也投入了很多资源,可能是试图在商族孕育之地实现王朝的再度更新;但与此同时,颠覆商朝的创意也正在殷商内部萌生。}



% new chapter mark, for the convience of reading
\section{姜太公与周方伯}
\rjpWxReaderNote{但上述内容都属于战国说客故事的翻版,和商周之际的真实历史完全不同。在身份世袭的商周时代,不会有民间隐士而一举成为帝王师的,至少也要出自小酋长家族方有外出活动的资本。}

\rjpWxReaderNote{在人类的早期文明中,屠夫职业往往和贱民身份相联系。结合吕尚的羌人出身,他可能本是羌人吕氏部落的首领之子,年轻时被俘获而作为人牲献给商朝(或许是青年周昌的战果)。但被送到殷都后,又由于某些偶然原因,吕尚侥幸逃脱了被献祭的命运,并被某个从事屠宰业的贱民族群接纳,然后娶妻生子。这也可以解释为什么他的后人的名字有明显的商人特征。}

\rjpWxReaderNote{而《史记》记载的商纣王的荒淫无度和酒池肉林(“以酒为池,悬肉为林”),其实正是典型的商王祭祀场面,并不专属于纣王。}

\rjpWxReaderNote{在一开始,对于周邑和邑姜的恋情,双方的父亲应该都不会赞成。在吕尚眼里,周族是商朝的无耻帮凶和吕氏部族的仇家;而周昌则指望周邑联姻一家显赫的邦国,至少是苏妲己的母国苏国的级别,倘若下一任族长夫人出身殷都贱民,且她的父亲还是个老羌人,周族在商朝的形象会大打折扣。但看来周昌不是一个固执的人,对《易经》的钻研也让他明白,世界有多种可能性,不如先看看这女子家族的情况,也许会有意想不到的转机。}

\rjpWxReaderNote{暌卦的内容非常诡异,讲的是周昌的一次看上去让人莫名其妙的行程,像是去到了都市中的贫民窟(屠宰场),充斥着肮脏和混乱,而且无知的贱民也对这外来者充满着敌意。}

\rjpWxReaderNote{战国诸子及司马迁的叙事总喜欢把商朝的灭亡归因于纣王的残暴和道德堕落。}

\rjpWxReaderNote{《史记·殷本纪》还说,周昌获释之后,“献洛西之地”请求纣王不再使用“炮烙之刑”,“纣乃许之”。这实乃后世的一种道德叙事,并不符合当时的规则。}

\rjpWxReaderNote{到西周后,人们已经忘记商人的鬼神血祭文化,只剩纣王个人的种种荒淫故事保留了下来。}

\rjpWxReaderNote{根据考古和甲骨卜辞提供的真实历史背景,周昌可以被商朝称为“伯”,但说他被封为可以征伐列国的“西伯”,则应当是虚构的。毕竟,老牛坡有商人的崇侯之国,还轮不到周族来任意征伐西土。}

\rjpWxReaderNote{所谓“周公吐哺”,说的就是周公旦经常会吐出正在吃的饭食。已经遗忘了真实的商朝是怎样的后世人却对此进行了合理化解释,说是周公忙于招纳贤人所致。但事实很可能是因被迫吃掉长兄的肉酱,周公留下了严重的心理阴影。}

\rjpWxReaderNote{尤其上六爻的“婚媾有言”,说的是通婚的亲家有怨言,本书猜测,这可能也和伯邑考被献祭有关,因伯邑考之死,吕尚和周昌之间可能发生了某些争执。一旦失去周邑和邑姜的婚姻纽带,两个家族的联系会变得非常微弱,直到周昌次子周发(武王)娶了这位嫂子。}

\rjpWxReaderNote{考虑到后世经学家对此诗的注解大都空泛而不切题,这里重新翻译如下}



% new chapter mark, for the convience of reading
\section{西土之人}
\rjpWxReaderNote{《史记》里的夏商往事,大多叙事过于程式化,或者说,其中的古代圣王往往言行幼稚,不近实情,如同写给儿童的启蒙故事。}

\rjpWxReaderNote{《史记·齐太公世家》说,吕尚给文王提供的主要是用兵的权谋和从内部颠覆商朝的分化瓦解之策。}

\rjpWxReaderNote{吕尚有机会见识商朝军队的征集、编练和实战。周族人只打过部落级别的猎俘战争,最需要的就是大规模部队的正规战争经验。}

\rjpWxReaderNote{吕尚可以获悉殷都宫廷中的诸多秘闻。}

\rjpWxReaderNote{尤其,吕尚和文王又是亲家,女儿邑姜现在是武王周发的夫人,对于商周更替来说,这桩婚姻意义重大。也正因此,周人对此事的沉默就更值得玩味。}

\rjpWxReaderNote{可能是因为吕尚一言难尽的来历,以及邑姜曾经更换过丈夫,再加上伯邑考在殷都的死因一直是周昌家族的隐痛,所以在文献中,邑姜王后只能被隐藏于幕后。}

\rjpWxReaderNote{在羌人的语言里,神灵所居之山是“太”(泰)山。周灭商后,不仅吕尚被分封到山东地区的齐国,吕氏部族的其他首领还有被分封到河南南阳地区的,如申国和吕国(也称为甫国),而这些吕氏诸侯国也把山岳崇拜带到了新的封地,比如,山东的泰山或许正因此得名。}

\rjpWxReaderNote{《诗经·大雅·皇矣》也记载了周昌改造过的上帝:}

\rjpWxReaderNote{周昌重新阐释上帝还有一个好处:这是商族的古老信仰,也利于在商人内部找到共鸣,获取商人贵族的好感。帝乙和纣王两代商王以“帝”自居,唯我独尊,侵害了很多商人贵族和宗室的利益,加上纣王又经常杀戮贵族献祭,使得商朝高层人人自危。}

\rjpWxReaderNote{其一,从周原去往殷都,必须渡过黄河,这对未来的远征军是个重大考验。《易经》六十四卦中有十卦的卦爻辞出现“利涉大川”或“不利涉大川”,可见文王一直在研究渡河的时机与方法。其二,周族的规模很小,仅凭自身是无力对抗庞大的商王朝的,所以,它必须争取尽可能多的同盟军。《易经》的蹇卦和解卦成对,内容皆与派使者联络西南的盟友有关。其三,一个关键的军事策略:“利建侯。其四,文王试图把商人的铸铜技术引进周原。《易经》的蒙卦记录的是文王在殷都和商朝上层的交往,其中,六三爻曰:“见金夫,不有躬,无攸利。}

\rjpWxReaderNote{从现代人的视角看,文王周昌为翦商而推演的“理论”,或许可以分为以下三个层面:一,宗教的,即他对商人“上帝”概念的重新诠释和利用。文王的身份类似犹太教的摩西、伊斯兰教的穆罕默德,身兼部族政治首领与神意传达者两重职能。二,巫术的,即他在《易经》里对商朝施展的各种诅咒、影射与禳解之术。在上古初民时代,这些行为往往和宗教混杂在一起,不易区分。三,理性的,或者说世俗的,即各种“富国强兵”的策略和行师用兵的战术。}

\rjpWxReaderNote{祖伊和纣王的一问一答都没提及周灭黎的威胁,很显然,这是一种离现实很远的道德叙事,但有一点很明显,其预设背景是:周灭黎,是对商朝的公然背叛,威胁极大。}

\rjpWxReaderNote{后人实在难以理解,在商朝的最后几年,纣王到底处于何种状态,他为何会对周人如此咄咄逼人的扩张态势毫无反应。这似乎是个千古之谜,尤其我们试图复原这段历史时,会愈发感到其中的荒谬和不近情理。}

\rjpWxReaderNote{总之,随着周人势力的膨胀,应该有越来越多的商人朝贵开始把希望寄托在周族身上。他们或许想的是,倘若能趁着周族的叛乱搞垮纣王,扶植一位正常的新王上位,商朝应该能够回到往日的正轨。}

\rjpWxReaderNote{文王受命第六年,灭崇侯虎的崇国。}

\rjpWxReaderNote{也许,《史记》所载的文王征伐的顺序并不完全准确,而且汉唐注家对黎地和邘地的解释也未必符合文王时代的地理。概而言之,文王的扩张历程可能已经湮没在时光中,永远无法如实呈现了。}

\rjpWxReaderNote{现代人已经很难理解这种怪异的商周关系,史书文献也并未提供更多的信息,倘若非要强行给出一种貌似合理的解释,我们大概也只能说:纣王的朝廷已经无法正常履行职能。}



% new chapter mark, for the convience of reading
\section{牧野鹰扬}
\rjpWxReaderNote{在传世的儒家经典中,周灭商可以说是顺天应人,毫无悬念。但《逸周书》不同,在它的叙事中,周武王充满着对翦商事业的恐惧,经常向弟弟周公旦寻求建议和安慰。}

\rjpWxReaderNote{除却对上帝是否存在以及周邦实力的担心,武王还有一个隐忧:目前的盟友太少,只要不公开与商朝为敌,就不可能吸引更多的盟军,但过早公开,又可能招来灭顶之灾。这让武王左右为难,夜不成寐。}

\rjpWxReaderNote{如果说武王的使命是成为帝王、翦商和建设人间秩序,那么,周公的使命就是做这位帝王的心理辅导师,塑造和维护他的神武形象,如此便于愿足矣。}

\rjpWxReaderNote{或许是看到周军义无反顾的冲锋,商军中的密谋者终于鼓起勇气,倒戈杀向纣王中军。接着,西土联军全部投入了混战。}

\rjpWxReaderNote{在这之前,周族首领就已经在一定程度上“商化”了,对他们来说,语言交流和表达的难度不算大,但要用商朝人的宗教理念来解释周灭商,则需要更深地进入商人的宗教思维。正是在思考这个问题的过程中,武王对商人宗教的依赖也越来越强,可以说,灭商使得武王更加“商化”了。}

\rjpWxReaderNote{文王创制关于上帝的宗教原理,周公探索关于“德”的理论创新,但武王却与他们不同,他没有父亲的创新能力,也从未真正信服弟弟的理论,所以只能沿用强大的商朝宗教传统。换句话说,在翦商的过程中,武王自己也完成了商化。}

\rjpWxReaderNote{周公对上帝和鬼神有自己一套基于“德”的理解,曾无数次用这套理论宽慰从噩梦中惊醒的兄长。}



% new chapter mark, for the convience of reading
\section{周公新时代}
\rjpWxReaderNote{辅政期间,周公平定了叛乱,还实行了一系列重要举措来巩固新生的周王朝,比如,拆解商人社会,分封周人诸侯,等等。其中,有一项非常重要但后世已经完全忘却的举措,就是废止商朝的人祭文化。}

\rjpWxReaderNote{武王灭商,虽三月告成,但其实只是开端,因为商朝解体后,大量商人氏族还保留着武装,尤其东南夷人地区的商人势力更是毫发无损。周公这次东征,历时三年,才算是彻底消灭了商人的军事实力,把周朝的统治推进到原商朝的全部疆域。}

\rjpWxReaderNote{周公做了两方面的工作:一,拆分商人族群,消灭其军事实力和人祭宗教;二,分封各种诸侯国,统治、同化新征服的东方地区}

\rjpWxReaderNote{和周人谈话时,周公讲得最多的是“德”。}

\rjpWxReaderNote{但在对商人布置任务时,周公却又会频频谈及上帝,因为商人笃信上帝,不用上帝很难震慑他们}

\rjpWxReaderNote{而忘却是比禁止更根本的解决方式。为此,首先必须毁灭殷都,拆分商人族群,销毁商王的甲骨记录;其次,自古公亶父以来,周人曾经为商朝捕猎羌俘,这段不光彩的历史也应当被永久埋葬;再次,长兄伯邑考在殷都死于献祭,他的父亲和弟弟们还参与并分享了肉食,这段惨痛的经历也必须被遗忘。}

\rjpWxReaderNote{不仅如此,以周公为首的周朝上层还要重构新版本的历史:夏人、商人和周人没有什么区别,从来不存在人祭行为,王朝的更替只是因为末代君王的德行缺陷。在周公的诰命里,他一遍遍地重复这套新版的历史解释,终于成为西周官方定论。}

\rjpWxReaderNote{在真正的现场发言中,叔侄二人也许会对“殷多士”直言必须禁止人祭行为,但在整理笔录文献的时候,人祭却被替换成了“惟虐”“殄戮”等相对含糊的字眼,似乎殷商贵族们只是喜欢滥施刑罚而已。}

\rjpWxReaderNote{这是蛮荒上古时代的技术条件决定的:人口很少,交通通信不发达,很难用官僚制的地方层级政府管理远方,只能采用武装殖民、世袭统治的方式,也就是封邦建国的所谓“封建制”。}

\rjpWxReaderNote{册封诸侯的活动在周公主政时期才真正大规模铺开。}

\rjpWxReaderNote{被分封的各姬姓和姜姓侯国皆设立在新征服的东方地区。}

\rjpWxReaderNote{看来王朝重臣的封国都要设在最遥远的前方,这似乎是周公分封的一项原则。}

\rjpWxReaderNote{伴随着周人大分封运动的,是广泛而持续的民族融合。由此,新的华夏族逐渐成形。}

\rjpWxReaderNote{周王室以及周公后裔的鲁国,都经常和商王后裔的宋国通婚。}

\rjpWxReaderNote{新兴的周文化,是西土周族传统文化和商文化的融合:一,它继承了商人的文字体系,但部分语言习惯来自周族;二,它继承了商人的“上帝”观念,但又逐渐将其淡化为含义模糊的“天”;三,它严厉禁止商人的人祭宗教,拉远人和神界的距离,拒绝诸神直接干预人间事务;四,周人谨慎,谦恭,重集体,富于忧患意识,这些都成了新华夏族的样板品格。}

\rjpWxReaderNote{西周的主要成就是它的诸侯封国在东方发芽成长,北到燕山,南到淮河,东到山东,西到陇山,形成了以中原为中心的政治文化圈。}

\rjpWxReaderNote{晋国,最初分封时,它还只是运城盆地中各姬姓诸侯中的一个,并没有受到格外优待,所以也没人会预料到它在四百年后的急剧扩张。}

\rjpWxReaderNote{可以这么说,西周——春秋时的中原,开发程度还很低,各诸侯国的城邑就像散布在荒野中的零星孤岛,有各种土著或东来戎人部族穿插点缀其间。}

\rjpWxReaderNote{随着统治阶层的繁衍,周朝特色的贵族制度逐渐得以形成,其中,最首要的是“宗法”家族制,核心则是嫡长子一系的独尊地位。}

\rjpWxReaderNote{这套基于血缘宗法制的贵族等级和封建政治秩序,周人称之为“礼”}

\rjpWxReaderNote{典礼可以在不同层次举行,如诸侯国或大夫家,但基本原则一致。贵族的冠礼、婚礼、丧礼和祭礼也都有各等级的标准规范,几乎所有礼仪场合都有乐队伴奏,而乐队的规模和演奏的乐曲也都有相应规范。所以,周人贵族文化又被称为“礼乐文明”。}

\rjpWxReaderNote{后世周人认为,这套礼乐文明是由周公创立的,到孔子的儒家学派出现后,“周公制礼作乐”的观念则更加流行。其实,周公当政时最关注的是新兴周王朝的各种军政大事,如废除血祭、拆分商人和大分封等,还来不及注意过于细节的层面,所以礼乐制度实则是在西周朝逐渐积累和规范起来的,到春秋乃至孔子的时代都还在继续发展。}

\rjpWxReaderNote{到周幽王时期,贵族诸侯间的派系之争则更为激化。}



% new chapter mark, for the convience of reading
\section{诸神远去之后}
\rjpWxReaderNote{相比于商代,周代考古带给我们的新奇和震撼要少得多,它不再有毫无征兆而突然崛起的巨大城市,也不再有庞大而用途不详的仓储设施,当然,更没有了堆积大量尸骨的祭祀场。}

\rjpWxReaderNote{西周遗址也更符合后世人观念里的“正常”标准:}

\rjpWxReaderNote{后世人的这种“正常”观念,正是周人开创的。}

\rjpWxReaderNote{从司马子鱼的话来看,当时的宋国早就已经不用人祭祀了,而且已经重构了一套“古代”的仁义祭祀模式——在这种版本的叙事中,商人自然是不用人祭祀的。可以说,宋襄公兄弟二人的言行正是官方和暗黑两种历史共存的表现。}

\rjpWxReaderNote{当然,春秋的人祭回潮并未成为主流,可能有以下两个原因:其一,战国时期的社会重组和政治变革。}

\rjpWxReaderNote{其二,以孔子为代表的儒家逐渐兴起,开始提倡仁政和爱人。当时还有制作陶人俑随葬和埋入祭祀坑的习俗,结果遭到孔子诅咒:“始作俑者,其无后乎!”}



% new chapter mark, for the convience of reading
\section*{尾声:周公到孔子}
\addcontentsline{toc}{section}{尾声:周公到孔子}
\rjpWxReaderNote{后世人对周公的认识,有事功和制度文化两方面:事功,主要是周公辅佐成王、平定三监之乱,为西周王朝奠定开局;制度文化,主要是周公“制礼作乐”,确立西周的政体,包括诸侯列国分封格局和贵族等级制度。}

\rjpWxReaderNote{但事实上,周公最重要的工作是消灭商人的人祭宗教,以及与之配套的弱肉强食的宗教价值体系。}

\rjpWxReaderNote{为了填补人祭宗教退场造成的真空,周公发展出了一套新的历史叙事、道德体系和宗教理念。}

\rjpWxReaderNote{孔子和儒家最推崇周公,而周公思想是儒家文化的源头。周公思想的产生和形成,主要源于对人祭宗教的恐惧,以及消灭人祭宗教的需要。}

\rjpWxReaderNote{前文所述,周公销毁了商朝诸多甲骨文记录,也禁止殷周贵族书写真实的历史。但周公唯一不敢销毁的,是文王留下的《易经》。这不仅是出于对父亲的尊重,也因为《易经》是周人对翦商事业(起步)的记录,里面很可能包含着父亲获得的天机,销毁它也就是对父亲和诸神的不敬。}

\rjpWxReaderNote{周公的办法是对《易经》进行再解释,具体方法则是在文王创作的卦爻辞后面加上一段象传进行说明。象传不再鼓励任何投机和以下犯上的非分之想,全是君子应当如何朝乾夕惕,履行社会责任的励志说教,和文王卦爻辞的本意完全不同。比如,乾卦的象传是“天行健,君子以自强不息”,坤卦的象传是“地势坤,君子以厚德载物”,远比文王卦爻辞清晰易懂,而且富于积极和励志的色调。}

\rjpWxReaderNote{文王《易经》的内容本就很晦涩,所以春秋时期的贵族用它占算时,大都已经不知道或者说不再关注它的本意。其中比较明显的例子就是《易经》中的“贞”字,它的本意是甲骨卜辞中的“占”,但春秋时人却已经将其误解为“贞正”“贞操”之意了:}

\rjpWxReaderNote{孔子虽是鲁国人,但他的先祖出自宋国国君家族,所以他是商人后裔。}

\rjpWxReaderNote{考虑到孔子是专职搜集历史文献的学者,待他收集起足够多的关于商朝的碎片化知识后,是有可能逐渐拼合出一些“非官方”版本的真实历史的。}

\rjpWxReaderNote{《史记》与此相关的记载是,孔子晚年频繁地研读《易经》,结果编竹简的皮条磨损严重,经常断裂,所谓“韦编三绝”。}

\rjpWxReaderNote{这种理解已经很接近真实的文王时期,而和周公的《象辞》很不一致,说明五十七岁之后的孔子已发掘出越来越多当年被周公隐藏的真相(商朝的血祭文化)。但是,孔子没有继续点破真相,而是频繁地翻检《易经》,以至“韦编三绝”。}

\rjpWxReaderNote{孔子是商王族后裔,他应该会感念周公给了商人生存的机会,还替他们抹去了血腥人祭的记忆,让子孙后代不必活在羞辱中。周公的这些宽容而伟大的事迹,被他自己掩埋五百年,又终被孔子再次破译。这或许才是他衷心服膺周公的根本原因。}

\rjpWxReaderNote{在传统的历史叙事框架中,孔子梦周公这件事不是那么容易让人理解,甚至有些人还会觉得这是不是有点虚假,但将其放在真实的商周之际的历史背景中就好理解了:越是接近商文化的残酷真相,孔子就越是对周公有真正的理解和感激。}

\rjpWxReaderNote{《诗经》记载了周族从姜嫄、后稷以来的多篇史诗,包括周族早期历史、文王确立翦商大计、武王的灭商战争、周公平定三监叛乱以及对商文化的改造等,属于经过周公修订的官方正式版本。}

\rjpWxReaderNote{这些稍有违背周公精神的历史篇章,并没有被孔子收录进《尚书》,但经过孔门的汇集、抄写和校勘,也形成了一个汇编本,被命名为《逸周书》,意思是“未能收入《尚书·周书》的文献”。}

\rjpWxReaderNote{至于《易经》,孔子虽然很精准地选择了文王而非春秋时流行的其他版本(如本书所述,文王的《易经》隐含的商末往事很多,周公的《象辞》其实是对文王原意的掩盖和曲解),但却继续奉行周公的原则,在给弟子讲授《易经》时,尽量避开商末的真实历史,重点从《易经》文本引申出宇宙秩序和社会伦理。这些讲授被他的学生整理成《文言》《系辞》《说卦》《序卦》等篇章,与周公《象传》合编在一起,被称为《易传》(对《易经》的解释)。[插图]文王的《易经》和之后的《易传》,则被后人合称为《周易》。因此,要还原文王时代的历史,必须研究文王《易经》本身,而非《易传》,这样才能避免周公和孔子刻意制造的误导}

\rjpWxReaderNote{周公在事实上扭转了历史进程,改变了人们的认知;孔子则把这一切文本成果汇总起来,形成盖棺定论的“六经”经典,传递给后世:华夏文明的源头就是如此,再无其他。}

\rjpWxReaderNote{概而言之,周公时代变革的最大结果,是神权退场,这让中国的文化过于“早熟”;战国时代变革的最大结果,是贵族退场,这让中国的政治过于“早熟”。而在其他诸人类文明中,神权和贵族政治的退场,都发生在公元1500年之后的所谓近现代时期。}



% new chapter mark, for the convience of reading
\section*{后记}
\addcontentsline{toc}{section}{后记}
\rjpWxReaderNote{按我最初的计划,写上古,就不能再局限于商周之际,要从新石器开始,把中国早期文明产生的全过程,以及人祭宗教的来龙去脉都写出来。这意味着考古学的内容会占一大半,难度很大,毕竟进入一个新领域需要时间成本,像王国维、郭沫若、陈梦家等先贤“触类旁通”的学科拓荒时代早已过去,现代学术的数量积累已经很大,学者的研究方向也都变得深而窄,学术生涯大都只能在博士论文的基础上生发、拓展,进而成为特定领域的“专家”。换句话说,到中晚年又另起炉灶、做大跨度跳跃的可能性,已经很低了。我曾几度尝试,只感到无暇亦无力再进入新石器与夏商的考古世界。不过,当时也形成了少量文字积累,如本书中关于藁城台西商代遗址的一章。}

\rjpWxReaderNote{2020年疫情初起时,我辞去教职,获得了自由时间,先在安阳、洛阳小住过一段时间,看过殷墟和二里头遗址后,搬进了成都郊外的一处租住房屋,再次进入了新石器和上古世界。}

\rjpWxReaderNote{那时也经常问自己,用一辈子里这么长一段时间,搞这种阴沉苦闷的工作,值得吗?无奈中也安慰自己:写史写到这种状态,怕也是一种难得的经历……}

\rjpWxReaderNote{也许,人不应当凝视深渊;虽然深渊就在那里。}