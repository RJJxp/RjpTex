\renewcommand\abstractname{\Large\textbf{序}}
\begin{abstract}
    局外人(L 'Etranger)是法国作家加缪短小精悍的代表作, 二零一五年十一月二十四日购买此书, 当时读完印象深刻是后期审判中人治与法治的冲突, 可能和当时一些社会事件有关; 今天又读了一遍, 有种说不出来的感伤. 主角什么都不在乎, 社会主流观念无法束缚他,因此他永远是自由的, 但却成了局外人. 母亲去世毫不悲伤, 甚至记不清母亲年龄; 老板加薪升职一脸冷淡态度, 绝不在乎, 冷静表达自己的态度; 女友问他爱不爱我, 他只觉得不耐烦. 最后在死刑前与牧师的对峙中, 他愤怒了, 全书在这一段爆发了主角情感的疯狂宣泄, 是一个高潮. 
    
    读完对我现在也有所启发: 如果自己有想法, 立场坚定, 很多人的看法根本不用在意, 就像书中的主角, 准确表达自己的感受反而舒服点. 自己好像个局外人, 对学院的共青团的活动, 课业成绩, 小论文质量, 提不上心. 一旦目标明确, 杂七杂八的事情统统滚出脑袋吧. 人生在世, 要开心.
\end{abstract}

\addcontentsline{toc}{section}{序}
\pagenumbering{roman}