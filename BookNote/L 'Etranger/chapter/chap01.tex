\paragraph*{page~006-007}
\begin{quotation}
    \itshape
    在小停尸房里, 他告诉我说, 他进养老院是因为穷. 自己身体结实, 所以就自荐当门房. 我向他指出, 归根结底, 他也要算是养老院收容的人. 对我这个说法, 他表示不同意. 在此之前, 我就觉得诧异, 他说到院里的养老者时, 总是称之为 ``他们'', ``那些人'', 有时也称之为 ``老人们'', 其实养老者之中有一些并不比他年长. 显然, 他以此表示, 自己跟养老者不是一码事. 他, 是门房, 在某种意义上, 他还管着他们呢.
\end{quotation}

\paragraph*{page~014}
\begin{quotation}
    \itshape
    我环顾周围的田野, 一排排柏树延伸到天边的山岭上, 田野的颜色红绿相间, 房屋稀疏零散, 却也错落有致, 见到如此景象, 我对妈妈有了理解. 
\end{quotation}

\paragraph*{page~034}
\begin{quotation}
    \itshape
    整栋楼房一片寂静, 从楼梯洞的深处升上来一股不易察觉的潮湿的气息. 我只听见血液的流动正在我耳鼓里嗡嗡作响, 我站在那里没有动. 
\end{quotation}

\paragraph*{page~041-042}
\begin{quotation}
    \itshape
    他计划在巴黎设一个办事处, 负责市场业务, 直接与那些大公司做生意, 他想知道我是否愿意被派往那儿去工作. 这份差事可以使我生活在巴黎, 每年还可以旅行旅行, ``你正年轻, 我觉得这样的生活你会喜欢的.'' 我回答说, 的确如此, 不过对我来说, 实在是可有可无. 于是, 他就问我是否不大愿意改变改变生活. 我回答说, 人们永远也无法改变生活, 什么样的生活都差不多, 而我在这里的生活并不使我厌烦. 老板显得有些扫兴, 他说我经常是答非所问, 而且缺乏雄心壮志, 这对做生意是糟糕的. 他说完, 我又回去工作了. 我本想不扫他的兴, 但我实在看不出有什么理由要改变我的生活. 仔细想来, 我还算不上是个不幸者. 当我念大学的时候, 有过不少这类雄心大志. 但当我辍学之后, 很快就懂得了, 这一切实际并不重要. 
\end{quotation}

\paragraph*{page~042}
\begin{quotation}
    \itshape
    晚上, 玛丽来找我, 问我是否愿意跟她结婚. 我说结不结婚都行, 如果她要, 我们就结. 她又问我是否爱她, 我像上次那样回答了她, 说这个问题毫无意义, 但可以肯定我并不爱她. ``那你为什么要娶我?'' 她反问. 我给他解释说这无关紧要, 如果她希望结婚, 那我们就结. 再说, 是她要跟我结婚的, 我不过说了一声同意. 
\end{quotation}

\paragraph*{page~058}
\begin{quotation}
    \itshape
    我一直陪伴着他走到木屋, 他登上木台阶的时候, 我却在最低一级的前面站住了. 我脑袋已被太阳晒得嗡嗡作响, 一想到还要费劲地爬上台阶, 然后又要去跟两位妇女周旋, 心里就泄了气. 但是天气酷热, 刺眼的阳光像大雨一样从空中洒落而下, 即便站在那里一动不动, 我也感到很难受. 待在原地或者到别处走走, 反正都是一样. 稍过了一会儿, 我转身向海滩走去.
    
    海滩上也是火热地阳光. 大海在急速而憋闷地喘息着, 层层细浪拍击着沙岸. 我漫步走向那片岩石, 感到脑袋在太阳照射下膨胀起来了. 周围的酷热都聚焦在我的身上, 叫我举步维艰. 每一阵热风扑面而来, 我就要咬紧牙关, 攥紧裤口袋里的拳头, 全身绷紧, 为的是能战胜太阳与它倾泻给我的那种昏昏然的迷幻感. 从砂砾上, 从白色贝壳上, 从玻璃碎片上, 投射出来的反光像一道道利剑, 刺得我睁不开眼, 不得不牙关紧缩. 就这样我走了好久. 
\end{quotation}

\paragraph*{page~060}
\begin{quotation}
    \itshape
    太阳晒得我脸颊发烫, 我觉得眉头上已经聚满了汗珠. 这太阳和我安葬妈妈那天得太阳一样, 我的头也像那天一样难受, 皮肤底下得血管都在一齐跳动.

    ... 
    
    刀刃闪闪发光, 我觉得就像有一把耀眼的长剑直逼脑门. 这是聚集在眉头的汗珠, 一股脑儿流到眼皮上, 给眼睛蒙上了一层温热, 稠厚的水幕. 在汗水的遮挡下, 我的视线一片模糊. 我只觉得太阳像铙(nao2)钹(bo2)一样压在我头上, 那把刀闪亮的锋芒总是隐隐约约威逼着我. 灼热的刀尖刺穿我的睫毛, 戳得我的两眼发痛. 此时此刻, 天旋地转. 大海吐出了一大口气, 沉重而炽热. 我觉得天门大开, 天火倾泻而下. 我全身紧绷, 手里紧握着那把枪. 扳机扣动了, 我手触光滑的枪托, 那一瞬间, 猛然一声震耳欲聋的巨响, 一切从这时开始了. 我把汗水与阳光全都抖掉了. 我意识到我打破了这一天的平衡, 打破了海滩上不寻常的寂静, 在这种平衡与寂静中, 我原本是幸福自在的. 接着, 我有对准那具尸体开了四枪, 子弹打进去, 没有显露出什么, 这就像我在苦难之门上急促地叩响了四下. 
\end{quotation}

\paragraph*{page~065}
\begin{quotation}
    \itshape
    毫无疑问, 我很爱妈妈, 但这并不说明什么. 所有身心健康的人, 都或多或少设想期待过自己所爱的人的死亡.
\end{quotation}

\paragraph*{page~067}
\begin{quotation}
    \itshape
    首先人家把我描述成一个性格孤僻, 沉默寡言的人, 他想知道我对此有何看法. 我回答说: ``这是因为我从来没有什么值得一说, 于是我就不说.''
\end{quotation}

\paragraph*{page~069-070}
\begin{quotation}
    \itshape
    我正要对他说, 他讲的这点并不那么重要, 他如此钻牛角尖实在没有道理. 但他打断了我, 挺直了身子, 又一次对我进行说教, 问我是否信仰上帝. 我回答说不相信, 他愤怒地坐下. 他反驳我说这是不可能的, 所有的人都信仰上帝, 甚至那些背叛了上帝的人也信仰. 这就是他的信念, 如果他对此也持怀疑态度的话, 那么他的生活也就失去意义了. 他嚷道: ``您难道要使我的生活失去意义吗?'' 在我看来, 这是他自己的事, 与我无关. 我把这话对他说了. 但他已经越过桌子把刻着基督受难像的十字架杵到我眼皮底下, 疯狂叫喊道: ``我, 我是基督徒, 我祈求基督宽恕你的过错, 你怎么能不相信他是为你而上十字架的?'' 我清楚地注意到他已经称呼我为 ``你'', 而不是 ``您''了, 但我对他的一套已经腻烦了. 房间里愈来愈热. 像往常那样, 当我听某个人说话听烦了, 想要摆脱他时, 就装出欣然同意的样子. 出乎我的意料, 他竟以为自己大获全胜, 得意洋洋起来: ``你瞧, 你瞧, 你现在不是也信上帝了? 你是不是要把真话告诉他啦?'' 我又一次说了声 ``不''. 他颓然往椅子上一倒. 
\end{quotation}


\paragraph*{page~071}
\begin{quotation}
    \itshape
    法官站起身来, 好像是告诉我审讯已结束. 他的样子显得有点儿厌倦, 只是问我是否对自己的犯案感到悔恨, 我沉思了一下, 回答说与其说是正真的悔恨, 不如说我感到某种厌烦. 
\end{quotation}


\paragraph*{page~073}
\begin{quotation}
    \itshape
    我被捕得那天, 先被关在一个已经有几个囚犯得牢房里, 他们多数是阿拉伯人, 看见我进来都笑了, 接着就问我犯了什么事, 我说我杀了一个阿拉伯人, 他们一听就不再吭声了. 
\end{quotation}

\paragraph*{page~078-079}
\begin{quotation}
    \itshape
    头几个月的确很艰难, 但我所做出的努力使我渡过了难关. 例如, 我老想女人, 想的很苦. 这很自然, 我还年轻嘛. 我从来都不特别想玛丽, 但我想某一个女人, 想某一些女人, 想我曾经认识的女人, 想我爱过她们得种种情况, 想得那么厉害, 以至我的牢房里都充满了她们得形象, 到处都萌动着我的性欲. 从某种意义上来说, 这使得我精神骚动不安, 从另一种意义上说, 却又帮我消磨了时间. 
\end{quotation}

\paragraph*{page~081}
\begin{quotation}
    \itshape
    于是我悟出了, 一个人即使只生活过一天, 他也可以在监狱里呆上一百年而不至于难以度日, 他有足够得东西可供回忆, 决不会感到烦闷无聊. 从某种意义上来说, 这也是一种愉快. 
\end{quotation}

\paragraph*{page~085}
\begin{quotation}
    \itshape
    这时, 我才看清我面前有一排面孔, 他们都盯着我, 我明白了, 这些人都是陪审员, 但我说不清这些面孔彼此之间有何区别. 我只是觉得自己似乎是在电车上, 对面座位上有一排不认识得乘客, 他们审视着新上车得人, 想在他们身上发现有什么可笑之处. 我马上意识到我这种联想很荒唐, 因为我面前这些人不是在找可笑之处, 而是在找罪行. 不过, 两者的区别也并不大, 反正我就是这么想的. 
\end{quotation}

\paragraph*{page~090}
\begin{quotation}
    \itshape
    他问我, 为什么要把妈妈送进养老院, 我回答说, 因为没有钱雇人照顾她得生活起居. 他又问我, 就我个人而言, 这样做是否使我心里难过, 我回答说, 不论是我妈妈还是我自己, 并不期望从对方那里得到什么, 而且也不期望从任何人那里得到什么, 我们两人都已经习惯我们这种新式的生活.     
\end{quotation}

\paragraph*{page~101}
\begin{quotation}
    \itshape
    虽然我顾虑重重, 我有时仍想插进去讲一讲, 但这时我的律师就这么对我说: ``别做声, 这样对您的案子更有利.'' 可以说, 人们好像是把我完全撇开的情况下处理这桩案子. 所有这一切都是在没有我参与的情况下进行的. 我的命运由他们决定, 而根本不征求我的意见. 时不时, 我真想打断大家的话, 这样说: ``归根到底, 究竟谁是被告? 被告才是至关重要的. 我本人有话说要!'' 但经过考虑, 我又没什么要说了. 
\end{quotation}

\paragraph*{page~120-121}
\begin{quotation}
    \itshape
    但是, 他突然抬起头来, 两眼直盯着我, 问道: ``您为什么多次拒绝我来探望?'' 我回答说我不信上帝. 他想知道我对此是否有绝对把握, 我说我没有必要去考虑, 我觉得这个问题并不重要. 他于是把身子往后一仰, 背靠在墙上, 两手放在大腿上, 好像不是在对我说话, 说他曾经注意到有的人总自以为有把握, 实际上他并没有把握. 我听了没有做声. 他盯着我发问: ``您对此有何想法?'' 我回答说有这种可能. 不过, 无论如何, 对于我真正感兴趣的事我也许没有绝对把握, 但对于我不感兴趣的事我是有绝对把握的, 恰好, 他跟我谈的事情正是我不感兴趣的.
\end{quotation}

\paragraph*{page~125-126}
\begin{quotation}
    \itshape
    这时, 不知是为什么, 好像我身上有什么东西爆裂开来, 我扯着嗓子直嚷, 我叫他不要为我祈祷, 我抓住他长袍的领子, 把我内心深处的喜怒哀乐猛地一股脑儿倾倒在他头上, 他的神气不是那么确信有把握吗? 但他的确信不值女人的一根头发, 他甚至连自己是否活着都没有把握, 因为他干脆就像行尸走肉. 而我, 我好像是两手空空, 一无所有, 但我对自己很有把握, 对我所有的一切都有把握, 比他有把握得多, 对我的生命, 对我即将来到的死亡, 都有把握. 使得, 我只有这份把握, 但至少我掌握了这个真理, 正如这个真理抓住了我一样. 我以前有理, 现在有理, 将来永远有理. 我以这种方式生活过, 我也可能以另一种方式生活. 我干过这, 没有干过那, 我做过这样的事, 而没有做过那样的事. 而以后呢? 似乎我过去一直等待的就是这一分钟, 就是我也许会被判无罪的黎明. 没有任何东西, 没有任何东西是有重要性的, 我很明白是为什么. 他也知道是为什么. 我在我所度过的整个那段荒诞生活期间, 一种阴暗的气息从我未来前途的深处向我扑面而来, 它穿越了尚未来到的岁月, 所到之处, 使人们曾经向我建议的所有一切彼此之间不再有高低优劣的差别了, 未来的生活也并不比我已往的生活更真切实在。 其他人的死, 母亲的爱, 对我有什么重要? 既然注定只有一种命运选择了我, 而成千上万的生活幸运儿都像他这位神甫一样跟我称兄道弟, 那么他们所选择的生活, 他们所确定的命运, 他们所尊奉的上帝, 对我又有什么重要? 
\end{quotation}

\paragraph*{page~127}
\begin{quotation}
    \itshape
    他走了以后, 我也就静了下来. 我精疲力尽, 扑倒在床上. 我认为我是睡着了, 因为醒来时我发现满天星光洒落在我脸上. 田野上万籁作响, 直传到我耳际. 夜的气味, 土地的气味, 海水的气味, 使我两鬓生凉. 这夏夜奇妙的安静像潮水一样浸透了我的全身. 这时, 黑夜将尽, 汽笛鸣叫起来了, 它宣告着世人将开始新的行程, 他们要去的天地从此与我永远无关痛痒. 很久以来, 我第一次想起了妈妈. 我似乎理解她为什么要在晚年找一个 ``未婚夫'', 为什么又玩起了 ``重新开始'' 的游戏. 那边, 那边也一样, 在一个生命凄然而逝的养老院的周围, 夜晚就像是一个令人伤感的间隙. 如此接近死亡, 妈妈一定感受到了解脱, 因而准备再重新过一遍. 任何人, 任何人都没有权利哭她. 而我, 我现在也感到自己准备好把一切再过一遍. 好像刚才这场怒火清除了我心里的痛苦, 掏空了我的七情六欲一样, 现在我面对着这个充满了星光与默示的夜, 第一次向这个冷漠的世界敞开了我的心扉. 我体验到这个世界如此像我, 如此友爱融洽, 觉得自己过去曾经是幸福的, 现在仍然是幸福的. 为了善始善终, 功德圆满, 为了不感到自己属于另类, 我期望处决我的那天, 有很多人前来看热闹, 他们都向我发出仇恨的叫喊声. 
\end{quotation}
