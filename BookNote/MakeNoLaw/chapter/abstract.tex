\renewcommand\abstractname{\Large\textbf{序}}
\begin{abstract}
    20世纪60年代, 种族歧视在美国仍泛滥成灾, 平权运动兴起. 在当时的背景下, 纽约时报刊登了一则呼吁黑人抗争权利的广告, 为金博士应援. 广告中指责了阿拉巴马警察的镇压, 于是警察局长沙利文认为报纸诽谤了他, 跨州起诉, 提出高额赔偿. 荒唐的是, 对比在纽约几十万的销量, 纽约时报在阿拉巴马州只有不到400份的发行量, 而且在广告中并未提到沙利文的名字也并无任何暗示的意思, 警察局长沙利文仍然认为这篇文章对他名誉造成危害, 广告将他塑造成种族主义者, 镇压黑人平权运动. 倘若沙利文真如报纸广告所描述, 或者读者看了广告会产生沙利文是种族主义的印象, 在沙利文的亲戚朋友中, 这并没有损害他的名誉, 反而提高了他在圈子中的威望, 因为他恰恰就是这样的人. 纽约时报上的广告并不是完全如实描述, 但这些不实描述都无关紧要的一些细节, 外加阿拉巴马州州法院的法官也是种族歧视者, 抓住广告有不实之处, 判沙利文胜诉. 面临高额赔偿而破产倒闭的危机, 纽约时报上诉到最高法院, 在最高法的判决下, 赢了官司. 
    
    全书一共20章, 这是16章及之前的内容, 为了让读者理解``自由''的含义, 也讲了美国开国先贤与``防治煽动法''的故事. 17章之后是沙利文案对美国司法, 美国社会(新闻行业)的影响. 

    沙利文案最大的贡献和影响是, 改变了涉及诽谤案的举证责任. 原来的举证是被告证明刊登的文章句句属实, 现在变成了原告要去证明被告的文章确实让自己的名誉受损. 一旦要求报社自证句句属实, 无异于言论审查和自我思想阉割, 最高法的大法官们正是考虑到这一点对言论自由的影响, 判决纽约时报胜诉. 而最高法的大法官们之所以可以下这样的判决, 不光是因为他们是美国最智慧最有思辨精神的法官之一, 更是因为终身制, 解放了他们的思想, 他们不用顾虑判决结果会对自己的政治前途受到影响. 书中第\ref{mark01}章讲到``宪法的生命力之所以能恒久延续,源自法官们在适用与解释上的不断创新,以适应制宪先贤们未能预测到的社会变迁。''. 沙利文案是对第一修正案``不得立法侵犯……言论自由或出版自由''活力完美演绎. 

    法律是国家社会运行的规则. 理论上, 国家内所有活动的运作都要受到法律的约束, 然而同样的一句法律, 既可以正着说, 也可以反着讲, 或者说, 同一个案件, 可以用法律条文A辩护, 可能同时也可以用和A关联度不大, 甚至看起来相左的B条文解释. 

    法律的解释权最终归于法官. 最高法的大法官们也不非铁板一块, 有人偏保守, 有人更注重言论自由, 有人更关注联邦和州权的分配. 代表多方势力的大法官们争辩, 最后得出最终判决. 可见言论自由, 并非一种静态的, 好似前辈们奋斗的成果, 放在盘中等待我们这些后辈品尝享用. 言论自由动态的, 需要各方势力参与, 讨论争辩抗争妥协, 跟随着时代动态变化的产物. 民主社会必须要提供这样可供多方争辩抗争的环境和土壤. 

    看到阿拉巴马州跨省起诉纽约时报, 想起了多年前的鸿茅药酒案件, 警察跨省追捕``造谣''者. 仅仅警察跨省追捕并没有什么, 但到这么做之后, 没有人敢对中药或是国企进行批评. 任何实事求是的评价都无法求得对方证实, 在党领导的法官不能有自己的思考, 要以稳定为重, 自然认为这是造谣, 最后这么做的后果便是被寻衅滋事, 被颠覆政权, 被开启铁窗生活. 我们的国家没有言论自由. 吹哨人``李文亮''为什么动用新闻联播的宣传力量进行压制, 真可悲. 

    法律条文是僵硬的, 判例也带有深深的时代特色, 唯有考虑判决影响, 充满智慧的司法解释才能很好平衡法律规则和社会时代背景的冲突. 没有终身制的制度保障, 恐怕很难实现. 感触最深的一点是, 言论自由是多方势力抗争妥协出来的, 这样的抗争和妥协也需要社会的土壤. 书中最后提到膨胀的总统权力和变弱势的大法官, 环境在变. 看到沙利文案中, 州法官跨州起诉, 很自然想到鸿茅药酒和李文亮, 对此, 我仍保有悲观态度.

\end{abstract}

\addcontentsline{toc}{section}{序}
\pagenumbering{roman}