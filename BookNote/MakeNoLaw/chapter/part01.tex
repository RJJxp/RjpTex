\section*{关于言论自由的经典判词}
\addcontentsline{toc}{section}{关于言论自由的经典判词}
\rjpWxReaderNote{当报界轻率诋毁公众人物,谩骂诽谤恪尽职守的公职人员,并借用公共舆论对他们施加负面影响时,我们不能说媒体权力正被严重滥用,因为与开国先驱们当年遭受的人身攻击相比,这类言论根本算不上什么。
}

\rjpWxReaderNote{历史给我们这个国家的人民带来的启示是:尽管存在滥用自由现象,但从长远来看,这些自由在一个民主国家,对于促成开明的公民意见和正当的公民行为,可谓至关重要。}

\section*{批评的限度就是民主的尺度}
\addcontentsline{toc}{section}{批评的限度就是民主的尺度}
\rjpWxReaderNote{ 威廉·布伦南大法官撰写的本案判决,不仅适时挽救了《纽约时报》,还推动美国新闻界真正担负起监督政府、评判官员的职能,跃升为立法、行政、司法之外的“第四权”。[插图]近半个世纪之后,这起名为“《纽约时报》诉沙利文案”(New York Times v. Sullivan)的案件,仍影响着当代美国社会,与每一位普通美国人的生活息息相关}

\rjpWxReaderNote{对言论进行适当限制,当然大有必要,但是,从操作角度看,依法禁止某种言论并不可怕,可怕的是禁止者不给出明确的认定标准,想查禁什么言论,就随意给某种言论贴上禁止“标签”。}

\rjpWxReaderNote{靠更多言论矫正异议,而非强制他人噤声沉默}

\rjpWxReaderNote{它以“《纽约时报》诉沙利文案”为叙事主线,串接起美国言论自由的历史,涵盖了独立战争、制宪会议、南北战争、罗斯福新政、两次世界大战、民权运动、越南战争、“水门事件”、“伊朗门事件”等各个历史时期的重要人物与事件。书中既描述了美国建国之初的残酷党争,又涉及最高法院的人事变迁与判决内幕,完美刻画了美国法政人物群像。}

\rjpWxReaderNote{认真对照中国当下的社会现实,我们会发现,这些争议和问题,仍具有强烈的时代意义。比如,近年频繁发生的“诽谤”官员案件中,许多地方官员面对媒体或网络上的负面言论,表现出的常常是反戈一击的迅猛,跨省追捕的“豪情”,而非理性宽容与沉稳回应,甚至令“跨省”二字,都成为官员打压网络舆论的代名词。}

\rjpWxReaderNote{当原告是手握实权的政府官员时,舶来理论就不见踪影了。}

\rjpWxReaderNote{国务院总理温家宝近年更是多次强调:“国之命在人心,解决人民的怨气,实现人民的愿望,就必须创造条件,让人民批评和监督政府。”}

\section{关注他们的呐喊}
\rjpWxReaderNote{此时虽已是 20 世纪 60 年代,但南方腹地各州的种族隔离观念,仍根深蒂固。}

\section{蒙哥马利的反击}


\section{南方的忧伤}
\rjpWxReaderNote{1876 年大选中,民主党人塞缪尔·蒂尔登与共和党人拉瑟福德·海斯竞选总统,选举结果因计票纠纷而“难产”。最后,还是靠共和党人组成的特别委员会大力支持,海斯才以一张选举人票之差当选总统。这一结果,其实是政治妥协的产物。作为获胜的回报,共和党允许南方自行处理种族事务}

\rjpWxReaderNote{旧南部邦联在参众两院的十一名议员联合发布“南方宣言”,声称“最高法院在没有任何法律依据的情况下,滥用司法权力,将个人政治理念与社会立场凌驾于国家法度之上”。}

\rjpWxReaderNote{1956 年 12 月 21 日,公车上的“黑白同乘”终于成为现实。}

\rjpWxReaderNote{阿拉巴马与密西西比两州对种族平权的抵制最为强烈。1956 年,阿拉巴马州屈从于联邦压力,不得不在高等教育领域取消了种族隔离措施。黑人女性奥瑟琳·露西在联邦法院指令的护佑下,得以进入阿拉巴马州立学院就读。但是,由于其他学生骚动抗议,校方不得不将她逐出校园。露西再次向联邦法院求助,希望能重返校园。校方被她以诉讼维权的做法激怒,干脆以“道德败坏”为由开除了她。艾森豪威尔政府始终无所作为,这起案件亦无果而终。}

\section{初审失利}
\rjpWxReaderNote{在阿拉巴马州,连找一位合作律师都如此困难,显然是种族因素在作祟。}

\rjpWxReaderNote{在这种情势下,没人乐意与《纽约时报》扯上关系,哪怕是经常为罪大恶极的被告辩护的律师们,也惟恐避之不及,不敢蹚这摊浑水。}


\section{媒体噤声}
\rjpWxReaderNote{琼斯法官的判决,使报道南方种族主义真相的行为,在 20 世纪 60 年代变得险象环生,报纸必须背负支付巨额诽谤赔偿的风险。哪怕相隔千里,只要你在阿拉巴马州派驻了记者,发行了几份报纸,或者承接了一些广告义务,都可能被强制到该州法院受审。一个在报道或广告中连姓名都没出现过的政府官员,可以轻易说服陪审团相信,报纸关于当地情况的报道是在影射他。如果上述文字令他感觉名誉受损,将被推定为不实言论。报纸若想逃避这一指控,必须举证证明报道所有细节都准确无误。至于原告受到多大损害,根本没有衡量标准,即使没有证据表明原告因不实报道受到伤害,当地陪审团一样可以随意确定赔偿金额。}

\rjpWxReaderNote{或许是有意为之,蒙哥马利《广告报》关于此案报道的标题一语道破玄机:“州法院成功追责州外媒体”。这就是本案的效果,也是沙利文与帕特森州长的真实目的。}

\rjpWxReaderNote{他们将传统上用来挽回个人名誉的诽谤诉讼,成功转化为挟制媒体的政治利器。这些人哪里是想打击什么不实报道,其根本目的,就是阻止媒体对白人至上社会丑陋现实的揭露:争取投票权利者遭遇私刑惩处;无良法官利用州法压制宪法权利;警察局长纵容警犬攻击商场内的黑人男女。他们认为,通过威慑吓阻,就能令全国媒体——报纸、杂志和电视——对民权事务的报道敬而远之。}

\rjpWxReaderNote{通过民事诽谤诉讼打压媒体的策略,很快被各地复制。哥伦比亚广播公司(CBS)仅仅因为报道黑人在蒙哥马利市获得选举权之难,就被人要求索赔 150 万美元。南方其他各州官员也开始效仿阿拉巴马州的做法。截至联邦最高法院 1964 年对“沙利文案”宣判之前,南方各州官员对媒体提起的诽谤诉讼总额,已高达 3 亿美元。}

\rjpWxReaderNote{过去,对多数持中立态度的人民而言,强制种族隔离与‘州权至上’和‘南方生活’一样,尚是抽象概念,如今,此起彼伏的各类暴行,却给了美国人民最及时、最直观的印象。原来,有这么多成年男女,正如此恶劣地对待他的同类,仅仅因为他们是有色人种。这充分说明,所谓南方道德,已经完全破产。}

\rjpWxReaderNote{1787 年制宪会议以来,美国人民即享有知悉执政者所作所为的自由,并可自由批评或更换当权者。1791 年增补的宪法第一修正案,禁止国会立法侵犯公民的言论自由和出版自由,成为表达自由(freedom of expression)的重要保障。}

\rjpWxReaderNote{后来,阿拉巴马州的种族歧视现象逐渐缓和,包括法院在内的政治生态也发生很大变化。1975 年,《纽约时报》的代理律师之一,埃里克·恩布里成功当选为阿拉巴马州最高法院大法官。}

\rjpWxReaderNote{针对陪审团关于广告“指涉旦关系到”沙利文的判断,州最高法院指出:“众所周知,诸如警察、消防之类的部门,均受政府控制、调遣,有时直接听命于一位市政专员。因此,对相关团队的赞美或批评,直接影响到人们对掌控团队者的评价。”上述关于“指涉且关系到”的认定标准,显然对新闻界乃至任何试图评点政府作为的个人、团体,都非常不利。如果按照这个标准,任何对阿拉巴马州公共事务的评论,如对“警察”的议论,都会被认定为对警察局长或相关官员的个人攻击,并因涉嫌诽谤而被处以巨额赔偿。}

\section{自由的含义}
\rjpWxReaderNote{制宪会议的目的,是建立一个人民高度自治的共和国。人民有权通过法案,选举领导人,也有权更换执政者。但是,大家并没有想出一个万无一失的方案,确保政府权力不被滥用。后来,会议采纳了麦迪逊提出的所谓“辅助性预防措施”,即分权制约策略。首先,各州保有主权,联邦政府仅享有特定权限,如管理外交事务、处理州际贸易。其次,联邦政府分为三个独立分支:立法、行政与司法。某一分支权力过分扩张时,另两个分支可施以反制,这种彼此制衡的权力设计,足以防止出现独裁政治。制宪先贤深信,三权分立能够防范暴政,从对权力架构的设想而言,他们当然有理由骄傲,然而,这些人却低估了当时的社情民意。}

\rjpWxReaderNote{是否可以更加直白地理解为,美国人可以畅所欲言,自由决定言说、出版的内容,而无须顾忌法律惩罚呢?当然,这样的情景,从未成为现实。果真如此,写勒索信就不会入罪,因为勒索者的行为只是说说写写。此外,循此逻辑,在法庭作伪证,当然也不会被判有罪。那么,言论与出版自由条款的含义究竟如何确定?}

\rjpWxReaderNote{制宪先贤们选择原则表述,而非精确界定,自有其良苦用心。他们这么做,是为避免后人受制于过于精确的条文。因为条文愈是细致,时代气息愈是浓厚,一旦时过境迁,反会成为阻碍后人与时俱进的枷锁。一部巨细靡遗的宪法,显然无法垂范久远,因此,制宪者只能用简略语言,给出权利保护的价值指向:“不得立法……侵犯言论自由或出版自由。”他们刻意令词义宽泛,正是为便于后人顺应时势,灵活解释。1819 年,马歇尔首席大法官就说过,宪法“注定流芳百世,并将不断回应解决人民面临的各种危机”。正是因为许多伟大法官能够准确把握宪法精神,适时解释宪法含义,这部宪法才至今仍被我们作为基本法遵守。}

\rjpWxReaderNote{许多专制政权用残酷刑罚令自己与异议绝缘。苏联解体前,刑法中即有“煽动颠覆苏联罪”,与“煽动诽谤政府罪”并无二致。}

\rjpWxReaderNote{早期的美国报纸,充满各种污蔑中伤之词。今天,那些认为自己受到媒体不公正对待的政治人物,若能回头看看两百年前的情形,可能觉得自己的遭遇不过是小儿科。}

\section{言者有罪}
\rjpWxReaderNote{“或许存在这么一条普遍规律:国家之所以压制某种自由,就是为了应对某种实际存在或者假想中的外部威胁。”1798 年 5 月 13 日,詹姆斯·麦迪逊在致副总统杰弗逊的信中,提出了上述论断。[插图]多年来,这条规律在美国历史上被反复印证,因为政客们压制公民自由的借口,多是对外部意识形态的畏惧。麦迪逊写这封信时,美国正弥漫着对法国革命思潮的恐惧。
}

\rjpWxReaderNote{当时,美国政党制度刚见雏形。制宪者没有预见到政党出现,未设计全民普选制度,总统靠选举人团选举产生。第一次总统大选,华盛顿自然无人匹敌,当选总统算得上众望所归。华盛顿执政期间,时任副总统亚当斯与国务卿杰弗逊身边分别形成两大政治势力。1796 年,华盛顿两届任期届满前,新总统大选拉开序幕,亚当斯最终以 71 张选举人票对 66 张选举人票战胜杰弗逊,当选为美国第三任总统,杰弗逊成为副总统。(美国现行的总统、副总统选举制,由 1804 年通过的宪法第十二修正案确立。)}

\rjpWxReaderNote{他们深信,只要及时打压批评政府之声,尤其是共和党旗下媒体的言论,就足以遏制敌对势力。这一时期的美国,政府对舆论的压制,完全受政党利益驱动。}

\rjpWxReaderNote{更过分的是,现代历史研究者们发现,当年凡是涉及《防治煽动法》的案件,联邦法官和执法官都会在陪审团中安插许多联邦党人。
}

\rjpWxReaderNote{艾伦很轻易地将对政府的批评与煽动叛乱联系起来,事实上,报纸只是呼吁人民用选票来更换执政党,而不是煽动颠覆政府。}

\rjpWxReaderNote{在对《防治煽动法》的一片抗议声中,数麦迪逊的声音最为强劲。此法一经国会通过,他与杰弗逊立即决定在各州立法机构发起阻击。不过,一切行动都是秘密进行,以免授人以柄,反被检控——可悲啊,这两个人,一个是堂堂的宪法之父,一个是美国副总统,居然还得偷偷摸摸行事!}

\rjpWxReaderNote{麦迪逊所说的“自由检视公众人物和公共事务的权利”,引起各方广泛回应,并在未来数十年里,逐步成为美国政治体制的前提,并被后人称为“麦迪逊前提”。}

\rjpWxReaderNote{不管真实作者是谁,“少数派声明”的确完美阐释了《防治煽动法》的政治前提。作者认为政府弱不禁风,需要时刻防范“邪恶公民”的侵扰。这一观点与杰弗逊和麦迪逊的看法完全背道而驰。杰弗逊认为,民主政府足以经得起社会风险与变革的考验。麦迪逊坚持人民才是国家的主人,有权选择暂时执政者。然而,对“少数派声明”的支持者而言,政府权力至高无上,有权以各种方式自保。这完全是典型的英式论调。}

\rjpWxReaderNote{从 1798 年 7 月到 1801 年 3 月,短短三年内,有十四人因违反《防治煽动法》下狱。这些人多数是亲共和党的主要报纸的老总或编辑,如费城《曙光报》、波士顿《独立纪事报》、纽约的《百眼巨人报》、巴尔的摩的《美国人报》、里士满的《检查者报》。纽约州有两家报社,因受到《防治煽动法》指控而关门大吉。1800 年 4 月至 8 月,康涅狄格州新伦敦的《蜜蜂报》因编辑查尔斯·霍尔特入狱,停刊了四个月。绝大多数这类案件的审判,都发生在 1800 年,这可绝非巧合。因为 1800 年正好是大选年,亚当斯与杰弗逊再次角逐总统之位,亚当斯的国务卿蒂莫西·皮克林大兴文字狱,正是为了逼迫杰弗逊阵营的报纸在大选时保持沉默。}

\rjpWxReaderNote{1801 年 3 月 4 日,杰弗逊宣誓就职时,《防治煽动法》埋下的积怨,正逐步被一种更为理性、健康的传统所替代。杰弗逊在就职演说中说道:“我们都是共和党人,我们也都是联邦党人。如果我们当中有任何人试图令联邦解体,或者改变共和政体,就让他们不受任何干扰地畅所欲言吧。容忍错误意见的存在,让不同观点辩驳交锋,正是我们得享安全的基石所在。”至此,美国人因发表政见而被追究刑责的时代已经结束,或者说,看上去仿佛如此。}
