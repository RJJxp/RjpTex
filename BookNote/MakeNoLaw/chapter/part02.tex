\section{“人生就是一场实验”}
\rjpWxReaderNote{霍姆斯认为,就算编辑辩称自己句句属实,但越这么说,越具有社会危害性,因为这样“极大妨害了司法权威”。参照上述标准,只要被扣上“危害社会安全”的大帽子,言论自由简直就形同虚设。}

\rjpWxReaderNote{汉德在这起案件中的判决意见,具有里程碑式的意义,这是美国法官首次在判决书中阐释言论自由和出版自由的意义。汉德法官说,《群众》杂志刊载的文章和漫画,“虽对战争充满敌意”,“令人不快”,试图激起反战情绪,但是,“不管这些作品是适度的政治推论,还是过激、不当的谩骂,在美国这个以言论自由为权力最终根源的国家里,人人皆享有批评政府的自由……钳制这些可能动摇人民意志的言论,无异于镇压所有对立评论和意见……言论只有在直接煽动叛乱、反抗等行为时,才构成犯罪,如果把合法的政治言论当作调唆煽动,就是驱逐了民主政治的守护神,是最大的不宽容。”在和平言论被普遍打压的年代里,作出这样的判决,需要很大的勇气。}

\rjpWxReaderNote{根据文森特·伯雷西教授的说法,汉德最大的创举,是将有敌意的批评也列入言论自由范畴,并将之视为民主社会中的“权力最终根源”。}

\rjpWxReaderNote{那些被政府官员们深恶痛绝的批评意见,反而是赋予政府合法性的权力之源。}

\rjpWxReaderNote{因为在俄亥俄州的这场演说,德布斯被以妨碍征兵为由,根据《防治间谍法》判处十年监禁。他最终只服了三年刑。(五十年后,越南战争爆发,当时的反战言论比德布斯的演说更加激进、刺耳,却无人因言获罪。)}

\section{伟大的异议者}


\section{“三天过去了,共和国安然无恙!”}
\rjpWxReaderNote{当报界轻率诋毁公众人物,谩骂诽谤恪尽职守的公职人员,并借用公共舆论对他们施加负面影响时,我们不能说媒体权力正被严重滥用,因为与开国先驱们当年遭受的人身攻击相比,这类言论根本算不上什么。如今,我们政府的行政架构已愈加叠床架屋,渎职、贪腐几率陡增,犯罪率屡创新高。玩忽职守的官员与黑帮分子狼狈为奸、包庇犯罪,将对人民的生命和财产安全构成极大威胁,因此,对勇敢、警觉的媒体之需要,显得尤为迫切,在大都市里更是如此。}

\rjpWxReaderNote{“五角大楼文件案”显示出,事前限制是比事后追惩更加危险的措施。政府原打算将《纽约时报》的编辑层和管理层全部送进班房,最终反而自取其辱,令自己陷入政治泥潭。申请禁令固然简单,但让所有媒体集体沉默,可没那么容易。}

\rjpWxReaderNote{“尼尔诉明尼苏达州案”将美国人的表达权利,彻底从英国式的事前限制中解放出来。}



\section{向最高法院进军}
\rjpWxReaderNote{牧师们在申请书中告诉最高法院,“布朗诉教育委员会案”判决八年后,阿拉巴马州的公立学校仍然维持种族隔离政策。}

\rjpWxReaderNote{诉状援引了杰弗逊 1804 年写给阿比盖尔·亚当斯的信,信中说:“就算认定《防治煽动法》违宪,也未必能遏制混淆是非的诽谤洪流,因为限制言论自由的权力,已落入各州立法机构之手。”}



\section{“永远都不是时候”}
\rjpWxReaderNote{一份精心制作的诉状,首先应流畅易读,就像讲述一段精彩故事,要令读者欲罢不能,恨不得一气读完。最高法院大法官们平日阅文无数,最厌烦枯燥冗长的法律论证。所以,起草者必须去芜存菁,及时呈现最有用的观点。}

\rjpWxReaderNote{惩罚性赔偿多在侵权案件中适用,当然也包括诽谤案件,这类赔偿的目的不是为补偿原告损失,而是借此形成威慑,使其他人不敢再从事同类违法行为。}

\rjpWxReaderNote{《芝加哥论坛报》的意见书列举了几起美国政府试图以诽谤法逼迫媒体噤声的判例。一起是韦克斯勒在诉状中提到的“芝加哥市诉《芝加哥论坛报》公司案”。意见书说,“芝加哥市长威廉·汤普森贪污渎职,无恶不作,当初想尽千方百计,压制本报对他的批评。”这份意见书还以罕见篇幅,详尽描述了 17 世纪英国的煽动诽谤政府法,对违反者施加的残酷刑罚:削鼻、割耳、绞刑、肢解……还把阿拉巴马州的判决称作“煽动诽谤政府法投胎转世”。}

\rjpWxReaderNote{如果“对官员的批评之所以猛烈”,是因为“批评者内心坚信自己所言皆为真相”,这样的言论就应当受到第一修正案保障。}

\rjpWxReaderNote{在关于政治话题的激烈争论过程中,难免会有人情绪激动,甚至有过激之举,大家唇枪舌剑,激烈攻辩,多基于各自有限认知,这些认知当时或许被笃信不疑,事后却可能被证明是断章取义、错谬曲解。}

\rjpWxReaderNote{有时,往往是那些合理却又无法证明的怀疑,又或未经确认的“内部消息”,暴露出官员的无能、过失或恶行……如果用诽谤诉讼威胁那些因真诚相信错误事实而批评官员者,或者要求批评官员的出版物必须证明自己提到的每一处细节都绝对真实,必将扼杀所有对政府或官员的批评。}

\rjpWxReaderNote{意见书写道:“就算这是一起诽谤案件,可是,《纽约时报》仅仅因为一则政治广告,就涉嫌诽谤,并被判巨额赔偿。如果连报纸都会因广告中的无心之失而付出惨痛代价,还有哪个异议团体敢借助出版,表达他们对公共事务的看法?”}



\section{最高司法殿堂上的交锋}
\rjpWxReaderNote{布朗奈担任过艾森豪威尔政府的司法部长,在任期间,协助总统提名过三位最高法院大法官,并在其中起到重要作用。这三位大法官分别是:首席大法官厄尔·沃伦、联席大法官约翰·马歇尔·哈伦二世和小威廉·布伦南。班克罗夫特后来在备忘录中,记录了与洛布的谈话内容:“我说,《纽约时报》应尽量避嫌,以免给人们造成这么一种感觉,好像我们故意请了一位在艾森豪威尔时期与最高法院颇有渊源的人物,而且,此人还在几位大法官提名过程中起过关键作用。我说,这么做或许显得过于谨慎,但我们必须如此。”}

\rjpWxReaderNote{若有幸目睹大法官们的法庭提问,会发现他们思维开阔、开诚布公,毫无矫揉造作之气,在充斥官僚主义、繁文缛节的首都,最高法院貌似守旧、弱小,却自有一股凛然气势。}

\rjpWxReaderNote{审判席上,首席大法官居中而坐,其他大法官以资历为序,在两侧就座:资历最深者居首席大法官右侧,第二资深者居首席大法官左侧,依此类推。}

\rjpWxReaderNote{布莱克大法官明明是最高法院最支持言论自由的人,此时为何对韦克斯勒步步紧逼,逼迫他承认陪审团认定广告内容侵犯到沙利文名誉是合情合理的呢?布莱克大法官很重视陪审团的作用,他自己就曾是一位技艺娴熟的庭辩律师。在最高法院,他经常呼吁大家尊重陪审团的裁断。他之所以不断追问,其实另有深意,他希望这起案件能被看作一起批评政府官员的案件,这样一来,最高法院就能直面韦克斯勒提出的更广泛意义上的解释。布莱克的最终目的,是想让最高法院承认,即使是对政府官员的直接批评,也是受第一修正案保护的。}

\rjpWxReaderNote{本案“虽伪装成民事诽谤诉讼,却是对新闻自由、言论自由和集会自由赤裸裸的侵犯”}



\section{批评官员的自由}
\rjpWxReaderNote{任何事务一旦实际运转,总难避免某种程度上的滥用,这类情形在新闻界体现得尤为明显。}

\rjpWxReaderNote{尽管存在滥用自由现象,但从长远来看,这些自由在一个民主国家,对于促成开明的公民意见和正当的公民行为,可谓至关重要。宪法第一修正案从来不拒绝对不恰当的、甚至错误的言论进行保护。}

\rjpWxReaderNote{政府官员名誉受损,并不意味着我们要以压制自由言论为代价进行救济。}

\rjpWxReaderNote{以及麦迪逊稍早前在众议院说过的话:“如果我们留意共和政府的性质,我们不难发现:如果有检查言论的权力,那也应当是人民检查政府的言论,而不是政府检查人民的言论。”}

\rjpWxReaderNote{在这份判决意见中,布伦南大法官的做法显然异乎寻常:他宣布一部一百六十三年前就已失效的国会法律违反宪法。他用麦迪逊、杰弗逊、约翰·尼古拉斯和所有对《防治煽动法》奋起抗争的共和党人的观点,作为自己的论证依据。这些民主先驱们做梦也不会想到,他们的这场斗争,居然会在一百多年后的最高法院内开花结果。}

\rjpWxReaderNote{批评政府官员的行为应享有绝对豁免权,哪怕其事实有误,甚至错得离谱。}



\section{“这是值得当街起舞的时刻”}
\rjpWxReaderNote{我们拥有一部成文宪法,并仰赖其至始未变之本质,为这个瞬息万变的社会,注入安定之力。然而,宪法的生命力之所以能恒久延续,源自法官们在适用与解释上的不断创新,以适应制宪先贤们未能预测到的社会变迁。}\label{mark01}

\rjpWxReaderNote{一百七十多年来,诽谤法一直只是个人名誉受损时的救济措施。但是,南方的官员、陪审团和法官们,居然为了政治目的,扭曲诽谤法的本质,使之沦为打压批评种族隔离言论的工具。}

\rjpWxReaderNote{对公共事务的讨论不只是一种自我表达,更是人民自治的基础。}

\rjpWxReaderNote{沙利文案”最重要的变革之一,是改变了诽谤诉讼中的举证责任,其次则是引入了“过错”要件。}

\rjpWxReaderNote{“沙利文案”判决彻底改造了普通法中的诽谤诉讼程序。从此以后,原告必须证明被告明知所言不实,存在重大过错,或者罔顾真相,明显不负责任。}

\rjpWxReaderNote{最高法院同样认为,言论自由应容忍错误存在,甚至是一些严重错误。判决提出,仅仅保护实事求是的陈述是不够的,因为人们有时会因害怕犯错,而不敢对政府提出批评。因此,为了防止人们自我审查,必须允许他们存在“犯错的空间”。}

\rjpWxReaderNote{不能把对政府的批评,等同于对官员的诽谤,如今已是不言自明的真理,但在 1964 年,情况却并非如此,而在那些没有保护言论自由传统的国家,现实情况要更加糟糕。}

\rjpWxReaderNote{最高法院要求败诉方承担此案的文件印刷费及其他杂项费用,共计 13000 美元。沙利文向最高法院提出,希望与《纽约时报》分摊这笔费用,但被法院拒绝。}



\section{判决背后的纷争}
\rjpWxReaderNote{我国曾对一项原则作出过深远承诺,那就是:对公共事务的辩论应当不受抑制、充满活力并广泛公开,它很可能包含了对政府或官员的激烈、刻薄,甚至尖锐的攻击}

\rjpWxReaderNote{经历过 1930 年代的“罗斯福新政”和第二次世界大战,政治权力重心逐步移至华盛顿。为遏制全国性“大萧条”,罗斯福总统和国会果断行动,推行新的经济措施,随着罗斯福任命的大法官陆续进入最高法院,最高法院逐步认可了这些措施的合宪性。联邦政府的权力触角,陆续进入许多过去不能轻易触及的领域,如管制最高工时与劳资关系、限制农民种植面积、发放福利津贴,等等。各州政府亦因此被人们讥讽为“无能的古董”。}



\section{连锁反应}
\rjpWxReaderNote{他说,之所以不能仅仅因为一些言论有不实成分,就解除对它们的宪法保障,主要基于两个原因:首先,在自由辩论过程中,错误在所难免。其次,在公共讨论领域,“真相”是一个难以界定的概念,如杲让事先已存偏见的陪审团去认定何为“真相”,很可能催生新闻审查制度。任何一个经历过“斯寇普斯审判”的国家,都不会因陪审团认定的事实有误,就急匆匆查禁某种思想。}

\rjpWxReaderNote{如果我们想确定一种思想是否真理,就应让它在思想市场的竞争中接受检验。}



\section{“舞已结束”}
\rjpWxReaderNote{ 报道说,经此一役,《电讯报》斗志尽丧,从此放弃报道政府的不法行为,并要求记者给任何采访对象发函前,必先征求编辑意见,甚至销毁了所有日后可能引发诽谤诉讼的信函、便条。[插图]一次,有人向《电讯报》爆料说,当地一位警长涉嫌滥用职权,编辑不仅放弃这一选题,还语重心长地告诫记者:“这次还是让别人去冒险吧。”}

\rjpWxReaderNote{这起案件充分显示了诽谤官司的威慑作用,尤其是对那些不敢担当的出版商而言。所以说,不是法院,也不是政府在打压此书,是出版商自己在吓自己。}

\rjpWxReaderNote{纽约时报》诉沙利文案”二十年后,诽谤诉讼竟成为美国一项欣欣向荣的产业。从演艺明星、商业大佬,到将军、州长、参议员,都踊跃成为诽谤官司原告。}

\rjpWxReaderNote{地产名人唐纳德·特朗普因一篇嘲弄其建筑规划的专栏文章,愤而起诉《芝加哥论坛报》和相关评论员,并索要 5000 万美元的赔偿。幸好,特朗普最终输了官司。不过,照这个趋势,迟早有人会提起标的额超过 10 亿美元的诽谤诉讼。}

\rjpWxReaderNote{针对诽谤诉讼成风的现象,有人提供了这么一种解释:“沙利文案”之后,媒体对政治人物的批评愈发不留情面,行文谋篇更是刀刀见血。}

\rjpWxReaderNote{20 世纪后半期的美国人,普遍信奉法律中的“免费午餐”理论。他们认为,只要自己的不幸可归咎于他人,即可获得金钱补偿。如果恰好是诽谤案件,可能会比普通侵权诉讼捞得更多。}

\rjpWxReaderNote{而且,即使发生这类情形,它给法律系统带来的负面影响,远远不及赋予媒体保密特权来得大。}

\rjpWxReaderNote{“伊穆诺公司案”的一波三折,让人们见证了“沙利文案”二十五年后,各州诽谤法的变化。我们看到,除非法官敏锐地选择即决裁判,否则,财大气粗的公司、财团将用拖沓冗长的诉讼程序、令人咋舌的索赔金额,将拒绝妥协的被告逼到破产境地。}

\rjpWxReaderNote{在公共讨论领域,‘真相’是一个难以界定的概念,如果让事先已存偏见的陪审团去认定何为‘真相’,很可能催生新闻审查制度。}

\rjpWxReaderNote{“威斯特摩兰案”之所以被广泛宣扬,是因为人们对诽谤法日益不满,或者说,是对“《纽约时报》诉沙利文案”确立的宪法规则不满。媒体和富有公共意识的公民们发现,虽然“沙利文案”判决倡导言者无罪,可他们仍有可能遭受诽谤官司与巨额诉讼费用的打击、拖累。}



\section{重绘蓝图?}
\rjpWxReaderNote{来信者是《纽约时报》副主编莱斯特·马克尔。马克尔是位很有主见、自律甚严的编辑,他说,虽然报社在“沙利文案”中取胜,但自己却非常忧虑,担心“此例一开,媒体愈发缺少责任感,也愈发不受约束”}

\rjpWxReaderNote{韦克斯勒在回信中指出,根据最高法院的判决,官员只能在媒体蓄意造假或罔顾真相,刊出不实报道时,才能获取赔偿。但是,他仍然认为,媒体应拥有绝对豁免权。在他看来,官员能否在媒体蓄意造假的情况下获取赔偿,取决于“陪审团掌握信息的可靠性”。}

\rjpWxReaderNote{经常充当被告的《费城询问者报》编辑尤金·罗伯茨后来说,“沙利文案”判决催生出一种打压媒体的新做法:“让媒体闭嘴沉默的现代方法,就是用漫长的诽谤官司拖垮他们。”}

\rjpWxReaderNote{哈伦大法官曾在“罗森布鲁姆案”中指出:“诽谤法的功能,主要是补偿个人经受的,可以计量的实际损害。……如果原告并未因不实报道受到任何损害,陪审团却判令被告赔偿……只会起到与第一修正案目的相抵触的效果。”这一评价实在再恰当不过。在一个倡导自由讨论的国度,诽谤法本应仅限于补偿损失,绝不能成为威胁新闻自由或打压改革言论的工具。但是,理想终究照不进现实。在多数情况下,诽谤诉讼中的赔偿金额要远远超过原告的实际损失。}

\rjpWxReaderNote{本案中,由于罗伯特·韦尔奇主办的《美国主张》杂志给民权律师埃尔默·葛茨贴上“列宁主义者”的标签,陪审团判令杂志赔偿葛茨 40 万美元。七十六岁的葛茨性格风趣,对判罚金额也十分满意。他决定用这笔钱与妻子一同乘游轮环游世界,他在临行前说:“我打算每到一处港口,就给韦尔奇先生发一封电报。”}

\rjpWxReaderNote{“沙利文案”虽然阻止了某些人压制舆论的努力,却使一些试图从谎言、污蔑中挽救个人名誉的人灰心丧气。}

\rjpWxReaderNote{“《好色客》诉福尔韦尔案”对言论自由的意义十分重大。它充分显示,在最高法院,即便是保守派大法官,也深刻意识到,宪法要求美国社会对公共事务的讨论保持宽容态度——这其中,不仅包括被画成驴的乔治·华盛顿,也包括“与母亲在洗手间乱伦”的福尔韦尔。}



\section{乐观主义者}
\rjpWxReaderNote{第一修正案 1791 年并入宪法后,一百多年来,它对言论、出版自由的保护功能,完全处于休眠状态。第一次世界大战期间,各州与联邦纷纷出台立法,压制言论自由,最高法院对此亦麻木不仁,甚至容许追惩反战或激进言论。之后四十年间,最高法院才逐步适用言论、出版自由条款,保护某些异见或非主流言论。但是,对那些颠覆或挑战现行秩序的言论,大法官们仍刻意排斥,不愿将之纳入第一修正案的保护范围。但是,“沙利文案”之后,最高法院开始彻底践行第一修正案的承诺:在美国,“不得立法侵犯……言论自由或出版自由。”}

\rjpWxReaderNote{他说,焚烧国旗的举动,也是表达政治立场的一种沟通方式,并指出:}

\rjpWxReaderNote{他在裁定中写道:“国家安全并非自由堡垒内的唯一价值。安全必须建立在自由体制的价值之上。为了人民的表达自由和知情权等更为重要的价值,政府必须容忍一个不断找茬的新闻界,一个顽固倔强的新闻界,一个无所不在的新闻界。”}

\rjpWxReaderNote{事实上,政府完全夸大了国家安全体系内的保密需要。}

\rjpWxReaderNote{“任何经常接触机密文件的人都清楚,许多政府文件的密级都被刻意提高,分类标准根本不是什么国家安全,而是为了掩饰政府的尴尬与窘况。……这是我们在‘五角大楼文件案’中的深刻体会。}

\rjpWxReaderNote{最高法院之所以如此迁就政府关于国家安全的主张,无疑是总统权在 20 世纪不断扩张的反映。}

\rjpWxReaderNote{上述秘密仿佛印证了麦迪逊那句名言:独裁和低效,往往是缺乏公开辩论和舆论监督的产物。}

\rjpWxReaderNote{我们应当对某种做法时刻保持警惕,那就是对那些我们深恶痛绝,甚至认为罪该万死的言论的不当遏制。}

\rjpWxReaderNote{他认为,许多国家不会像美国人那样,对打着纳粹旗帜穿街过巷之类的“极端政治表达”保持宽容态度,这是因为,美国人受自身历史影响,具备欧洲人无法拥有的个性,那就是:“根深蒂固的社会和历史乐观主义”。}

\rjpWxReaderNote{当最高法院依循麦迪逊的美好理想,去审理“《纽约时报》诉沙利文案”时,大法官们虽未明示,但他们的内心深处,也一定洋溢着这样的乐观主义。}