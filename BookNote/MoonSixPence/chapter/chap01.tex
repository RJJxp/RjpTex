\section{第一部分}

\paragraph*{page~001}
\begin{quotation}
    \itshape
    我所谓的伟大不是走红运的政治家或是立战功的军人的伟大; 这种人显赫一时, 与其说是他们本身的特质倒不如说沾了他们地位的光, 一旦事过境迁, 他们的伟大也就黯然失色了. 人们常常发现一位离了职的首相当年只不过是个大言不惭的演说家; 一个卸甲归田的将军无非是个平淡无味的市井英雄. 但是查理斯$\cdot$思特里克兰德的伟大确是真正的伟大. 
\end{quotation}

\paragraph*{page~002}
\begin{quotation}
    \itshape
    崇拜者对他的赞颂同贬低者对他的诋毁固然都可能出于偏颇和任性, 但有一点是不容置疑的, 那就是它具有天才.

    而那位克里特岛画家的作品却有一种肉欲和悲剧性的美, 仿佛作为永恒的牺牲似地把自己灵魂的秘密呈现出来.
\end{quotation}

\paragraph*{page~005}
\begin{quotation}
    \itshape
    制造神话是人类的天性. 对那些出类拔萃的人物, 如果他们生活中有什么令人感到诧异或者迷惑不解的事件, 人们就会如饥似渴地抓住不放, 编造出种种神话, 而且深信不疑, 近乎狂热. 这可以说是浪漫主义对平凡暗淡的生活的一种抗议. 
\end{quotation}

\paragraph*{page~008}
\begin{quotation}
    \itshape
    但他的目光敏锐, 一眼就望穿了隐含在一些天真无邪的行为下的可鄙的动机. 他既是一个艺术研究者, 又是一个心理--病理学家. 他对一个人的潜意识了如指掌. 没有哪个探索心灵秘密的人能够像他那样透过普通事物看到更深邃的意义. 探索心灵秘密的人能够看到不好用语言表达出来的东西, 心理病理学家却看到了根本不能表达的事物.
\end{quotation}

\paragraph*{page~011}
\begin{quotation}
    \itshape
    我不记得是谁曾建议过, 为了使灵魂宁静, 一个人每天要做两件他不喜欢的事. 说这句话的是个聪明人, 我也一直在一丝不苟地按照这条格言行事: 因为我每天早上都起床, 每天也都上床睡觉.
\end{quotation}

\paragraph*{page~012}
\begin{quotation}
    \itshape
    战争来了, 战争也带来了新的生活态度. 年轻人求助于我们老一代人过去不了解的一些神祗, 已经看得出继我们之后而来的人要向哪个方向活动了. 年轻的一代意识到自己的力量, 吵吵嚷嚷, 早已经不再叩击门扉了. 他们已经闯进房子里来, 坐到我们的宝座上, 空中早已充满了他们喧闹的喊叫声. 老一代的人有的也模仿年轻人的滑稽动作, 努力叫自己相信他们的日子还没有过去; 这些人同那些最活跃的年轻人比赛喉咙, 但是他们发出的呐喊听起来却那么空洞, 他们有如一些可怜的浪荡女人, 虽年华已过, 却仍然希望靠涂脂抹粉, 靠轻狂浮荡来恢复青春的幻影. 聪明一点儿的则摆出一副端庄文雅的姿态. 他们的莞尔一笑中流露着一种宽容的讥诮. 他们记起了自己当初也曾经把一代高踞宝座的人践踏在脚下, 也正是这样大喊大叫, 傲慢不逊; 他们预见这些高举火把的勇士们有朝一日也同要让位于他人. 谁说的话也不能算最后拍板. 说这些豪言壮语的人可能还觉得他们在说一些前任未曾道过的真理, 但是实际上连他们说话的腔调前任也已经用过一百次, 而且丝毫没有变化. 钟摆摆过来又荡过去, 这一旅程反复循环. 
\end{quotation}

\paragraph*{page~013}
\begin{quotation}
    \itshape
    但是像济慈同华兹华斯写的颂歌, 柯勒律治的一两首诗, 雪莱的更多的几首, 确实发现了前任未曾探索过的广阔精神领域. 布莱布先生已经陈腐过时了, 但是克莱布先生还是孜孜不倦地继续写他的押韵对句诗. 我也断断续续读了一些我们这一时代的年轻人的诗作, 他们当中可能有一位更炽情的济慈或者梗一尘不染的雪莱, 而且已经发表了世界将长久记忆的诗章, 这我说不定. 我赞赏他们的优美词句--尽管他们还年轻, 却已才华横溢, 因此如果仅仅说他们很有希望, 就显得荒唐可笑了---, 我惊叹他们精巧的文体; 但是虽然他们用词丰富, 却没有告诉我们什么新鲜东西. 在我看来, 他们知道的太多, 感觉过于肤浅; 对于他们拍我肩膀的那股亲热劲儿同闯进我怀抱时的那种感情, 我实在受不了. 我觉得他们的热情似乎没有血色, 他们的梦想也有些平淡. 我不喜欢他们. 我已经是过时的老古董了.
\end{quotation}

\paragraph*{page~015}
\begin{quotation}
    \itshape
    我想在过去的日子里我们都羞于使自己的感情外露, 因为怕人嘲笑, 所以都约束着自己不给人以傲慢自大的形象. 我并不认为当时风雅放浪的诗人作家执身如何端肃, 但我却不记得那时候文艺界有今天这么多风流韵事. 我们对自己一些荒诞不经的行为遮上一层保持体面的缄默, 并不认为这是虚伪. 我们讲话讲究含蓄, 并不总是口无遮拦, 说什么都直言不讳.
\end{quotation}

\paragraph*{page~017}
\begin{quotation}
    \itshape
    艺术家较之其他行业的人有一个有利的地方, 他们不仅可以讥笑朋友的性格和仪表, 而且可以嘲弄他们的著作. 他们的评论恰到好处, 话语滔滔不绝, 我实在望尘莫及. 在那个时代谈话仍然被看作是一种需要下功夫陶冶的艺术, 一句巧妙的对答比锅子底下噼啪爆响的荆棘更受人赏识, 格言警句当时还不是痴笨的人利用来冒充聪敏的工具, 风雅人物的闲谈中随便使用几句会使得谈话妙趣横生. 
\end{quotation}

\paragraph*{page~025}
\begin{quotation}
    \itshape
    思特里克兰德太太是很会同情人的. 同情体贴本是一种很难得的本领, 但是却常常被那些知道自己有这种本领的人滥用了. 他们一看到自己的朋友有什么不幸就恶狠狠地扑到人们身上, 把自己地全部才能施展出来, 这就未免太可怕了. 同情心应该像一口油井一样喷薄而出; 惯爱表同情的人让它纵情奔放, 反而使那些受难者非常困窘. 
\end{quotation}

\paragraph*{page~027}
\begin{quotation}
    \itshape
    她笑了, 她的笑容很甜, 脸上微微泛起一层红晕; 像她这样年纪的女人竟这么容易脸红, 是很少有的. 也许她最迷人的地方就在于她的纯真.

    她用这个词一点儿也没有贬抑的意思, 相反地, 倒是怀着一股深情, 好像由她自己说出他最大的缺点就可以保护他不受她朋友们的挖苦似的.
\end{quotation}

\paragraph*{page~030}
\begin{quotation}
    \itshape
    文明社会这样消磨自己的心智, 把短促的生命浪费在无聊的应酬上实在令人莫解.
\end{quotation}

\paragraph*{page~032}
\begin{quotation}
    \itshape
    很清楚, 他一点儿也没有社交的本领, 但这也不一定人人都要有的. 他甚至没有什么奇行怪癖, 使他免于平凡庸俗之嫌. 他只不过是个一个忠厚老实, 索然无味的普通人. 一个人可以钦佩他的为人, 却不愿意同他待在一起, 他是一个毫不引人注意的人. 他可能是一个令人起敬的社会成员, 一个诚实的经纪人, 一个恪尽职责的丈夫和父亲, 但是在他身上你没有任何必要浪费时间.
\end{quotation}

\paragraph*{page~034}
\begin{quotation}
    \itshape
    如果纯粹从善于辞令这一角度衡量一个人智慧, 也许查理斯$\cdot$思特里克兰德不算聪明, 但是在他自己的哪个环境里, 他的智慧还是绰绰有余的, 这不仅是事业成功的敲门砖, 而且是生活幸福的保障.

    这一定是世间无数夫妻的故事. 这种生活模式给人以安详亲切之感. 它使人想到一条平静的小河, 蜿蜒流过绿茸茸的牧场, 与郁郁的树荫交相掩映, 直到最后泻入烟波浩瀚的大海中. 
\end{quotation}

\paragraph*{page~048}
\begin{quotation}
    \itshape
    尽管她悲痛的感情是真实的, 却没忘记使自己的衣着合乎她脑子里的礼规叫她扮演的角色. 
\end{quotation}

\paragraph*{page~}
\begin{quotation}
    \itshape
    
\end{quotation}

\paragraph*{page~}
\begin{quotation}
    \itshape
    
\end{quotation}

\paragraph*{page~}
\begin{quotation}
    \itshape
    
\end{quotation}

\paragraph*{page~}
\begin{quotation}
    \itshape
    
\end{quotation}

\paragraph*{page~}
\begin{quotation}
    \itshape
    
\end{quotation}

\paragraph*{page~}
\begin{quotation}
    \itshape
    
\end{quotation}

\paragraph*{page~}
\begin{quotation}
    \itshape
    
\end{quotation}

\paragraph*{page~}
\begin{quotation}
    \itshape
    
\end{quotation}

\paragraph*{page~}
\begin{quotation}
    \itshape
    
\end{quotation}

\paragraph*{page~}
\begin{quotation}
    \itshape
    
\end{quotation}




