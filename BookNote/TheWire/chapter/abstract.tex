\renewcommand\abstractname{\Large\textbf{序}}
\begin{abstract}
    本科多看日剧, 略微涉猎美剧, 看过无耻之徒(Shameless), 绝命毒师(Breaking Bad), 风骚律师(Better Call Saul)等, 这些剧大多不会引起我二刷三刷的冲动, 而火线不同, 它没有无耻之徒毫无逻辑刻意追求笑点和剧情的低俗跳脱, 相比绝命风骚两部精致的画面, 天马行空的剧情, 火线过强的写实风格让我沉醉其中, 欲罢不能. 去年七八月在嘉定看完后, 南桥四平的通勤中第二遍, 最近三刷完成. 从第一季抓捕街头毒贩, 第二季港口的衰败, 第三季政治主旋律与Hamsterdan, 第四季教育, 第五季衰落的报社, 整整五季从警局, 港口, 政治, 教育, 新闻大方面全方位地剖析了一番巴尔的摩, 同时也刻画出众多栩栩如生地角色: 流落街头最终自力更生摆脱毒瘾地Bubbles; 立志改革, 重整巴尔的摩城市面貌终被体制同化的Carcatti; 成为局长却又因不想与数字游戏同流合污而下台的警务专家Cedric; 身处体制却又游离于规则的灰色地带, 不畏惧任何权威的Jimmy McNauty; 无法忍受所有警员都在为指标做一些无谓的警务工作开启毒品合法区的Bunny; 还有我认为剧中最伟大的人物Frank Sobotka, 封面便是他, 在码头存亡危急关头, 用帮国际毒贩走私的资金游说政客复兴码头, 最后牺牲自己, 落得割喉下场. 还有其他令人印象深刻的人物, 喜剧双人组Carver和Herc, 富国论毒贩Stringer, 天生政客Clay, 神探Lester, 坚守底线的Bunk, 黑道大侠Omar等等. 
    
    这部剧教会了我太多, 每每想起剧中人物, 回味无穷. 我从不谙世事的小孩子忽然蜕变成官僚达人, 美妙的Chain of Command. 感谢我兄弟的推荐, 不仅火线这部电视剧还有叫魂这本书. 我找到了中意的影视题材.
    
    二零二一年六月十六日, 也是时候整理火线中喜爱的台词了.
\end{abstract}

\addcontentsline{toc}{section}{序}
\pagenumbering{roman}