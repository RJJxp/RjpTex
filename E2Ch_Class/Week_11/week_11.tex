\documentclass[a4paper, UTF8, 12pt]{article}

\usepackage{ctex}
\usepackage{geometry}
\usepackage{graphicx}
\usepackage{float}
\usepackage{caption}
\usepackage{enumerate}
\usepackage{paracol}
\usepackage{longtable}
\usepackage{array}
\usepackage{multirow}

\usepackage{hyperref}

\hypersetup{
    colorlinks=true,   % false, ,链接黑色, 外有红框
    linkcolor=black, % 目录颜色, 脚注颜色
    filecolor=blue, % 链接本地文件的链接颜色
    urlcolor=cyan, % 网页链接颜色
    anchorcolor=cyan,   % 锚点颜色
    citecolor=yellow    % 参考文献颜色
}

\geometry{
    textheight=230mm,
    textwidth=180mm
}

\begin{document}
\title{\Huge 英译汉课程}
\author{\Large 
        第十一周 \\[12pt]
        1931991 任家平 \\[12pt]
        同济大学 \\[12pt]
        测绘与地理信息学院}
\date{2019-11-11}

\maketitle

\thispagestyle{empty}

\newpage
\pagenumbering{Roman}
\tableofcontents
\addtocontents{toc}{\protect\vspace{10pt}}

\newpage
\pagenumbering{arabic}

\begin{paracol}{2}[\section{翻译}]

    \switchcolumn*
    \paragraph{} \quad {\bfseries 发展中国家云的使用对云经济的提供商和用户有潜在影响.}
    \switchcolumn
    \paragraph{}
    {\bfseries Cloud adoption in developing countries has potential implications for both the supply and the user side of the cloud economy.} 
    
    \switchcolumn*
    云服务为发展中国家能带来的最有益的活动和潜在的供应机会有如下: (a) 数据中心与相关云服务的提供. (b) 为本地用户群(本地企业和居民)提供的云服务的发展与供应. (c) 云聚合, 系统集成和经济业务相关的服务. 除了上述明确与基于云的活动有关, 云服务对于本国的通讯业务(电信运营商和网络服务提供商)是一个极好的发展机会, 他们可以从增长的数据流量获益. 尽管国际云服务提供商已经占尽先机, 但在发展中国家, 由于对私有云的需求增加, 国家数据保护法规或是公司要求数据在国内监管, 国外网络无法稳定访问等因素, 其本地或区域数据中心仍有发展空间.
    \switchcolumn
    The most significant activities and potential supply opportunities for enterprises in developing countries are concerned with: (a) data-centre and related cloud provision; (b) the development and provision of local cloud services for groups of customers, including local businesses and individual citizens; (c) cloud aggregation, system integration, brokerage and related services. In addition to these explicitly cloud-based areas of activity, opportunities exist for national communications businesses (telecommunications operators and ISPs) which can gain from increased data traffic using their networks. Despite the advantages of global cloud service providers, there are some factors that offer scope for local or regional data centres to expand in developing countries, such as growing demand for private cloud solutions, national data-protection laws or corporate policies requiring data to be kept within national jurisdictions, and high costs of or unreliable international broadband connectivity. 

    \switchcolumn*
    在发展中国家, 有很多人使用免费的云服务(网络邮件, 在线社交网络). 几乎在所有国家, 尤其是网络使用水平高和云服务可用度高的国家, 皆是如此. 最受欢迎的云服务应用基本都是全球性应用. 在云服务刚开始使用的低收入国家中, 基础设施即服务经常是第一个出现的云服务. 随着基础设施状况的改善以及中小型企业部门的扩大, 发展中国家的软件即服务市场将变得越来越重要, 并且最终将占据发达国家的主导地位. 在发达国家, 软件即服务已经占据主导位置.
    \switchcolumn
    There has been extensive adoption by individuals in developing countries of free cloud services such as webmail and online social networks. This is true in almost all countries, in particular those with higher levels of Internet use and cloud readiness. The most popular cloud-based applications are generally those provided at a global level. In low-income countries at a nascent stage of cloud readiness, IaaS is often the first category of cloud services to emerge. As the infrastructure situation improves and if the SME sector expands, the market for SaaS in developing countries will become more important and eventually dominant as it already is in developed countries.

    \switchcolumn*
    其母公司的全球网络的一部分, 外国子公司在发展中国家提供的云服务被广泛使用. 同时发展中国家政府也谨慎地将自己的服务转向云端. 其中有些政府推出了系统性的云战略作为其国家整体信息通讯产业发展战略的一部分或与其并驾齐驱. 相比于公共云, 政府部门和大企业一直更偏向使用私有云. 尽管有很多人支持云服务, 但由于政策原因, 政府会对国内企业使用的云服务进行一定的比例限制.
    \switchcolumn
    Foreign affiliates in developing countries make extensive use of the cloud as part of their parent companies’ global networks. With some wariness, Governments in developing countries are also moving towards the cloud. Some are developing systematic cloud strategies, as part of broader ICT strategies or sometimes alongside these. Where government departments and larger corporations are concerned, there is so far a general preference for private over public cloud approaches. There is planned adoption of the cloud in domestic enterprises, although less extensive than anticipated by cloud advocates. 
    
    \switchcolumn*
    云计算在发展中国家发展过于迅速, 由于缺少资料, 通过严格循证评估其影响力较为困难. 企业政府和其他组织在通过云提升自己的管理和提供服务的同时, 也要注意云服务潜在的风险. 只有当云服务优势巨大, 并且所带来风险可被优势适当缓冲时, 他们才会将其数据和服务迁移到云端, 使用云服务. 无论是公共云方案或是私有云方案, 都应考虑隐私和数据安全性.
    \switchcolumn
    Experience of cloud computing in developing countries is too recent for there to be a strongly established evidence base on which to assess impacts. Businesses, Governments and other organizations should carefully examine the potential for cloud services to improve their management and service delivery. They should only migrate data and services to the cloud when they are confident that the cloud offers significant benefits and that attendant risks can be appropriately mitigated. Both public and private cloud solutions should be considered in this context, taking into account implications for data security and privacy. 


    \switchcolumn*
    \paragraph{} \quad {\bfseries 基础设施薄弱严重阻碍发展中国家从云计算中获益.}
    \switchcolumn
    \paragraph{}
    {\bfseries Infrastructure deficiencies seriously ha-mper the uptake of and benefits from cloud computing in many developing countries.}

    \switchcolumn*
    低收入和中等收入国家和较发达国家可用的云服务差距较大, 其原因错综复杂. 其中主要原因有以下三点. 一是云计算基础设施缺失或是质量低劣, 二是云计算花费较高, 三是没有完善的法规框架去保护用户数据和隐私.
    \switchcolumn
    For several reasons, the options for cloud adoption available in low- and middle-income countries look different from those in more advanced economies. Critical factors relate, among other things, to the availability and quality of cloud-related infrastructure, cost considerations and inadequate legal and regulatory frameworks to address data protection and privacy concerns. 
    
    \switchcolumn*
    关于云服务的获取和可用这方面, 尽管发展中国家已经在宽带连接稳定性已有极大改善, 但发展中和发达国家的差距仍进一步扩大. 平均固定宽带的使用率在发达国家高于28\%, 但在发展中国家为6\%, 最不发达国家为0.2\%. 移动宽带的差距也十分巨大, 在发达国家, 发展中国家使用率分别为67\%, 14\%, 最不发达国家连0.2\%都不到.
    \switchcolumn
    As regards access to and availability of cloud-related infrastructure, and despite significant improvements in broadband connectivity in many developing economies, the gap between developed and developing countries keeps widening. Average fixed broadband penetration is now more than 28 subscriptions per 100 people in developed economies, 6 in developing countries and only 0.2 in the least developed countries (LDCs). In the case of mobile broadband, the gap is also significant. The average number of subscriptions in 2012 was about 67 per 100 people in developed countries, 14 in developing countries and below 2 in the LDCs. 

    \switchcolumn*
    除此以外, 在收入最低的国家里, 移动宽带网络服务一般低速且高延迟, 现阶段无法理想提供云服务, 尤其是高端云服务. 在通讯网和电网经常故障的国家里, 云服务的净值较低. 由于缺少网络交换中心, 廉价可靠的电力和稳定的光纤骨干网络等基础设施, 国家数据中心的部署受到极大影响. 事实上, 发达国家85\%的数据中心都提供服务器托管的服务. 该服务反映了 `数据中心鸿沟' 这一现象; 在2011年, 发达国家每一百万居民就有1000个安全数据服务器, 而最不发达国家每一百万居民只有一个.
    \switchcolumn
    \hypertarget{par:01}{}
    In addition, in most low-income countries, mobile broadband networks are characterized by low speed and high latency and are therefore currently not ideal for cloud service provision, especially of the more advanced kinds. The net value of cloud-based solutions will be lower in countries with a heightened risk of communication- and power-network outages. The lack of supporting infrastructure, such as Internet exchange points (IXPs), reliable and inexpensive electricity and robust fibre-optic backbones also affect the deployment of national data centres. Indeed, as much as 85 percent of data centres offering colocation services are in developed economies. This “data centre divide” is reflected in the availability of servers; whereas there were in 2011 more than 1,000 secure data servers per million inhabitants in highincome economies, there was only one such server per million inhabitants in LDCs.

    \switchcolumn*
    通讯的花费是发展中国家建设云服务的一大阻碍. 在提供云服务的总花费中, 发展中国家用户向云服务提供商, 网络服务提供商和涉及的软硬件服务所付的开销, 其比例要高于发达国家. 极少的国家数据中心和高价的国际宽带通讯对依赖云方案的净值产生影响.
    \switchcolumn
    The cost of communication remains another critical obstacle for adoption of cloud services in many developing countries. The fees paid to cloud service providers and for broadband access and usage, charges by the ISP and the hardware and software costs incurred are likely to form a much higher proportion of the total costs of cloud provisioning than in advanced economies. The combination of few national data centres and high costs of international broadband communications further weighs on the net value of relying on cloud solutions. 

    \switchcolumn*
    \paragraph{} \quad {\bfseries 云带来了法规方面的挑战, 尤其是数据保护和隐私方面.}
    \switchcolumn
    \paragraph{}
    {\bfseries The cloud raises legal and regulatory challenges, especially concerning data protection and privacy.}

    \switchcolumn*
    云计算的迅速崛起引发了关于其法律法规影响方面的担忧. 数据保护和安全是发展中国家和发达国家潜在云云用户最关心的话题. 2013年国家监控计划曝光, 媒体报道了国家执法机构访问云服务提供商托管的数据后, 这种担忧达到了极点. 政府这么做, 其声称是出于国家利益, 保护公民; 然而云服务提供商需要稳定的机制去谋求创新吸引投资, 政府此种做法会让公司名声扫地; 且用户需要服务商保证其数据隐私和数据安全, 才会信任服务商, 继续使用云服务. 对此, 相关政策反响巨大, 有人持保守态度, 什么都不做; 有人持激进态度, 要求出台云相关的法律.
    \switchcolumn
    \hypertarget{par:02}{}
    The rapid emergence of cloud computing has raised concerns about its legal and regulatory implications. Issues of data protection and security are among the concerns most frequently mentioned by potential cloud customers in both developed and developing countries. Such concerns have intensified following the disclosure in 2013 of national surveillance programmes and reports on access by law-enforcement agencies to data hosted by global cloud service providers. Governments need to protect national interests and their citizens; service providers require a stable framework to facilitate innovation and investment; and users require assurance and trust to encourage the take-up of such services. Policy responses may range from a do-nothing attitude to the adoption of cloud-specific laws. 

    \switchcolumn*
    公共法确实对于确保端用户的基本权利十分重要. 但现在为云计算制定相关的法律法规并不是迫在眉睫, 因为需要改革的领域相对比较清楚: 隐私, 数据保护, 信息安全和网络犯罪. 对于发展中国家的政府来说, 制定适当的相关领域法律法规显得较为重要. 在2013年, 有99个国家制定了数据隐私保护方面的法律. 迄今为止, 墨西哥是唯一一个在数据保护方面采用了特殊云服务条款的国家. 目前并没有全球性统一隐私框架来规范数据跨境传输, 发展中国家的国内云服务便可从国内强力的隐私政策中获益.
    \switchcolumn
    \hypertarget{par:03}{}
    Public law is essential to secure the basic rights of end users. While there is no imperative to develop specific laws or regulations on cloud computing, areas requiring reform are relatively clear: privacy, data protection, information security and cybercrime. For Governments of developing countries, it is essential that appropriate laws and regulations are adopted and enforced in these areas. As of 2013, there were 99 countries with data-privacy laws. As far as is known, Mexico is the only country which has adopted cloudspecific provisions in relation to data protection. There is no international harmonized privacy framework regulating data transfers across borders, but developing countries could benefit from implementing strong domestic-privacy regim-es.

    \switchcolumn*
    除了公共法, 云服务的用户条款也会极大影响云经济的运作和效益. 用户条款规定, 在某些特殊情况下, 以契约自由原则的监管干预对于保护公众利益也是必要的. 对云中储存的数据进行一定的监管干预才能消除关于个人隐私, 商业机密或是国家安全的担忧. 比如, 根据数据保护法规定, 云服务提供商至少要保证用户数据安全并及时告知安全漏洞, 这可以提高漏洞的透明度并及时处理风险. 
    \switchcolumn
    In addition to public law, contractual agreements between cloud service providers and cloud service customers also greatly impact on the operation and effects of the cloud economy. In some circumstances, regulatory intervention in the freedom to contract may be necessary to protect the public interest. The placement of data in the cloud may require regulatory intervention to address concerns related to personal privacy, commercial secrecy or national security. For example, within data protection laws, imposing minimum responsibilities on the cloud service providers – to ensure the security of customer data and to notify its customers if there is a security breach – could help to provide greater transparency about vulnerabilities and to enable mitigation in a timely manner. 

    \switchcolumn*
    民众可能对过度依赖于外国辖区服务商提供的云服务表示担心, 确实, 本国法律法规可能无法监管外国公司, 这样通过监管介入来落实市场一些关于云安全的一些政策比较困难. 另一种政策是通过向国外投资者建立一个良好的环境去建设本地的基础设施(如数据中心), 或是让本国企业参与云经济的供应方, 来达到建设本国云服务的目的. 尽管这样的政策会涉及到监管的成分, 例如强行 `本地化' 的要求, 但这些要求理应是去促进云服务的提供, 并非限制云服务, 这样就适得其反. 发展中国家的政府正在组建一个政府云来满足本国政府或是其他政府的需求. 在欧洲, 已经有建立安全的欧洲云的呼声, 并启动了国家云计划作为另一个云服务提供源.
    \switchcolumn
    Where there are apprehensions on relying hea-vily on cloud services offered by providers based in a foreign jurisdiction, it may be difficult to address this market reality through regulatory intervention. An alternative policy response may be to encourage the establishment of domestic cloud services, either by offering foreign investors a favorable environment to invest in the building of local infrastructure (such as data centres) or encouraging domestic enterprises to enter the supply side of the cloud economy. Whereas such measures may involve regulatory components, such as imposing “localization” requirements, they would be designed to facilitate the provision of cloud services rather than to constrain them. Several Governments of developing countries are building government clouds to serve the needs of the Government itself and sometimes others. In Europe, there have been calls for the development of a secure European cloud and some national cloud initiatives have been launched to offer an alternative source of cloud service provision. 

    \switchcolumn*
    \paragraph{} \quad {\bfseries 政府在从云经济获益的同时也要警惕云经济的陷阱.}
    \switchcolumn
    \paragraph{}
    {\bfseries Governments should facilitate benefits from the cloud economy but be aware of pitfalls.}

    \switchcolumn*
    尽管云经济在发展中国家还是起步阶段, 决策者仍要不遗余力地加深云计算如何影响经济和社会的理解, 以便于做出明智的决定. 在详细评估云技术优缺点, 并透彻了解各国对现有信息通讯技术和云技术后, 再进行决策. 政府必须要加强对云商业模式和云服务的多元化, 云客户的多样性和云经济生态的复杂性的理解. 考虑到公共服务和企业竞争的相关性, 政府应当将所有云策略纳入国家整体发展计划并提早为其制定执行, 监管和估值的计划, 这十分重要. 每个国家因其具体国情, 政策大不相同, 但其政策定会和经济发展, 信息通讯技术的国家整体战略框架保持一致.
    \switchcolumn
    Although cloud adoption is still at a nascent stage in developing countries, policymakers should waste no time in enhancing their understanding of how it may affect their economies and societies, in order to be able to make informed policy decisions. Government policies should be based on an assessment of the pros and cons of cloud solutions and be rooted in a thorough understanding of existing ICT and cloud use within countries. Governments need to recognize the diversity of business models and services within the cloud, the multiplicity of customers of cloud services, and the complexity of the cloud economy ecosystem. In view of its relevance for both public service delivery and business competitiveness, it is important to integrate any cloud strategy in the overall national development plan, and to plan for its execution, monitoring and evaluation. Policy approaches should be tailored to the circumstances of individual economies, and be consistent with the overall strategic framework for national economic development and for leveraging ICTs.

    \switchcolumn*
    总体来说, 政府应极力欢迎和支持云经济并采用云服务. 原则上, 没有政府政策和法规阻止向云端迁移数据和服务. 更准确的说, 政策和法规要去创造一个让公司和组织可以轻易安全将自己数据和服务迁移到云端的框架系统. 但这并不意味着, 云解决方案一定优于其他方案. 除此之外, 有多种使用云技术的方式, 如在国家范围, 地区范围或是全球范围的公共云, 私有云, 混合云. 各国政府应设法促进在特定情况下让云发挥更大经济利益的方式.
    \switchcolumn
    On the whole, Governments should broadly welcome and support the development of a cloud economy and the adoption of cloud services. In principle, there is no general case for government policy and regulation to discourage migration towards the cloud. Rather, policies and regulatory approaches should seek to create an enabling framework that supports firms and organizations that wish to migrate data and services to the cloud so that they may do so easily and safely. However, this does not mean that cloudbased solutions are always preferable to alternative approaches. In addition, there are multiple ways of making use of cloud technology – using public, private or hybrid clouds at national, regional or global levels. Governments should seek to facilitate those approaches that seem most likely to deliver wider economic benefits in their particular context. 

    \switchcolumn*
    政府应考虑可将云技术潜力化为切实经济效益的方法. 就范围层面, 面对国家级决策, 可以考虑涉及以下方面的措施:
    \switchcolumn
    A number of steps could be considered by Governments that wish to translate the potential of the cloud into tangible development gains. In terms of scope, at the national level policymaking would be advised to consider measures related to the following areas:

    \begin{itemize}
        \switchcolumn*
        \item {\bfseries 评估该国家的云成熟度.} 政府应该从仔细评估本国现状开始, 确认现阶段对于有效利用云技术的瓶颈和缺陷何在, 并决定哪一种云方案收益最高.
        \switchcolumn
        \item {\bfseries Assess the cloud readiness of the country.} Governments should start by carefully assessing the current situation in their countries, to identify bottlenecks and weaknesses that need to be addressed if the cloud is to be effectively exploited, and clarify what kind of cloud solutions are most propitious.  
        
        \switchcolumn*
        \item {\bfseries 确定国家云战略.} 在确认云成熟度基础上, 国家云战略要么是一份单独的文件, 要么是国家信息通讯技术战略的一部分.
        \switchcolumn
        \item {\bfseries Develop a national	cloud strategy.} Bas-ed on the readiness assessment, a national cloud strategy could be drafted either as a stand-alone policy document or as an integral part of the national ICT strategy. 
        
        \switchcolumn*
        \item {\bfseries 解决基础设施的问题.} 这涉及到提供可靠并且价格合理的宽带基础设施服务. 有效的通信法规直观重要, 还要注意网络交换点和电力供应.
        \switchcolumn
        \item {\bfseries Address the infrastructure challenge.} This would involve measures to improve the provision of reliable and affordable broadband infrastructure and to monitor regularly the quality of broadband services. Effective communications regulations are here of the essence. Attention should also be given to the role of IXPs and the provision of electricity.
        
        \switchcolumn*
        \item {\bfseries 解决相关法律法规问题以确保云用户的利益得到合理保护} 关键点包括数据定位, 电子交易和网络犯罪. 应加大制定新法规, 来反映国际最佳实践.
        \switchcolumn
        \hypertarget{par:04}{}
        \item {\bfseries Address relevant legal	and	regulatory issues related to cloud	adoption to	ensure that cloud service	users’ interests are properly protected.} Key areas include the location of data, e-transactions and cybercrime. Efforts should be made to reflect international best practice in the development of new legislation. 
        
        \switchcolumn*
        \item {\bfseries 在云经济的供给方寻求机会.} 有三点值得特别注意: 国家信息中心的建立, 云聚合服务的潜力, 新型云服务的开发.
        \switchcolumn
        \item {\bfseries Map opportunities in the supply side of the	cloud economy.} Three key areas deserve particular attention: the development of national data centres, the potential for cloud aggregation services, and the development of new cloud services. 

        \switchcolumn*
        \item {\bfseries 解决人力资源的需求.} 具有管理云服务的迁移和整合方面的信息技术和软件技术的人才愈发重要. 具有管理和组织技能的人才用于对业务流程的重组织和二次工程化十分重要. 同时具有法律技能和采购技能的人才也很重要.
        \switchcolumn
        \hypertarget{par:05}{}
        \item {\bfseries Address the need for human resources.} Skill areas that are likely to become increasingly important include those related to the IT and software skills needed to manage the migration and integration of cloud services; management and organizational skills to handle the reorganization and reengineer-ing of business processes; and legal and procurement skills. 
        
        \switchcolumn*
        \item {\bfseries 政府对云服务的使用.} 由于信息经济对发展中国家十分重要, 应探讨政府在建立国家数据中心, 电子政务系统和相关公共采购中扮演什么样的角色.
        \switchcolumn
        \item {\bfseries Government use of cloud services.} Giv-en their important role in the information economy in many developing countries, the role of Governments should be explored with regard to the establishment of national data centres, e-government systems and related public procurement. 
    
    \end{itemize}

    \switchcolumn*
    \paragraph{} \quad {\bfseries 云经济发展战略伙伴应与政府合作}
    \switchcolumn
    {\bfseries Development partners should work with Governments in responding to the cloud economy.}

    \switchcolumn*
    发展中国家想从蓬勃发展的云经济中获益, 所面临的挑战有多领域的专业性和巨大财力支撑. 这时便需要发展战略伙伴的帮助, 主要承包云发展中的困难与挑战, 来减少向云经济转向导致数据鸿沟扩大的风险. 他们也可以在国家范围内宽带基础设施建设上提供资金, 在建立合适的法律法规框架也可以做出贡献, 或是相关领域的能力建设.
    \switchcolumn
    Addressing the many challenges that developing countries face in seeking to benefit from the evolving cloud economy will require both expertise in various fields and financial resources. Development partners could help in that respect, by ensuring that cloudrelated development challenges are incorporated in their agendas to reduce the risk that the move towards the cloud economy may result in a widening of the digital divide. They may also provide support at the country level in contributing to financing broadband infrastructure, establishing appropriate legal and regulatory frameworks, and building capacity in relevant areas.

    \switchcolumn*
    国际机构可以通过它们已有的一些项目提供协助. 比如, 贸发会议和其他国际组织可以向国发展国家提供发展的经验. 这样发展中国家在制定政策时, 便可以绕过云经济的陷阱并从中获利.
    \switchcolumn
    International agencies could facilitate this assistance through some of their existing activities. UNCTAD and other international organizations can, for example, facilitate an exchange of experiences with regard to the policy challenges that developing countries face to derive benefits from the cloud economy and avoid pitfalls. 

    \switchcolumn*
    发展战略伙伴可在云服务的国际标准上帮上大忙. 这对于促进云服务的相互操作性并帮助客户理解他们为什么付账有重大意义. 标准化论坛应考虑如何确认发展中国家与其用户的需求是否被满足. 在某些地区, 还是需要对各种云的采用方式作出详细的评估. 数据库规模质变后, 才能去用数据说话, 准确评估经济增长, 就业率, 生产和交易的宏观影响.
    \switchcolumn
    Another key area in which development partners can play a role concerns international standards for cloud services, which are essential to facilitate interoperability and to help customers understand what they are purchasing. Standardization forums should consider how to engage developing countries and their users to ensure that their specific needs and requirements are addressed. More research is also needed in a number of areas to allow for a more comprehensive assessment of the impact of different forms of cloud adoption. As the evidence base expands, it will become feasible to assess macroeconomic implications for economic growth, employment, productivity and trade. 

    \switchcolumn*
    对比其他信息通讯技术领域, 云技术和云市场瞬息万变. 本篇报告中的云发展经历介绍是基于现状. 但是云经济或是云服务其云的本质决定, 云的发展十分迅速, 五年后的云和现在的云可能会完全不同. 政府, 企业和其战略发展伙伴一定要记住云是在随着时间变化, 所以要经常性重新评估现阶段策略政策是否与时俱进, 以保证市民, 企业和云顾客可以获取最大的经济效益同时将风险最小化.
    \switchcolumn
    As with other ICT areas, the pace of change in cloud technology and markets is rapid. The experiences described in this report relate to present circumstances. The nature of cloud services and of the cloud economy will continue to develop fast, and may be very different in five years’ time. Governments, businesses and development partners need to bear these changes in mind, and to re-evaluate their policies and strategies concerning the cloud regularly to ensure that they continue to maximize potential benefits and minimize potential risks to their citizens, businesses and customers.

\end{paracol}

\section{反思}

课程讨论内容不再赘述. 对翻译中比较有争议的地方进行记录.

\begin{enumerate}
    \item The net value of cloud-based solutions will be lower in countries with a heightened risk of communication- and power-network outages. 翻译改变原文侧重点. \hyperlink{par:01}{\underline{\emph 原文及译文}}
    \item The rapid emergence of cloud computing has raised concerns about its legal and regulatory implications. 不清楚 `implication' 如何翻译. \hyperlink{par:02}{\underline{\emph 原文及译文}}
    \item Such concerns have intensified following the disclosure in 2013 of national surveillance programmes and reports on access by law-enforcement agencies to data hosted by global cloud service providers. Governments need to protect national interests and their citizens; 按照自己的思路翻译. \hyperlink{par:02}{\underline{\emph 原文及译文}}
    \item Policy responses may range from a do-nothing attitude to the adoption of cloud-specific laws. 按照自己的思路翻译. \hyperlink{par:02}{\underline{\emph 原文及译文}}
    \item This “data centre divide” is reflected in the availability of servers; 按照自己的思路翻译. \hyperlink{par:01}{\underline{\emph 原文及译文}}
    \item Governments need to protect national interests and their citizens; service providers require a stable framework to facilitate innovation and investment; and users require assurance and trust to encourage the take-up of such services. Policy responses may range from a do-nothing attitude to the adoption of cloud-specific laws. 加了自己的理解. \hyperlink{par:02}{\underline{\emph 原文及译文}}
    \item There is no international harmonized privacy framework regulating data transfers across borders, but developing countries could benefit from implementing strong domestic-privacy regim-es. 加了自己的理解. \hyperlink{par:03}{\underline{\emph 原文及译文}}
    \item Efforts should be made to reflect international best practice in the development of new legislation. 不会翻译. \hyperlink{par:04}{\underline{\emph 原文及译文}}
    \item Address the need for human resources. 加了自己的理解. \hyperlink{par:05}{\underline{\emph 原文及译文}}

\end{enumerate}


\end{document}