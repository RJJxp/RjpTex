\documentclass[a4paper, UTF8, 12pt]{article}

\usepackage{ctex}
\usepackage{geometry}
\usepackage{graphicx}
\usepackage{float}
\usepackage{caption}
\usepackage{enumerate}
\usepackage{paracol}
\usepackage{longtable}
\usepackage{array}
\usepackage{multirow}

\usepackage{hyperref}

\hypersetup{
    colorlinks=true,   % false, ,链接黑色, 外有红框
    linkcolor=black, % 目录颜色, 脚注颜色
    filecolor=blue, % 链接本地文件的链接颜色
    urlcolor=cyan, % 网页链接颜色
    anchorcolor=cyan,   % 锚点颜色
    citecolor=yellow    % 参考文献颜色
}

\geometry{
    textheight=230mm,
    textwidth=180mm
}

\begin{document}
\title{\Huge 英译汉课程}
\author{\Large 
        第十三周 \\[12pt]
        1931991 任家平 \\[12pt]
        同济大学 \\[12pt]
        测绘与地理信息学院}
\date{2019-11-25}

\maketitle

\thispagestyle{empty}

\newpage
\pagenumbering{Roman}
\tableofcontents
\addtocontents{toc}{\protect\vspace{10pt}}

\newpage
\pagenumbering{arabic}

\begin{paracol}{2}[\section{翻译}]

    \switchcolumn*
    \paragraph{} \quad {\bfseries 发展中国家云的使用对云经济的提供商和用户有潜在影响.}
    \switchcolumn
    \paragraph{}
    {\bfseries Cloud adoption in developing countries has potential implications for both the supply and the user side of the cloud economy.} 
    
    \switchcolumn*
    在晚期资本主义社会中, 不再像过去一样, 需要非条文式公共道德机制去规范个人行为与社会关系; 而要在基本符合人性契约原则下, 并用法律机制去编纂一套可以规范社会关系的规则与措施.
    \switchcolumn
    The need to ground individual actions and social relations in the informal mechanisms of a community ethos is, in late capitalist societies, superseded by the need to codify a set of rules and prodcedures as legal mechanisms which formalize social relations as basically contractrual in nature.

    

\end{paracol}

\section{资料}


\section{反思}

课程讨论内容不再赘述. 对翻译中比较有争议的地方进行记录.

\begin{enumerate}
    \item The net value of cloud-based solutions will be lower in countries with a heightened risk of communication- and power-network outages. 翻译改变原文侧重点. 
    

\end{enumerate}


\end{document}