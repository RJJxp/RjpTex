\documentclass[a4paper, 12pt, UTF8]{article}

\usepackage{ctex}
\usepackage{geometry}
\usepackage{graphicx}
\usepackage{float}
\usepackage{caption}
\usepackage{enumerate}
\usepackage[colorlinks, 
            linkcolor=blue,
            anchorcolor=blue,
            citecolor=blue]{hyperref}

\geometry{
    textheight=230mm
}

\begin{document}

\title{\Huge 英译汉课程}
\author{\Large 
        第三周 \\[12pt]
        1931991 任家平 \\[12pt]
        同济大学 \\[12pt]
        测绘与地理信息学院}
\date{2019-09-16}
\maketitle

\thispagestyle{empty}

\newpage
\pagenumbering{Roman}
\tableofcontents
\addtocontents{toc}{\protect\vspace{10pt}}

\newpage
\pagenumbering{arabic}
\section{正文}
\begin{bfseries}
    \Large
    Grand openings
    \paragraph*{}
    \large
    Changes that will bring scientific discovery more freely into the public domain are happening. About time too.
\end{bfseries}

\paragraph*{}
    IN 2001 a meeting on scientific publishing held in Budapest by what was then called the Open Society Institute (now the Open Society Foundation) coined the phrase “open access”. The gathering official statement asked the world to “share the learning of the rich with the poor and the poor with the rich, make this literature as useful as it can be, and lay the foundation for uniting humanity in a common intellectual conversation” --- in other words, to make scientific papers free to users.

\paragraph*{}
    A noble aspiration, but one which cynics might have thought had little chance of coming to fruition. The rich, they would observe, include academic publishers, who have enjoyed three centuries of dominion over the dissemination of scientific work and who often have profit margins approaching 40\%. They had every incentive to scupper change.

\paragraph*{}
    Cynicism, however, is not always correct. The open-access movement which the meeting helped spawn now looks unstoppable. All seven of Britain’s research councils, for example, now require that the results of the work they pay for are open-access in some way. So does the Well-come Trust, a British charity whose medical-research budget exceeds that of many scientifically successful countries. And by 2016 every penny of public money given to British universities by the government will carry the same requirement.

\paragraph*{}
    Elsewhere, the story is the same. In 2013, after years of wrangling in America’s Congress, the White House stepped in to require federal agencies that spend more than \$100m a year on research to publish the results where they can be read for free. Countless universities, societies and funding bodies in other countries have similar requirements.

\paragraph*{}
    Publishers, though they have often dragged their feet, are adjusting. This week the oldest, the Royal Society, and arguably the most prestigious, Nature Publishing Group (NPG)—both based in London—joined in. Each will now publish a journal that readers do not have to pay to look at.

\paragraph*{}
    \begin{bfseries}
        \large
        Who will publish, and who perish?
    \end{bfseries}

\paragraph*{}
    Open-access publishing actually began in 2000, a year before the Budapest meeting, with the launch, in Britain, of BioMed Central, and in America, of the Public Library of Science (PLOS). The open-access business model shifts the cost of producing the journal from subscribers, such as university libraries, to researchers themselves, who pay an article-processing charge (APC) to appear in print or its electronic equivalent. Either way, the taxpayer picks up the bill in the end. But open publication makes research more widely accessible, which is a public good in its own right.

\paragraph*{}
    One problem open access brings is a shift in incentives. More published articles means more revenue from processing charges. Rejection rates at high-end paid-for journals often exceed 90\%. A commercial open-access publisher (which PLOS, a charity, is not) might be tempted to publish anything that came his way, in order to pocket the APC—and many have, giving open access a fly-by-night feel to some academics.

\paragraph*{}
    For example, in a survey conducted by NPG, of 27,000 authors of papers in its journals, 44\% expressed some concern about the quality of open-access publishing and more than a third agreed with the notion that it was associated with less prestige. Yet those perceptions may be misguided. A study of Nature Communications (which was, until this week’s announcement, a hybrid between open access and traditional subscription, but is now pure open access) shows that its open-access papers enjoyed a slightly higher number of citations and significantly more downloads and online views than their non-open-access brethren.

\paragraph*{}
    To foster prestige, champions of open access have launched efforts such as eLife, an online journal with an array of academic bigwigs at its helm. They hope to create a top-tier publication by sheer weight of bigwiggery. But eLife and its kind cannot force the change alone, because publishing with the requisite due diligence is not cheap.

\paragraph*{}
    The Public Library of Science has found a way out of this by copying the more-the-merrier approach, but in a controlled way. Until 2006 it was a producer of high-impact but loss-making publications. Then it started PLOS ONE, a different kind of journal altogether. Instead of acting as an arbiter of the importance of scientific work, PLOS ONE claims only to ensure that articles are scientifically sound. With less effort going into peer review, PLOS ONE publishes many more papers (in 2013, it carried more than 31,000 articles, 36 times as many as the next-most-prolific PLOS journal) while simultaneously charging less for them and becoming a cash-cow that helps pay for the rest of the outfit. The Royal Society hopes to recapitulate this idea with its new offering, Royal Society Open Science.

\paragraph*{}
    \begin{bfseries}
        \large
        Free hits
    \end{bfseries}

\paragraph*{}
    Whether the experience of Nature Communications will overcome researchers’ misgivings remains to be seen. Despite the Wellcome Trust’s requirement that its grantees publish in open-access journals, only 70\% do so. To ensure compliance, the trust has had to introduce punitive measures, such as withholding money.

\paragraph*{}
    Some researchers just don’t care, though. A survey by Taylor \& Francis, a publishing firm, asked American and British scientists if they had published under open-access policies; 44\% and 32\%, respectively, did not know. More than half responded they did not know if they would in future. But many do know, and resist.

\paragraph*{}
    The point about prestige is not mere snobbery. The league table of journals is as finely graded as that of football, and, at the moment, has far less scope for promotion and demotion. Grant-awarding bodies and appointment committees know this and, in a wonderful display of doublethink, promote open access but also promote those who eschew it by publishing in top-notch, non-open journals

\paragraph*{}
    Only when that changes will open access’s victory be complete. This could happen either by new open-access journals acquiring the necessary kudos, or by old ones, seeing the game is up, becoming open access themselves. Though Nature Communications is a successful and well-regarded publication, it is not NPG’s top product. And Royal Society Open Science is untested. At the moment, then, both the Royal Society and NPG seem to be hedging their bets. When the society’s Proceedings, and NPG’s eponymous flagship, Nature, are both free for anyone to read, then open access’s partisans really will be able to declare victory and go home.

\paragraph*{}
    \href{https://www.economist.com/science-and-technology/2014/09/27/grand-openings}{\emph{\small The Economist, Sep 27th, 2014}}


\newpage
\section{补充材料}
\subsection{peer review}
\paragraph*{}
    \emph{\small 来自} \href{https://zh.wikipedia.org/wiki/同行評審}{\emph{\small 维基百科}}

同行评审(peer review,在某些学术领域亦称refereeing)或译为同侪审查,是一种学术成果审查程序, 即一位作者的学术著作或计划被同一领域的其他专家学者评审. 一般学术出版单位主要以同行评审的方法来选择与筛选所投送的稿件录取与否, 而学术研究资金提供机构, 也广泛以同行评审的方式来决定研究是否授予资金, 奖金等.

同行评审程序的主要目的是确保作者的著作水平符合一般学术与该学科领域的标准. 在许多领域著作的出版或者研究奖金的颁发, 如果没有以同行评审的方式来进行就可能比较会遭人质疑, 甚至成为某出版物,作品是否可以被称为学术出版物的主要标准.

\subsection{eLife}
\paragraph*{}
    \emph{\small 来自} \href{https://zh.wikipedia.org/wiki/ELife}{\emph{\small 维基百科}}

eLife是一个生物医学及生命科学方向的需要同行评审的开放获取的科学期刊. 2010年, 在一次霍华德·休斯医学研究所珍利亚农场研究园区举办的学术研讨会后, 由霍华德·休斯医学研究所, 马克斯·普朗克学会, 以及惠康基金会决定共同赞助创办. 该期刊正式创立于2012年底. 第一年该杂志共发表287篇文章. 这些组织共同提供了初始资金来支持商业和出版业务, 并且在2016年, 这些组织承诺投入2600万美元继续出版该期刊.

\subsection{PLOS ONE}
\paragraph*{}
    \emph{\small 来自} \href{https://en.wikipedia.org/wiki/PLOS_One}{\emph{\small 维基百科}}

公共科学图书馆:综合(PLOS ONE, 原名PLoS ONE)为一份同行评审的开放获取科学期刊, 由公共科学图书馆(Public Library of Science, PLOS)自2006年发行. PLOS ONE为全世界文章刊载数量最多的期刊.

PLOS ONE所刊载的文章包含科学及医学各领域的基础研究. 所有的文章在刊登前必须经过学术编辑的同行审查, 学术编辑则可以征求外部的意见. 刊登的文章并不会因为缺乏重要性而不被刊登, 除非其缺乏立论基础, 因为PLOS ONE线上平台的特色是提供一个\lq\lq 先刊登,后评论\rq\rq\ 的场域,使科学家可以即时提供文章反馈.


\newpage
\section{翻译}
学术进展的新发现将更容易进入公共领域, 也是时候做出这样的改变了.

2001年, 在布达佩斯举行了一场以学术刊物发表为主题的会议, 当时它被称作开放社会研究所(现在被称作开放社会基金). 在这个会议上提出了\lq 开放获取\rq\ 的概念. 这个具有号召力的官方声明呼吁全世界间不同阶级相互分享, 更应让社会底层拥有接触学术文献的机会, 让文献尽其所用, 这为之后人类的学术交流打下了坚实的基础. 换句话说, 让学术刊物对使用者免费开放.

这是一个十分崇高的理念. 但仍有不少挑刺者认为它不可能实现. 他们认为那些包括出版商在内的权贵们已经在三个世纪中享有学术成果的传播决定权, 并从中获取逼近40\%的利益. 因此, 学术权贵们理所当然会拒绝改变. 挑刺者有足够的理由担心开放获取运动沦为泡影.\footnote[1]{之前将 \emph{they} 当做\lq 学术权贵\rq, 现修改为\lq 挑刺者\rq.}

然而这些挑刺者不总是对的. 这个被会议促成的\lq 开放获取\rq\ 运动无人能挡. 比如, 英国七大研究委员会要求他们的学术成果在某种程度上免费获取. Well-come Trust是一家英国的慈善机构. 它的医学研究预算已经超过了多个科学发达国家. 在2016年之前, 他们要求有英国政府补助的科研结果也应该免费向大众开放\footnote[2]{在\lq 有英国政府补助...\rq\ 前添加\lq 他们要求\rq.}.

其他地方情况也差不多. 在2013年, 经过美国国会多年的激烈讨论, 白宫介入, 要求联邦机构每年多花100万美金去让科技成果文献免费为大众开放. 其他国家很多的大学, 社团和出资机构\footnote[3]{之前译为\lq 基金会\rq.}都如此要求.

出版商们并不想将学术刊物免费提供大众, 但外界压力逼得他们不得不做出改变. 这周在伦敦, 最老的学术出版商英国皇家学会和富有争议性的著名学术出版商自然杂志出版集团(NPG)也加入了开放获取运动, 他们承诺会发布向公众免费的学术刊物.

\paragraph*{\large 谁出版? 谁遭罪?\footnote[4]{Who will publish and who perish?}} \hspace{10pt} \\

开放获取运动其实早在布达佩斯会议一年之前的2000年就开始了. 当时是由英国的生物制药中心和美国的公共科学图书馆发起. 开放学术刊物出版的商业模式与传统学术刊物出版不同\footnote[5]{之前译为\lq 开放获取运动使学术出版的商业模式发生了改变\rq.}, 传统刊物由大学图书馆之类的订阅者来承担它的印刷费, 而开放刊物现在由科研工作者付一笔出版费, 用于在学术期刊或其电子版上发表其文章. 无论哪种商业模式, 最后还是由公众买单. 但相比之下, 开放获取运动还是更易让学术刊物被大众获取, 造福大众.

但开放获取运动带来的一个问题就是出版社刊登文章动机的转变. 现在出版社的营业收入与出版费挂钩, 刊登更多的论文意味着更高的收益. 高端付费的学术刊物对投稿者的拒绝率高达90\%. 盈利性的开放获取出版商可能经不起诱惑, 为了从出版费捞一笔, 别人投什么他就发什么. 这让某些学者觉得开放获取运动不可靠.\footnote[6]{删除\lq 发表了一些不可靠低质量的论文\rq. 添加本句.}.

例如, 在自然杂志出版社集团的一个问卷中, 对在其旗下杂志表过文章的27000名学者进行了调查, 44\%的人表达了对开放获取刊物中论文质量的担忧, 多于三分之一的人认为开放获取运动没有坚持初衷, 威望大打折扣. 然而这些投稿人的看法可能有点狭隘片面. 根据自然通讯的一个研究显示, 开放刊物的论文引用量稍稍多于传统出版商刊物, 下载量和在线访问量更要多得多.

为了提高论文质量\footnote[7]{之前直译为\lq 维护声誉\rq.}, 开放获取出版商中的佼佼者 eLife 做出了一系列努力. 它是一家以学术大佬撑腰的在线刊物. 它想仅凭学术大佬们的力量搞一个顶尖刊物. 但毕竟这么干很花钱, 仅靠 eLife 单打独斗远远不够.

公共科学图书馆出版社创造出了一种质量有保证的薄利多销经营方式. 2006年之前, 它虽是一个有影响力的出版商, 但其实一直在亏钱. 之后它们发行了 PLOS ONE, 一个与众不同的刊物. PLOS ONE 不去要求论文的高质量, 只要不是粗制滥造就好. 这样可以大大减少同行审查的成本. 结果 PLOS ONE 文章刊登量显著提高. 在2013年, 它发表了31000多篇论文, 比第二多产的 PLOS 的36倍还要多. 与此同时, 向投稿者收的出版费比其他刊物少. 薄利多销, PLOS ONE 变成了公共科学图书馆的摇钱树, 把 PLOS 部分的亏损也补上了. 皇家学会也想这么搞, 于是推出了皇家学会开放学术.

\end{document}