\documentclass[a4paper, UTF8, 12pt]{article}

\usepackage{ctex}
\usepackage{geometry}
\usepackage{graphicx}
\usepackage{float}
\usepackage{caption}
\usepackage{enumerate}
\usepackage{paracol}

\usepackage{hyperref}

\hypersetup{
    colorlinks=true,   % false, ,链接黑色, 外有红框
    linkcolor=black, % 目录颜色, 脚注颜色
    filecolor=blue, % 链接本地文件的链接颜色
    urlcolor=cyan, % 网页链接颜色
    anchorcolor=blue,
    citecolor=yellow    % 参考文献颜色
}

\geometry{
    textheight=230mm,
    textwidth=180mm
}

\begin{document}
\title{\Huge 英译汉课程}
\author{\Large 
        第五周 第六周 \\[12pt]
        1931991 任家平 \\[12pt]
        同济大学 \\[12pt]
        测绘与地理信息学院}
\date{2019-10-07}

\maketitle
\thispagestyle{empty}

\newpage
\pagenumbering{Roman}
\tableofcontents
\addtocontents{toc}{\protect\vspace{10pt}}

\newpage
\pagenumbering{arabic}

\begin{paracol}{2}[\section{NOTE}]
    
    \switchcolumn*
    第一段翻译

    \switchcolumn
    Within the UNCTAD Division on Technology and Logistics, the ICT Analysis Section carries out policy-oriented analytical work on the development implications of information and communication technologies (ICTs). It is responsible for the preparation of the Information Economy Report. The ICT Analysis Section promotes international dialogue on issues related to ICTs for development, and contributes to building developing countries’ capacities to measure the information economy and to design and implement relevant policies and legal frameworks. 

    \switchcolumn*
    第二段翻译

    \switchcolumn
    In this Report, the terms country/economy refer, as appropriate, to territories or areas. The designations employed and the presentation of the material do not imply the expression of any opinion whatsoever on the part of the Secretariat of the United Nations concerning the legal status of any country, territory, city or area or of its authorities, or concerning the delimitation of its frontiers or boundaries. In addition, the designations of country groups are intended solely for statistical or analytical convenience and do not necessarily express a judgement about the stage of development reached by a particular country or area in the development process. The major country groupings used in this Report follow the classification of the United Nations Statistical Office. These are: 

    Developed countries: the member countries of the Organization for Economic Cooperation and Development (OECD) (other than Chile, Mexico, the Republic of Korea and Turkey), plus the new European Union member countries that are not OECD members (Bulgaria, Cyprus, Latvia, Lithuania, Malta and Romania), plus Andorra, Liechtenstein, Monaco and San Marino. Countries with economies in transition: South-East Europe and the Commonwealth of Independent States. Developing economies: in general, all the economies that are not specified above. For statistical purposes, the data for China do not include those for Hong Kong Special Administrative Region (Hong Kong, China), Macao Special Administrative Region (Macao, China), or Taiwan Province of China. 

    Reference to companies and their activities should not be construed as an endorsement by UNCTAD of those companies or their activities.

    The following symbols have been used in the tables: 
    \begin{itemize}
        \item Two dots(..)indicate that data are not available or are not separately reported. Rows in tables have been omitted in those cases where no data are available for any of the elements in the row; 
        \item A dash (-) indicates that the item is equal to zero or its value is negligible;
        \item A blank in a table indicates that the item is not applicable, unless otherwise indicated; 
        \item A slash (/) between dates representing years, for example, 1994/95, indicates a financial year; 
        \item Use of an en dash (–) between dates representing years, for example, 1994–1995, signifies the full period involved, including the beginning and end years; 
        \item Reference to “dollars” (\$) means United States of America dollars, unless otherwise indicated; 
        \item Annual rates of growth or change, unless otherwise stated, refer to annual compound rates; 
        \item Details and percentages in tables do not necessarily add up to the totals because of rounding. 
    \end{itemize}

    The material contained in this study may be freely quoted with appropriate acknowledgement.

    \begin{center}
        UNITED NATIONS PUBLICATION 

        UNCTAD/IER/2013 

        Sales No. E.13.II.D.6

        ISSN 2075-4396 

        ISBN 978-92-1-112869-7 

        e-ISBN 978-92-1-054154-1 

        Copyright © United Nations, 2013

        All rights reserved. Printed in Switzerland
    \end{center}

\end{paracol}
\columnratio{0.7}
\begin{paracol}{2}[\section{PREFACE}]
    Innovation in the realm of information technology continues its rapid pace, with cloud computing representing one of the latest advances. Significant improvements in the capacity to process, transmit and store data are making cloud computing increasingly important in the delivery of public and private services. This has considerable potential for economic and social development, in particular our efforts to achieve the Millennium Development Goals and define a bold agenda for a prosperous, sustainable and equitable future. 

    The Information Economy Report 2013 marks the first time the United Nations is examining the economic potential of cloud computing for low- and middle-income countries, where rates of adoption are currently low. With governments, businesses and other organizations in the developing world considering whether to migrate some or all of their data and activities to the cloud, this publication is especially timely. I commend its information and analysis to all those interested in learning more about the benefits and risks of the cloud economy.
    \begin{center}
        BAN Ki-moon 
        
        Secretary-General
        
        United Nations
    \end{center}
    
\end{paracol}

\begin{paracol}{2}[\section{ACKNOWLEDGEMENTS}]
   The Information Economy Report 2013 was prepared by a team comprising Torbjörn Fredriksson (team leader), Cécile Barayre, Shubhangi Denblyden, Scarlett Fondeur Gil, Suwan Jang, Diana Korka, Smita Lakhe and Marie Sicat under the direction of Anne Miroux, Director of the Division on Technology and Logistics. 
   
   The report benefited from major substantive inputs provided by Michael Minges, David Souter, Ian Walden and Shazna Zuhyle. Research ICT Africa provided original research for five country case studies. Additional inputs were contributed by Tiziana Bonapace, Axel Daiber, Nir Kshetri, Rémi Lang and Howard Williams. 
   
   Comments on the initial outline of the report were provided by experts attending a brainstorming meeting organized in Geneva in February 2013, including Jamil Chawki, Alison Gillwald, Abi Jagun, Martin Labbé, Juuso Moisander, Jason Munyan, Jorge Navarro, Thao Nguyen, Marta Pérez Cusó and Lucas von Zallinger. Valuable feedback on various parts of the text was also given by experts attending a peer review meeting organized in Geneva in July 2013, including Chris Connolly, Bernd Friedrich, Alison Gillwald, Angel González-Sanz, Nir Kshetri, Matthias Langenegger, Mpho Moyo, Tansuğ Ok, Daniel Ramos and Carlos Razo. 
   
   Additional comments were received at various stages of the production of the report from Dimo Calovski, Padmashree Gehl Sampath, Esperanza Magpantay, Markie Muryawan and Marco Obiso. Ngozi Onodugo provided helpful assistance and inputs during her internship with UNCTAD. 
   
   UNCTAD is grateful for the sharing of data by national statistical offices and responses received to UNCTAD’s annual survey questionnaire on ICT usage by enterprises and on the ICT sector. The sharing of data for this report by the International Telecommunication Union, LIRNEasia, Research ICT Africa and TeleGeography is highly appreciated. 
   
   The cover was done by Sophie Combette. Desktop publishing was done by Nathalie Loriot, graphics were carried out by Stephane Porzi and Christian Rosé and the Information Economy Report 2013 was edited by Maritza Ascencios, Lucy Annette Deleze-Black and John Rogers. 
   
   Financial support from the Government of Finland and the Republic of Korea is gratefully acknowledged. 
\end{paracol}
W
\begin{paracol}{2}[\section{LIST OF ABBREVIATIONS}]
    \begin{description}
        \item[3G] third generation (refers to mobile phones
        \item[ACP] The African, Caribbean and Pacific Group of States 
        \item[ADSL] asymmetric digital subscriber line 
        \item[API] application programming interface 
         
    \end{description}
    % \begin{tabbing}
    %     3G\quad \= third generation (refers to mobile phones)\kill
    %     3G\quad \= third generation (refers to mobile phones)\\[10pt]
    %     ACP\quad \= The African, Caribbean and Pacific Group of States\\[10pt] 
    % \end{tabbing}
    % \begin{table}{ll}
    %     Abbreviation & Full Name \\[5pt]
    %     3G & third generation (refers to mobile phones)\\
    %     ACP & The African, Caribbean and Pacific Group of States\\
    %     ADSL & asymmetric digital subscriber line\\
    %     API & application programming interface
    % \end{table}
\end{paracol}



\section{翻译反思}

\end{document}