\documentclass[a4paper, UTF8, 12pt]{article}

\usepackage{ctex}
\usepackage{geometry}
\usepackage{graphicx}
\usepackage{float}
\usepackage{caption}
\usepackage{enumerate}
\usepackage{paracol}
\usepackage{longtable}
\usepackage{array}
\usepackage{multirow}

\usepackage{hyperref}

\hypersetup{
    colorlinks=true,   % false, ,链接黑色, 外有红框
    linkcolor=black, % 目录颜色, 脚注颜色
    filecolor=blue, % 链接本地文件的链接颜色
    urlcolor=cyan, % 网页链接颜色
    anchorcolor=blue,
    citecolor=yellow    % 参考文献颜色
}

\geometry{
    textheight=230mm,
    textwidth=180mm
}

\begin{document}
\title{\Huge 英译汉课程}
\author{\Large 
        第七周 \\[12pt]
        1931991 任家平 \\[12pt]
        同济大学 \\[12pt]
        测绘与地理信息学院}
\date{2019-10-15}

\maketitle

\thispagestyle{empty}

\newpage
\pagenumbering{Roman}
\tableofcontents
\addtocontents{toc}{\protect\vspace{10pt}}

\newpage
\pagenumbering{arabic}

\begin{paracol}{2}[\section{OVERVIEW(概述)}]
    
    \switchcolumn*
    \paragraph{} \quad {\bfseries 云计算着重强调数字鸿沟的质量层面.}
    \switchcolumn
    {\bfseries Cloud computing accentuates the quality dimension of the digital divide}

    \switchcolumn*
    长久以来, 数字鸿沟, 即各个国家接触和使用信息通讯技术的差距, 是各国政府和国际社会的关注点之一. 随着时代发展, 其含义也发生了改变. 之前其含义是基础电话服务普及的困难, 随着电话服务的大规模使用, 现在数字鸿沟指代的是网络服务普及的困难, 尤指宽带服务. 宽带服务容量和质量的数字鸿沟会对一些国家或地区的个人, 企业, 经济体或是社会团体能否借助信息通信技术创新应用产生影响. 
    \switchcolumn
    The differential between countries in access to and use of information and communication technologies (ICTs) – the digital divide – has long been a significant concern of Governments and the international community. Over time, its nature has changed. The gap in access to basic telephone services, once very substantial, is now significantly diminished and expected to shrink further in the next few years. In its place has come a gap in access to the Internet and, particularly, in access to broadband services. The digital divide in broadband capacity and quality leads in turn to a divide between countries and regions in the extent to which individuals, businesses, economies and societies are able to take advantage of new ICT innovations and applications. 
    
    \switchcolumn*
    云计算是近年来蓬勃发展信息通信技术领域的一个重要表现. 其潜力无穷, 对政府和企业愈发重要. 简单来说, 它能让用户通过因特网或是其他数字网络访问无限可扩展的数据存储池或计算资源. 有些人认为云技术是二十年中最有颠覆性的技术, 会让市场, 经济体或是社会团体产生天翻地覆的变化. 在此背景下, 《经济信息报告2013》为发展中国家对云技术可能产生影响进行了客观的分析.
    \switchcolumn
    Cloud computing is a recent manifestation of this evolving ICT landscape. Given its potential, it is becoming increasingly important for Governments and enterprises. In simple terms, cloud computing enables users, through the Internet or other digital networks, to access a scalable and elastic pool of data storage and computing resources, as and when required. Some predict that cloud technology will be among the most significant disruptive technologies over the next two decades, with major implications for markets, economies and societies. Against this background, the Information Economy Report 2013 provides an objective analysis of the possible implications for developing countries of the evolving cloud economy.
    
    \switchcolumn*
    \paragraph{} \quad {\bfseries 数据存储, 处理和传输能力的巨大提升让云经济成为可能.}
    \switchcolumn
    \paragraph{}
    {\bfseries Massive improvements in storage, processing and transmission capacity have pav-ed the way for the cloud economy.} 
    
    \switchcolumn*
    `云' 的比喻可能有不太准确, 它指的并不是天空上无形的物体, 而是以硬件, 网络, 存储, 服务和为提供计算服务的接口组合而成的一个系统. 云计算的特征之一是它会将数据传送到第三方控制的服务器中.
    \switchcolumn
    The metaphor of the “cloud” can be misleading. Rather than representing an amorphous phenomenon in the sky, cloud computing is well anchored on the ground by the combination of physical hardware, networks, storage, services and interfaces that are needed to deliver computing as a service. A key feature of cloud computing is that it often involves transferring data to a server controlled by a third party. 

    \switchcolumn*
    计算能力, 数据储存和高速传输能力使云计算的迅速发展, 它意味着电信, 企业, 社会之间关系发生了巨大变化. 比如由因特尔制造的22纳米中央处理器运行速度比其1971年制造的处理器快了将近4000倍. 在1986年到2007年之间, 计算机的内存容量每三年大致要翻一番. 比如在1993年, 互联网浏览器才被发明, 当时理论上拨号连接的最大速度是56kb每秒. 但在2013年, 消费级的宽带可达到2gb每秒, 几乎是拨号网络的36000倍. 主要的云服务提供商现拥有成千上百的服务器坐落于全球各个大数据中心.
    \switchcolumn
    The shift that is taking place towards the cloud represents a step change in the relationship between telecommunications, business and society, and has been enabled by massively enhanced processing power, data storage and higher transmission speeds. For example, Intel’s 22-nanometre central processing unit is 4,000 times faster than that which the same company introduced in 1971, and between 1986 and 2007 the world’s “technological memory” roughly doubled every three years. Meanwhile, the fastest theoretical speed of a dial-up connection in 1993, the year the Internet browser was introduced, was 56 kilobits per second (kbit/s); as of 2013, consumer broadband packages of 2 gigabits per second (Gbit/s) are available, almost 36,000 times faster than dial-up. Major cloud service providers today have hundreds of thousands of servers located in massive data centres in different parts of the world. 

    \switchcolumn*
    根据2013年国际电信联盟和国际标准组织的定义, 云计算是一个可通过网络访问的无限扩展范式模型, 它根据客户需求, 提供分享式物理的数据存储和虚拟的计算自服务与管理. 云服务为可被用户在任意时间任意联网设备上使用云计算的服务. 两者对于经济发展的影响将在本报告云经济部分谈到.
    \switchcolumn
    According to definitions proposed in April 2013 by the International Telecommunication Un-ion (ITU) and the International Organization for Standardization (ISO), cloud computing is a para-digm for enabling network access to a scalable and elastic pool of shareable physical or virtual resources with on-demand selfservice provisioning and administration. Cloud services are defined as services that are provided and used by clients on demand at any time, through any access network, using any connected devices that use cloud computing technologies. The implications of cloud computing and cloud services on wider economic development are discussed in this report in the context of the cloud economy. 

    \switchcolumn*
    \paragraph{} \quad {\bfseries 云经济由多种云服务组成并部署.}
    \switchcolumn
    \paragraph{}
    {\bfseries The cloud economy comprises various cloud service categories and deployments.}

    \switchcolumn*
    然而, `云' 和 `云服务' 并非同一类产品, 他们在配置上截然不同. 云服务有三大类, 基础设施即服务, 平台即服务, 软件即服务, 它们组成了现阶段被使用的云服务. 每种云服务的特性是虚拟方式的云计算或是由用户向服务商租用订, 可远程使用的信息科技设备.
    \switchcolumn
    However, the “cloud” and “cloud services” are not homogenous products but come in different shapes and configurations. Three categories of cloud services – infrastructure as a service (IaaS), platform as a service (PaaS) and software as a service (SaaS) – are commonly used to encompass the whole range of cloud service categories that are currently available. The defining characteristic of each of these variations of the cloud is the type of computing or information technology (IT) facilities that is made available remotely to a cloud service customer, on a rental or subscription basis, by a cloud service provider: 

    \switchcolumn*
    对于基础设施即服务来说, 提供商要提供数据计算与储存, 网络和其他基础运算资源的部署和用户可使用的软件. 基础设施即服务的可扩展性让社会组织或是企业可以灵活并及时地使用计算基础设施.
    \switchcolumn
    In the case of IaaS, the cloud provider’s processing, storage, networks and other fundamental computing resources allow the cloud customer to deploy and run software. The elasticity of IaaS allows an organization or enterprise to access computing infrastructure in a flexible and timely manner.  

    \switchcolumn*
    在平台即服务方面, 客户将自己的应用程序和数据部署在属于云提供商并由其管理的平台工具(包括编程工具)上
    \switchcolumn
    In the case of PaaS, the cloud customer deploys its own applications and data on platform tools, including programming tools, belonging to and managed by the cloud provider

    \switchcolumn*
    对于软件即服务, 客户可将自己的程序在提供商的服务器上运行, 无需使用自己的设备. 客户所需要的应用可通过简单的客户接口, 如网页浏览器(网页版邮箱), 或是程序接口访问云端应用.
    \switchcolumn
    With SaaS, the cloud customer takes advantage of software running on the cloud-provider’s infrastructure rather than on the customer’s own hardware. The applications required are accessible from various client devices through either a thin client interface, such as a web browser (for example, web-based email), or a program interface. 

    \switchcolumn*
    云服务也可通过多种方式部署, 比较常见的如下所示:
    \switchcolumn
    Cloud services can also be deployed to users in a variety of ways, the most significant of which are summarized below: 

    \begin{itemize}
        \switchcolumn*
        \item {\bfseries 公共云端} 可通过网络向公众提供的开放性服务. 很多被个人广泛使用的大众市场服务, 如网络邮件, 在线储存和网络社交都是公共云服务的一种.
        \switchcolumn
        \item {\bfseries Public clouds}: open resources that offer services over a network that is open for public use. Many mass market services widely used by individuals, such as webmail, online storage and social media are public cloud services.
        
        \switchcolumn*
        \item {\bfseries 私人云端} 由内部或是第三方管理和托管, 为单个组织提供私的人云服务, 如政府或大企业.
        \switchcolumn 
        \item {\bfseries Private clouds}: proprietary resources provided for a single organization (for example, a Government or large enterprise), managed and hosted internally or by a third-party. 
        
        \switchcolumn*
        \item {\bfseries community clouds} 由内部或是第三方管理和托管, 为少量客服提供的可共享服务.
        \switchcolumn 
        \item {\bfseries 社区云端} resources/services provided for and shared between a limited range of clients/ users, managed and hosted internally or by a third-party. 
        
        \switchcolumn*
        \item {\bfseries 混合云端} 上述云端混合体, 比如, 提供公共私人云服务.
        \switchcolumn 
        \item {\bfseries Hybrid clouds}: a mix of the deployment models described above, for example, public and private cloud provision. 
    \end{itemize}

    \switchcolumn*
    {\bfseries 不同的云配置为潜在的云服务客户带来机遇和风险.}
    \switchcolumn
    {\bfseries Different cloud configurations offer both opportunities and risks for potential cloud service customers.}

    \switchcolumn*
    作为分析的基础, 本报告使用了云经济生态系统的概念, 该概念着重说明了云计算和云服务在更广泛的信息经济中的部署和影响, 以及它们与国民经济发展的相关性. 云经济生态包含了技术与企业, 管理与创新, 生产与消费等一些列复杂关系. 正是这种生态的发展模式而不是单单技术本身的潜力, 决定了发展中国家的发展成果.
    \switchcolumn
    As a basis for the analysis, the Information Economy Report 2013 uses the concept of the cloud economy ecosystem, which highlights the deployment and impacts of cloud computing and cloud services within the wider information economy and, thereby, their relevance to national economic development. The cloud economy ecosystem includes a complex set of relationships between technology and business, governance and innovation, production and consumption. It is how this ecosystem evolves, rather than the potential of the technology alone, that will determine the outcomes for developing countries.

    \switchcolumn*
    发展中国家的政府, 企业或是其他组织考虑是否要将部分或是全部数据和业务迁移到云端, 他们需要评估这么做潜在的收益和风险.
    \switchcolumn
    As Governments, enterprises and other organizations in the developing world consider whether to migrate some or all of their data and activities to the cloud, they need to assess the potential advantages and risks of such a move. 

    \switchcolumn*
    收益如下:
    \switchcolumn
    Potential advantages include:
    \begin{itemize}
        \switchcolumn*
        \item 相比使用公司设备和相关管理费, 使用云服务可以降低硬件软件的成本.
        \switchcolumn
        \item Reduced costs for rented IT hardware and software compared to in-house equipment and IT management; 
        
        \switchcolumn*
        \item 云服务科根据需求提供出色的数据储存处理的能力的灵活性
        \switchcolumn
        \item Enhanced elasticity of storage/processing capacity as required by demand;
        
        \switchcolumn*
        \item 访问数据与服务的灵活与便携.
        \switchcolumn
        \item Greater flexibility and mobility of access to data and services;
        
        \switchcolumn*
        \item 软件的免费即使更新.
        \switchcolumn
        \item Immediate and cost-free upgrading of software; 
        
        \switchcolumn*
        \item 数据管理和服务的安全可靠.
        \switchcolumn
        \item Enhanced reliability/security of data management and services. 
    \end{itemize}

    \switchcolumn*
    潜在风险如下:
    \switchcolumn
    Potential risks or disadvantages include:
    \begin{itemize}
        \switchcolumn*
        \item 增加了网络供应商的压力.
        \switchcolumn
        \item Increased costs of communications (to tele-communication operators/Internet service p-roviders (ISPs)); 
        
        \switchcolumn*
        \item 增加了数据迁移融合的成本.
        \switchcolumn
        \item Increased costs for migration and integration;
        
        \switchcolumn*
        \item 减少了客户对数据和应用的可控性.
        \switchcolumn
        \item Reduced control over data and applications;
        
        \switchcolumn*
        \item 数据安全性和对隐私的担忧
        \switchcolumn
        \item Data security and privacy concerns; 
        
        \switchcolumn*
        \item 可能会出现由于信息技术落后或是电力基础设施不足而导致服务无妨访问.
        \switchcolumn
        \item Risk of services being inaccessible, for example, due to inadequate ICT or power infrastructure;
        
        \switchcolumn*
        \item 在垄断的云经济市场, 可能出现被云服务供应商掐断服务的情况(有限的互操作性和低数据可迁移性).
        \switchcolumn
        \item Risk of lock-in (limited interoperability and data portability) with providers in uncompetitive cloud markets.
    \end{itemize}
    
    \switchcolumn*
    云服务提高效率的潜力激励着私人或是公共领域将其活动迁移到云端. 与此同时也要考虑云服务的缺点, 比如, 虽然它节省了成本, 但同时也有数据安全和隐私的风险. 云服务的利弊对不同的云服务客户影响不同, 因此客户所采取的策略也不同. 有些企业, 政府和其他组织由于其活动或是商业模式的性质, 可以更多的享受云经济带来的好处, 或是可以比其他企业获得更大优势. 尤其对于维护自身IT部分已耗资巨大的企业, 由于他们要一直使用IT硬件软件, 对IT资源有各种各样的大量需求, 云服务非常适合他们, 对可以从对数据和市场机遇的合理利用获得更多附加价值的企业也很适合.
    \switchcolumn
    The cloud’s potential to improve efficiency is a strong incentive for organizations in the private and public sectors to transfer activities to the cloud. At the same time, there are important trade-offs to be made, for example, between cost savings on the one hand and considerations related to data security and privacy on the other. Various cloud customers will assess the opportunities and risks associated with the cloud differently, therefore opting for different solutions. Some businesses, Governments and other organizations are better positioned to reap the benefits of a shift to the cloud, or can gain greater advantage than others because of the nature of their activities or business model. This is the case, for example, for those that have high fixed costs in maintaining inhouse IT departments, recurrently need IT software and hardware, face large or unpredictable variations in demand for IT resources or can gain substantial added value from more efficient exploitation of data and market opportunities. 

\end{paracol}

\section{LIST OF ABBREVIATIONS(缩略词)}
\begin{longtable}{m{2cm}m{3cm}m{3cm}m{4cm}}
    \hline
    {\bfseries 简称} & {\bfseries 全称} & {\bfseries 翻译} & {\bfseries 备注}\\[5pt] \hline
    3G & third generation (refers to mobile phones) & 第三代移动通信技术 & \\
    \hline
    ACP & The African, Caribbean and Pacific Group of States & 太平洋地区国家集团 & \\
    \hline
    ADSL & asymmetric digital subscriber line & 非对称数字用户线路 & \\  
    \hline
    API & application programming interface & 应用编程接口 & \\
    \hline
    BPaaS & business process as a service & 业务流程即服务 & 云计算交付模式之一 \\
    \hline
    BPO & business process outsourcing & 业务流程外包 & 将职能部门的全部功能都转移给供应商 \\
    \hline
    bps & bits per second & 比特每秒, 波特率 & \\
    \hline
    BRICS & Brazil, the Russian Federation, India, China and South Africa & 金砖五国 & 巴西, 俄罗斯, 印度, 中国, 南非\\
    \hline
    CaaS & communication as a service & 通信即服务 & \\
    \hline
    CERT & computer emergency response team & 计算机应急响应小组 &\\
    \hline
    CIO & chief information officer & 首席信息官 & \\
    \hline
    CPC & Central Product Classification & 中心产品目录 & \\
    \hline
    CPU & entral processing unit & 中央处理器 & \\
    \hline
    CRM & client customer relationship management & 客户关系管理 &\\
    \hline
    ERP & enterprise resource planning & 企业资源规划 & \\
    \hline
    GATS & General Agreement on Trade in Services & 服务贸易总协议 & \\
    \hline
    GB & gigabyte & 吉字节 & \\
    \hline
    Gbit/s, Gbps & gigabits per second & 千兆位每秒 & \\
    \hline
    GDP & gross domestic product & 国内生产总值 & \\
    \hline
    IaaS & infrastructure as a service & 基础设施即服务 & 消费者使用处理、储存、网络以及各种基础运算资源,部署与执行操作系统或应用程序等各种软件\\
    \hline
    ICT & information and communication technology & 信息与通信技术 & \\
    \hline
    IDC & International Data Corporation & 国际数据公司 & \\
    \hline
    IP & Internet protocol & 网际协议 & \\
    \hline
    ISO & International Organization for Standardization & 国际标准组织 & \\
    \hline
    ISP & Internet service provider & 互联网服务供应商 & \\
    \hline
    IT & information technology & 信息技术 & \\
    \hline
    ITU & International Telecommunication Union & (联合国)国际电信联盟 &\\
    \hline
    ITU-T & ITU Telecommunication Standardization Sector & 国际电信联盟电信标准部 & \\
    \hline
    IXP & Internet exchange point & 互联网交换中心 & \\
    \hline
    kbit/s, kbps & kilobits per second & 比特率, 千比特每秒 & \\
    \hline
    LDC & least developed country & 最不发达国家 & \\
    \hline
    LTE & long-term evolution & 长期演进技术 & \\
    \hline
    m2m & mothers-2-mothers organization & 艾滋母亲互助协会 & \\
    \hline
    Mbit/s, Mbps & megabits per second & 兆字节每秒 & \\
    \hline
    ms & millisecond & 毫秒 & \\
    \hline
    NCIA & National Computing and Information Agency (Republic of Korea) & 综合计算机中心 & \\
    \hline
    NDC & national data centre & 国家数据中心 & \\
    \hline
    NGO & non-governmental organization & 非政府组织 & \\
    \hline
    NIST & National Institute of Standards and Technology & 国家标准与技术协会 & \\
    \hline
    NTT & Nippon Telegraph and Telephone Corporation & 日本电报电话公司 & \\
    \hline
    OECD & Organization for Economic Cooperation and Development & 经济合作与发展组织 & \\
    \hline
    PaaS & platform as a service & 平台即服务 & \\
    \hline
    PC & personal computer & 个人电脑 & \\
    \hline
    PPP & public–private partnership & 公私合作, 公私合营 & \\
    \hline
    PUE & power usage effectiveness & 能源使用效率 & \\
    \hline
    QoS & quality of service & 服务质量 & \\
    \hline
    RTT & round-trip time & 往返时延 & \\
    \hline
    SaaS & software as a service & 软件即服务 & \\
    \hline
    SLA & service level agreement & 服务级别协议 & \\ 
    \hline
    SME & small and medium-sized enterprise & 中小型企业 & \\
    \hline
    SMS & short message service & 短信服务 & \\
    \hline
    Tbps & terabits per second & 兆兆每秒 & \\
    \hline
    TDF & transborder data flow & 跨境数据流 & \\ 
    \hline
    TNC & transnational corporation & 跨国公司 & \\
    \hline
    UNCTAD & United Nations Conference on Trade and Development & 联合国贸易与发展会议 & \\
    \hline
    WTO & World Trade Organization & 世界贸易组织 & \\
    \hline
    XaaS & x as a service & 一切皆服务 & \\
    \hline
\end{longtable}

\section{反思}
缩略词部分按照格式随便写了写. 我还是比较看重文章能否理解, 对于这些格式问题真不大感兴趣. 课堂上, 大家好像很纠结 `m2m' 的翻译, 在网上查了组织的定义, 就翻译成 `艾滋母亲互助协会'.

记录一些翻译中遇到的问题.

\begin{itemize}
    \item \emph{quality dimension} 质量方面; 一看到 `dimension' 第一反应是 `维度'. 所以在大脑中难以形成该词语的直观印象, 卡了一会.
    \item \emph{digital divide} 不会翻译, 网上查到的翻译是 `数字鸿沟'. 其指的是, 在全球数字化进程中, 不同国家, 地区, 行业, 企业, 社区之间,由于对信息, 网络技术的拥有程度, 应用程度以及创新能力的差别而造成的信息落差及贫富进一步两极分化的趋势.
    \item \emph{a scalable and elastic pool} 直接按照单词原意翻译会显得莫名其妙, 所以在百度上查找了云计算的特点, 只有 `无限可扩展性' 和两个词语的意思能对的上. 在我个人理解, 对于云服务的用户来讲, 他只需要将自己的数据上传, 不需要考虑数据结构等等对服务器的占用, 因此云端服务器对他来讲是无限可扩展.
    \item \emph{technological memory} 在原文中, 这两个词语是带引号, 一般 memory 指的是电脑的内存, 所以我直接翻译成了内存, 正好这一段是讲计算机性能提升, 所以应该是内存吧.
    \item \emph{physical or virtual resources} 个人理解 `physical resources' 指的是云计算可以让用户将数据存储到云端的服务器, `virtual resources' 指的应该是可以在云端运行自己的软件, 让提供商计算服务.
    \item \emph{The cloud economy ecosystem includes a complex set of relationships between technology and business, governance and innovation, production and consumption.} 翻译的时候注意语序问题.
    \item \emph{lock-in} 个人理解应该是, 云服务提供服务商对用户如限速等的一系列限制, 具体可参考百度网盘.
\end{itemize}

以下为比较难翻译的部分. 总而言之, 句子太长了, 我搞不清修饰词修饰谁.

\begin{enumerate}
    \item The digital divide in broadband capacity and quality leads in turn to a divide between countries and regions in the extent to which individuals, businesses, economies and societies are able to take advantage of new ICT innovations and applications. 在此句子中, `extent' 指代的应该是 `方面' 和上文所述 `dimension' 相似. 可以对官方翻译的类似词语做一个总结.
    
    \item  Some businesses, Governments and other organizations are better positioned to reap the benefits of a shift to the cloud, or can gain greater advantage than others because of the nature of their activities or business model. 此处 `or' 与 `better positioned to ...' 并列, 之前搞不清并列关系, 卡住.
    
    \item This is the case, for example, for those that have high fixed costs in maintaining inhouse IT departments, recurrently need IT software and hardware, face large or unpredictable variations in demand for IT resources or can gain substantial added value from more efficient exploitation of data and market opportunities. 出现了两次 `or', 第一次 `or'前后的形容词是用来形容 `variation'. 第二次的 `or' 应该修饰 `for those who'. 
    
    \item cloud computing is a paradigm for enabling network access to a scalable and elastic pool of shareable physical or virtual resources with on-demand selfservice provisioning and administration. 后面 `with' 所带的部分卡住, 发现 `selfservice provisioning' 和 `administration' 并列.
    
    \item resources/services provided for and shared between a limited range of clients/ users, managed and hosted internally or by a third-party. 一开始在翻译的时候卡住, 因为没搞清楚 `managed and hosted' 修饰的是谁.
\end{enumerate}

这翻译真是太难了. 长句子堆砌, 经常会卡在一个长句子, 或是明明读懂了, 中文语序很难组织. 翻译出来牛头不对马嘴, 看不出上下文逻辑, 很像机翻.
\end{document}