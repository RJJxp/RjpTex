\documentclass[a4paper, UTF8, 12pt]{article}

\usepackage{ctex}
\usepackage{geometry}
\usepackage{graphicx}
\usepackage{float}
\usepackage{caption}
\usepackage{enumerate}
\usepackage{paracol}
\usepackage{longtable}
\usepackage{array}
\usepackage{multirow}

\usepackage{hyperref}

\hypersetup{
    colorlinks=true,   % false, ,链接黑色, 外有红框
    linkcolor=black, % 目录颜色, 脚注颜色
    filecolor=blue, % 链接本地文件的链接颜色
    urlcolor=cyan, % 网页链接颜色
    anchorcolor=blue,
    citecolor=yellow    % 参考文献颜色
}

\geometry{
    textheight=230mm,
    textwidth=180mm
}

\begin{document}
\title{\Huge 英译汉课程}
\author{\Large 
        第八周 \\[12pt]
        1931991 任家平 \\[12pt]
        同济大学 \\[12pt]
        测绘与地理信息学院}
\date{2019-10-21}

\maketitle

\thispagestyle{empty}

\newpage
\pagenumbering{Roman}
\tableofcontents
\addtocontents{toc}{\protect\vspace{10pt}}

\newpage
\pagenumbering{arabic}

\begin{paracol}{2}[\section{OVERVIEW PART\uppercase\expandafter{\romannumeral2}}]
    
    \switchcolumn*
    \paragraph{} \quad {\bfseries 云经济增长迅速但规模依然很小}
    \switchcolumn
    {\bfseries The cloud economy is expanding fast but is still small.}

    \switchcolumn*
    云服务的市场规模有多种预测结果. 预计在2015年, 提供基础设施即服务, 平台即服务和软件即服务的公共云服务费用产生的营业额大致在430亿到940亿美元之间, 这包括了对用户免费使用的网页云服务应用的广告收入. 广告的收入远大于公共云服务. 私有云的估值结果差别较大, 为50亿到500亿美元. 不同的预测方法会有不同的结果, 但大多数结果表明云服务的使用率会近年来会快速提高.
    \switchcolumn
    There are various estimates of the size of the market for cloud services. Fee-generated revenues from public provision of IaaS, PaaS and SaaS have been forecast to reach somewhere between \$43 billion and \$94 billion by 2015. To this can be added the revenue generated through advertising on cloud-enabled web applications that are available at no cost to the user. Such revenue is currently considerably larger than the fees generated from public cloud provision. Estimates of the value of private cloud services also vary greatly – from about \$5 billion to about \$50 billion. Discrepancies in projections reflect different methodologies, but most forecasts agree that cloud adoption will continue to expand rapidly over the next few years.
    
    \switchcolumn*
    全球信息通讯技术在2011年的产业估值大致有4万亿美元, 云服务所占份额非常小. 然而, 信息通讯技术的大部分领域都在一定程度上受到了云计算的影响. 对带宽的需求将推动电信服务收入的增长, 随着越来越多的人转而使用基于网络协议的云语音服务, 传统语音服务的收入可能会受到影响. 随着越来越多的服务迁移到云中,对设备和计算机硬件(尤其是数据服务器和网络设备)的需求将上升。
    \switchcolumn
    This is still very small compared with the revenue of the global ICT sector, which was estimated at about \$4 trillion in 2011. However, most segments of the ICT sector are in some way affected by cloud computing. The demand for bandwidth will drive telecommunication services revenue, although revenues from voice services could be affected as more people switch to cloud-based voice over Internet protocol applications. Demand for equipment and computer hardware, particularly data servers and network equipment, will rise as more services move to the cloud. 

    \switchcolumn*
    向云的转移正在极大地促进数据流量的增长. 在2012年的平均时间里, 谷歌收到了200万次搜索请求, 脸书用户分享了大约七十万条动态, 而推特发送了十万条推文. 2012年, 互联网上此类云流量的60\%来自欧洲和北美. 亚太地区占了另外三分之一, 而拉丁美洲, 中东和非洲加在一起只占了5\%.但是, 预计未来几年中中东和非洲的增长率最高.
    \switchcolumn
    The shift to the cloud is generating considerable growth in data traffic. During an average minute in 2012, Google received two million search requests, Facebook users shared around 700,000 content items and Twitter sent out 100,0-00 tweets. In 2012, 60 per cent of such cloud traffic on the Internet emanated from Europe and North America. Asia–Pacific was responsible for another third while Latin America and the Middle East and Africa together accounted for only 5 per cent. However, the highest growth rates in the next few years are expected in the Middle East and Africa. 

    \switchcolumn*
    在服务的提供方面, 云经济目前由一些势力强大的云服务提供商垄断, 这些提供商总部都在美国. 他们尽早进入云计算领域为他们提供了先发优势, 尤其是在建立大型用户网络以及海量数据存储和处理能力方面. 主要的云计算产业所需的绝对投资水平很高; 一个数据中心集群的成本可能超过五亿美元.
    \switchcolumn
    On the supply side, the cloud economy is currently dominated by a few very large cloud service providers, almost all headquartered in the United States. Their early entry into cloud computing gave them firstmover advantages, not least in terms of building large networks of users and massive data storage and processing capacity. The absolute levels of investment required for major cloud-computing estates are very high; it can cost more than half a billion dollars for a cluster of data centres.

    \switchcolumn*
    尽管云服务的提供仍然还是在掌控在少数科技公司手中, 仍有些对国家或地区企业有利的因素. 比如, 有的政府和企业被要求(根据法律或是公司政策)亦或是出于对数据安全或地理政治的考虑, 只能将其数据放在国内的云服务商中. 迄今大公司和政府对私有云情有独钟, 虽然价格稍贵于公共云, 但数据和服务的安全性都有所保证. 近期有关于数据监管的国际宣传更让大公司倾向选择私有云.
    \switchcolumn
    While the cloud service provider market is likely to continue to be dominated by a small number of global IT businesses, some factors may favour national or regional players. Some Governments and enterprises are required (by law or corporate policy) to locate their data within national jurisdictions, or prefer to do so for security or geopolitical reasons. Large corporations and Governments have hitherto shown a preference for private over public clouds, eschewing some cost saving to ensure a greater sense of security and control over their data and services. Recent international publicity concerning data surveillance may have reinforced such a preference. 
    
    \switchcolumn*
    \paragraph{} \quad {\bfseries 发展中国家云的使用对云经济的提供商和用户有潜在影响.}
    \switchcolumn
    \paragraph{}
    {\bfseries Cloud adoption in developing countries has potential implications for both the supply and the user side of the cloud economy.} 
    
    \switchcolumn*
    云服务为发展中国家能带来的最有益的活动和潜在的供应机会有如下: (a) 数据中心与相关云服务的提供. (b) 为本地用户群(本地企业和居民)提供的云服务的发展与供应. (c) 云聚合, 系统集成和经济业务相关的服务. 除了上述明确与基于云的活动有关, 云服务对于本国的通讯业务(电信运营商和网络服务提供商)是一个极好的发展机会, 他们可以从增长的数据流量获益. 尽管国际云服务提供商已经占尽先机, 但在发展中国家, 由于对私有云的需求增加, 国家数据保护法规或是公司要求数据在国内监管, 国外网络无法稳定访问等因素, 其本地或区域数据中心仍有发展空间.
    \switchcolumn
    The most significant activities and potential supply opportunities for enterprises in developing countries are concerned with: (a) data-centre and related cloud provision; (b) the development and provision of local cloud services for groups of customers, including local businesses and individual citizens; (c) cloud aggregation, system integration, brokerage and related services. In addition to these explicitly cloud-based areas of activity, opportunities exist for national communications businesses (telecommunications operators and ISPs) which can gain from increased data traffic using their networks. Despite the advantages of global cloud service providers, there are some factors that offer scope for local or regional data centres to expand in developing countries, such as growing demand for private cloud solutions, national data-protection laws or corporate policies requiring data to be kept within national jurisdictions, and high costs of or unreliable international broadband connectivity. 

    \switchcolumn*
    在发展中国家, 有很多人使用免费的云服务(网络邮件, 在线社交网络). 几乎在所有国家, 尤其是网络使用水平高和云服务可用度高的国家, 皆是如此. 最受欢迎的云服务应用基本都是全球性应用. 在云服务刚开始使用的低收入国家中, 基础设施即服务经常是第一个出现的云服务. 随着基础设施状况的改善以及中小型企业部门的扩大, 发展中国家的软件即服务市场将变得越来越重要, 并且最终将占据发达国家的主导地位. 在发达国家, 软件即服务已经占据主导位置.
    \switchcolumn
    There has been extensive adoption by individuals in developing countries of free cloud services such as webmail and online social networks. This is true in almost all countries, in particular those with higher levels of Internet use and cloud readiness. The most popular cloud-based applications are generally those provided at a global level. In low-income countries at a nascent stage of cloud readiness, IaaS is often the first category of cloud services to emerge. As the infrastructure situation improves and if the SME sector expands, the market for SaaS in developing countries will become more important and eventually dominant as it already is in developed countries.

\end{paracol}

\section{LIST OF ABBREVIATIONS(缩略词)}
\begin{longtable}{m{2cm}m{3cm}m{3cm}m{4cm}}
    \hline
    {\bfseries 简称} & {\bfseries 全称} & {\bfseries 翻译} & {\bfseries 备注}\\[5pt] \hline
    3G & third generation (refers to mobile phones) & 第三代移动通信技术 & \\
    \hline
    ACP & The African, Caribbean and Pacific Group of States & 太平洋地区国家集团 & \\
    \hline
    ADSL & asymmetric digital subscriber line & 非对称数字用户线路 & \\  
    \hline
    BPaaS & business process as a service & 业务流程即服务 & 云计算交付模式之一 \\
    \hline
    BPO & business process outsourcing & 业务流程外包 & 将职能部门的全部功能都转移给供应商 \\
    \hline
    BRICS & Brazil, the Russian Federation, India, China and South Africa & 金砖五国 & 巴西, 俄罗斯, 印度, 中国, 南非\\
    \hline
    CaaS & communication as a service & 通信即服务 & \\
    \hline
    CERT & computer emergency response team & 计算机应急响应小组 &\\
    \hline
    CIO & chief information officer & 首席信息官 & \\
    \hline
    CPC & Central Product Classification & 中心产品目录 & \\
    \hline
    CRM & client customer relationship management & 客户关系管理 &\\
    \hline
    --- & data traffic & 数据流量 & \\
    \hline
    ERP & enterprise resource planning & 企业资源规划 & \\
    \hline
    GATS & General Agreement on Trade in Services & 服务贸易总协议 & \\
    \hline
    GDP & gross domestic product & 国内生产总值 & \\
    \hline
    IaaS & infrastructure as a service & 基础设施即服务 & 消费者使用处理、储存、网络以及各种基础运算资源,部署与执行操作系统或应用程序等各种软件\\
    \hline
    ICT & information and communication technology & 信息与通信技术 & \\
    \hline
    IDC & International Data Corporation & 国际数据公司 & \\
    \hline
    IP & Internet protocol & 网际协议 & \\
    \hline
    ISO & International Organization for Standardization & 国际标准组织 & \\
    \hline
    ISP & Internet service provider & 互联网服务供应商 & \\
    \hline
    IT & information technology & 信息技术 & \\
    \hline
    ITU & International Telecommunication Union & (联合国)国际电信联盟 &\\
    \hline
    ITU-T & ITU Telecommunication Standardization Sector & 国际电信联盟电信标准部 & \\
    \hline
    IXP & Internet exchange point & 互联网交换中心 & \\
    \hline
    LDC & least developed country & 最不发达国家 & \\
    \hline
    LTE & long-term evolution & 长期演进技术 & \\
    \hline
    m2m & mothers-2-mothers organization & 艾滋母亲互助协会 & \\
    \hline
    NCIA & National Computing and Information Agency (Republic of Korea) & 综合计算机中心 & \\
    \hline
    NDC & national data centre & 国家数据中心 & \\
    \hline
    NGO & non-governmental organization & 非政府组织 & \\
    \hline
    NIST & National Institute of Standards and Technology & 国家标准与技术协会 & \\
    \hline
    NTT & Nippon Telegraph and Telephone Corporation & 日本电报电话公司 & \\
    \hline
    OECD & Organization for Economic Cooperation and Development & 经济合作与发展组织 & \\
    \hline
    PaaS & platform as a service & 平台即服务 & \\
    \hline
    PUE & power usage effectiveness & 能源使用效率 & \\
    \hline
    QoS & quality of service & 服务质量 & \\
    \hline
    RTT & round-trip time & 往返时延 & \\
    \hline
    SaaS & software as a service & 软件即服务 & \\
    \hline
    SLA & service level agreement & 服务级别协议 & \\ 
    \hline
    SME & small and medium-sized enterprise & 中小型企业 & \\
    \hline
    SMS & short message service & 短信服务 & \\
    \hline
    TDF & transborder data flow & 跨境数据流 & \\ 
    \hline
    TNC & transnational corporation & 跨国公司 & \\
    \hline
    UNCTAD & United Nations Conference on Trade and Development & 联合国贸易与发展会议 & \\
    \hline
    XaaS & x as a service & 一切皆服务 & \\
    \hline
\end{longtable}

\section{反思}

根据自己的理解进, 把术语表内容进行了一定的删除与添加.

记录一些翻译中遇到的问题.

\begin{itemize}
    \item \emph{Fee-generated} 产生费用的.
    \item \emph{global ICT sector} 将 sector 译为行业. 
    \item \emph{data traffic} 数据流量.
    \item \emph{firstmover advantages} 先发优势.
    \item \emph{not least} 尤其.
    \item \emph{telecommunications operators} 电信供应商
\end{itemize}

难翻译的句子.

\begin{enumerate}
    \item In addition to these explicitly cloud-based areas of activity, opportunities exist for national communications businesses (telecommunications operators and ISPs) which can gain from increased data traffic using their networks. `in addition to ' 翻译为 `除了..., 还有'. 所以 `in addition to' 所指带的是上文所述云服务的一些活动和给予, 然后笔锋一转, 开始叙述云服务以外能为本地所带来的益处.  
\end{enumerate}

对比官方译文

\begin{enumerate}
    \item This is still very small compared with the revenue of the global ICT sector, which was es-timated at about \$4 trillion in 2011. 官方译文 `据估计, 全球信通技术部门2011年的收入约为4万亿美元'. 我误将 `revenue' 翻成了 `估值'.
    \item eschewing some cost saving 官方译文 `不惜成本', 我翻译为 `虽然价格稍贵与公共云'. 没有官方翻译的准确.
    \item Recent international publicity concerning data surveillance may have reinforced such a preference. 官方译为 `最近国际上对数据监控问题的曝光可能更加深了这种偏爱', 我在翻译时, 对 `偏爱' 进行了解释, 翻译为 `近期有关于数据监管的国际宣传更让大公司倾向选择私有云.'
    \item , there are some factors that offer scope for local or regional data centres to expand in developing countries,such as growing demand for private cloud solu-tions, national data-protection laws or corporate policies requiring data to be kept within national jurisdictions, and high costs of or unreliable international broadband connectivity. 对于官方译为, 我将 `因素' 后面的修饰作为了 `因素' 的定语.
    \item cloud readiness 官方译文为 `云条件成熟', 我翻译为 `云翻译可用度比较高'. 明显官方译文更加简洁.
\end{enumerate}

\end{document}