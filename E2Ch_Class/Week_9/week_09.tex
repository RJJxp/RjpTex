\documentclass[a4paper, UTF8, 12pt]{article}

\usepackage{ctex}
\usepackage{geometry}
\usepackage{graphicx}
\usepackage{float}
\usepackage{caption}
\usepackage{enumerate}
\usepackage{paracol}
\usepackage{longtable}
\usepackage{array}
\usepackage{multirow}
\usepackage{color}

\usepackage{hyperref}

\hypersetup{
    colorlinks=true,   % false, ,链接黑色, 外有红框
    linkcolor=black, % 目录颜色, 脚注颜色
    filecolor=blue, % 链接本地文件的链接颜色
    urlcolor=cyan, % 网页链接颜色
    anchorcolor=blue,
    citecolor=yellow    % 参考文献颜色
}

\geometry{
    textheight=230mm,
    textwidth=180mm
}

\begin{document}
\title{\Huge 英译汉课程}
\author{\Large 
        第九周 \\[12pt]
        1931991 任家平 \\[12pt]
        同济大学 \\[12pt]
        测绘与地理信息学院}
\date{2019-10-28}

\maketitle


\begin{abstract}
    为了凸显精校部分, 对格式进行调整, 同时对错别字进行改正.英文原文部分, 使用单列模式. 精校部分, 左侧原译文, 右侧精校译文, 修改部分用红色标记, 争议部分用蓝色标记. 删除术语表部分.
\end{abstract}


\thispagestyle{empty}

\newpage
\pagenumbering{Roman}
\tableofcontents
\addtocontents{toc}{\protect\vspace{10pt}}

\newpage
\pagenumbering{arabic}

\section{英文原文}
\paragraph*{}
    \begin{bfseries}
        \large 
        The cloud economy is expanding fast but is still small.
    \end{bfseries}

\paragraph*{}
    There are various estimates of the size of the market for cloud services. Fee-generated revenues from public provision of IaaS, PaaS and SaaS have been forecast to reach somewhere between \$43 billion and \$94 billion by 2015. To this can be added the revenue generated through advertising on cloud-enabled web applications that are available at no cost to the user. Such revenue is currently considerably larger than the fees generated from public cloud provision. Estimates of the value of private cloud services also vary greatly – from about \$5 billion to about \$50 billion. Discrepancies in projections reflect different methodologies, but most forecasts agree that cloud adoption will continue to expand rapidly over the next few years.

\paragraph*{}
    This is still very small compared with the revenue of the global ICT sector, which was estimated at about \$4 trillion in 2011. However, most segments of the ICT sector are in some way affected by cloud computing. The demand for bandwidth will drive telecommunication services revenue, although revenues from voice services could be affected as more people switch to cloud-based voice over Internet protocol applications. Demand for equipment and computer hardware, particularly data servers and network equipment, will rise as more services move to the cloud. 

\paragraph*{}
    This is still very small compared with the revenue of the global ICT sector, which was estimated at about \$4 trillion in 2011. However, most segments of the ICT sector are in some way affected by cloud computing. The demand for bandwidth will drive telecommunication services revenue, although revenues from voice services could be affected as more people switch to cloud-based voice over Internet protocol applications. Demand for equipment and computer hardware, particularly data servers and network equipment, will rise as more services move to the cloud. 

\paragraph*{}
    The shift to the cloud is generating considerable growth in data traffic. During an average minute in 2012, Google received two million search requests, Facebook users shared around 700,000 content items and Twitter sent out 100,0-00 tweets. In 2012, 60 per cent of such cloud traffic on the Internet emanated from Europe and North America. Asia–Pacific was responsible for another third while Latin America and the Middle East and Africa together accounted for only 5 per cent. However, the highest growth rates in the next few years are expected in the Middle East and Africa. 

\paragraph*{}
    On the supply side, the cloud economy is currently dominated by a few very large cloud service providers, almost all headquartered in the United States. Their early entry into cloud computing gave them firstmover advantages, not least in terms of building large networks of users and massive data storage and processing capacity. The absolute levels of investment required for major cloud-computing estates are very high; it can cost more than half a billion dollars for a cluster of data centres.

\paragraph*{}
    While the cloud service provider market is likely to continue to be dominated by a small number of global IT businesses, some factors may favour national or regional players. Some Governments and enterprises are required (by law or corporate policy) to locate their data within national jurisdictions, or prefer to do so for security or geopolitical reasons. Large corporations and Governments have hitherto shown a preference for private over public clouds, eschewing some cost saving to ensure a greater sense of security and control over their data and services. Recent international publicity concerning data surveillance may have reinforced such a preference. 

\paragraph*{}
    \begin{bfseries}
        \large 
        Cloud adoption in developing countries has potential implications for both the supply and the user side of the cloud economy.
    \end{bfseries}

\paragraph*{}
    The most significant activities and potential supply opportunities for enterprises in developing countries are concerned with: (a) data-centre and related cloud provision; (b) the development and provision of local cloud services for groups of customers, including local businesses and individual citizens; (c) cloud aggregation, system integration, brokerage and related services. In addition to these explicitly cloud-based areas of activity, opportunities exist for national communications businesses (telecommunications operators and ISPs) which can gain from increased data traffic using their networks. Despite the advantages of global cloud service providers, there are some factors that offer scope for local or regional data centres to expand in developing countries, such as growing demand for private cloud solutions, national data-protection laws or corporate policies requiring data to be kept within national jurisdictions, and high costs of or unreliable international broadband connectivity. 

\paragraph*{}
    There has been extensive adoption by individuals in developing countries of free cloud services such as webmail and online social networks. This is true in almost all countries, in particular those with higher levels of Internet use and cloud readiness. The most popular cloud-based applications are generally those provided at a global level. In low-income countries at a nascent stage of cloud readiness, IaaS is often the first category of cloud services to emerge. As the infrastructure situation improves and if the SME sector expands, the market for SaaS in developing countries will become more important and eventually dominant as it already is in developed countries.


\begin{paracol}{2}[\section{翻译对比}]
    
    \switchcolumn*
    \paragraph{} \quad {\bfseries 云经济增长迅速但规模依然很小}
    \switchcolumn
    {\bfseries 云经济增长迅速但规模依然很小}

    \switchcolumn*
    {\color{red} 对于云服务的市场规模, 有很多预测结果}. 预计在2015年, 提供基础设施即服务, 平台即服务和软件即服务的公共云服务费用产生的营业额大致在430亿到940亿美元之间, 这包括了对用户免费使用的网页云服务应用的广告收入. {\color{red} 其中的广告}的收入远大于公共云服务. 私有云的估值结果差别较大, 为50亿到500亿美元. {\color{blue} 不同的预测方法会有不同的结果}, 但大多数结果表明云服务的使用率会近年来会快速提高.
    \switchcolumn
    {\color{red} 云服务的市场规模有多种预测结果}. 预计在2015年, 提供基础设施即服务, 平台即服务和软件即服务的公共云服务费用产生的营业额大致在430亿到940亿美元之间, 这包括了对用户免费使用的网页云服务应用的广告收入. {\color{red} 广告}的收入远大于公共云服务. 私有云的估值结果差别较大, 为50亿到500亿美元.  {\color{blue} 不同的预测方法会有不同的结果}, 但大多数结果表明云服务的使用率会近年来会快速提高.
    
    \switchcolumn*
    全球信息通讯技术在2011年的产业{\color{red} 收入}大致有4万亿美元, 云服务所占{\color{red} 比例}非常小. 然而, 信息通讯技术的大部分领域都在一定程度上受到了云计算的影响. 对带宽的需求将推动电信服务收入的增长, 随着越来越多的人转而使用基于网络协议的云语音服务, 传统语音服务的收入可能会受到影响. {\color{red} \sout{随着}}越来越多的服务迁移到云中,对设备和计算机硬件(尤其是数据服务器和网络设备)的需求将上升。
    \switchcolumn
    全球信息通讯技术在2011年的产业{\color{red} 估值}大致有4万亿美元, 云服务所占{\color{red} 份额}非常小. 然而, 信息通讯技术的大部分领域都在一定程度上受到了云计算的影响. 对带宽的需求将推动电信服务收入的增长, 随着越来越多的人转而使用基于网络协议的云语音服务, 传统语音服务的收入可能会受到影响. {\color{red} 随着}越来越多的服务迁移到云中,对设备和计算机硬件(尤其是数据服务器和网络设备)的需求将上升。

    \switchcolumn*
    向云的转移正在极大地促进数据流量的增长. 在2012年的平均时间里, 谷歌收到了200万次搜索请求, 脸书用户分享了大约七十万条动态, 而推特发送了十万条推文. 2012年, 互联网上此类云流量的60\%来自欧洲和北美. 亚太地区占了另外三分之一, 而拉丁美洲, 中东和非洲加在一起只占了5\%.但是, 预计未来几年中中东和非洲的增长率最高.
    \switchcolumn
    向云的转移正在极大地促进数据流量的增长. 在2012年的平均时间里, 谷歌收到了200万次搜索请求, 脸书用户分享了大约七十万条动态, 而推特发送了十万条推文. 2012年, 互联网上此类云流量的60\%来自欧洲和北美. 亚太地区占了另外三分之一, 而拉丁美洲, 中东和非洲加在一起只占了5\%.但是, 预计未来几年中中东和非洲的增长率最高.

    \switchcolumn*
    在服务的提供方面, 云经济目前由一些势力强大的云服务提供商垄断, 这些提供商总部都在美国. 他们尽早进入云计算领域为他们提供了先发优势, 尤其是在建立大型用户网络以及海量数据存储和处理能力方面. 主要的云计算产业所需的绝对投资水平很高; 一个数据中心集群的成本可能超过五亿美元.
    \switchcolumn
    在服务的提供方面, 云经济目前由一些势力强大的云服务提供商垄断, 这些提供商总部都在美国. 他们尽早进入云计算领域为他们提供了先发优势, 尤其是在建立大型用户网络以及海量数据存储和处理能力方面. 主要的云计算产业所需的绝对投资水平很高; 一个数据中心集群的成本可能超过五亿美元.

    \switchcolumn*
    尽管云服务的提供仍然还是在掌控在少数科技公司手中, 仍有些对国家或地区企业有利的因素. 比如, 有的政府和企业被要求(根据法律或是公司政策)亦或是出于对数据安全或地理政治的考虑, 只能将其数据放在国内的云服务商中. 迄今大公司和政府对私有云情有独钟, {\color{blue} 虽然价格稍贵于公共云}, 但数据和服务的安全性都有所保证. 近期有关于数据{\color{red} 监控}的国际宣传更让大公司倾向选择私有云.
    \switchcolumn
    尽管云服务的提供仍然还是在掌控在少数科技公司手中, 仍有些对国家或地区企业有利的因素. 比如, 有的政府和企业被要求(根据法律或是公司政策)亦或是出于对数据安全或地理政治的考虑, 只能将其数据放在国内的云服务商中. 迄今大公司和政府对私有云情有独钟, {\color{blue} 虽然价格稍贵于公共云}, 但数据和服务的安全性都有所保证. 近期有关于数据{\color{red} 监管}的国际宣传更让大公司倾向选择私有云.
    
    \switchcolumn*
    \paragraph{} \quad {\bfseries 发展中国家云的使用对云经济的提供商和用户有潜在影响.}
    \switchcolumn
    \paragraph{}
    {\bfseries 发展中国家云的使用对云经济的提供商和用户有潜在影响.} 
    
    \switchcolumn*
    云服务为发展中国家能带来的最有益的活动和潜在的供应机会有如下: (a) 数据中心与相关云服务的提供. (b) 为本地用户群(本地企业和居民)提供的云服务的发展与供应. (c) 云聚合, 系统集成和经济业务相关的服务. 除了上述明确与基于云的活动有关, 云服务对于本国的通讯业务(电信运营商和网络服务提供商)是一个极好的发展机会, 他们可以从增长的数据流量获益. {\color{blue} 尽管国际云服务提供商已经占尽先机, 但在发展中国家, 由于对私有云的需求增加, 国家数据保护法规或是公司要求数据在国内监管, 国外网络无法稳定访问等因素, 其本地或区域数据中心仍有发展空间.}
    \switchcolumn
    云服务为发展中国家能带来的最有益的活动和潜在的供应机会有如下: (a) 数据中心与相关云服务的提供. (b) 为本地用户群(本地企业和居民)提供的云服务的发展与供应. (c) 云聚合, 系统集成和经济业务相关的服务. 除了上述明确与基于云的活动有关, 云服务对于本国的通讯业务(电信运营商和网络服务提供商)是一个极好的发展机会, 他们可以从增长的数据流量获益. {\color{blue} 尽管国际云服务提供商已经占尽先机, 但在发展中国家, 由于对私有云的需求增加, 国家数据保护法规或是公司要求数据在国内监管, 国外网络无法稳定访问等因素, 其本地或区域数据中心仍有发展空间.}

    \switchcolumn*
    在发展中国家, 有很多人使用免费的云服务(网络邮件, 在线社交网络). 几乎在所有国家, 尤其是网络使用水平高和{\color{red} 云条件成熟}的国家, 皆是如此. 最受欢迎的云服务应用基本都是全球性应用. 在云服务刚开始使用的低收入国家中, 基础设施即服务经常是第一个出现的云服务. 随着基础设施状况的改善以及中小型企业部门的扩大, 发展中国家的软件即服务市场将变得越来越重要, 并且最终将占据发达国家的主导地位. 在发达国家, 软件即服务已经占据主导位置.
    \switchcolumn
    在发展中国家, 有很多人使用免费的云服务(网络邮件, 在线社交网络). 几乎在所有国家, 尤其是网络使用水平高和{\color{red} 云服务可用度高}的国家, 皆是如此. 最受欢迎的云服务应用基本都是全球性应用. 在云服务刚开始使用的低收入国家中, 基础设施即服务经常是第一个出现的云服务. 随着基础设施状况的改善以及中小型企业部门的扩大, 发展中国家的软件即服务市场将变得越来越重要, 并且最终将占据发达国家的主导地位. 在发达国家, 软件即服务已经占据主导位置.

\end{paracol}

\section{反思}
第九周的课程学到了两点东西, `情感色彩' 和 `侧重点'. 引人深思的是, 我翻译究竟是给谁看? 


\end{document}