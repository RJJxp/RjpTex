\documentclass[a4paper, UTF8, 12pt]{article}

\usepackage{ctex}
\usepackage{geometry}
\usepackage{graphicx}
\usepackage{float}
\usepackage{caption}
\usepackage{enumerate}
\usepackage{paracol}
\usepackage{longtable}
\usepackage{array}
\usepackage{multirow}

\usepackage{hyperref}

\hypersetup{
    colorlinks=true,   % false, ,链接黑色, 外有红框
    linkcolor=black, % 目录颜色, 脚注颜色
    filecolor=blue, % 链接本地文件的链接颜色
    urlcolor=cyan, % 网页链接颜色
    anchorcolor=cyan,   % 锚点颜色
    citecolor=yellow    % 参考文献颜色
}

\geometry{
    textheight=230mm,
    textwidth=180mm
}

\begin{document}
\title{\Huge 英译汉课程}
\author{\Large 
        第八周 \\[12pt]
        1931991 任家平 \\[12pt]
        同济大学 \\[12pt]
        测绘与地理信息学院}
\date{2019-11-04}

\maketitle

\thispagestyle{empty}

\newpage
\pagenumbering{Roman}
\tableofcontents
\addtocontents{toc}{\protect\vspace{10pt}}

\newpage
\pagenumbering{arabic}

\begin{paracol}{2}[\section{翻译}]

    \switchcolumn*
    \paragraph{} \quad {\bfseries 发展中国家云的使用对云经济的提供商和用户有潜在影响.}
    \switchcolumn
    \paragraph{}
    {\bfseries Cloud adoption in developing countries has potential implications for both the supply and the user side of the cloud economy.} 
    
    \switchcolumn*
    云服务为发展中国家能带来的最有益的活动和潜在的供应机会有如下: (a) 数据中心与相关云服务的提供. (b) 为本地用户群(本地企业和居民)提供的云服务的发展与供应. (c) 云聚合, 系统集成和经济业务相关的服务. 除了上述明确与基于云的活动有关, 云服务对于本国的通讯业务(电信运营商和网络服务提供商)是一个极好的发展机会, 他们可以从增长的数据流量获益. 尽管国际云服务提供商已经占尽先机, 但在发展中国家, 由于对私有云的需求增加, 国家数据保护法规或是公司要求数据在国内监管, 国外网络无法稳定访问等因素, 其本地或区域数据中心仍有发展空间.
    \switchcolumn
    The most significant activities and potential supply opportunities for enterprises in developing countries are concerned with: (a) data-centre and related cloud provision; (b) the development and provision of local cloud services for groups of customers, including local businesses and individual citizens; (c) cloud aggregation, system integration, brokerage and related services. In addition to these explicitly cloud-based areas of activity, opportunities exist for national communications businesses (telecommunications operators and ISPs) which can gain from increased data traffic using their networks. Despite the advantages of global cloud service providers, there are some factors that offer scope for local or regional data centres to expand in developing countries, such as growing demand for private cloud solutions, national data-protection laws or corporate policies requiring data to be kept within national jurisdictions, and high costs of or unreliable international broadband connectivity. 

    \switchcolumn*
    在发展中国家, 有很多人使用免费的云服务(网络邮件, 在线社交网络). 几乎在所有国家, 尤其是网络使用水平高和云服务可用度高的国家, 皆是如此. 最受欢迎的云服务应用基本都是全球性应用. 在云服务刚开始使用的低收入国家中, 基础设施即服务经常是第一个出现的云服务. 随着基础设施状况的改善以及中小型企业部门的扩大, 发展中国家的软件即服务市场将变得越来越重要, 并且最终将占据发达国家的主导地位. 在发达国家, 软件即服务已经占据主导位置.
    \switchcolumn
    There has been extensive adoption by individuals in developing countries of free cloud services such as webmail and online social networks. This is true in almost all countries, in particular those with higher levels of Internet use and cloud readiness. The most popular cloud-based applications are generally those provided at a global level. In low-income countries at a nascent stage of cloud readiness, IaaS is often the first category of cloud services to emerge. As the infrastructure situation improves and if the SME sector expands, the market for SaaS in developing countries will become more important and eventually dominant as it already is in developed countries.

    \switchcolumn*
    外国子公司在发展中国家提供的云服务扩大了其母公司的全球网络业务. 同时, 出于安全考虑, 发展中国家政府也谨慎地将自己的服务转向云端. 其中有些政府推出了系统性的云战略作为其国家整体信息通讯产业发展战略的一部分或与其并驾齐驱. 相比于公共云, 政府部门和大企业一直更偏向使用私有云. 尽管有很多人支持云服务, 但由于政策原因, 政府会对国内企业使用的云服务进行一定的比例限制.
    \switchcolumn
    \hypertarget{par:01}{}
    Foreign affiliates in developing countries make extensive use of the cloud as part of their parent companies’ global networks. With some wariness, Governments in developing countries are also moving towards the cloud. Some are developing systematic cloud strategies, as part of broader ICT strategies or sometimes alongside these. Where government departments and larger corporations are concerned, there is so far a general preference for private over public cloud approaches. There is planned adoption of the cloud in domestic enterprises, although less extensive than anticipated by cloud advocates. 
    
    \switchcolumn*
    云计算在发展中国家发展过于迅速, 由于缺少资料, 通过严格循证评估其影响力较为困难. 企业政府和其他组织在通过云提升自己的管理和提供服务的同时, 也要注意云服务潜在的风险. 只有当使用云服务优势巨大, 并且所带来风险可被优势适当缓冲时, 他们才会使用云服务, 将其数据和服务迁移到云端. 在本文的所给建议中, 无论是公共云或是私有云, 都应考虑隐私和数据安全性.
    \switchcolumn
    \hypertarget{par:02}{}
    Experience of cloud computing in developing countries is too recent for there to be a strongly established evidence base on which to assess impacts. Businesses, Governments and other organizations should carefully examine the potential for cloud services to improve their management and service delivery. They should only migrate data and services to the cloud when they are confident that the cloud offers significant benefits and that attendant risks can be appropriately mitigated. Both public and private cloud solutions should be considered in this context, taking into account implications for data security and privacy. 


    \switchcolumn*
    \paragraph{} \quad {\bfseries 翻译}
    \switchcolumn
    {\bfseries Infrastructure deficiencies seriously ha-mper the uptake of and benefits from cloud computing in many developing countries.}

    \switchcolumn*
    翻译
    \switchcolumn
    For several reasons, the options for cloud adoption available in low- and middle-income countries look different from those in more advanced economies. Critical factors relate, among other things, to the availability and quality of cloud-related infrastructure, cost considerations and inadequate legal and regulatory frameworks to address data protection and privacy concerns. 
    
    \switchcolumn*
    翻译
    \switchcolumn
    As regards access to and availability of cloud-related infrastructure, and despite significant improvements in broadband connectivity in many developing economies, the gap between developed and developing countries keeps widening. Average fixed broadband penetration is now more than 28 subscriptions per 100 people in developed economies, 6 in developing countries and only 0.2 in the least developed countries (LDCs). In the case of mobile broadband, the gap is also significant. The average number of subscriptions in 2012 was about 67 per 100 people in developed countries, 14 in developing countries and below 2 in the LDCs. 

    \switchcolumn*
    翻译
    \switchcolumn
    In addition, in most low-income countries, mobile broadband networks are characterized by low speed and high latency and are therefore currently not ideal for cloud service provision, especially of the more advanced kinds. The net value of cloud-based solutions will be lower in countries with a heightened risk of communication- and power-network outages. The lack of supporting infrastructure, such as Internet exchange points (IXPs), reliable and inexpensive electricity and robust fibre-optic backbones also affect the deployment of national data centres. Indeed, as much as 85 per cent of data centres offering colocation services are in developed economies. This “data centre divide” is reflected in the availability of servers; whereas there were in 2011 more than 1,000 secure data servers per million inhabitants in highincome economies, there was only one such server per million inhabitants in LDCs.

    \switchcolumn*
    翻译
    \switchcolumn
    The cost of communication remains another critical obstacle for adoption of cloud services in many developing countries. The fees paid to cloud service providers and for broadband access and usage, charges by the ISP and the hardware and software costs incurred are likely to form a much higher proportion of the total costs of cloud provisioning than in advanced economies. The combination of few national data centres and high costs of international broadband communications further weighs on the net value of relying on cloud solutions. 

    
    

    \switchcolumn*
    \paragraph{} \quad {\bfseries 翻译}
    \switchcolumn
    {\bfseries The cloud raises legal and regulatory challenges, especially concerning data protection and privacy.}


    \switchcolumn*
    翻译
    \switchcolumn
    The rapid emergence of cloud computing has raised concerns about its legal and regulatory implications. Issues of data protection and security are among the concerns most frequently mentioned by potential cloud customers in both developed and developing countries. Such concerns have intensified following the disclosure in 2013 of national surveillance programmes and reports on access by law-enforcement agencies to data hosted by global cloud service providers. Governments need to protect national interests and their citizens; service providers require a stable framework to facilitate innovation and investment; and users require assurance and trust to encourage the take-up of such services. Policy responses may range from a do-nothing attitude to the adoption of cloud-specific laws. 

    \switchcolumn*
    翻译
    \switchcolumn
    Public law is essential to secure the basic rights of end users. While there is no imperative to develop specific laws or regulations on cloud computing, areas requiring reform are relatively clear: privacy, data protection, information security and cybercrime. For Governments of developing countries, it is essential that appropriate laws and regulations are adopted and enforced in these areas. As of 2013, there were 99 countries with data-privacy laws. As far as is known, Mexico is the only country which has adopted cloudspecific provisions in relation to data protection. There is no international harmonized privacy framework regulating data transfers across borders, but developing countries could benefit from implementing strong domestic-privacy regim-es.

    \switchcolumn*
    翻译
    \switchcolumn
    In addition to public law, contractual agreements between cloud service providers and cloud service customers also greatly impact on the operation and effects of the cloud economy. In some circumstances, regulatory intervention in the freedom to contract may be necessary to protect the public interest. The placement of data in the cloud may require regulatory intervention to address concerns related to personal privacy, commercial secrecy or national security. For example, within data protection laws, imposing minimum responsibilities on the cloud service providers – to ensure the security of customer data and to notify its customers if there is a security breach – could help to provide greater transparency about vulnerabilities and to enable mitigation in a timely manner. 

    \switchcolumn*
    翻译
    \switchcolumn
    Where there are apprehensions on relying hea-vily on cloud services offered by providers based in a foreign jurisdiction, it may be difficult to address this market reality through regulatory intervention. An alternative policy response may be to encourage the establishment of domestic cloud services, either by offering foreign investors a favorable environment to invest in the building of local infrastructure (such as data centres) or encouraging domestic enterprises to enter the supply side of the cloud economy. Whereas such measures may involve regulatory components, such as imposing “localization” requirements, they would be designed to facilitate the provision of cloud services rather than to constrain them. Several Governments of developing countries are building government clouds to serve the needs of the Government itself and sometimes others. In Europe, there have been calls for the development of a secure European cloud and some national cloud initiatives have been launched to offer an alternative source of cloud service provision. 

    \switchcolumn*
    \paragraph{} \quad {\bfseries 翻译}
    \switchcolumn
    {\bfseries Governments should facilitate benefits from the cloud economy but be aware of pitfalls.}

    \switchcolumn*
    翻译
    \switchcolumn
    Although cloud adoption is still at a nascent stage in developing countries, policymakers should waste no time in enhancing their understanding of how it may affect their economies and societies, in order to be able to make informed policy decisions. Government policies should be based on an assessment of the pros and cons of cloud solutions and be rooted in a thorough understanding of existing ICT and cloud use within countries. Governments need to recognize the diversity of business models and services within the cloud, the multiplicity of customers of cloud services, and the complexity of the cloud economy ecosystem. In view of its relevance for both public service delivery and business competitiveness, it is important to integrate any cloud strategy in the overall national development plan, and to plan for its execution, monitoring and evaluation. Policy approaches should be tailored to the circumstances of individual economies, and be consistent with the overall strategic framework for national economic development and for leveraging ICTs.

    \switchcolumn*
    翻译
    \switchcolumn
    On the whole, Governments should broadly welcome and support the development of a cloud economy and the adoption of cloud services. In principle, there is no general case for government policy and regulation to discourage migration towards the cloud. Rather, policies and regulatory approaches should seek to create an enabling framework that supports firms and organizations that wish to migrate data and services to the cloud so that they may do so easily and safely. However, this does not mean that cloudbased solutions are always preferable to alternative approaches. In addition, there are multiple ways of making use of cloud technology – using public, private or hybrid clouds at national, regional or global levels. Governments should seek to facilitate those approaches that seem most likely to deliver wider economic benefits in their particular context. 

    \switchcolumn*
    翻译
    \switchcolumn
    A number of steps could be considered by Governments that wish to translate the potential of the cloud into tangible development gains. In terms of scope, at the national level policymaking would be advised to consider measures related to the following areas:

    \begin{itemize}
        \switchcolumn*
        \item 翻译
        \switchcolumn
        \item {\bfseries Assess the cloud readiness of	the	country.} Governments should start by carefully assessing the current situation in their countries, to identify bottlenecks and weaknesses that need to be addressed if the cloud is to be effectively exploited, and clarify what kind of cloud solutions are most propitious.  
        
        \switchcolumn*
        \item 翻译
        \switchcolumn
        \item {\bfseries Develop a national	cloud strategy.} Bas-ed on the readiness assessment, a national cloud strategy could be drafted either as a stand-alone policy document or as an integral part of the national ICT strategy. 
        
        \switchcolumn*
        \item 翻译
        \switchcolumn
        \item {\bfseries Address the infrastructure challenge.} This would involve measures to improve the provision of reliable and affordable broadband infrastructure and to monitor regularly the quality of broadband services. Effective communications regulations are here of the essence. Attention should also be given to the role of IXPs and the provision of electricity.
        
        \switchcolumn*
        \item 翻译
        \switchcolumn
        \item {\bfseries Address relevant legal	and	regulatory	issues related to cloud	adoption to	ensure that	cloud	service	users’ interests are properly protected.} Key areas include the location of data, e-transactions and cybercrime. Efforts should be made to reflect international best practice in the development of new legislation. 
        
        \switchcolumn*
        \item 翻译
        \switchcolumn
        \item {\bfseries Map opportunities in the supply side of the	cloud economy.} Three key areas deserve particular attention: the development of national data centres, the potential for cloud aggregation services, and the development of new cloud services. 

        \switchcolumn*
        \item 翻译
        \switchcolumn
        \item {\bfseries Address the need for human resources.} Skill areas that are likely to become increasingly important include those related to the IT and software skills needed to manage the migration and integration of cloud services; management and organizational skills to handle the reorganization and reengineer-ing of business processes; and legal and procurement skills. 
        
        \switchcolumn*
        \item 翻译
        \switchcolumn
        \item {\bfseries Government use of cloud services.} Giv-en their important role in the information economy in many developing countries, the role of Governments should be explored with regard to the establishment of national data centres, e-government systems and related public procurement. 
    
    \end{itemize}

    \switchcolumn*
    \paragraph{} \quad {\bfseries 翻译}
    \switchcolumn
    {\bfseries Development partners should work with Governments in responding to the cloud economy.}

    \switchcolumn*
    翻译
    \switchcolumn
    Addressing the many challenges that developing countries face in seeking to benefit from the evolving cloud economy will require both expertise in various fields and financial resources. Development partners could help in that respect, by ensuring that cloudrelated development challenges are incorporated in their agendas to reduce the risk that the move towards the cloud economy may result in a widening of the digital divide. They may also provide support at the country level in contributing to financing broadband infrastructure, establishing appropriate legal and regulatory frameworks, and building capacity in relevant areas.

    \switchcolumn*
    翻译
    \switchcolumn
    International agencies could facilitate this assistance through some of their existing activities. UNCTAD and other international organizations can, for example, facilitate an exchange of experiences with regard to the policy challenges that developing countries face to derive benefits from the cloud economy and avoid pitfalls. 

    \switchcolumn*
    翻译
    \switchcolumn
    Another key area in which development partners can play a role concerns international standards for cloud services, which are essential to facilitate interoperability and to help customers understand what they are purchasing. Standardization forums should consider how to engage developing countries and their users to ensure that their specific needs and requirements are addressed. More research is also needed in a number of areas to allow for a more comprehensive assessment of the impact of different forms of cloud adoption. As the evidence base expands, it will become feasible to assess macroeconomic implications for economic growth, employment, productivity and trade. 

    \switchcolumn*
    翻译
    \switchcolumn
    As with other ICT areas, the pace of change in cloud technology and markets is rapid. The experiences described in this report relate to present circumstances. The nature of cloud services and of the cloud economy will continue to develop fast, and may be very different in five years’ time. Governments, businesses and development partners need to bear these changes in mind, and to re-evaluate their policies and strategies concerning the cloud regularly to ensure that they continue to maximize potential benefits and minimize potential risks to their citizens, businesses and customers.

\end{paracol}

\section{术语表}
\begin{longtable}{m{2cm}m{3cm}m{3cm}m{4cm}}
    \hline
    {\bfseries 简称} & {\bfseries 全称} & {\bfseries 翻译} & {\bfseries 备注}\\[5pt] \hline
    BPaaS & business process as a service & 业务流程即服务 & 云计算交付模式之一 \\
    \hline
    BPO & business process outsourcing & 业务流程外包 & 将职能部门的全部功能都转移给供应商 \\
    \hline
    BRICS & Brazil, the Russian Federation, India, China and South Africa & 金砖五国 & 巴西, 俄罗斯, 印度, 中国, 南非\\
    \hline
    CaaS & communication as a service & 通信即服务 & \\
    \hline
    CIO & chief information officer & 首席信息官 & \\
    \hline
    --- & data traffic & 数据流量 & \\
    \hline
    GDP & gross domestic product & 国内生产总值 & \\
    \hline
    IaaS & infrastructure as a service & 基础设施即服务 & 消费者使用处理、储存、网络以及各种基础运算资源,部署与执行操作系统或应用程序等各种软件\\
    \hline
    ICT & information and communication technology & 信息与通信技术 & \\
    \hline
    IP & Internet protocol & 网际协议 & \\
    \hline
    ISO & International Organization for Standardization & 国际标准组织 & \\
    \hline
    ISP & Internet service provider & 互联网服务供应商 & \\
    \hline
    IXP & Internet exchange point & 互联网交换中心 & \\
    \hline
    NCIA & National Computing and Information Agency (Republic of Korea) & 综合计算机中心 & \\
    \hline
    NDC & national data centre & 国家数据中心 & \\
    \hline
    NGO & non-governmental organization & 非政府组织 & \\
    \hline
    OECD & Organization for Economic Cooperation and Development & 经济合作与发展组织 & \\
    \hline
    PaaS & platform as a service & 平台即服务 & \\
    \hline
    PUE & power usage effectiveness & 能源使用效率 & \\
    \hline
    RTT & round-trip time & 往返时延 & \\
    \hline
    SaaS & software as a service & 软件即服务 & \\
    \hline
    SLA & service level agreement & 服务级别协议 & \\ 
    \hline
    SME & small and medium-sized enterprise & 中小型企业 & \\
    \hline
    TDF & transborder data flow & 跨境数据流 & \\ 
    \hline
    UNCTAD & United Nations Conference on Trade and Development & 联合国贸易与发展会议 & \\
    \hline
    XaaS & x as a service & 一切皆服务 & \\
    \hline
\end{longtable}

\section{反思}
根据自己的理解, 把术语表内容进行了一定的删除与添加.

记录一些翻译中遇到的问题.

\begin{itemize}
    \item \emph{planned adoption} `planned' 直译为 `按计划的', 但在此可以将 `planned' 单独拿出来译为 `受政策调控或是由于政策原因' 比较恰当. \hyperlink{par:01}{\underline{\emph 原文}}
    \item \emph{strongly established evidence base} 在维基百科上查找到 `Evidence-based practice' 翻译为 `循证实践'. 所以将 `evidence base' 翻译为动词, `循证'. 整个翻译为 `严格循证'. \hyperlink{par:02}{\underline{\emph 原文}}
   
\end{itemize}

对一些翻译的想法.

\begin{enumerate}
    \item \emph{... sometimes alongside these.} 一开始不明白什么意思, 主要是 `these' 的指代不是很清楚, 且 `or' 和谁并列也不清楚. 但想了想, `these' 指代的应该是前面的 `ICT strategies'. `or' 是和 `part of' 并列. 因此翻译为 `并驾齐驱'. \hyperlink{par:01}{\underline{\emph 原文}}
    \item  \item \emph{... examine the potential ...} 这里的 `potential' 指的更多是 `潜在的风险'. \hyperlink{par:02}{\underline{\emph 原文}}
    \item \emph{... attendant risks can be appropriately mitigated.} `be mitigated' 的理解是, 由于云计算的优势巨大, 所以相比之下, 所带来的风险可以被适当的降低. \hyperlink{par:02}{\underline{\emph 原文}}
    \item \emph{Both public and private cloud solutions should be considered in this context, taking into account implications for data security and privacy.} 完全没按照原文的语序来翻译. 将 `solutions' 和 `considered in this context' 联系起来, 译为 `根据本文所提出的建议', `both' 译为 `无论'. \hyperlink{par:02}{\underline{\emph 原文}}
\end{enumerate}

对比官方译文

\begin{enumerate}
    \item This is still very small compared with the revenue of the global ICT sector, which was es-timated at about \$4 trillion in 2011. 官方译文 `据估计, 全球信通技术部门2011年的收入约为4万亿美元'. 我误将 `revenue' 翻成了 `估值'.
\end{enumerate}

\end{document}