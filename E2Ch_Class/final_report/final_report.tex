\documentclass[a4paper, 12pt, UTF8]{article}

\usepackage{ctex}

\begin{document}
\title{\Huge E2CN 期末总结}
\author{\Large 
        1931991 任家平 \\[12pt]
        同济大学 \\[12pt]
        测绘与地理信息学院}
\date{2019-12-16}
\maketitle
\thispagestyle{empty}

\newpage
\pagenumbering{Roman}
\tableofcontents

\newpage
\pagenumbering{arabic}
\section{感想总结}
简单的写几句对于翻译与该课程的所思所想.

生活中接触到翻译的地方还挺多. 美剧日剧等字幕组作品, 翻译大佬Fall\_Art的没品笑话, 英文书(线性代数, C++, 机器人路径规划等), 经济学人等杂志. 课堂上三个翻译内容有经济学人2014年一篇文章, 2013信息经济报告和专业性很强的社科文章. 

对于经济学人杂志, 开学`立志'每周读一篇算是很打脸. 理想很丰满, 现实很骨感. 既要忙着手中的活, 还要坚持锻炼, 时间所剩无几. 不过幸好英语作业写的认真, 虽不一定对, 但也算是等效于没有英语课坚持读杂志了, 正好课上也算是对自己作业的一种纠错. 

杂志之类算是通识性的英语翻译. 相比之下, 信息经济报告就是专业性的翻译. 虽然老师经常会在课上说, 翻译最重要的是准确. 但我更偏执地认为, 这是倾向对翻译从业者的要求. 信息经济报告涉及到很多计算机方面的词汇, 并非专业词汇, 对于报告中出现类似专业词汇的引申, 如 `thin client interface' 译为 `瘦用户界面', 根据上下文, 意思绝非如此. 这说明译者专业知识的缺失, 对此, 我觉得不能单打独斗, 更不能自己踩坑, 遇到问题直接找顾问, 高效解决问题.

第三部分的社科文章翻译, 自己的参与度不是很高. 一方面是因为出差请假缺了一次课, 二是因为翻译材料少, 课下不需要怎么看, 三是项目逼得紧, 也幸好最近几周没有作业, 要不然留了肯定也没时间写(拒绝通宵). 对此我感到十分抱歉. 但就但从翻译文章来看, 可见译者在该领域具有一定的专业性, 否则很难用学术写作的方式把原文翻译出来. 

上完了课后, 没空练习, 基本功没怎么变, 但翻译意识确实长进不少. 之前的翻译一般是 `看懂', 虽然知道自己不是机翻, 但翻译的时候还是习惯性全盘按照英文的思路翻译, 在疯狂写翻译作业的过程中, 疯狂翻译, 突然就顿悟了. 翻译应该是彻底看懂原文后(说起来简单, 做起来特别难), 打破原文的思路, 用自己的思路再将原文所讲内容复述一遍, 只是要准确. 同时翻译还要注意很多情感色彩, 褒贬之类的细节.

更重要的是一个问题, 我究竟为何人翻译? 如果类似学术或是报告那样, 确实要准确, 一个字也不能错. 但如果是类似 4chan 论坛或没品笑话之类, 只要梗到位能够惹人发笑即可, 准确倒也不是那么重要. 我自己也对英语笑话里面的一些地道的俚语和双关比较感兴趣, 会在第二部分稍微展示一些自己收藏的没品笑话.

总体来说, 收获很多, 反面教材也没白当, 感谢老师, 感谢同学, 蛤蛤.

\section{记录}
一些自己看过的作品.

1. 2015年NHK日本大河剧(类似中国历史剧, 在日本每年一部)真田丸 第34集
. 故事背景: 1600年五月, 德川家康搬入秀赖所在大阪城, 几乎将天下收入囊中. 返回会津的上衫景胜被怀疑谋反之事. 景胜断然拒绝了家康让其上洛的命令. 直江兼续向家康寄送了一封对冲其猛烈批判的文书, 历史上称`直江状'. 

在剧中片段为涅槃字幕组翻译作品. 原文(日文)可在网络找到, 其译文如下: 

\begin{quote}
    诚然, 吾等确实在收撰战之器物, 佛如上方武士喜好手机茶具器皿, 吾等乡野村夫不过把玩火枪弓矢. 大人拘与此等雕虫末技, 实负天下众望所托之量. 吾等殷殷之情所表绝无二心, 虽大人 `若无逆心, 则可上洛' 之言, 实乃强词夺理, 不可理喻. 诚待大人慧眼拨乱之时, 方是吾等上洛之日. 吾等非前恭后倨之徒, 对太阁殿下遗言阳奉阴违, 视起请文书似废篇草纸, 将秀赖公置如无物. 此等行径, 即便坐怀天下, 亦为人所万世唾骂, 子子孙孙亦受此辱, 至末方修.
\end{quote}

对于特定题材的翻译, 比如日本大河剧, 译者不止要对日语比较了解, 也要对古代日语很了解, 并且对中文素质要求很高.

2. 4chan 网页发帖. 原文如下. 4chan匿名社区则被称为网络的下限, 但在其中, 偶尔也会见到智慧与温暖的光芒.
\begin{quote}
    >Spend all my life being told by my parents that I'm smart. 
    
    >Teachers tell me I'm smart. 
    
    >Social retard, but get good grades. 
    
    >Fall for the "must just be a tortured genius" meme. 
    
    >Double down on academics and completely ignore social development since l've memed myself in to thinking others are just too dumb to get me. 

    >Get straight As all through school, which only reinforces this. 
    
    >By college, still being a virgin weighs on me, but at this point l've resigned myself to the life of a scholar so l ignore it and work even harder. 
    
    >At least being smart makes me special. 
    
    >Get straight As through undergrad, majored in chemistry. 
    
    >Do great on the MCAT 
    
    >Get in to med school. 
    
    >Surprise. Everyone here is smart as fuck. I'm probably dead average comparatively. No longer special at all. 
    
    >Begin to realize that I wasn't actually all that smart at all, I just worked like an autist all the time on getting good grades since I didn't have a social life. 
    
    >okay, well at least l'll be surrounded by other social outcasts, right? 
    
    >No. Med school is full of Chads and Stacies who are not only smarter than me, but are socially graceful. 
    
    >they've got plenty of stories about all the fun they had in undergrad and high school. 
    
    >l've got nothing. Being the smartest person in the room was my "thing." It defined me. 
    
    >Really sinking in now that I'm not special at all. 
    
    >Barely keeping up with the material, even with all of my autism directed at it. 
    
    >meanwhile I've got classmates who are getting laid, staying in shape, and keeping up fine. 
    
    \emph{I wish I never went to medical school some days. Yeah, it's it's good job, but it's absolutely shattered the idea that I'm exceptional.}
\end{quote}

    翻译大佬 Fall\_Art 的译文如下:

\begin{quote}
    >从小就被父母说我很聪明.
    
    >老师们也都说我很聪明.
    
    >社交上是个智障, 但成绩很不错 
    
    >相信了 ``我一定只是个受难的天才'' 的梗. 
    
    >于是加倍努力读书, 完全不顾社交上的成长, 因为我让自己相信其他孩子们只是太靠了才无法理解我
    
    >学校里一直是全A, 于是再度强化这一观念. 
    
    >到了大学, 虽然自己是处男这件事让上我心负重担, 但这时我已经接受了自己成为学术研究者的人生, 于是我无视了这一点, 更加努力. 
    
    >至少我的聪明让我很特别. 
    
    >本科也是全A, 读的是化学专业. 
    
    > MCAT医学院入学考试很顺利
    
    >进了医学院
    
    >惊不惊喜意不意外, 这里所有人都聪明得像鬼, 我的竞争力根本就很普通, 一点也不特别了. 
    
    >开始意识到自己其实根本没有那么聪明, 只是因为我一直都没有社交生活, 于是所有时间都像自闭狂魔一样拼命读书才获得了好成绩.
    
    >好吧, 至少现在我周围的人应该也都是社交上的被抛弃者们了对不对?
    
    >错, 医学院里满满的都是不仅比我聪明而且社交上也如鱼得水的渣男骚女们. 
    
    >他们有许多在本科和高中里玩得很开心的回忆和故事. 
    
    >我什么也没有. 我唯一的“特长”就是我是房间里最聪明的人, 那一点定义了我的一切.
    
    >现在才终于明白自己完全不特别.
    
    >学业勉强才能跟上, 明明已经把我所有的自闭力倾注在上面了.
    
    >与此同时我的同学们则在嘿味, 健身, 学业轻松无压力
    
    \emph{有时我希望自己从没去读医学院. 是的, 工作很好, 但这段经历彻底击碎了我前得白己与众不同的想法.}
\end{quote}

这里, 关于 `Chads and Stacies' 是 `不仅成绩优秀还很社交的同学' 的代名词. 由于 4chan 社区里面都是自闭的宅男, 所以将 `Chads and Stacies' 译为 `渣男骚女' 充分表现了对其的嫉妒. 译者在翻译的时候, 考虑到中文读者可能对 4chan 社区不是很了解, 要对类似俚语的名词做出解释.

3. Fall\_Art 大佬翻译的没品笑话集, 在此只抽出几个翻译的比较好的笑话. Sickipedia网站是由英国人所开办的百无禁忌笑话聚合站, 其中各类种族/疾病/状况歧视段子最受到欢迎.

第一则: 注意译文中 `黏' 与 `撵' 的谐音, 恰好与 `over' 的双关对应.

\begin{quote}
    I remember when my ex-girlfriends were all over me.
    
    我还记得前女友们当初有多黏我。
    
    Now they’re all over me.
    
    现在她们都只想撵我……
    
    (all over连用,极指爱得形影不离、无孔不入,over本身有释怀、忘记、不在意的意思,例如get over it)
    
    ——TilakGrey
\end{quote}

第二则: 与美国本土的校园枪击文化有关.

\begin{quote}
    学校里常常有人问我,为什么总是这么安静。
    
    啧,我琢磨着怎么大杀四方呢,这事儿能说出声吗?
    
    ——sam23
    
    //英语中有一句谚语“It’s the quiet ones you have to watch.”专指那些平时看起来闷声不响、孤僻安静的人,最有可能做出可怕的事情来,在各种校园枪击案中似乎尤其符合事实。
\end{quote}

第三则: 单纯的谐音梗.

\begin{quote}
    I phoned my work this morning and said, “Sorry boss, I can’t come in today, I have a wee cough.”
    
    今早打电话给公司:“对不起啊老板,今天不能来了,我有点小感冒。”
    
    He said, “You have a wee cough?”
    
    他说:“你有点小感冒?!” (谐音“You have a week off.”——“放你一星期假。”)
    
    I said, “Really? Cheers boss, see you next week!”
    
    我赶紧回答:“真的?太好了老板,那下周再见!!”
    
    ——jonnydelanio

\end{quote}

第四则: 没品笑话里面经常把老婆比作洗衣机或是洗碗机.

\begin{quote}
    A new survey shows that a fifth of British men have no idea how to turn on the washing machine.
    
    一项最新调查反映,有1/5的不列颠男性不知道该怎么让洗衣机开动起来。
    
    I find chocolates or flowers usually do the trick.
    
    个人经验,巧克力或者鲜花通常有效。(turn on用在人身上时,有“取悦、使……兴奋/性奋”的意思)
    
    ——stevo21
    
\end{quote}

\end{document}