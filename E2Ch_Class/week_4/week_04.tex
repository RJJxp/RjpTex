\documentclass[a4paper, 12pt, UTF8]{article}

\usepackage{ctex}
\usepackage{geometry}
\usepackage{graphicx}
\usepackage{float}
\usepackage{caption}
\usepackage{enumerate}

\usepackage{hyperref}

\hypersetup{
    colorlinks=true,   % false, ,链接黑色, 外有红框
    linkcolor=black, % 目录颜色, 脚注颜色
    filecolor=blue, % 链接本地文件的链接颜色
    urlcolor=cyan, % 网页链接颜色
    anchorcolor=blue,
    citecolor=yellow    % 参考文献颜色
}

\geometry{
    textheight=230mm
}

\begin{document}
\title{\Huge 英译汉课程}
\author{\Large 
        第四周 \\[12pt]
        1931991 任家平 \\[12pt]
        同济大学 \\[12pt]
        测绘与地理信息学院}
\date{2019-09-27}
% \maketitle

\maketitle
\thispagestyle{empty}


\newpage
\pagenumbering{Roman}
\tableofcontents
\addtocontents{toc}{\protect\vspace{10pt}}

\newpage
\pagenumbering{arabic}
\section{原文}
\begin{bfseries}
    \Large
    Grand openings
    \paragraph*{}
    \large
    Changes that will bring scientific discovery more freely into the public domain are happening. About time too.
\end{bfseries}

\paragraph*{}
    IN 2001 a meeting on scientific publishing held in Budapest by what was then called the Open Society Institute (now the Open Society Foundation) coined the phrase “open access”. The gathering official statement asked the world to “share the learning of the rich with the poor and the poor with the rich, make this literature as useful as it can be, and lay the foundation for uniting humanity in a common intellectual conversation” --- in other words, to make scientific papers free to users.

\paragraph*{}
    A noble aspiration, but one which cynics might have thought had little chance of coming to fruition. The rich, they would observe, include academic publishers, who have enjoyed three centuries of dominion over the dissemination of scientific work and who often have profit margins approaching 40\%. They had every incentive to scupper change.

\paragraph*{}
    Cynicism, however, is not always correct. The open-access movement which the meeting helped spawn now looks unstoppable. All seven of Britain’s research councils, for example, now require that the results of the work they pay for are open-access in some way. So does the Well-come Trust, a British charity whose medical-research budget exceeds that of many scientifically successful countries. And by 2016 every penny of public money given to British universities by the government will carry the same requirement.

\paragraph*{}
    Elsewhere, the story is the same. In 2013, after years of wrangling in America’s Congress, the White House stepped in to require federal agencies that spend more than \$100m a year on research to publish the results where they can be read for free. Countless universities, societies and funding bodies in other countries have similar requirements.

\paragraph*{}
    Publishers, though they have often dragged their feet, are adjusting. This week the oldest, the Royal Society, and arguably the most prestigious, Nature Publishing Group (NPG)—both based in London—joined in. Each will now publish a journal that readers do not have to pay to look at.

\paragraph*{}
    \begin{bfseries}
        \large
        Who will publish, and who perish?
    \end{bfseries}

\paragraph*{}
    Open-access publishing actually began in 2000, a year before the Budapest meeting, with the launch, in Britain, of BioMed Central, and in America, of the Public Library of Science (PLOS). The open-access business model shifts the cost of producing the journal from subscribers, such as university libraries, to researchers themselves, who pay an article-processing charge (APC) to appear in print or its electronic equivalent. Either way, the taxpayer picks up the bill in the end. But open publication makes research more widely accessible, which is a public good in its own right.

\paragraph*{}
    One problem open access brings is a shift in incentives. More published articles means more revenue from processing charges. Rejection rates at high-end paid-for journals often exceed 90\%. A commercial open-access publisher (which PLOS, a charity, is not) might be tempted to publish anything that came his way, in order to pocket the APC—and many have, giving open access a fly-by-night feel to some academics.

\paragraph*{}
    For example, in a survey conducted by NPG, of 27,000 authors of papers in its journals, 44\% expressed some concern about the quality of open-access publishing and more than a third agreed with the notion that it was associated with less prestige. Yet those perceptions may be misguided. A study of Nature Communications (which was, until this week’s announcement, a hybrid between open access and traditional subscription, but is now pure open access) shows that its open-access papers enjoyed a slightly higher number of citations and significantly more downloads and online views than their non-open-access brethren.

\paragraph*{}
    To foster prestige, champions of open access have launched efforts such as eLife, an online journal with an array of academic bigwigs at its helm. They hope to create a top-tier publication by sheer weight of bigwiggery. But eLife and its kind cannot force the change alone, because publishing with the requisite due diligence is not cheap.

\paragraph*{}
    The Public Library of Science has found a way out of this by copying the more-the-merrier approach, but in a controlled way. Until 2006 it was a producer of high-impact but loss-making publications. Then it started PLOS ONE, a different kind of journal altogether. Instead of acting as an arbiter of the importance of scientific work, PLOS ONE claims only to ensure that articles are scientifically sound. With less effort going into peer review, PLOS ONE publishes many more papers (in 2013, it carried more than 31,000 articles, 36 times as many as the next-most-prolific PLOS journal) while simultaneously charging less for them and becoming a cash-cow that helps pay for the rest of the outfit. The Royal Society hopes to recapitulate this idea with its new offering, Royal Society Open Science.

\paragraph*{}
    \begin{bfseries}
        \large
        Free hits
    \end{bfseries}

\paragraph*{}
    Whether the experience of Nature Communications will overcome researchers’ misgivings remains to be seen. Despite the Wellcome Trust’s requirement that its grantees publish in open-access journals, only 70\% do so. To ensure compliance, the trust has had to introduce punitive measures, such as withholding money.

\paragraph*{}
    Some researchers just don’t care, though. A survey by Taylor \& Francis, a publishing firm, asked American and British scientists if they had published under open-access policies; 44\% and 32\%, respectively, did not know. More than half responded they did not know if they would in future. But many do know, and resist.

\paragraph*{}
    The point about prestige is not mere snobbery. The league table of journals is as finely graded as that of football, and, at the moment, has far less scope for promotion and demotion. Grant-awarding bodies and appointment committees know this and, in a wonderful display of doublethink, promote open access but also promote those who eschew it by publishing in top-notch, non-open journals

\paragraph*{}
    Only when that changes will open access’s victory be complete. This could happen either by new open-access journals acquiring the necessary kudos, or by old ones, seeing the game is up, becoming open access themselves. Though Nature Communications is a successful and well-regarded publication, it is not NPG’s top product. And Royal Society Open Science is untested. At the moment, then, both the Royal Society and NPG seem to be hedging their bets. When the society’s Proceedings, and NPG’s eponymous flagship, Nature, are both free for anyone to read, then open access’s partisans really will be able to declare victory and go home.

\paragraph*{}
    \href{https://www.economist.com/science-and-technology/2014/09/27/grand-openings}{\emph{\small The Economist, Sep 27th, 2014}}

\section{译文}
学术进展的新发现将更容易进入公共领域, 也是时候做出这样的改变了.

2001年, 在布达佩斯举行了一场以学术刊物发表为主题的会议, 当时它被称作开放社会研究所(现在被称作开放社会基金). 在这个会议上提出了\lq 开放获取\rq\ 的概念. 这个具有号召力的官方声明呼吁全世界间不同阶级相互分享, 更应让社会底层拥有接触学术文献的机会, 让文献尽其所用, 这为之后人类的学术交流打下了坚实的基础. 换句话说, 让学术刊物对使用者免费开放.

这是一个十分崇高的理念. 但仍有不少挑刺者认为它不可能实现. 他们认为那些包括出版商在内的权贵们已经在三个世纪中享有学术成果的传播决定权, 并从中获取逼近40\%的利益. 因此, 学术权贵们理所当然会拒绝改变. 挑刺者有足够的理由担心开放获取运动沦为泡影.

然而这些挑刺者不总是对的. 这个被会议促成的\lq 开放获取\rq\ 运动无人能挡. 比如, 英国七大研究委员会要求他们的学术成果在某种程度上免费获取. 维康信托是一家英国的慈善机构. 它的医学研究预算已经超过了多个科学发达国家. 在2016年之前, 他们要求有英国政府补助的科研结果也应该免费向大众开放.

其他地方情况也差不多. 在2013年, 经过美国国会多年的激烈讨论, 白宫介入, 要求联邦机构每年多花100万美金去让科技成果文献免费为大众开放. 其他国家很多的大学, 社团和出资机构都如此要求.

出版商们并不想将学术刊物免费提供大众, 但外界压力逼得他们不得不做出改变. 这周在伦敦, 最老的学术出版商英国皇家学会和富有争议性的著名学术出版商自然杂志出版集团(NPG)也加入了开放获取运动, 他们承诺会发布向公众免费的学术刊物.

\paragraph*{\large 谁出版? 谁遭罪?\footnote[1]{Who will publish and who perish? 谁为出版买单, 谁的利益受损}} \hspace{10pt} \\

开放获取运动其实早在布达佩斯会议一年之前的2000年就开始了. 当时是由英国的生物制药中心和美国的公共科学图书馆发起. 开放学术刊物出版的商业模式与传统学术刊物出版不同, 传统刊物由大学图书馆之类的订阅者来承担它的印刷费, 而开放刊物现在由科研工作者付一笔出版费, 用于在学术期刊或其电子版上发表其文章. 无论哪种商业模式, 最后还是由公众买单. 但相比之下, 开放获取运动还是更易让学术刊物被大众获取, 造福大众.

开放获取运动带来的一个问题就是出版社刊登文章动机的转变. 开放出版社的营业收入与出版费挂钩, 刊登更多的论文意味着更高的收益. 高端付费的学术刊物对投稿者的拒绝率高达90\%. 盈利性的开放获取出版商可能经不起诱惑, 为了从出版费捞一笔, 别人投什么他就发什么. 这让某些学者觉得开放获取运动不可靠.

例如, 在自然杂志出版社集团的一个问卷中, 对在其旗下杂志表过文章的27000名学者进行了调查, 44\%的人表达了对开放获取刊物中论文质量的担忧, 多于三分之一的人认为开放获取运动没有坚持初衷, 威望大打折扣. 然而这些投稿人的看法可能有点狭隘片面. 根据自然通讯的一个研究显示, 开放刊物的论文引用量稍稍多于传统出版商刊物, 下载量和在线访问量更要多得多.

为了提高论文质量, 开放获取出版商中的佼佼者 eLife 做出了一系列努力. 它是一家以学术大佬撑腰的在线刊物. 它想仅凭学术大佬们的力量打造一款顶级学术刊物. 但这么做连最基本调查的资金可能都不够, 所以仅靠 eLife 单打独斗远远不够.

公共科学图书馆出版社创造出了一种质量有保证且薄利多销的经营方式. 2006年之前, 它虽是一个有影响力的出版商, 但其实一直在亏损. 之后它们发行了 PLOS ONE, 一个与众不同的刊物. PLOS ONE 不去要求论文的高质量, 这样可以大大减少同行审查的成本. 只要论文不是粗制滥造, 缺乏立论基础, 就可以发表在其期刊.  结果 PLOS ONE 文章刊登量显著提高. 在2013年, 它发表了31000多篇论文, 比第二多产的 PLOS 的36倍还要多. 与此同时, 由于一直以来出版费比其他刊物少. 薄利多销, PLOS ONE 变成了公共科学图书馆的摇钱树, 把 PLOS 部分的亏损也补上了. 皇家学会也想推出这类摇钱树开放刊物, 于是推出了皇家学会开放学术.

\paragraph*{\large 路在何方?\footnote[2]{之前译为`消除质疑'}} \hspace{10pt} \\

自然通讯的调查能否打消学者们的顾虑仍是未知. 尽管 Wellcome-Trust 要求它的被资助人在开放获取刊物上发表文章, 也只有70\%的人这么做. 为了达到这一目的, Wellcome-Trust 不得不弄出来一套扣钱之类的惩罚措施来确保被资助者在开放获取刊物上发表论文.

尽管如此, 还是有学者根本不在乎是否一定要在开放刊物发表文章. 泰勒弗朗西斯出版公司对英美学者是否在开放获取的刊物发表过文章做了调查, 44\%的美国学者和32\%的英国学者表示自己根本都不知道还有开放获刊物取这种东西. 在了解开放获取运动后, 这之中多半学者仍不确定自己会不会向开放刊物投稿, 也有很多学者明确自己的抵制态度.

学术刊物地位的决定性因素并不是学者们的势利.\footnote[3]{之前译为`学术刊物的名望往往很世俗, 其地位也对名望有极大影响'}. 学术刊物也有严格的等级划分. 不同学术刊物地位的排名和足球球队积分表排名有相似之处, 往往比较固定, 几乎没有上升或是下降的可能. 政府的拨款机构和提名委员会对此心知肚明, 他们表彰在开放运动的同时\footnote[4]{之前译为`他们表彰在开放刊物发论文的学者同时'}, 也同时表彰在那些非开放的顶尖学术刊物发表文章的学者. 真是一场完美绝伦的双标秀.

只有开放获取刊物的地位真正被政府机构和提名委员会认可, 开放获取运动才会迎来完整的胜利. 要么现有的开放刊物获取了足够的认可, 要么传统的不开放刊物见大势已去, 自己改革成开放刊物. 尽管自然通讯很成功且学术地位相对较高, 但它并不是自然杂志出版社集团最拿手的刊物. 皇家学会开放学术还没有经得起时间的考验, 是个未知数. 现在看来, 皇家学会和自然杂志出版社集团想及时止损, 采取观望态度.\footnote[5]{之前译为`现在看来, 皇家学会和自然杂志出版社集团想及时止损, 不想花大量精力做开放获取刊物'}. 大概只有当海军的 Proceedings 杂志和与自然杂志出版社集团齐名王牌刊物 Nature 全都免费开放时, 开放获取运动才真的成功, 它的铁杆粉丝们就能凯旋而归了.

\section{反思}
\begin{itemize}
    \item \emph{requisite due diligence} 直译为`应尽的尽职调查', 在此应该翻译为`代价'
    \item \emph{Free Hits} 觉得同学说的很好, 其直译为`任意球', 但她所说`任意球'是方向不确定, 和开放运动未来不确定有一定相似之处. 但查了百科之后,貌似任意球并没有表示方向不确定的意思. 但却给了我启发, 翻译标题重点应该点明本段大意的同时尽量简短. 因此译为`路在何方'.
    \item \emph{the point about prestige is not mere snobbery.} 承上启下的判断没有错, 启下的部分判断还算正确, 但承上就跑的没边了. 作者想表达的是, 在读过前两部分后, 你会对学术发表产生一些疑虑, 既然开放运动利国利民, 为什么还有学者要在非开放的期刊发表文章呢? 在本段就会做出回答. 但仅凭这篇文章很片面的学术期刊介绍, 会很难理解整个学术期刊的商业模式. 而且我觉得这里作者的逻辑也出了一点问题. 学术期刊地位应该是一直由自然, 科学这样的大佬牵头, 所有期刊地位都是固定的, 如果把这些东西提前说明, 就不会在有上文提到的疑问. 而且科研工作者发期刊往往是慕名而发, 正是学术刊物的地位导致学者们的势利, 而不是 `the point about prestige is not mere snobbery', 感觉有点像是作者把因果关系搞反了. 应该是 `the point about snobbery is not mere prestige'
    \item \emph{promote open access butalso promote those who eschew it by publishing in top-notch, non-open journals} 一般见到 `and' 我都是把其前后翻译成等价, 之前的翻译中, 因为 `and'后面主语是学者, 所以我把 `and' 前面的主语也翻译成学者. 但貌似在这种情况下并非如此. 还需打破惯性思维.
    \item \emph{hedging their bets} 搭配的本身意思是 `及时止损'. 但在此文中, 是说各大学术出版社现在在权衡利弊, 如果开放运动势头再涨, 可能会在开放运动上多些投资, 以保持自己的政治正确, 如果不能给自己带来很大收入, 反而让自己亏损, 那不如放弃. 所以翻译成 `采取观望态度'. 两边不贬也不褒, 也表现了一种对开放出版不确定的态度.
    \item \emph{in a wonderful display of doublethink} 我自己觉得应该翻译成 `双标秀'.  国外都是这种习惯, 为了大众福利, 盈利性代码开源, 如谷歌为代表. 随着大公司都这么做, 如果有的大公司不这么做, 便会被贴上 `政治不正确' 的标签. 作者第一部分写道政府大力推进开放运动, 其实很大部分原因也是出于政治正确的考虑, 换句话说, 可以提高政府的形象. 上文提到关于开放运动, 政府各个机构和部分都是大力推进, 很大部分都是基于 `面子', 所以当此时作者应该是一种揭穿政客虚伪面纱的一种心态写下 `a wonderful display of doublethink', 意在讽刺政府的虚伪, 并不是真心支持开放运动. 但是站在政府的立场, 其实非开放期刊才是学术的顶梁柱, 政治家们心知肚明, 鼓吹非开放期刊仅仅是博得大众信任的一个举措而已. 而且开放运动的提出人是 `索罗斯', 其政治意味也就不言而喻了. 为了讽刺, 所以意味 `双标秀'.
\end{itemize}

这篇文章挺难. 一方面是先验知识的匮乏, 二是翻译能力不够, 往往能意识到文章层次结构, 遇到有关词语, 联系上下文就会彻底搞乱, 经验不够. 另外是这篇文章的写法问题, 为了吸引人, 总是把事实一部分一部分挤牙膏地写出来, 是可以引发读者的兴趣, 但这就意味着想读懂要顺着作者的思路, 戴着 `镣铐' 读文章.

\section{机译}
\emph{只对作业要求部分机译}


\paragraph{1} 正在发生使科学发现更自由地进入公共领域的变化。 大约也是时间。

\paragraph{2}2001年,当时称为开放社会研究所(现为开放社会基金会)在布达佩斯举行的一次科学出版会议上创造了“开放获取”一词。 大会的正式声明要求世界“与穷人分享富人的知识,并与穷人分享穷人的知识,使这些文学尽可能地有用,并为在共同的知识分子对话中团结人类奠定基础” – -换句话说,免费向用户提供科学论文。

\paragraph{3}一种崇高的抱负,但愤世嫉俗的人可能以为这一抱负几乎没有实现的机会。 他们会观察到,这些富人包括学术出版商,他们在传播科学作品方面享有三个世纪的统治权,他们的利润率通常接近40%。 他们竭尽全力遏制变革。

\paragraph{4}然而,犬儒主义并不总是正确的。 会议帮助催生的开放获取运动现在势不可挡。 例如,英国的所有七个研究委员会现在都要求以某种方式开放其所支付的工作成果。 英国慈善机构Well-come Trust也是如此。 到2016年,政府向英国大学提供的每分钱公费将承担相同的要求。

\paragraph{5}在其他地方,故事是一样的。 2013年,在美国国会经过多年的争论之后,白宫介入了这项工作,要求每年在研究上花费超过1亿美元的联邦机构发布结果,以供免费阅读。 其他国家的无数大学,社团和资助机构也有类似的要求。

\paragraph{6}出版商虽然经常拖延脚,但他们正在调整。 本周,历史最悠久的皇家学会(Royal Society)和可以说是最负盛名的自然出版集团(Natural Publishing Group,NPG)(都位于伦敦)也加入了进来。现在,每个出版集团都将出版一本期刊,读者不必花钱看。

\paragraph{7}Nature Communications的经验能否克服研究人员的疑虑还有待观察。 尽管惠康基金会(Wellcome Trust)要求受赠人在开放获取期刊上发表文章,但只有70%这样做。 为了确保合规,该信托必须采取惩罚性措施,例如预扣资金。	

\paragraph{8}不过,有些研究人员不在乎。 出版公司泰勒和弗朗西斯(Taylor&Francis)进行的一项调查询问了美国和英国科学家是否按照开放获取政策发表了论文。 不知道的分别为44%和32%。 超过一半的受访者表示他们不知道将来是否会这样做。 但是许多人确实知道,并且反抗。

\paragraph{9}关于声望的意义不仅仅在于势利。 期刊的联赛表与足球的联赛表一样好,目前,晋升和降级的空间要小得多。 授予资助的机构和任命委员会都知道这一点,并且以双重思想很好地展示了这一点,既促进了开放获取,又通过在一流的非开放期刊上发表论文来促进那些避开它的人。

\paragraph{10}只有在这种变化之后,开放访问的胜利才能完成。 这可以通过新的开放获取期刊获得必要的荣誉来实现,也可以通过旧的期刊看到游戏结束,自己成为开放获取来实现。 尽管《自然通讯》是成功且备受推崇的出版物,但它并不是NPG的顶级产品。 皇家学会开放科学未经测试。 当时,皇家学会和NPG似乎都在对冲他们的赌注。 当该学会的会议记录以及NPG的同名旗舰产品Nature都可以免费供任何人阅读时,那么开放获取的游击队员真的就能宣告胜利并回家。

\section{机译评价}
机译来自谷歌翻译, 我不知道怎么去评价机翻结果. 逐字逐句太繁琐, 所以逐段对机翻结果写写自己的想法. 总体来说, 机翻结果基本不能看, 但其中也有不可忽视的亮点.

\paragraph{1} 全段属于机翻直译, `公共领域' 翻译比较准确.

\paragraph{2} `coin the phrase' 首次提出, 谷歌翻译为 `创造'. 直译 `“share the learning of the rich with the poor and the poor with the rich', `literature' 翻译成 `文学', 以及后面的翻译和我第一次翻译的一模一样. 很严重的直译.

\paragraph{3} 全段属于机翻直译, 翻译过程中, 没有对 `they' 进行仔细的翻译.

\paragraph{4} 将 `cynic' 译为 `犬儒主义者', 也属于机翻. 最后一句又是机翻

\paragraph{5} 全段机翻.

\paragraph{6} `dragging their feet' 翻译过来应引申为 `不想干某事', 而在此又是机翻. `可以说是最负盛名的' 我觉得翻译的不错, 英文原文是 ` arguably the most prestigious'. 在这个译文中以中文逻辑体现了 `arguably' 受争议的翻译. 后一句又是机翻.

\paragraph{7} `To ensure compliance' 直译为 `为了确保合规' 不大合适.
 
\paragraph{8} 本段整体翻译的还是可以的, 就是最后一句出戏, 没有联系上下文. 

\paragraph{9} 本段第一句 `The point about prestige is not mere snobbery' 是重点. 原来这所以不能翻译的另一理由是因为, 这么翻译和机翻没有太大区别. 剩下的基本是机翻, 前言不搭后语. 不过谷歌翻译在最后一句翻译 `发表论文...的人' 还是比较智能.

\paragraph{10} 本段, `seeing the game is up' 机翻为 `看到游戏结束'. `hedging their bets' 机翻为 `对冲赌注'. 非常生硬的机翻. 最后一局将 `partisans'(坚定的支持者) 翻译为 `游击队员', 比较滑稽. 

\paragraph{} 机翻前言不搭后语, 对于关键词的翻译永远是直译, 不联系上下文. 让人印象深刻的还是 `the point about prestige is not mere snobbery' 的直译居然和很多人的翻译一模一样, 让人脸红.

\end{document}