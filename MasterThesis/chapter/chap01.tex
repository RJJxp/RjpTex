\section{绪论}

\subsection{研究背景与意义}
从空中和太空观测地球获取影像是20世纪的重大成果之一, 几十年内, 遥感数据获取手段迅猛发展, 呈现三多(多平台, 多传感器, 多角度)和三高(高空间分辨率, 高光谱分辨率和高时相分辨率)的特点~\cite{lideren}. 遥感影像可用于环境监测, 灾害预防, 地物变化, 目标识别解译等用途, 而低分辨率遥感影像(下文所指分辨率均默认为空间分辨率)无法提供细致的地物特征, 实际应用效果不佳. 高分辨率遥感影像地物细节丰富, 可为遥感影像的各类研究带来更多的便利. 随着研究和应用的领域不断扩展, 人们对高分辨率遥感影像的需求也与日俱增. 

获得高分辨率影像是遥感技术领域追求目标之一, 最直接的方法是使用更高精度的影像传感器在硬件层面直接获取高分辨影像, 具体为缩小感光元件的单位尺寸或是增大光学焦距长度, 但这些方法受制于传感器体积, 制造成本和工艺等因素的约束, 难度较大且成本较高. 此外, 仅为增加遥感影像光学分辨率而单独发射一颗卫星, 在现实生活中不可行. 因此, 如何在现有的硬件条件下, 以最小的成本, 最大限度地提高遥感影像分辨率已经成为一个迫切需要得以解决的问题.

超分辨率(Super Resolution)技术是通过现有的一张或者多张低分辨率图像, 利用图像处理技术, 获得一张包含更多地物细节的高分辨率图像的过程, 本质上是通过图像算法恢复成像系统在获取图像过程中丢失成像系统截止频率之上的高频信息~\cite{yangxin}. 根据超分辨率技术中低分辨率图像的数量, 可分为单幅影像超分(Single Image Super Resolution)和多幅影像超分(Multiple Image Super Resolution)的方法. 多幅影像超分方法可充分利用影像间的冗余互补信息, 获得更好的超分效果, 但在很多应用场景下, 直接获得严格空间配准后同一景象的多幅低分辨率影像难度较高. 因此, 鉴于应用的便利, 针对单幅影像的超分算法显得尤为重要. 

单幅影像超分算法主要有三种: 基于插值的超分方法, 基于重建的超分方法和基于学习的超分方法. 基于插值的超分方法无法恢复图像高频信息, 图像放大后细节模糊; 基于重建的方法对超分倍数较为敏感, 超分倍数越大, 重建效果逐渐降低且耗时过大; 基于学习的方法通过训练大量的图像样本, 建立从低分辨率到高分辨率图像的映射关系. 随着算力的提升, 基于深度学习的超分方法的出现, 彻底改善了传统浅层学习方法的不足, 使图像超分的效果大幅度提升. 

高分辨率遥感影像价格昂贵, 基于实用的角度, 如何使用免费开源的低分辨率影像, 通过深度学习超分辨率的方法获得细节纹理丰富高分辨率遥感影像, 提升影像分类解译等任务的精度, 具有重大研究意义. 

\subsection{国内外研究现状}
SRCNN~\cite{SRCNN2014}第一个将端到端深度学习引入超分领域, 文中先将低分辨率图片双三次插值上采样到高分辨率图片, 再输入网络, 用三层纯卷积网络进行特征映射, 最后使用MSE来作为重建损失函数进行训练. 针对三层网络无法挖掘足够特征而导致重建效果不佳的问题, VDSR~\cite{VDSR2016}采用20层卷积的深层网络进行特征挖掘, 并使用残差学习和自适应梯度裁剪来加速模型的训练; DRCN~\cite{DRCN2016}借鉴了VDSR~\cite{VDSR2016}的残差思想, 提出了一种递归神经网络模型; DRRN~\cite{DRRN2017}对DRCN~\cite{DRCN2016}进行了改进, 提出了局部残差连接和残差单元的递归学习的方法, 在常用超分数据集中取得不错效果. 以上方法与SRCNN~\cite{SRCNN2014}相似, 先对低分辨率图像按照缩放因子上采样, 再输入网络模型训练, 只是对网络深度和模型进行了改进. 但将上采样的图片作为网络输入, 引入大量噪声, 模糊了图片原有特征, 相当于先验确定了未知的图像退化模型, 同时数据量变大, 网络训练也变得更加耗时. 

为了克服以先上采样后输入网络方法的缺陷, FSRCNN~\cite{FSRCNN2016}首次使用了反卷积层(Deconvolution Layer), 网络直接输入低分辨率图片, 网络最后用反卷积进行上采样, 不需要对输入图像进行插值的预处理, 模型直接进行端到端的学习低分辨率输入与高分辨率输出之间的映射关系. 同样使用反卷积层, SRDenseNet~\cite{SRDenseNet2017}将稠密块结构应用到了超分辨率问题上, 用多个稠密块学习高层的特征, 然后通过几个反卷积层学到上采样滤波器参数, 最后通过一个卷积层生成高分辨率图片输出. 

除了反卷积层, ESPCN~\cite{ESPCN2016}首次提出基于亚像素卷积层(Sub-Pixel Convolution Layer)后端放大的超分辨率网络, 使得在低分辨率空间保留更多的纹理区域. 相比反卷积层, 亚像素卷积层是一种高效, 快速的像素重排列的上采样方式, 拥有更大的感受野, 对于细节信息的重建效果更好. SRResNet~\cite{SRGAN2017}使用残差网络和亚像素卷积层进行超分, EDSR~\cite{EDSR2017}对SRResNet~\cite{SRGAN2017}进行了改进, 基于超分辨率低级视觉任务的考虑, 去掉了残差网络中的批规范化处理层(Batch Normalization Layer), 使模型更紧凑, 训练时占用内存小速度变快, 使用亚像素卷积层进行最后的上采样与高分辨率图片生成. RDN~\cite{RDN2018}结合残差模块和密集模块的优势, 提出残差密集模块; RCAN~\cite{RCAN2018}在深度残差网络中引入通道注意力机制, 通过特征通道之间的相互依赖性来重新调整特征权重; MSRN~\cite{MSRN2018}通过不同大小卷积核提取不同尺度特征; SAN~\cite{SAN2019}提出了用于图像超分的深度二阶注意力网络, 更关注图像的高频信息, 并利用了协方差归一化的方法来加速网络的训练. 使用亚像素卷积层进行网络后部上采样是现在的主流方法. 

除了先对低分辨率图片上采样预处理再输入网络训练和基于后置上采样模块直接输入低分辨率图片训练的超分方法, LapSRN~\cite{LapSRN2017}使用拉普拉斯金字塔的结构, 结合残差网络渐进式地进行上采样超分预测; DBPN~\cite{DBPN2018}通过借鉴传统方法中的反向投影(Back Projection)方法, 构造了迭代式升降采样的方法, 关注在不同的深度的上采样率,并且将重建的损失分布到各个阶段中. 

上述方法训练网络时用像素级别损失函数, 虽然能够获得很高的峰值信噪比, 但当缩放因子较大时, 超分重建出的高分辨率图片趋于平滑, 丢失高频信息严重, 使人不能有良好的视觉感受. 针对此问题, SRGAN~\cite{SRGAN2017}首次将生成对抗网络(Generative Adversarial Networks, GAN~\cite{GAN})引入图像超分辨率领域, 以SRResNet作为生成器, 使用对抗损失提高了生成图片的真实感, 并通过内容损失函数提升视觉感受. 除了图像域的判别器, SRFeat~\cite{SRFeat2018}在SRGAN~\cite{SRGAN2017}基础上增加了特征域的判别器, 使生成图片高层特征尽可能与真实图片相似. ESRGAN~\cite{ESRGAN2018}删除了SRGAN~\cite{SRGAN2017}中SRResNet中批规范化处理层, 并用RRDB模块代替原有的残差模块, 借鉴了相对生成对抗网络(relativistic GAN)让判别器预测相对的真实度而不是绝对的值, 生成的超分图片效果更好. 

随着基于对抗生成网络的超分模型不断涌现, 超分方法日渐成熟, 其研究方向逐渐转向提升实际应用的效果. 

现阶段基于对抗生成网络的超分方法主要为监督学习, 较为依赖超分数据集. 超分数据集据其低分辨率图像获取方法, 主要分为两大类, 第一类如DIV2K~\cite{DIV2K}, BSDS500~\cite{BSDS500}, Set5~\cite{Set5}, Set14~\cite{Set14}, Urban100~\cite{Urban100}等人造低分辨率超分数据集, 这些数据集只含有高分辨率图像, 在进行训练之前, 需要通过先验的图像退化模型进行下采样生成所对应的低分辨率图像, 和原始的高分辨率图像组合成图像对. 在以上自然影像超分数据集制作过程中, 使用的是较为经典的退化模型~\cite{classicDM}, 一般对高分辨率图像进行模糊化(Blur), 噪声模拟(Noise), 下采样(Downsample)后得到低分辨率图像. 然而真实世界的退化模型十分复杂, 经典退化模型过于简单, 无法胜任模拟现实中高分辨率到低分辨率图像的退化过程的任务. 因此, Fritsche等人~\cite{DSGAN}, Wang等人~\cite{DASR}, 构建数据集时, 使用对抗生成网络学习低分辨率图像的数据分布来得到退化模型, 但通过此种方法训练出的退化模型受限制于其训练数据集数据分布, 如果实际数据分布不同于训练集数据分布, 其训练出退化模型效果不佳. 无论经典退化模型或是训练出的退化模型, 由于模拟生成的低分辨率数据和真实的低分辨率数据在特征空间中总是存在明显差异, 数据鸿沟现象无法避免, 因此往往通过人造低分辨率图像训练得到的模型在现实应用场景效果大打折扣~\cite{SupER}.

与之相对应, 第二类构建超分辨率数据集的方法则是通过调整硬件设备的拍摄条件, 直接从获取真实世界中的高分辨率低分辨率图像对, 如Real-SR~\cite{realSR}, SR-RAW~\cite{SR-RAW}等. 而制作数据集过高的人力成本不言而喻. 

鉴于过高的数据集制作成本, 又为避免数据鸿沟现象, 借助于CycleGAN~\cite{CycleGAN2017}在图像翻译领域的成功, CinCycleGAN~\cite{CinCycleGAN2018}将其引入到超分辨率领域, 使用四个生成器和两个判别器, 用不成对的高低分辨率图片训练进行非监督式的超分模型训练, 学习超分辨率中的退化模型, 超分问题中更关注图像中细节信息和纹理特征, 而CinCycleGAN更偏向于风格迁移, 这会造成超分结果中的高频信息丢失. 因此, 为了达到较好的实用效果, 必须制作真实的超分数据集. 

<现在遥感超分方法, 大家用哪些数据集>

<任意比例的引入>

<评价指标的研究>



\subsection{论文研究内容与章节安排}
todo: