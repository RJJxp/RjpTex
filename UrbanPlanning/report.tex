\documentclass[a4paper, 12pt, UTF8]{article}

\usepackage{ctex}
\usepackage{abstract}
\usepackage{geometry}
\usepackage{graphicx}
\usepackage{float}
\usepackage{caption}
\usepackage{enumerate}
\usepackage{natbib}

\usepackage{hyperref}
\hypersetup{
    colorlinks=true,   % false, ,链接黑色, 外有红框
    linkcolor=black, % 目录颜色, 脚注颜色
    filecolor=blue, % 链接本地文件的链接颜色
    urlcolor=cyan, % 网页链接颜色
    anchorcolor=blue,
    citecolor=yellow    % 参考文献颜色
}


\begin{document}
\title{\Huge 陕西省老龄化问题研究}
\author{\Large 
        1931991 任家平 \\[12pt]
        同济大学 \\[12pt]
        测绘与地理信息学院}
\date{2019-12-27}
\maketitle
\thispagestyle{empty}

\newpage
\renewcommand\abstractname{\Large\textbf{摘要}}
\begin{abstract}

    人口老龄化是当今许多国家面临的一个重要问题. 我国老年人口规模大, 老龄化进程发展迅速, 2000年正式进入老龄化社会, 成为较早进入老龄化社会的发展中国家, 也是老龄化发展最快的发展中国家. 近年来, 在各政府部门和全社会的共同努力下, 我国虽然建立起了较为完善的社会养老服务体系, 但仍存在较大不足, 需要加以大力完善, 陕西省作为西部经济大省, 人口老龄化问题对社会经济的发展带来不小的影响和挑战. 因此, 积极面对人口老龄化问题对陕西省社会经济发展十分重要.  

    \par\textbf{关键字:} 老龄化, 城乡规划, 陕西省

\end{abstract}
\thispagestyle{empty}

\newpage
\pagenumbering{Roman}
\tableofcontents

\newpage
\pagenumbering{arabic}
\section{人口老龄化现状}
\paragraph{老龄化程度不断加深}

根据陕西省历年统计年鉴显示, 2000年陕西省65岁及以上老年人为216.45万人, 占全省人口总数5.94\%, 未超过7\%的界限. 到2002年, 65岁及以上老年人口占比增长至8.01\%, 陕西省从此进入老龄化社会. 从2002年开始, 陕西省老年人口规模持续增长, 老龄化程度不断加深, 到2017年65岁及以上老年人有414.23万人, 较 2000年增长近200万人, 占全省总人口10.80\%.  

\paragraph{老年人口抚养比例持续增长}

老年人口抚养比是指65岁及以上老年人占15~64岁劳动人口的比例, 用以表示社会负担的大小. 根据陕西省历年统计年鉴显示, 2000年65岁及以上老年人口抚养比为 8.60\%, 即约每12个劳动力人口负担1个老年人. 到2002年, 65岁及以上老年人口抚养比增长3个百分点, 即约每9个劳动力人口负担1个老年人. 到2017年, 65岁及以上老年人口抚养比为增长至 14.43\%, 约每7个劳动力人口负担1个老年人. 老年人口抚养比持续增长, 劳动力人口抚养老年人口压力显著增大.  

\paragraph{各地区老龄化程度差异大}
受到政治, 经济, 文化, 社会发展水平以及人口发展过程中的各种因素的综合影 响, 各地区老龄化程度存在较大差异. 根据人口普查和全国1\%人口调查数据显示, 2000年陕西省11个市区中只有汉中, 安康两个城市的老年人口占比超过7\%, 其余城市皆未进入老龄化社会, 其中西安市老年人口占比趋向7\%的临界值, 占比最高的汉中 市比占比最低的延安市高2.81\%. 到2005年, 除商洛市和杨凌示范区外, 其余9个城市均先后进入老龄化社会, 其中汉中市仍然是老年人口占比最高的城市, 老龄化增长最快的是咸阳市. 2010年各地区老龄化进程呈缓慢增长态势, 其中汉中市老年人口占比突破10\%, 远高于全省平均水平, 延安市和杨凌示范区低于7\%. 到2015年全省11个市区 全部进入老龄化社会, 其中有6个城市老年人口占比超过10\%, 汉中市老年人口占比更是高达13.04\%, 趋向14\%高龄社会的临界值. 从2000年到2015年, 老龄化发展速度最快的是宝鸡市和咸阳市, 平均年增长率均在5\%以上, 发展速度最慢的是杨凌示范区, 平均年增长率仅为2.14\%.  

\paragraph{农村老龄化程度高于城镇}
随着社会经济的迅猛发展和城镇化的不断推进, 人口迁移加速了乡村老龄化进程. 根据人口普查和全国1\%人口调查数据显示, 2000年陕西省无论是城市, 镇还是乡村, 65岁及以上老年人占比都未超过7\%, 属于典 型的成年型人口. 2005年, 陕西省城市, 镇 和乡村全部都已迅速进入老龄化社会, 老年人口占比均超过7\%, 其中城市老龄化程度 最高, 其次是乡村, 最后是镇. 而到了2010年, 乡村65岁及以上老年人占比增长至9.34\%, 反超城市老龄化程度. 到2015年乡村老龄化程度高达11.9\%, 在15年间年平均增长4.32\%, 远高于城市和镇的老龄化速度.

\section{人口老龄化影响}
\subsection{老龄化对劳动力市场的影响}

劳动力的供给一般是由总人口中处于劳动年龄人口的多少来决定的. 劳动年龄人口的数量及其在总人口中的比重, 对一个国家或地区的经济发展至关重要. 人口老龄化实际上在人口年龄结构上的变化可以表现在三个方面即总人口中老年人口的比例增加, 老龄人口中高龄人口的比例增加(高龄化), 劳动力人口中老年劳动人口(45~59)的比例增加. 

人口老龄化意味着在经过一段年龄推移之后劳动力年龄人口比例会相对下降, 劳动力资源相对减少, 这在一定程度上会影响社会生产和开发, 不利于经济的发展. 另外, 人口老龄化随着时间的推移会造成劳动力人口老化, 而老龄劳动力在接受新知识和科学技术方面要比轻壮年慢, 对新兴产业和就业岗位的适应能力也相对较弱一些, 企业的新产品开发和技术革新也受到一定影响, 对科学技术迅速发展和经济的增长显然是很不利的. 

虽然计划生育政策所导致的人口快速转变使得少儿人口抚养比快速下降, 而与此同时, 老年人口的抚养比不断上升, 但其上升速度远远低于少儿人口抚养比的下降速度, 因此, 总抚养比将呈现快速下降的趋势. 因此在相当长的时期内, 陕西省的劳动力供给将是过剩而不是不足. 陕西的人口态势显示我省出现了劳动力增长连续超过总人口增长速度的现象. 第三次至第五次人口普查资料数据显示, 陕西省1982年劳动力人口数量为1802.82万人, 占陕西省总人口的62.37\%. 1990年, 陕西省劳动力人口数量达2169.32万人, 比1982年增加366.54万人, 增长20.33\%了, 年平均增长速度为2.34\%. 而且由于人口的惯性作用, 第一、二次生育高峰期出生的人口逐渐成长为劳动年龄人口, 使劳动力人口占总人口的比重有所上升, 1990年上升到65.97\%, 比1982年高个3.6百分点. 2000年陕西省的总人口达到3644.01万人, 陕西省的劳动力人口达到2446.09万人. 与1990年相比, 从数量上看, 无论是总人口还是劳动力人口都在持续增加从增长速度上看, 年平均增长速度为1.21\%, 比1982年至1990年间的年均增长速度下降了1.13个百分点, 这主要是总人口增长速度放慢的结果. 总起来说, 这一时期劳动力人口的年增长速度快于总人口增长速度, 因此劳动力人口占总人口的比重越来越高, 到2000年达到67.13\%.

预测显示, 未来陕西劳动力资源将十分丰富. 15岁-64岁劳动人口占总人口的比重基本上在65\%以上, 2020年以前基本上保持在70\%以上. 根据预测, 劳动年龄人口在2010年达到2855万人, 2016年达到峰值2884万人, 占总人口的比重为71.24\%. 适龄劳动人口数量的增大, 将使人口的抚养比降低. 

应当说, 在低抚养比时期, 我省的劳动力资源是十分丰富的, 较长时期内还是处在劳动力供给大于需求的状况. 然而由于人口结构变化的影响具有一定的滞后性, 虽然现阶段我们并没有感受到老龄化对劳动力供给的影响. 但事实上, 伴随着人口老龄化快速发展的过程, 必然会导致劳动力人口年龄结构的变化, 即劳动力中年轻人的比重降低, 而年长者的比重将会上升, 即劳动年龄人口老化必然会出现劳动年龄人口比重的下降, 从而影响到劳动力资源的有效供给. 同时, 劳动年龄人口本身也会出现老化, 长此以往, 人口老龄化将使劳动力供给出现困难局面, 当劳动力年龄人口的比重下降到一定程度时, 可能导致劳动力供给数量的短缺, 而劳动年龄人口的老化及劳动力资源供给的减少, 将影响社会劳动生产率的提高, 对经济的发展产生严重的影响. 在一些高度老龄化的国家和地区, 如西欧, 北欧以及美国, 日本, 澳大利亚等已经产生了劳动力短缺的现象, 并且成为制约其经济发展的一个重要因素. 

人口老龄化使得劳动力的年龄结构发生变化, 青年劳动力在劳动年龄人口中的比重下降, 年长劳动力比重增加. 而人到中年以后, 生理机能将逐渐衰退, 体力和精力开始下降, 同时心理方面也会发生一些变化. 一般来说, 老年劳动力的反应不如年轻劳动力快, 不能适应快节奏的生产活动, 接受新事物、掌握新技能的能力也随之降低, 生产效率往往低于年轻劳动力, 特别是在劳动密集型的生产中, 不利于劳动生产率的提高. 当然, 年长的劳动力在工作经验和工作阅历上具有一定的优势, 是社会生产中不可缺少的一部分, 然而从整体趋势来看, 人口老龄化对整个社会劳动力的供给和劳动生产率的影响还是负面性居多.

\subsection{老龄化对储蓄水平的影响}
根据经济学理论, 储蓄等于投资, 国内储蓄是资本积累的重要资金来源, 一般来说, 长期保持较高的投资率和相应的储蓄率, 使投资率和储蓄率经常处于均衡的状态, 才`能促进经济的持续发展. 储蓄对于投资与扩大再生产密切相关. 储蓄减少等于投资减少, 从而对经济发展产生不利影响. 而人口老龄化又对储蓄有一定的影响, 因此, 人口老龄化与储蓄两者之间的关系是人老龄化与经济发展之间关系的一个重要方面. 

美国经济学家克拉克和斯彭格勒在年出版的《个人与人口老龄化经济学》一书中指出对于个人来说, 进入老年以后由于收入的来源和数量的变化, 会带来个人储蓄的减少. 而弗朗科莫迪里亚尼在研究人口老龄化与储蓄之间的关系时, 提出了``储蓄的生命周期假说''. 他认为, 人们在花费自己的收入时, 总要结合生命循环的过程来考虑, 特别是在退休前总是倾向积极储蓄以备退休后使用, 而退休后基本上停止储蓄, 储蓄的目的是为退休后的消费提供保障. 即退休后每年预期需要的消费, 要在退休前进行储蓄. 因此一般认为, 退休前储蓄率往往呈上升趋势, 而由于退休后老年人的收入下降, 导致个人无储蓄或减少储蓄, 使储蓄率下降. 但是, 人口老龄化对储蓄、投资的影响也应该结合陕西的实际分阶段地进行分析. 首先, 在人口老龄化的初始阶段, 人口老龄化对储蓄的影响相对比较小. 具体到我省的情况, 虽然老年人口不断增加, 但在今后5-10年内以下因素将会抑制我省储蓄率的下降, 第一虽然我省老年人口比重不断上升, 但在今后年内, 劳动年龄人口比重仍将保持上升势头, 这一因素在一定程度上会抑制储蓄率的下降. 第二养老保障制度所采取的强制性储蓄制度, 在一定程度上可以增加储蓄. 第三目前养老制度改革, 医疗制度改革的不确定性将增加人们的储备性储蓄, 从而扩大投资, 刺激经济发展. 但是从长期趋势来看, 由于老年人口的储蓄水平相对较低, 且陕西老龄者的实际收入一贯较低, 因此伴随着人口老龄化程度的加深将会带来总储蓄水平的降低. 总体上看这不仅减少了资本的积累和储蓄而且会对生产基金的积累产生消极影响, 限制了社会扩大再生产, 进而对经济发展产生不利的影响. 

\subsection{老龄化对消费的影响}
消费是社会经济运行过程的四个环节之一, 它为生产创造出更多的新的需求, 从而促使生产不断向前发展. 人作为消费者, 是消费的主体与直接承担者, 社会的发展总是为了满足人的某种需要. 一定社会的消费水平, 消费结构以及由此形成的产业结构总是与这个社会的人口构成因素密切相关. 不同年龄的人群有不同的消费倾向和消费量. 因而人口年龄结构的变化会引起消费水平和消费结构的变化, 从而影响到经济的发展. 

随着人口老龄化的发展, 老年人口消费在消费市场中所占的份额越来越大. 根据对我省人口年龄结构发展趋势的预测, 我们可以知道在未来的几十年内由于未成年人口的消费负担指数下降的速度没有老年人口消费负担指数上升的快, 这将使我省总的负担人口的消费指数在未来几十年内呈上升的发展趋势. 这说明, 快速发展的人口老龄化不仅会改变我省的消费结构, 而且会加重社会的负担, 使得国民收入中用于消费的基金需求不断上涨, 而使得积累基金的比重受到消极影响, 甚至影响到整个社会经济的发展. 另一方面, 人口老龄化对微观消费水平的影响更为明显. 就一个国家的整体消费而言, 在老龄化过程中, 消费支出呈现逐渐扩大趋势, 但是在老龄化社会, 随着人口老龄化的迅速发展, 由于年老者的收入水平相对较低, 其人均消费额会随着不断的衰老减少, 例如会对住宅, 电视等耐用或价格昂贵的消费品的需求减少, 消费支出则呈现减少趋势. 随着人口老龄化的迅速发展, 这就在某种程度上也抑制了经济的发展. 

此外, 人口老龄化的迅速发展意味着需求结构将会发生很大的变化. 按消费经济学的观点, 当老年人口的绝对数量增加到一定程度时, 满足具有老年人特色的衣, 食, 住, 行, 乐, 医等各方面极大的物质和文化需要, 将形成一个崭新的庞大消费市场, 促进一个``阳光产业''一老年产业. 为了更好的满足迅速增长的老年人口的各种需求, 需要大力发展老龄产业. 

\subsection{老龄化对产业结构的影响}
产业结构的本质是产业间经济技术的联系, 反映为产业内部和产业间的比例关系. 

人口老龄化的快速发展要求满足老年人口对物质和精神文化生活的特殊需求, 从而在供给和需求两个方面对现有产业结构提出调整和升级的需求. 老年人的生理和心理特点决定他们更多需要的是第三产业创造的直接服务于他们的各种劳动, 而不仅仅是第一,  二产业提供的一般消费品. 因而, 未来人口老龄化的进展必然带动第三产业的发展, 有力地促进剩余劳动力, 尤其是农业剩余劳动力向第三产业的转移, 从而实现劳动力就业的产业结构调整. 

如上所述, 由于人口老龄化在今后相当长的一段时间里还不会影响到陕西省劳动年龄人口的数量供给, 因而首先要重视发展有比较优势的劳动密集型产业, 兼顾劳动密集,  资本密集,  技术密集产业和企业的协调发展及合理布局, 特别是发展就业容量大的服务业. 

人口老龄化所导致的老年人口比重的上升意味着劳动适龄人口在经过一段年龄推移之后, 其中的青年劳动适龄人口比例下降, 年老劳动适龄人口比例上升. 这就要求经济结构中的技术结构要完成由以劳动密集型为主体向以知识技术密集型为主体的方向转化, 以便使活劳动支出中体力消耗下降, 脑力消耗提高, 主动与年老型劳动适龄人口的年龄结构相适应. 另外, 由于不同年龄段的消费结构,  消费数量及消费方式的不同, 因此老年人口比重的上升和数量的增多, 也将为老年产业的兴起和发展提供基础. 而在老年产业体系中, 大多数行业属于第三产业的范畴, 因而人口老龄化引起的人口结构状况和变动趋势必将通过消费结构的变化引起产业生产结构的调整, 并带动第三产业的发展.


\subsection{老龄化对社会保障的影响}
社会养老保障是国家依法对老年人基本生活予以保障的社会安全制度。它包括老年社会保险养老保险、医疗保险、老年社会福利、老年社会救济,其中老年社会保险是其核心,这三者构成完整的老年社会保障。

随着老年人口的增加,一方面由于老年人收入水平较低,其纳税水平也较低,另一方面,由于老年人口数量的快速增加,因此用于老年人口的养老支出,将随着老年人口规模的扩大及比重的上升而不断增加。

\paragraph{养老金的支出} 
随着我省人口老龄化的进一步加快,离退休职工也迅速增加,相应的养老金支出也以更快的速度增长,根据2002年陕西统计年鉴可知,陕西省离退休人员占在职人员的比例1990年为14.5\%,2001年上升到31.2\%,10年之间翻了一翻,国有单位离休、退休和退职职工保险福利费394151.46万,占在岗职工工资总额2963544.8万元的。另外,用于敬老院、老年医疗费用支出、老年职工死亡丧葬费的数额也十分巨大。随着我省人口老龄化程度的不断加深,人口老龄化高峰将在21世纪40年代达到高峰,人口负担的老年系数迅速上升,将成为我省养老保险成本上升的主要因素。

\paragraph{医疗保障费用}
养老金支出只是人口老龄化成本的一个方面, 而另一项不容忽视的支出则是老年人的医疗费用. 人口老龄化过程中公共医疗费用的上升主要是两个因素相互作用的结果, 一是老年人口规模的扩大和比重的迅速膨胀, 二是老年人口人均医疗费用的增长. 按照人的生理演变, 人的一生的80\%医疗费用在60岁以后. 从发达国家的情况来看, 65岁以上人口与65岁以下人口的人均医疗费用比例约为3:1-5:1, 特别是75岁以上人口疗费用增长更快. 

目前我省现有的老年保障体系尚不完善, 社会保障制度没有做到应保尽保, 覆盖面非常有限, 不适应快速到来的人口老龄化的需要. 虽然我省在党政机关、全民所有制企事业单位普遍建立起了退休制度, 年老退休人员的晚年生活完全由国家和企业包下来, 在广大农村地区则普遍实行“五保制度”. 但现实情况却是, 由于党政机关和事业单位的保障资金由财政部门支付, 用于企业的保障资金由企业自身支付. 这是一种单位保障和企业保障, 社会化程度极低. 目前城镇中只有一部分的老年人可以领取到退休金, 而还有相当一部分下岗职工尤其是老年人无社会保障农村老年人口中, 90\%以上的农村老年人被摒弃于社会保障体系之外, 这些老年人全是靠自养和子女供养. 事实上, 我省现行的社会保障制度完全排斥了农村人口, 基本排斥了农民工群体, 广大农民及农民工仍然依靠自我保障. 

因此, 随着人口老龄化的加剧, 使家庭、企业、社会和国家的负担日益加重, 用于老年人的社会保障和福利费用支出将大幅度上升, 从而给政府带来沉重的负担, 降低政府的积累和投资能力, 影响到社会经济的发展. 

\subsection{老龄化对养老模式的影响}
``老有所养''除了指经济方面的保障外, 也包括精神慰籍及生活照料. 在我国, 家庭养老是中国的优良传统, 仍然是解决老有所养的主要模式. 与家人同住, 便于子女供养照料老人, 也是我国老年人比较认同的生活方式. 从省老龄委对全省老年人状况的调查结果可以得知, 我省以上的老年人属于家庭养老.

但是, 随着我省人口老龄化程度的不断加深, 传统的家庭养老模式受到影响. 计划生育政策的实施, 形成了庞大的独生子女和少子女家庭. 家庭结构发生了变化, 不少地方形成了`4-2-1'的家庭结构, 一对夫妇要供养4个老人. 不论是目前的家庭收入, 还是夫妇能够用于赡养的精力和时间, 都难以满足老年人的养老需求. 今后, 这些独生子女父母的生活照料和经济保障将成为一个严重的社会问题, 也将对家庭养老产生负面影响. 并且, 持续下降的生育率, 使家庭户规模进一步减少, 趋向小型化、核心化. 家庭结构的脆弱, 使负担老人的能力大大降低, 难以承受人口老龄化和老年人口高龄化的冲击. 另外, 空巢家庭增多加剧了家庭养老的难度, 从我省的情况来看, 尽管多数老年人生活在大家庭中, 但是子女与父母分居的情况越来越多第三, 随着人口寿命的延长, 一方面, 使高龄老人不断增加, 家庭内的代际数将相应增加另一方面, 伴随父母年龄的提高, 子女的年龄也相应提高, 导致低龄老人供养高龄老人的局面产生, 家庭供养能力会有所下降. 所有这些转变毫无疑问都将会对家庭养老产生负面影响, 但就我省目前的情况而言, 社会养老的发展面临较多的制约因素, 如社会保障体系的覆盖面低、缺乏系统的老年服务体系、养老设施严重不足、老年人口的收入水平偏低等等, 社会养老在很大程度上只能作为家庭养老的一个补充. 

\section{应对人口老龄化的不足}
\paragraph{认识方面存在误区}
在应对人口老龄化的过程中有一部分人在对待这一现象时存在观念上的误区, 认为老龄化的出现将是社会的倒退, 是社会经济发展的阻力, 而老年人则是社会发展进步的包袱, 绊脚石. 这一错误观念的存在, 必将影响我们正确地应对人口老龄化. 因此, 我们应积极转变观念, 一是要正确看待人口老龄化问题, 不要把老龄化仅仅看成是危机, 要把老龄化看成是社会的重大成就. 要看到老年型社会象征着人类社会的成熟, 在人口老龄化的过程中, 社会经济的发展并未停滞, 而是日新月异. 人口老龄化可以与社会经济协调发展二是要正确看待老年人口问题. 不要将老年人口作为社会负担, 老年群体是蕴藏着技能, 经验, 智慧的人才宝库, 挖掘老年人潜能, 是建设美好未来社会的重要组成部分. 忽视这一群体的作用, 会导致人才资源的极大浪费, 影响社会经济的持续发展. 

\paragraph{法律保障体系不健全}
目前我国属于社会主义市场经济. 市场经济是法制经济, 在市场经济条件下, 解决老龄化问题也应逐步实现法制化. 解决人口老龄化问题是一项战略任务, 与之相适应的应该是从法制的高度采取对策, 积极地探索人口老龄化问题法制化的道路. 然而, 目前我省有关保护老年人权益的法律法规尚不健全, 侵害老年人各方面合法权益的事时有发生, 因此我们首先要加大有关老年法律法规的执法力度, 法律部门要坚决制裁侵害老年人合法权益的行为, 依法合理调整老年群体与其他群体, 老年人之间的关系, 加强民事调解工作, 促进家庭和睦与社会稳定. 其次, 要加快完善老年立法步伐. 在老年人的权益, 老年事业的发展, 老年人参与社会和经济发展, 老龄工作的性质, 地位, 职能等方面, 加强立法和执法, 形成一个科学的法律机制, 使老龄化问题得以正确有效的解决. 同时要在“十一五”期间, 要尽快出台养老保险, 医疗保险, 社会救济, 老年人福利等有关社会保障方面的法律法规, 使老年人的生活得到切实保障, 形成以老年人权益保障法为基本法的老年法律体系第三, 要进一步弘扬中华民族的传统美德, 加大宣传普及老年法的力度, 将老年人法规列入国家普法教育计划, 加强执法检查力度, 积极开展老年人的守法教一育和思想政治教育工作, 表彰敬老养老先进典型, 依法惩治残害和虐待老人行为, 营造出健康老龄化的良好社会环境. 

\paragraph{老龄工作机构不完善}
应对人口老龄化的发展, 需要有健全的工作机构来完成繁重复杂的工作任务. 然而目前我省的老龄工作机构大多为临时性的机构, 具有不稳定性, 其工作职能也较虚, 无权威性, 形同虚设. 此外, 目前我省现存的老年机构五花八门, 不统一. 老年管理服务分散, 运作低效, 缺乏强有力的调控机制. 陕西省的老干部门, 人事部门, 劳动社会保障部门, 民政部门, 甚至工会组织等多个部门都参与了有关离退休职工的管理工作. 城乡基层老年群众组织大多由老龄办事机构管理. 这种多部门, 多层面的老年管理格局形成的分散管理与现代化大生产越来越不相适应, 甚至在一定程度上对整体老龄保障事业造成负面影响. 这些问题的存在, 随着时间的推移将会逐渐突出, 影响老龄工作的发展. 随着人口老龄化的发展, 老龄机构不管以什么形式存在, 都必将是一个强有力的部门, 我们应尽快建立和完善老龄工作机构. 

\paragraph{农村老龄问题堪忧}
如前所述, 我省人口老龄化具有农村的老龄化的进程快于城镇, 农村人口比城市人口更老的特点. 因此, 随着人口老龄化的发展, 由于大量农村青牙上年劳动力外流而引起的劳动力老化, 农村高龄老人, 病残老人的供养, 照料, 护理, 精神慰籍等越来越成为一个严重的问题. 而我省农村绝大部分地区尚未建立社会养老保险制度, 农村新型合作医疗制度目前还处在试点阶段, 农民的养老, 医疗都缺乏必要的社会保障, 农村的养老, 医疗等方面的压力相对城镇将更加突出. 此外, 农村人口老龄化的快速发展, 使得看管孩子和下地种田的大多为老年人, 而“老年农业”现象的发展无疑也会对农业现代化的发展产生不利影响, 这些应引起我们的足够重视. 

\paragraph{对银色资源开发利用不足}
虽然总的来说, 人口老龄化在宏观和主导方向不利于经济发展, 但老年人口作为``次要劳动力''不是纯消费者, 并非仅是受社会供养而无所作为的群体, 他们中间仍有一部分人有能力从事经济活动, 老年人长期积累的知识和劳动经验相对丰富, 对社会, 经济, 文化科技等的发展, 都有着很重要的作用, 可以弥补体力的不足, 这样在一定程度上有助于经济的发展. 同时老有所为也是一种积极的养老, 部分老年人要求而又能够“老有所为”, 这样对国家, 社会, 老年人都有好处, 可以说是一举三得. 然而从有关资料统计和抽样调查来看, 我省60岁-69岁的低龄老人, 占老年人口总数的61.54\%, 且相当一部分老年人还有工作的能力和工作的愿望, 但高达84.2\%的老年人认为社会上适合自己工作的机会很少, 有89.8\%的老年人没有从事有经济收入的工作. 因此, 我省在老龄人力资源开发利用方面不足, 仍大有文章可做. 我们应该在政策方面对老年人才作适当的调整, 以充分利用老年人力资源为
陕西的经济建设做贡献. 

\section{人口老龄化对策}
人口的发展应该是与资源, 环境, 经济和社会发展相统一的. 从可持续发展理论看, 人口老龄化是社会经济发展不可避免的趋势, 越来越成为一个重要的社会问题. 通过上面的分析, 我们可以看出, 人口老龄化将会从多个方面影响我省社会经济的发展. 面对人口老龄化的挑战, 我们既不能视而不见, 采取不承认的态度, 也不能盲目悲观, 消极害怕. 我们应该根据我省老龄人口发展的趋势和可能带来的后果, 以科学的态度对待人口年龄结构老龄化问题. 我们应该认识到, 人口老龄化对社会经济的可持续发展而言, 是一把双刃剑, 可能促进也可能阻碍社会经济的发展. 在老龄化的不同阶段对社会经济发展的影响是不同的. 在人口老龄化的初始阶段, 也就是社会负担相对较轻的“人口红利期”, 如果我们有效利用初始阶段的红利效应, 做好充分的准备, 在人口老龄化的重度阶段也可以实现与社会经济的协调发展. 反之, 如果我们不能积极采取措施充分利用这一有利时机, 没有做好应对人口老龄化的准备, 就可能使人口老龄化的重度阶段即“人口亏损期”的负面影响倍增, 从而阻碍社会经济的发展进程. 因此, 我们既要积极采取措施决策, 充分利用人口老龄化前期总抚养比较低的有利形势, 大力发展我省的社会经济又要认真研究人口老龄化所带来的负面影响和消极后果, 尽量避免或减少人口老龄化对我省社会经济发展所造成的不良影响.

\subsection{加快经济发展}
在社会经济的运行过程中, 生产居于支配地位, 起决定和主导作用, 生产出来的物质财富的种类和多少, 决定了可供分配, 交换和消费的产品的种类和数量. 如果生产发展不上去, 就谈不上社会物质财富的极大丰富, 也就谈不上改善人们的生活. 只有生产发展了, 社会物质财富的种类和数量才会充足, 才能有足够的实力应对人口老龄化给社会经济发展带来的种种问题. 西方发达国家的经验也证明, 他们出现的``社会保障危机''是由于经济发展速度缓慢和严重的失业问题所造成的, 而解决人口老龄化问题的根本出路在于加快经济的发展. 

我省的人口老龄化是在经济尚不发达的情况下出现的, 人口老龄化的加快发展超前于我省经济发展的客观现实则更加要求我们应加快经济的发展, 以强大的经济实力作后盾来解决老龄化所带来的社会问题. 决定经济增长主要有三个因素, 即资金的投入, 劳动的投入和科技水平. 而人口老龄化会使社会消费基金的比重上升, 积累基金的比重下降, 社会适龄劳动人口减少. 这些会使资金投入和劳动投入受到限制, 从而会降低经济增长的速度. 所以人口老龄化日趋严重的情况下, 要促进经济的快速发展, 就应注重教育的发展, 注重科技的进步, 不断提高生产要素的质量, 合理配置生产要素, 这样单位投入的产出就会增加, 也会使社会产出总量增加, 从而能保持经济的持续增长并满足人口老龄化的各种需要. 综上所述, 要应对人口老龄化给我省带来的挑战, 其根本途径在于大力推进科学技术进步, 保证经济的快速, 稳定, 持续发展, 为人口老龄化奠定坚实的物质基础. 

\subsection{优化劳动力资源配置}
人口老龄化意味着在经过一段年龄推移之后劳动力年龄人口比例会相对下降, 劳动力资源相对减少, 这在一定程度上会影响社会生产和开发, 不利于经济的发展. 另外, 人口老龄化随着时间的推移会造成劳动力人口老化, 而老龄劳动力在接受新知识和科学技术方面要比轻壮年慢, 对新兴产业和就业岗位的适应能力也相对较弱一些, 企业的新产品开发和技术革新也受到一定影响, 对科学技术迅速发展和经济的增长显然是很不利的. 为了有效解决人口老龄化对劳动力资源的影响, 应做到以下几个方面:

\paragraph{有效利用丰富的劳动力资源}
人口老龄化对我省劳动力供给的总的影响趋势是减少劳动力的供给. 因此, 我们必须抓住时机, 有效地利用我省丰富的劳动力资源, 要改革就业制度和用工制度, 大力开放劳动力市场, 实行多渠道, 多形式的就业, 使适龄劳动人口都能及时地转变为经济劳动人口, 以推动我省经济的持续发展. 在城市中发挥劳动力资源的优势, 要广开就业渠道, 妥善地安排停产, 半停产企业的下岗人员在农村要为广大的剩余劳动力寻找出路. 只有做好这项工作, 才能把劳动力资源优势转变为经济发展的优势.

\paragraph{重视劳动力资源开发,  提高生产率}
按照经济学家哈比森的说法, 人力资源是国民财富的最终基础, 资本和自然资源都是被动的生产因素. 人作为资本积累, 是开发自然资源, 建立社会经济和政治组织并推动社会发展的主动力量. 人力资源是以劳动者的数量和质量来表示的存在于人的生命有机体的一种国民经济资源. 经济增长不仅与投入的劳动力数量有关, 更重要的是取决于劳动力质量的提高. 特别是世纪是知识, 经济和高科技发展的时期, 知识更新和高科技发展导致产业结构和职业结构的变化, 对劳动者智力的需求高于对体力的需求. 如发达国家和地区以技术密集型产业为主, 提高劳动生产率主要靠科学技术, 对职工的脑力的需求高于对体力的需求. 劳动力人口老龄化对提高劳动生产率的不利影响较小. 所以, 在发挥我省劳动力资源优势的时候, 不仅重视它的数量资源, 更要注重提高劳动者自身的素质. 因此, 要对付日益严重的人口老龄化, 必须重视劳动力资源的开发. 通过提高科技水平和人口素质, 增大其对生产要素中劳动力数量的替代作用. 要落实科技兴省的战略, 为不同年龄层次的人口提供针对性的教育机会, 即要抓复合型人才`的培养, 造就更多面向现代化, 面向世界, 面向未来的高素质新人, 又要抓在职人员和下岗待业人员的职业技术教育培训, 特别是要加强对年长劳动力的培训, 加快其知识更新, 提高就业竞争力. 总之应以科技带动社会生产力的发展, 逐渐削弱人口老龄化对劳动生产率和劳动供给的负面影响. 

我们应该看到, 随着知识经济的发展, 加快了社会生产力的发展, 为包括老年人口在内的有劳动能力人口提供了就业机会今后随着生命科学技术的进步, 人口平均寿命延长, 老年健康状况逐步改善, 将使老年劳动力对经济发展做出比以往更大的贡献. 因此, 应该通过鼓励他们积极参与劳动力市场竞争重新工作, 使他们的经验和才`干得到充分的运用. 对于老年人兴办实体, 兴教办学, 种植, 养殖等, 各级政府部门都应在注册, 税收等方面制定优惠政策, 积极扶持, 以推动“老有所为”事业的发展, 以便合理开发利用老年人力资源, 减轻人口老龄化对社会经济发展的不利因素. 

\subsection{稳定储蓄率,  保障投资}
根据前文的分析, 虽然在人口老龄化的初始阶段, 陕西省人口老龄化对储蓄的影响相对较小, 但是从长期趋势来看, 由于老年人口的储蓄水平相对较低, 老年人口的增加会带来总储蓄水平的降低, 因此人口老龄化对储蓄率的提高影响是消极的. 随着人口老龄化的加速, 特别是当老年人口内部的高龄老人增加时, 势必要影响储蓄, 从而对经济的发展产生不利影响. 尤其在人口老龄化的成熟和稳定阶段, 人口老龄化对储蓄的影响将逐步发生变化. 由于在业人口总量的相对减少, 会带来相应的储蓄量的减少. 因此, 强化个人储蓄养老, 筹措养老基金成为资本投资的重要来源. 发达国家通过税收的形式从员工的工资中扣除一部分作为退休后的养老基金, 即采取强制性储蓄制度, 用工资税来筹集养老基金. 现在美国75\%新投资的来自养老基金, 促进了美国经济的发展. 因此我们应借鉴发达国家的这一经验模式, 建立完善的养老保障制度, 从而通过强制性储蓄制度稳定储蓄率, 从而达到扩大投资, 刺激经济发展的目的.

\subsection{调整产业结构,  加快发展第三产业}
在发达国家和地区第三产业中, 社区, 家庭和个人服务行业从业人员一般占50\%以上. 而我省第三产业水平较低, 因此就业发展前景广阔, 此外, 人口老龄化带动发展的第三产业主要集中在劳动密集型的行业, 如社区服务业, 家庭服务业, 个人服务娱乐等. 这些行业服务项目比较简单易做, 对劳动力质量的要求比较低, 适合质量不高的劳动力就业需求. 而根据省委就业调研小组的调研报告的统计, 2002年我省劳动适龄人口2228万人, 劳动力2109万人. 其中城镇劳动力478万人, 农村劳动力1631万人. 城镇劳动力中处于失业状态的41.8万人, 农村``剩余劳动力''则为370万人. 因此顺应人口老龄化, 家庭小型化以及家务劳动和家庭生活化的大趋势, 努力发展以服务业为主的第三产业, 不仅有助于调整我省国民经济增加值的产业结构, 而且有利于促进劳动力的产业转移, 特别是促进农业剩余劳动力向第三产业转移, 以实现劳动力就业的产业结构调整. 这将对未来陕西省社会经济的持续发展和稳定产生极大的影响. 

\subsection{大力发展老龄产业}
老龄产业是为老年人提供商品和服务的产业, 是为了满足老年人物质和精神生活需求而形成的产业, 既包括生产性产业, 也包括服务性产业, 是解决人口老龄化问题的重要手段. 老龄产业作为老年保障事业的重要组成部分, 它的发展标志着一个国家文明和社会进步程度. 它的发展取决于以一下三个方面的因素一是取决于经济的发展, 老年人收入的提高和赡养老年人的一子女和亲属的收入水平的提高二是取决于老年人特殊的物质, 精神生活的需要, 如果老年人没有特殊的需要就不会有老龄产业三是取决于政府是否鼓励和支持发展老龄产业. 

老龄产业是世纪商家, 企业的前瞻性产业, 有着广阔的市场, 潜力巨大, 有可能成为另一个经济增长点. 然而目前老龄产业在我省的发展是很薄弱的, 老龄产业现今还无法满足老年人的需求, 几乎是一片空白. 我们缺乏专门为老年人服务的专业医护和服务人员缺乏足够的养老机构缺乏上门服务的保健机构, 而这些服务产业和岗位将创造许多新的工作机会, 缓解社会就业压力. 再看看老年消费市场老年人很难买到合适的服装鞋帽, 老年食品也很少研究开发, 老年药品, 老年保健护理用品以及其他各种老年商品都处于匾乏状态. 这一方面是由于我省经济发展相对落后, 老年人及赡养者的收入较低, 不大可能对市场提供很多关于老年人的需要, 因而难以达到形成一个产业部门的要求. 另一方面, 由于我省目前还仍是以年轻老人为主, 因此对发展老龄产业的需求也不是很迫切. 但是, 我们应该看到快速发展的人口老龄化进程和不断增加的老年人口规模以及我省经济持续, 稳定的增长, 将为我省老龄产业的发展创造条件. 我们必须审时度势, 抓住这一千载难逢的机遇, 积极推动老龄产业的发展. 而且当老年人成为一个庞大的群体, 其特殊的需求必然会对我省经济, 社会发展各方面产生特殊的影响, 同时这也蕴藏着巨大的机遇和良好的市场. 

在推动老龄产业发展的过程中, 政府应起主导作用. 政府应该研究和制定积极促进老龄产业发展的政策和规划, 为老龄产业的发展营造一个有利的大环境. 将老龄产业的发展纳入国民经济发展的整体规划之中, 并采取倾斜政策, 优先发展. 同时发展老龄产业, 必须要针对我省具体省情, 把握老龄产业发展的阶段性特征, 认真分析老年人的消费需求特征, 以市场机制带动老龄产业发展, 根据人口老龄化发展趋势, 围绕老年人物质需求和精神需求, 对一些产业进行结构调整, 开发生产适用对路的各种老年用品, 鼓励和引导老年产品市场的发展, 实现老龄产业从规模到结构, 从速度到效益, 从产品到服务, 更加充分, 全面的发展. 此外, 经济管理部门应运用市场机制, 制定一些必要的优惠政策, 充分利用金融杠杆作用, 在税费征收政策上, 考虑扶持老年服务产业的发展.

\subsection{建立健全全社会养老保障制度}
从国际经验看, 在人口老化高峰的至少20-30前年, 必须建立起相应的社会养老保障制度, 进行足够的资金储备, 以应对人口老化高峰给经济发展和社会稳定带来的压力. 而陕西20世纪60-70年代生育高峰时出生的人口将在2030年前后进入老年期, 届时人口老化进程突然加速, 达到人口老化的高峰. 我们准备的时间恰好有二, 三十年. 因而从现在起就要抓住未来20年人口抚养负担轻, 劳动力资源丰富的有利契机, 抓紧建立社会养老保障体系, 完善老年社区保障服务, 大力发展银色产业, 多层次, 全方位地推进老龄化事业和产业的快速发展. 

由于陕西省人口老龄化的速度比较快, 并且属于“未富先老”的这一事实, 我们没有条件去实施类似于西方国家那样覆盖全社会的养老保障制度也鉴于当代西方国家福利政策的教训, 我们也不该走那条路, 我们应该从我省实际出发, 走渐进式, 逐步扩展完善的养老保障之路. 我们应该在我省人口老龄化推进较慢的时期内, 逐步形成包括集养老保险, 社会福利, 社会救济于一体的养老保障框架在人口老龄化加速推进进入严重阶段, 主要应致力于完善养老保障体系和制度建设而当人口老龄化速度减慢并趋于稳定的阶段时, 养老保障制度也应相对稳定, 只是对个别不适宜的部分作出相应的调整. 毫无疑问, 构建全社会的养老保障体系需要加大财政投资, 政府扮演主力军的角色. 然而从目前人口老龄化的实际及社会经济持续发展的需要出发, 我们还必须调动各方面的积极性, 挖掘各种潜力, 盘活各方面的资产, 建立一个城乡有别, 地区有别, 多层次的既能解决人口老龄化带来的种种问题又能推动社会持续稳定发展的社会养老保障制度. 而不能走单一的道路解决养老问题. 因为目前政府无法包揽所有老人的养老, 而完全依赖家庭养老又面临新的困难. 所以要走以国家, 社区, 家庭和个人相结合的养老道路. 要大力推进社区老年福利服务, 卫生保健, 文化体育等各项事业发展. 特别是要加快以社区为中心的老年照料服务体系的建设. 在社区建立综合性, 多功能的服务站, 依托社区服务设施, 采取上门, 定点等多种服务形式. 要建立和完善老年社会保障体系, 必须按照因地制宜的原则, 结合城乡老年人不同的特点和需要, 进一步建立和完善老年社会保障体系. 

首先, 在城镇要实行由社会统筹和个人帐户相结合的覆盖各类从业人员的基本养老保险制度, 鼓励企业开展补充养老保险, 提倡发展个人储蓄性养老保险和商业保险, 形成基本养老保险, 补充养老保险, 个人储蓄养老保险与商业保险相结合的养老保险体系. 确保离休人员养老金的按时足额发放. 并建立养老金的随机调整的合理机构, 使广大老年人共享经济和社会发展的成果. 

其次, 在农村重点要在建设社会主义新农村新形势下, 进一步完善社会救济和以保吃, 保住, 保穿, 保医, 保葬为内容的``五保''供养制度普遍实行农村新型合作医疗制度, 并在有条件的地方建立农村居民最低生活保障制度和试点推行社会养老保险制度. 要巩固和发挥家庭养老的功能, 探索和推行适合我省农村特点的养老救助制度, 在已试行的农村养老保险的基础上适时开展个人养老储蓄, 商业保险等多种形式的养老保险. 并逐步建立老年人福利补贴制度和特困, 高龄老人社会求助制度. 

第三, 以政府为主体, 完善以最低生活保障和大病医疗救助为主的城乡社会救助体系, 切实保障老年人最低的生活需求和医疗需求, 构筑起公平, 公正的社会救助网. 目前我省的社会救助项目覆盖范围不广, 救助力度小, 资金来源不稳定, 难以惠及大多数老年贫困人口, 特别是农村老年贫困人口. 今后应以政府投入为主体, 辅以社会捐助机制, 完善城乡统筹的最低生活保障制度和大病医疗救助制度, 实现``应保尽保''和``应救尽救''. 要在健全和完善城市社会养老保险, 医疗保险体系的同时, 大力推广城乡困难群众的医疗救助制度. 

\subsection{适时调整人口政策}
计划生育政策是我国的基本国策, 是为实现可持续发展战略目标而对人口进行调节的手段. 计划生育通过生育率的调控达到对人口数量和人口结构的调节. 自从计划生育政策实施以来, 在一定程度上缓解了人和资源的矛盾, 促进了经济和社会的可持续发展. 生育率迅速下降对陕西的经济增长做出了重要贡献. 

然而人口老龄化主要是由出生率下降引起的, 生育率下降的程度可以加速或延缓人口老龄化的过程. 我省的人口出生率由1970年的29.28\%下降到2004年的10.59\%, 妇女的总和生育率由1953年的5.3下降到2004年的1.68, 低于全国平均水平. 人口生育率的急速下降与我省实行的严格的生育政策紧密关联. 虽然人口转变是社会经济发展和计划生育相互作用的结果, 但社会经济因素在人口转变中起了重要的作用, 计划生育不是我省人口老化的根本原因, 因为即使没有实施计划生育政策, 陕西省的人口也会伴随着社会经济的发展而产生老龄化. 但不可否认的是, 计划生育是陕西人口加速老化的主要原因. 计划生育政策大大地加快了我省生育率水平的转变, 而加快了生育率的转变也就加快了人口的老龄化. 

在这样一种严峻的人口形势之下, 有关人口政策方面的争议始终就没有平息. 人口学家们大多认为, 现行政策是多年来历史经验的结晶, 不能轻易松动. 其主要理由有三一是人口数量增多会对经济发展有压力劳动就业压力, 失业人口压力二是人口数量对资源环境的巨大压力三是提高生育率调整结构并不能改变未来老年人口数量增加, 无助于老年人口问题的解决. 其中的核心在于人口数量增多不仅对缓解老龄化问题于事无补, 而且还会带来巨大的就业压力. 与此同时, 少数人口学家和几乎所有介入其中的经济学家都认为, 在保持控制人口基本国策不变的大前提下, 人口政策需要适时进行微调. 首先, 不能只看到人口对经济的负面影响, 更需要看到人口对经济的正面影响. 其次, 人口数量对资源环境的巨大压力会随着技术进步和经济社会发展而逐步化解最后, 提高生育率调整结构虽然并不能改变未来老年人口数量, 但却能够改变未来的人口结构, 从而有利于老年人口问题的解决. 

面对人口老龄化迅速发展的新形势, 我们的政府, 人口工作部门应该重新审视一下我们一贯的计划生育政策, 要为计划生育政策注入新的内容, 要针对新的形势和发展作出必要的调整. 但计划生育政策对人口的调整是要遵循人口自身规律的, 是要有前瞻性的. 我们主张应以全面, 协调, 可持续科学发展观为指导, 把现行的追求低生育率只盯在控制人口数量上的生育政策平稳过渡为“低生育率水平与调控人口年龄结构并举的政策”, 也就是说人口数量和人口结构问题并重. 通过人口数量, 素质, 结构的合理变动, 积极促进人口与经济, 社会以及资源, 环境的和谐发展. 因此, 在制定我省长期的人口战略规划时应避免两个方面的问题一是生育率过高造成人口增长过快另一方面是生育率过低造成的人口年龄结构老龄化过快. 然而对现行的人口计划生育政策适时的进行调整, 并不是说计划生育政策要松动. 因为众所周知, 陕西的低出生率是在陕西经济尚不发达, 广大群众的生育观还没有彻底改变的情况下, 通过政府的强控措施来实现控制生育的目的. 特别是在经济文化落后的农村地区, 情况更是这样. 所以, 计划生育工作在一些地区特别是落后的农村地区, 应继续搞好控制人口数量的工作, 保证低生育率水平另一方面, 对于生育率水平较低的地区, 特别是相对较发达的城市地区, 如果总生育率己经下降1.0到以下, 则不要继续维持这样低的生育水平, 而是应该采取措施, 刺激生育率回升, 以达到调控人口结构的目的. 只有这样, 陕西省的人口在未来才不至于过分老化, 才可以避免带来许多今天预想不到的严峻问题, 也只有这样, 才符合可持续发展的战略目标. 

\newpage
\nocite{*}
\bibliographystyle{plain}
\bibliography{ref}
\addcontentsline{toc}{section}{参考文献}

\end{document}