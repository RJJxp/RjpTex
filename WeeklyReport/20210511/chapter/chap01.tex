\section{基于植被指数时序谱类内差异特征的冬小麦遥感识别研究}

\subsection{选题背景与意义}
近年来,全球气候变化、生物能源发展以及城市扩张等对农业生产带来了前所未有的影响,全球农业可持续发展面临着巨大的威胁,粮食供应短缺风险不断加大,粮食安全问题被各国广泛关注。中国作为产粮大国,粮食基本实现了自给,但在新时期下依然面临严峻的粮食安全问题。中国的粮食安全问题,不仅是我国政府高度重视和亟需解决的首要问题之一,也是社会各界普遍关注的一个敏感话题。

小麦是一种全球性的重要粮食作物。准确地获取冬小麦种植面积及其空间分布信息是关系社会经济发展、国家粮食安全、农业政策制定以及生态系统功能和服务的重要问题,也为国民经济可持续发展提供科学管理决策依据。

传统的冬小麦种植面积信息获取途径主要是依靠地方行政单位统计,然后逐级上报或者采用一定比例的样方抽样调查获取。不但效率低下且需要耗费大量的人力、物力、财力资源。难以满足当前快速、准确地获取农作物相关信息的实际需求。

遥感技术其覆盖面积广、重访周期短、现势性强、数据获取相对容易和费用低廉等优点,为快速和准确地收集大范围的农业资源和农业生产的信息提供了强有力的技术手段,其在农业领域的应用也越来越受到人们的普遍关注。农情遥感也经历了从单一作物到多作物、从定性到定量分析、从田间试验研究到业务化运行应用,遥感也成为全面、快速、准确地监测地面农作物信息的主要途径。

随着遥感平台的不断发展及影像分辨率的不断提高,遥感技术在小麦监测领域中得到了广泛的应用。由于农作物类型复杂多样,不同作物之间存在明显的光谱重叠,利用单一时相遥感影像数据进行冬小麦识别经常会出现“错判、漏判”的现象,很难达到理想的分类精度。遥感影像时间分辨率的显著提高,为多时相的选择和时间序列数据的构建提供了丰富的数据源,也为因单一时相遥感影像进行农作物分类时出现的“错分、漏分”问题提供了有效的解决途径。

国产高分系列卫星的出现,为高分辨率植被指数时间序列的构建提供了可能,时序特征的获取为冬小麦关键特征识别研究提供了重要的识别手段但与此同时大量的细节信息会大大增加图像分类识别的难度,大量遥感数据的涌入对遥感特征提取方法和技术也有了更高的要求。基于目前己有的技术方法很难继续发挥兼具高时、高空间分辨率的遥感数据地面信息识别提取的优势。如何在大量的遥感数据中挖掘有用信息,针对研宄对象、目的与研究区特点,恰当选择卫星影像数据并合理用于农作物遥感识别与提取是当前冬小麦遥感识别研究的重要方向。因此,如何高
效地利用信息量大增的遥感数据源,综合表达冬小麦信息在新数据源的光谱特征、空间特征与时间特征中的表现形式,对冬小麦遥感分类识别等信息提取技术具有重要的现实意义。

\subsection{国内外研究进展}
遥感分类识别方法主要分为两大类,基于专家知识的人工目视解译方法和基于计算机的自动分类方法。

早期的遥感图像分类识别主要是基于目视解译的方法,根据己有的先验知识或者专家决策,通过目视解译遥感影像中各地物的特征进行遥感分类识别。

计算机自动分类方法是目前应用最广泛的遥感影像分类方法,技术和方法都已相对成熟,主要包括监督分类和非监督分类。常用的监督分类方法主要有最大似然分类、最小距离分类、光谱角制图、光谱信息散度分类、平行六边形。还有近来年不断发展的支持向量机、人工神经网络、随机森林、模糊聚类等智能化分类方法将遥感图像分类过程进一步推向自动化。

NDVI时间序列数据能够精确地反映植被物候信息(出苗、拔节、抽穗、成熟),有效削弱“同物异谱,同谱异物”现象,在农作物遥感分类识别与信息提取研究中发挥了重要作用。大尺度的农作物遥感分类识别研究多是基于中低分辨率遥感影像,但受影像空间分辨率的限制分类精度一直难以提高。

着近年来遥感技术的飞速发展,多源数据获取的更加容易,对遥感影像光谱特征、空间特征、时序特征、物候特征的选择以及对各种辅助数据特征的提取成为农作物遥感分类识别研究中的重点环节。而基于多光谱遥感数据的作物分类识别研究中,特征参量的选择主要从3个方面进行,光谱特征、空间特征与时序特征:基于光谱特征的各类指数,基于空间信息的各类纹理特征,以及基于时序差异的物候特征。

\paragraph*{基于光谱特征的农作物遥感分类识别研究}

基于光谱特征的农作物遥感分类识别最初是以目视解译为主,即凭借光谱规律、地学规律和解译者的经验,结合多光谱波段组合的假彩色影像识别不同的作物类型。随着遥感技术的发展,光谱植被指数被广泛应用于植被遥感分类,也成为基于光谱特征的农作物遥感分类的主要方式。

\paragraph*{基于时序特征的农作物遥感分类识别研究进展}
目前NDVI时间序列在冬小麦种植信息的识别、提取等方面的研究越来越多.

\paragraph*{基于物候特征的农作物遥感分类识别研究进展}

己有研究表明,不同作物类型因物候特征在多光谱影像中的表现不同,可以对其进行区分,而物候特征本质上是基于时序特征进一步分析得到的。在基于遥感影像信息获取作物的生长发育状态时,如何解决遥感信息与作物生长变化状态信息进行匹配和转换,并进一步将不同作物的信息进行分离,是将物候信息用于农作物遥感分类与识别的前提基础.

\paragraph*{基于空间特征的农作物遥感分类识别研宄进展}


\subsection{存在问题}
\begin{itemize}
    \item 由于冬小麦具有与其他地物不同的物候特征,开展植被指数时间序列冬小麦遥感识别时,已有的研宄方法没有充分利用其典型的物候特征形成的时序波谱特征
    \item 基于时间序列遥感影像进行冬小麦识别研究时,没有充分考虑时效性问题
    \item MODIS遥感数据产品是大区域冬小麦遥感制图的主要数据源,根据已有的研宄成果总结发现,基于MODIS数据的冬小麦识别精度一直维持在85%左右,难以满足日益增加的应用需求,所以基于MODIS的冬小麦识别精度仍有较大提升空间和需求.
    \item 新的遥感分类识别方法在不同空间分辨率影像上的适用性问题,给信息提取提出了新的挑战
\end{itemize}

\subsection{研究内容}

\begin{itemize}
    \item 基于N维矢量空间方向和距离的冬小麦遥感识别矢量分析模型。
    \item 冬小麦遥感识别矢量分析模型的时间序列谱段选择研究. 基于本文提出的冬小麦遥感识别矢量分析模型,就如何选择时序谱段、减少识别所需遥感影像问题开展进一步研究
    \item 考虑植被指数时序谱类内差异特征的冬小麦遥感识别矢量分析模型
    \item 考虑植被指数时序谱类内差异特征的冬小麦遥感识别矢董分析模型的适用性评价以及影响因素分析
\end{itemize}

















