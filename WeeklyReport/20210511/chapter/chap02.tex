\section{基于多季相Landsat8 OLI影像的多特征森林植被分类}

\subsection{研究背景与意义}
我国森林资源结构复杂,分布范围广泛,分布不均,森林类型多样,这给国家森林普查带来了巨大的困难,准确、快速、高效的掌握森林资源的结构和分布现状,是森林经营的必然要求。利用时序遥感数据进行快速、准确的森林植被分类,可以有效的缩短国家森林资源信息的更新周期,解决人力物力资源消耗大的问题,并为制定林业宏观规划、政策方针和森林经营方案提供了必要的数据支持.

传统森林资源清查是以地面调查为主,该方式周期较长,成本很高,且工作量很大,往往准确性和时效性很难满足实际应用需求. 如何在大量的数据中,快速、准确地提取所森林植被分类是我们需要解决
的重大问题之一。目前,对森林植被快速、准确的分类精度具有更高要求,期望能准确的获取森林植被分布的现状;而海量的数据对林业遥感的分类提出了更高的要求,因此,研宄基于多时序影像进行森林植被精细提取的方法具有重要应用价值和研究意义。

\subsection{国内外研究现状}
监督分类, 非监督分类. 人工神经网络, 支持向量机.

\paragraph*{多特征分类}
\begin{itemize}
    \item 植被指数特征: PVI、MSAVI、ARVI、NDVI
    和AVI等
    \item 纹理特征: 主要为遥感图像的视觉特征
    \item 时序特征: 植被指数时序分类可分为两类,一种是利用植被指数的光谱变化曲线直接分类,常采用非监督分类法;另一种则是将植被指数时间谱进行参量化,在此基础上进行分类,多采用决策树法.
\end{itemize}

\subsection{研究内容}
\begin{itemize}
    \item 时序Landsat-80LI影像的预处理
    \item 时序影像的不同类型特征提取,包括植被指数和纹理特征
    \item 基于随机森林(RF)的特征重要性排序
    \item 最佳特征数的确定
    \item 研宄区的植被分类
\end{itemize}



