\section{基于光谱时间序列拟合的中国南方水稻遥感识别及面积估算方法研究}

\subsection{研究背景与意义}
粮食安全关系着一个国家的社会稳定、经济稳定、政治稳定以及可持续发展. 受非农建设用地占用、自然灾害毁坏以及退耕还林还草,尤其
是耕地荒废等因素的影响,我国耕地面积逐年呈下降趋势, 必须通过有效手段掌握粮食种植面积,才能为粮食估产提供有效的数据。快速、准确、可靠的农作物种植面积监测与调查.

我国农作物种植面积调查与估算的传统方式主要依靠地方上报和抽样调查方法, 需要耗费大量人力物力,受样方数量和分布的影响,面积估算精度控制困难. 遥感技术在农作物识别和面积估算方面已经相对可靠.

\subsection{研究现状}
\begin{enumerate}
    \item 水稻遥感识别特征研究
    \item 水稻分类方法研究
    \item 农作物种植面积估算方法研究
\end{enumerate}
[1和2是一个内容] 监督分类, 非监督分类. 支持向量机、模糊分类、人工神经网络、随机森林和面向对象等各种分类方法.

农作物种植面积估算方法研究: 像元尺度农作物遥感面积估算, 基于混合像元分解农作物遥感面积估算, 多尺度遥感影像相结合的农作物面积估算, 基于高分辨率影像面积修正农作物面积估算


\subsection{研究内容}
\paragraph*{水稻识别方法}
\begin{itemize}
    \item 水稻识别特征时间序列拟合曲线数字化定量描述。
    \item 研究水稻(中稻、晚稻)遥感识别阈值范围确定方法 
\end{itemize}

\paragraph*{水稻面积估计}
统计出外业调查样方内的农作物真实种植面积,通过线性回归方法计算出水稻面积修正参数,对 GF-1 号 WFV 数据水稻的分类统计结果进行修正.
