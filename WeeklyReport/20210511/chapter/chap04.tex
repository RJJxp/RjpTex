\section{基于时序MODIS影像的农作物遥感识别方法研究}

\subsection{背景与意义}
农作物种植面积及空间分布信息对于粮食安全、全球变化以及农业可持续发展等都具有重要的现实意义. 及时、准确地获取农作物空间分布及其时空动态变化信息不仅是监测农作物长势、估测区域农作物产量、研究区域粮食平衡的核心数据源,也是农作物结构调整和布局优化的主要依据. 遥感技术因其覆盖范围大、探测周期短、现势性强和费用成本低等优点,已经广泛应用于对地观测活动中,为大范围的农作物空间分布信息快速和准确获取提供了新的技术手段.

由于遥感数据源种类繁杂、农作物严重的类内光谱变异性、耕地地块破碎等问题,使得农作物遥感识别的精度和效率受到了制约。开展具有强普适能力、高精度的区域作物遥感识别方法研究对于推动我国农情遥感监测和农业现代化发展都具有重要的现实意义。 

\subsection{研究现状}
基于时间序列影像的农作物遥感识别方法
\begin{itemize}
    \item 基于单一特征量的方法: 基于时间序列影像单一特征量的农作物遥感提取方法操作简单、效率高,其引入了农作物的时间变化特征,在不同农作物的区分和识别中具有明显优势,尤其对农作物种植结构比较简单的区域提取效果较好。然而,该方法也存在一些不足。如该方法通常选择 EVI 或 NDVI 作为特征量,特征量的选取具有较强的主观性,缺乏特征量的敏感性分析。另外,单一的 EVI 或 NDVI 特征量对于作物类型复杂多样的区域存在局限性,因为该特征量未必是所有作物识别的最优特征量,使得不同农作物提取的精度差异较大。
    \item 基于多特征量的方法: NDVI、EVI 以及单波段光谱特征以外的纹理特征、地形(如高程、坡度和坡向信息等)、土壤、作物分布环境等特征量,也逐渐引入到农作物遥感识别中,两者结合可以较好改善农作物空间分布信息的提取精度. 基于时间序列影像多特征量的农作物空间分布信息提取方法较好利用了多维特征向量的集成优势,可以有效地解决混合作物交界处和内部光谱混合或变异的问题,适用于农作物种植结构复杂的区域。然而,特征向量的增加会降低数据处理和运算的效率,也会带来误差的累积,因此,如何确定适宜的特征量数量以及选取合适的特征量是需要重点考虑的研究内容。此外,如何实现不同特征量,尤其是光谱和非光谱特征量之间的整合及尺度协同,也是该方法需重点解决的问题.
    \item 基于特征量的统计模型法. 基于``光谱-时序''特征量构建的统计模型一定程度上解决了混合像元问题,使得农作物种植面积提取精度更高。然而,这种经验或半经验模型的稳定性、普适性还需要进一步加强与完善,更好满足不同区域农作物种植结构提取的精度要求.
\end{itemize}

\subsection{存在问题}
\begin{itemize}
    \item 和国外已有的农作物空间分布遥感产品相比,国内农作物遥感提取覆盖的农作物类型数量偏少,主要集中于水稻、小麦和玉米等主要粮食作物
    \item 大区域尺度的农作物遥感提取能力薄弱. 基于遥感与统计数据融合的方法获取了一些全球农作物空间分布产品,但这些数据产品精度低,难以满足具体的应用需求
    \item 如何充分利用多源数据提高农作物遥感识别精度的研究不足. 空间分辨率高的影像往往难以有效进行作物区分,而时间和光谱分辨率高的影像空间分辨率低,混合像元现象普遍。因此,如何处理好光谱、时间和空间分辨率之间的矛盾关系,实现农作物空间分布信息的准确提取是需要重点考虑的问题
    \item 农作物遥感识别方法研究不深入. 预处理引入的不确定性, 光谱和时相选择问题, 分类器的合理选择.
\end{itemize}

\subsection{研究内容}
\begin{itemize}
    \item 光谱和时序特征对农作物遥感识别的影响评估. 构
    建不同数量和质量的光谱和时序特征集组合(单一光谱和单一时序,单一光谱和多时序,多光谱和单一时序,未优化的多光谱多时序,优化的多光谱和多时序等). 比较不同特征组合产生的精度,分析时序和光谱特征质量以及数量对于农作物分类的影响机制.
    \item 全局 SI 指数扩展方法研究以及在农作物特征选择中的潜力评估
    \item 一种基于 SI 指数的时序光谱特征自动优选方法研究  
    \item 一种融合遥感数据与统计数据提取亚像素农作物比例的方法研究
\end{itemize}




