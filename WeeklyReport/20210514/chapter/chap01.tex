\section{基于循环神经网络模型的遥感影像时间序列分类及变化检测方法研究}

\subsection{研究意义与选题背景}

\paragraph*{para01~
    \textcolor[RGB]{17, 205, 29}{土地资源重要性}}
\begin{quotation}
    \itshape
    土地资源作为人类赖以生存和发展的物质基础, 在人类生活, 社会生产中起着不可替代的作用. 监测土地资源的动态变化不仅可以让我们更好地了解自己生活的自然环境, 而且还能够指导我们进行农业生产, 商业布局, 城市建设, 环境保护等等关乎国计民生的事情. 
\end{quotation}

\paragraph*{para02~
    \textcolor[RGB]{17, 205, 29}{土地覆盖分类和变化检测的重要性. 有文献做引证.}}
\begin{quotation}
    \itshape
    获取准确的土地覆盖分类以及变化信息, 能够为其他科学研究提供基础资料, 对于自然资源监测和规划, 环境变化研究以及动植物生态栖息地分布研究具有十分重要的意义 (Gómez et al., 2016).
\end{quotation}

\paragraph*{para02~
    \textcolor[RGB]{17, 205, 29}{土地覆盖分类和变化检测的必要性. 有文献做引证.}}
\begin{quotation}
    \itshape
    随着人口增长和技术革新, 人类活动对自然环境的影响日益加剧, 地表的覆盖类型也随之发生加速变化(Ramankutty and Foley, 1999; Goldewijk, 2001), 如何及时准确地获取这些变化信息, 已经成为目前研究中亟待解决的一个问题
\end{quotation}

\paragraph*{para02~
    \textcolor[RGB]{17, 205, 29}{卫星影像可用于土地覆盖分类和变化检测. 有文献做引证.}}
\begin{quotation}
    \itshape
    卫星影像以其丰富的光谱特征, 宽覆盖范围以及长时间观测记录等特点, 逐渐成为监测土地资源动态变化的重要工具.
\end{quotation}

\paragraph*{para03~
    \textcolor[RGB]{17, 205, 29}{在分类方面, 时序影像优于单影像. 有文献做引证.}}
\begin{quotation}
    \itshape
    遥感影像时间序列由于蕴含显著的物候信息, 使得不同地物拥有不同的光谱反射率轨迹, 利用这些数据特点能够更加有效地区分地物类型, 时间序列在土地覆盖分类方面已经被证明优于单时相分类 (Franklin et al., 2015) 
\end{quotation}

\paragraph*{para03~
    \textcolor[RGB]{17, 205, 29}{在变化检测方面, 时序影像比双时相精度高且可以定量. 有文献做引证.}}
\begin{quotation}
    \itshape
    同样, 对于土地覆盖变化检测, 遥感影像时间序列也具有天然的优势, 基于双时相影像的变化检测方法在检测精度方面往往不如基于时间序列的变化检测精度高. 而且, 由于信息的缺失, 双时相影像无法定量描述变化的过程, 比如变化速率或者变化强度(Gillanders et al., 2008) 等. 
\end{quotation}

\paragraph*{para04~
    \textcolor[RGB]{17, 205, 29}{遥感影像已经有几十年的积累, 论文可行性中的数据方面进行论证. 有文献做引证.}}
\begin{quotation}
    \itshape
    目前, 随着遥感传感器资源的日益丰富, 遥感数据的时间分辨率有了显著提高, 较早的卫星传感器例如 MODIS, 其数据已经有了近几十年的积累, 2008 年陆地卫星 (Landsat) 数据的开放也使得中分辨率遥感影像时间序列的研究成为了热门领域 (Woodcock et al., 2008; Wulder et al., 2012) . 海量存储的多时相遥感数据已经满足了建立时间序列的要求, 基于这些长期积累的遥感数据, 建立相应的遥感影像时间序列数据集并进行应用已经成为了遥感图像处理的一种发展趋势. 
\end{quotation}

\paragraph*{para05~
    \textcolor[RGB]{17, 205, 29}{由于时序数据特点与遥感时序影像的特殊性, 现有数据挖掘算法不能满足需求. 有文献做引证.}}
\begin{quotation}
    \itshape
    时间序列数据的特征包括:数据规模大, 高维度, 持续更新 (Fu, 2011) , 这些自然属性决定了时间序列数据挖掘的难度. 同时遥感影像时间序列自身具有十分强烈的特殊性, 比如不等长, 采样间隔不等, 固有的时间轴畸变 (物候期根据每年的温度等自然条件有所提前或者延迟) , 以年为单位的自然周期性等, 复杂的内部数据结构无疑加大了信息挖掘的难度. 现有的时间序列处理方法不能完全适应遥感应用的需要, 针对这种情况, 有必要采用新的数据挖掘方法, 使之适用遥感影像时间序列处理的要求. 在此基础上, 进一步完成土地覆盖分类和变化检测, 实现对国土资源的长期动态监测.
\end{quotation}

\subsection{国内外研究现状}
\subsubsection{遥感影像时间序列分类方法}

\paragraph*{para01~
    \textcolor[RGB]{17, 205, 29}{遥感影像分类方法分类, 三大类: 数据源不同, 分类框架不同, 分类算法不同.}}
\begin{quotation}
    \itshape
    面向遥感影像时间序列的分类方法多种多样, 作为一项交叉学科研究内容, 流行于机器学习中的分类算法以及时间序列分析中的算法都可以应用在遥感影像时间序列中, 纷繁复杂的研究内容虽然各有特点, 但是其本质的区别在于三个方面:数据源的不同, 分类框架的不同以及分类算法的不同. 利用遥感影像进行土地覆盖分类需要综合考虑这三种因素. 
\end{quotation}

\paragraph*{para02~
    \textcolor[RGB]{17, 205, 29}{针对分类方法中数据源不同,  对数据源进行整体叙述, 时间周期内分布均匀的遥感影像更适合时序影像分类}}
\begin{quotation}
    \itshape
    数据源选取的依据主要是遥感影像的空间分辨率和时间分辨率, 遥感影像时间序列的独特性在于不仅包含地物的空间信息, 也包含地物随时间的变化信息, 相应地, 为了实现特定的应用目标既要考虑遥感影像的空间分辨率是否满足要求, 又要考虑遥感影像的时间间隔是否合理. 一般认为在一个时间周期内分布均匀的遥感影像能够提供更多的时空信息, 更有利于土地覆盖分类.
\end{quotation}

\paragraph*{para02~
    \textcolor[RGB]{17, 205, 29}{对MODIS和Landsat两个卫星的影像数据总结, 并对其不足之处, 介绍改进方法. 更多由于该论文实验数据来自这两颗卫星. 有文献做引证.}}
\begin{quotation}
    \itshape 
    对这两种具有代表性的时间序列进行总结可知, MODIS 时间序列具有低空间分辨率, 高时间分辨率的特点, Landsat 时间序列具有高空间分辨率, 低时间分辨率的特点. 为了弥补两种数据的不足之处, 一些研究人员进行了相关探索, 比如, 为了提高Landsat 影像的时间分辨率, 有人通过提出影像融合模型将 MODIS 时间序列数据和Landsat影像融合, 从而获取高时间分辨率的Landsat时间序列 (Hilker et al.,  2009) . 另外, 对于无法避免的噪声污染, 可以对遥感影像时间序列进行平滑处理, 如 Shao 对 MODIS-NDVI 时间序列在分类前进行了平滑处理, 通过平滑处理能够有效提高分类的精度 (Shao et al., 2016) .  
\end{quotation}

\paragraph*{para03~
    \textcolor[RGB]{17, 205, 29}{两种分类框架. 第一种先分类, 在分类结果基础上检测变化; 第二种对一部分数据分类, 在此基础上检测变化, 在原始数据基础上累加变化, 得到最终分类结果}}
\begin{quotation}
    \itshape
    根据分类流程的不同, 利用时间序列数据进行土地覆盖分类一般分为两种框架 (Gómez et al., 2016) . 第一种是 ``自上而下'' 的分类框架, 即对每个时间周期 (比如一年) 内的时间序列进行分类, 获取独立的土地覆盖分类结果, 然后对这些分类结果进行变化检测后处理, 以此获取土地覆盖的变化信息.

    ...

    第二种是 ``自下而上'' 的分类框架, 首先对一个时间周期内的时间序列进行分类并将其分类结果作为基础分类图, 然后提取时间序列的变化信息对基础分类图进行逐周期更新, 获取每个周期内的土地覆盖产品. 

\end{quotation}

\paragraph*{para04~
    \textcolor[RGB]{17, 205, 29}{分类算法部分, 非监督分类算法当数据维度高或数据量大时, 耗时且精度下降, 因此监督分类算法成为遥感影像分类中的主流. 引出监督分类算法. 有文献做引证.}}
\begin{quotation}
    \itshape
    当缺乏土地覆盖的先验知识时, 一般先利用遥感影像的特征相似性进行聚类分析, 然后通过后分类处理使得聚类结果和土地覆盖类型对应起来 (Eva et al., 2004; Wulder et al., 2008a; Chen and Gong, 2013) . 通常使用的聚类算法有 K-Means 和ISODATA, Petitjean 等 (2012) 将动态时间规整算法 (Dynamic  Time  Warping, DTW) 和 K-Means 算法相结合, 提出了一种面向遥感影像时间序列的聚类框架, Zhang 等 (2014) 将其应用于面向 MODIS  NDVI时间序列的聚类中, 最终取得了比较好的土地覆盖分类结果. 作为无监督分类的一种, 聚类算法的优点在于无需先验知识和训练样本, 算法自动化程度高, 但是当数据维度过高或者数据量过大时, 聚类算法会十分耗时, 聚类精度也有所下降, 并且如何合理地标记聚类结果也是一项十分复杂的工作. 相比之下, 监督分类算法通过从训练数据中学习一定的先验知识, 在分类时能够做到有的放矢, 使分类精度得到大大提高, 监督分类算法正逐渐成为土地覆盖分类中的主流算法 (Khatami  et  al., 2016) .
\end{quotation}

\paragraph*{para04~
    \textcolor[RGB]{17, 205, 29}{单时相的分类方法多可用于时序影像分类中, 主要对单时相分类方法进行了叙述. 单分类器: 决策树, 随机森林, SVM; 多分类器: 集成学习等等}}
\begin{quotation}
    \itshape
    目前用于单时相影像的监督分类方法, 大多数可以直接应用在遥感影像时间序列分类中, 比如, Gebhardt 等 (2014) 利用 10 层的 C5.0决策树对 Landsat7 ETM+  和 landsat5 TM L1T 组成的多时相数据进行分类, 得到每年的土地覆盖分类图. 在论文中, Gebhardt 等通过计算 Landsat 影像时间序列 (由 NDVI, EVI, SR, ARVI 以及 6 个缨帽主成分构成的 10 条时间序列) 的最小值, 最大值, 数值变化范围, 均值, 标准差, 将其作为特征输入决策树进行分类. Clark 等 (2012) 利用随机森林分类器对 2001 年至 2010 年的 MODIS 影像时间序列进行了土地覆盖分类, 并根据 2001 年和 2010 年的土地覆盖分类结果进行了分类后变化检测. 还有一些研究人员通过组合 MODIS 影像时间序列的光谱特征和时间信息特征, 对其进行 SVM 分类以获取土地覆盖分类结果, 并在研究中比较了光谱信息和时间信息在分类中的不同作用 (Carrão et al., 2011; Gonçalves et  al., 2011) . 除了上述介绍的单个分类器进行监督分类外, 一种成熟的分类框架是采用多个分类器对数据集分别进行分类, 通过 ``打分'' 或 ``投票'' 机制获取最终的分类结果, 以此提高整体的分类精度, 该分类框架即是在机器学习领域十分流行的集成学习框架. 集成学习中常用的方法有 Bagging 和 Boosting 方法, 例如, Friedl 等 (2010) 使用 Boosting 方法对 2001 年至 2010 年间的年际时间序列    进行了土地覆盖分类并获取了不错的分类结果, 其中, 基本分类器选择 C4.5 决策树, 输入特征为光谱特征和时间信息特征. 
\end{quotation}

\paragraph*{para05~
    \textcolor[RGB]{17, 205, 29}{单影像分类算法不足: 在单时相影像上表现效果良好的分类算法有时并不适合挖掘时间序列的时空信息. 因此有必要探索新的时序影像分类方法, 由此引出深度学习. 有文献引证.}}
\begin{quotation}
    \itshape
    基于遥感影像时间序列的分类算法和基于单时相影像的分类算法在使用规则, 使用方式等方面大体一致, 但是由于时间序列数据含有更多的信息, 比如地物随季节的周期性变化信息, 不可避免的时空信息噪声等, 在单时相影像上表现效果良好的分类算法有时并不适合挖掘时间序列的时空信息 (Bruzzone  and Demir, 2014) . 而且, 目前的算法大多是提取遥感影像时间序列的统计变量, 比如最大值, 最小值, 均值, 标准差等, 将其作为时间信息特征用于分类器的输入以实现分类. 此类方法没有考虑时间序列的节点与节点之间的内在关联性, 反映在遥感影像上, 即没有考虑地物随时间变化的内在规律, 潜在信息挖掘力度不足造成了遥感影像的时空信息不能得到充分利用, 针对这种情况, 有必要探索新的算法用于遥感影像时间序列分析. 
\end{quotation}

\paragraph*{para05~
    \textcolor[RGB]{17, 205, 29}{深度学习近年来在诸多领域的应用, 深度学习在时序数据分类的结果. 论文作者认为深度学习将成为时序分析的主流发展方向.}}
\begin{quotation}
    \itshape
    深度神经网络模型作为近几年炙手可热的研究算法, 已经成功应用在图像识别, 语言翻译, 语音识别等等诸多领域 (Krizhevsky et al.,   2012; He et al., 2016; Silver et al., 2016; Wu et al., 2016) . 在时间序列分类方面, Wang 等 (2016) 分别利用 MLP, FCN, ResNet 三种深度神经网络模型对 UCR 时间序列数据集 (UCR Time Series Classification Archive)  (Anthony et  al., 2018) 进行了分类, 并取得了令人满意的分类结果. 在遥感影像时间序列方面, 利用深度神经网络模型进行数据处理的研究目前正处于萌芽阶段, 虽然研究不多但是笔者认为利用深度学习工具挖掘遥感影像时间序列的有效信息, 进而用于土地覆盖分类, 将是一项具有重要意义的研究内容, 甚至以后可能会成为遥感影像时间序列分析的主流发展方向.  
\end{quotation}

\subsubsection{遥感影像时间序列变化检测方法}
\paragraph*{para01~
    \textcolor[RGB]{17, 205, 29}{变化检测定义和其应用的研究领域. 有文献引证.}}
\begin{quotation}
    \itshape
    遥感影像时间序列变化检测用于检测同一地区的地物在一段时间内的差异变化, 提供有关地物的空间分布及其变化的定性与定量信息. 目前基于时序影像的变化检测方法已经广泛地应用到各个研究方向, 比如城市扩张 (Yuan  et  al., 2015) , 火灾监测 (Lin et al., 2016) , 土地分类 (Wen et al., 2016) , 森林退化调查 (Shapiro et al., 2015) , 冰川融化 (Robson et al., 2016) 等等
\end{quotation}

\paragraph*{para01~
    \textcolor[RGB]{17, 205, 29}{时序影像变化检测方法四大类: 基于光谱时序特征, 基于时间序列轨迹, 基于分类, 基于模型}}
\begin{quotation}
    \itshape
    根据变化检测所基于的前提假设和基本原理的不同, 现有的遥感影像时间序列变化检测方法可以概括为四种类型, 分别是基于光谱-时间特征的方法, 基于时间序列轨迹的方法, 基于分类的方法, 基于模型的方法.
\end{quotation}

\paragraph*{para02
    \textcolor[RGB]{17, 205, 29}{基于光谱-时间特征的方法. 波段简单计算结果与阈值比较, 或是通过学习的方法提取特征进行变化检测. 有文献引证.}}
\begin{quotation}
    \itshape
    常见的阈值法便是该类方法的一种形式, 比如基于多时相影像的云检测方法一般是利用原始光谱反射率与阈值进行比较从而获取检测结果(Hagolle et al., 2010). 为了能够更加准确地检测出特定地类的变化信息, 针对检测目标常常使用一些具有物理意义的特征作为检测指标, 比如, 通过设定归一化差异植被指数 (Normalized Difference Vegetation Index, NDVI) 的阈值用于检测植被的变化情况 (Lee, 2008; Li et al., 2015) . Huang 等 (2010) 提出了一种植被变化追踪 (Vegetation Change Tracke, VCT) 算法, 该算法首先计算一种概率指数-森林综合 z 得分 (Integrated Forest Z-score, IFZ) , 然后通过设置阈值进行森林扰动自动判断. 同时, 诸如归一化差异干湿指数 (Normalized Difference Wetness Index,   NDWI) , 归一化燃烧指数 (Normalized Burn Ratio, NBR) , 变化向量 (Change Vector) , 紊乱指数 (Disturbance Index, DI) , 增强湿度 (Enhanced Wetness) 等在该类方法中也均有使用 (Jin and Sader, 2005; Bolton et al., 2015; Parker et al., 2015; Bovolo and Bruzzone, 2006; Fraser et al., 2009; Vorovencii, 2014; Neigh et al., 2014; Linke et al., 2015) . 
    
    另外, 经过一定的学习规则提取的遥感影像特征在识别变化区域, 类型方面往往比原始波段效果好, 常用的特征提取方法有主成分分析 (Principal  Component  Analysis, PCA) 方法, 多元变化检测 (Multivariate Alteration Detection, MAD) 方法, 子序列提取方法, 稀疏表达方法等 (Parmentier, 2014; Nielsen, 2007; Wu.C, 2013; Bouaraba, 2014) 
\end{quotation}

\paragraph*{para03
    \textcolor[RGB]{17, 205, 29}{基于时间序列轨迹的方法}}
\begin{quotation}
    \itshape
    不同地物的光谱反射率除了与光谱波段有关之外, 随季节也会发生周期性变化, 这就决定了不同地物的遥感影像时间序列轨迹各有特点, 对于发生了地物变化的时间序列, 其轨迹不同于已有的地物时间序列, 因此, 一种基本思想是通过拟合不同的时间序列轨迹来识别土地覆盖的变化过程 (汤冬梅等, 2017) 
\end{quotation}

\paragraph*{para04
    \textcolor[RGB]{17, 205, 29}{基于分类的方法: 基本定义, 方法优点, 明确土地覆盖类型变化信息, 缺点为变化检测精度依赖于分类精度, 易出现误检. 解决方案: 一致性检验, 时空数据挖掘技术引入.}}
\begin{quotation}
    \itshape
    分类后土地覆盖变化检测是指对经过几何配准的两个 (或多个) 不同时相遥感图像分别作分类处理后, 获得两个 (或多个) 分类图像, 并逐个像元进行比较, 生成变化图像, 并根据土地覆盖转移矩阵确定各变化像元的前后时相类型 (赵英时, 2013) 

    ...    

    基于分类的变化检测方法能够明确土地覆盖类型的转换关系, 为变化检测结果提供 ``from-to'' 的土地覆盖转化信息. 但是, 该类方法的变化检测精度与分类精度息息相关, 容易出现大量的误检以及不合理的变化检测结果. 为了消除这方面的影响, Li 等 (2015) 通过采用一致性检验的方法来提高最终的变化检测精度. 另外, 有一些学者将时空数据挖掘技术引入到了遥感影像时间序列数据的处理中, 能够自动或半自动地从海量时间序列数据中挖掘有意义的模式用于分类和变化检测 (Zurita-Milla et al., 2013) 
\end{quotation}

\paragraph*{para05
    \textcolor[RGB]{17, 205, 29}{基于模型的方法}}
\begin{quotation}
    \itshape
    基于模型的方法一般以统计假设为前提, 通过构建时空变化模型来实现变化检测, 其基本思想是通过拟合或平均的方法得到各种地类的 ``标准'' 时间序列或特征向量, 将新输入的时间序列 (或特征向量) 与标准序列 (或特征向量) 之间的差异作为变化检测的依据. 
\end{quotation}

\subsubsection{存在的问题}
\paragraph*{para02
    \textcolor[RGB]{17, 205, 29}{问题1模型泛化能力(针对分类): 遥感影像时间序列的复杂多样}}
\begin{quotation}
    \itshape
    但是, 复杂多变的地表情况和难以预估的气候年际变化, 造成了遥感影像时间序列的复杂多样, 使得同一地物的不同年际时间序列有所不同. 利用一定周期内获取的时间序列样本训练得到的模型, 往往对其他周期内获取的时间序列分类效果不好, 甚至出现分类精度巨幅下降的情况.
\end{quotation}

\paragraph*{para03
    \textcolor[RGB]{17, 205, 29}{问题2训练样本不足: 人主观性, 遥感地面复杂性.}}
\begin{quotation}
    \itshape
    目前获取遥感影像时间序列训练样本的方式主要是通过目视解译实现, 由于人的主观性以及遥感地面情况的复杂性, 短时间内很难选取理想的, 足够数量的训练样本, 尤其是对于变化检测训练样本的选取, 实际影像中发生土地类型变化的像元十分有限, 选取工作更加困难. 
\end{quotation}

\paragraph*{para04
    \textcolor[RGB]{17, 205, 29}{问题3序列对序列的对应关系}}
\begin{quotation}
    \itshape
    现有的面向遥感影像时间序列的分类和变化检测方法, 在标定输出时通常标定为离散的土地覆盖类别, 但是, 对于一条时间序列而言, 其表示的地物在一个时间序列周期内会有不同的状态, 因此对应的输出应该是一条能够表示地物状态变化的序列. 例如, 温带地区的植被随着季节变化而枯荣交替, 在一年内会有不同的状态变化. 通过获取植被的物候状态变化, 对于提高分类精度, 实现周期内的变化检测具有至关重要的作用. 目前基于监督学习的 ``序列到序列'' 算法研究还不多, 如何构建合适的 ``序列到序列'' 模型用于遥感影像时间序列分类和变化检测将是一项具有创新性的研究. 
\end{quotation}

\subsection{研究概述}

\subsubsection{研究对象}
\paragraph*{para01
    \textcolor[RGB]{17, 205, 29}{遥感影像时序分类和变化检测, 所用数据为MODIS和Landsat8}}
\begin{quotation}
    \itshape
    本文的研究对象为遥感影像时间序列分类以及变化检测, 以土地覆盖分类和变化检测为例. 遥感影像时间序列是指经过几何校正和配准的多时相影像按照时间顺序构成的数据集, 数据集中的每条时间序列代表了其对应的地物在一定周期内随时间发生的光谱特征变化. 本文中应用的遥感数据主要为 MODIS 时间序列和Landsat-8 时间序列. 
\end{quotation}

\subsubsection{研究目标}
\paragraph*{para01
    \textcolor[RGB]{17, 205, 29}{研究目标总结, 点题}}
\begin{quotation}
    \itshape
    本文的研究目标是通过构建合适的循环神经网络 (Recurrent Neural Network, RNN) 模型来挖掘遥感影像中的时空信息, 完成面向遥感影像时间序列的分类和变化检测, 从而提供研究区域的土地覆盖分类结果和变化检测结果. 具体包括以下目标: 
\end{quotation}

\paragraph*{para02
    \textcolor[RGB]{17, 205, 29}{目标1, 对应存在问题1模型泛化能力}}
\begin{quotation}
    \itshape
    构建合适的 RNN 模型用于 MODIS 时间序列分类, 能够在有限的训练样本集下实现分类模型的训练, 并取得鲁棒的分类结果, 要求分类精度能够满足土地覆盖制图的需要.
\end{quotation}

\paragraph*{para03
    \textcolor[RGB]{17, 205, 29}{目标2, 对应存在问题3序列到序列对应关系}}
\begin{quotation}
    \itshape
    将基于循环神经网络结构的 ``序列到序列'' 模型应用到 Landsat-8 时间序列变化检测中, 实现影像中变化信息的直接提取, 提供研究区域土地覆盖类型的变化情况. 
\end{quotation}

\paragraph*{para04
    \textcolor[RGB]{17, 205, 29}{目标3, 对应存在问题2训练样本不足}}
\begin{quotation}
    \itshape
    针对标定样本不足造成分类模型容易过拟合的难题, 探索合适的数据增强方法扩增时间序列, 增加样本数据的多样性, 满足训练模型的数据要求. 针对训练数据中缺乏混合像元时间序列和变化像元 (类型发生变化的像元) 时间序列的问题, 探索合适的序列合成方法构造混合像元, 变化像元的时间序列, 满足分类模型和变换检测模型对训练数据的要求.
\end{quotation}

\subsubsection{研究内容}

\paragraph*{para01
    \textcolor[RGB]{17, 205, 29}{内容1, 对应研究目标1: 构建合适的 RNN 模型用于 MODIS 时间序列分类}}
\begin{quotation}
    \itshape
    循环神经网路作为人工神经网络的一种, 通过使用带自反馈的神经元, 能够挖掘时间序列中的上下文信息, 并能够处理任意长度的序列数据, 对于序列数据建模, 分析有着天然的优势. 遥感影像时间序列表示地物随时间发生的变化情况, 时间节点之间相互影响, 具有一定的联系. 因此, 可以利用深度循环神经网络模型提取遥感影像时间序列蕴含的地物物候信息和周期性变化信息, 并用于土地覆盖分类. 本文主要研究内容之一是探究适于 MODIS 影像时间序列分类的深度RNN 模型, 设计合理的网络结构并确定合适的模型结构参数, 最后利用训练好的深度 RNN 模型实现研究区域不同年份的土地覆盖分类
\end{quotation}

\paragraph*{para02
    \textcolor[RGB]{17, 205, 29}{内容1, 对应研究目标3: 探索合适的数据增强方法扩增时间序列, 增加样本数据的多样性}}
\begin{quotation}
    \itshape
    深度神经网络模型的训练常常需要大量样本的支持, 并且样本的选取好坏对后续的土地覆盖分类有着至关重要的影响, 因为它直接决定了模型的识别能力. 针对标定样本不足造成模型容易过拟合的难题, 论文探索了面向遥感影像时间序列的数据增强方法, 通过研究如何对遥感影像时间序列进行不同方式的扩增, 以实现增加训练样本集多样性的目标, 从而解决模型的过拟合问题, 满足训练深度RNN 模型的数据要求.
\end{quotation}

\paragraph*{para03
    \textcolor[RGB]{17, 205, 29}{内容2, 应研究目标2: 序列到序列的对应关系}}
\begin{quotation}
    \itshape
    鉴于 ``序列到序列''(sequence-to-sequence, seq2seq)模型和注意力机制 (attention  mechanism) 在机器翻译等领域的成功应用, 本文有意将该模型应用在遥感影像时间序列变化检测中. 因此, 本文的另一个研究重点是探索带有注意力机制的 seq2seq 模型在 Landsat-8 时间序列变化检测中的应用, 通过构建合适的网络模型以及选择合适的模型结构参数, 实现研究区域针对特定类型的土地覆盖变化检测
\end{quotation}