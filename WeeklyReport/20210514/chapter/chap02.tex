\section{基于时序光学和雷达影像的中国海岸带盐沼植被分类研究}

\subsection{研究背景与意义}

\paragraph*{para01~
    \textcolor[RGB]{17, 205, 29}{盐沼及其分布. 有文献引证.}}
\begin{quotation}
    \itshape
    盐沼生态系统是海岸带区域最主要的生态类型之一,最主要的特征为其中生长的潮间带植被,因潮汐影响而周期性地受到淹没或排干。相比于分布在热带海岸的红树林生态系统,盐沼广泛分布于全球中纬度温带区域或北部高纬度的海岸,纬度最高处可至格陵兰岛等北极地区。盐沼通常出现在沿海的平原河口,或是海湾周围的岸线附近,这些区域同时受到潮汐和径流的作用,形成湿地与邻近地区之间的紧密耦合。另外,在水动力较弱且沉积物丰富的泻湖、港湾中也往往分布大片盐沼。盐沼植被多以耐盐草本植物或灌木为主,对于环境中较高盐度或较长时期的淹水均有其独特生物过程以适应(Scott et al., 2014;Day JR et al., 2012)。 
\end{quotation}

\paragraph*{para02~
    \textcolor[RGB]{17, 205, 29}{盐沼的生态服务功能, 其对沿海地区生态稳定的重要性. 有文献引证.}}
\begin{quotation}
    \itshape
    盐沼被广泛认为是沿海地区最具活力和价值的生态系统之一,可提供消浪护岸、水源净化、动植物栖息地、蓝碳碳汇、泥沙滞留、养分循环等多种生态服务功能(Costanza et al., 1997)。不同盐沼植被在各类生态服务功能上提供的价值具有明显差异,因此当讨论区域范围内盐沼湿地的生态服务价值时,必须将盐沼植被类型纳入考虑。例如,互花米草生物量大、植株高,能在较大程度上减少波浪的水动力能量,生态工程能力在平均水平的 2.5 倍以上(Ysebaert et al., 2011)。此外,互花米草群落下土壤有机碳和氮浓度值显著高于盐地碱蓬群落,证明其具有强大的碳汇能力(Zhang et al., 2010)。动植物栖息地提供方面,芦苇群落中的生物多样性明显高于互花米草群落,贝类等运动能力差的软体动物在密集的米草中活动困难,甚至会窒息死亡,从而造成鱼类及鸟类的食物资源减少,导致滩涂湿地的生物多样性显著下降(马强等,2017)。 
\end{quotation}

\paragraph*{para03~
    \textcolor[RGB]{17, 205, 29}{全球盐沼生态系统更替退化. 有文献引证.}}
\begin{quotation}
    \itshape
    过去的一个世纪里,受气候变化和人类活动影响,全球盐沼生态系统经历了更替和退化。人类活动包括围垦造陆、滩涂养殖、外来物种引入、水质污染、过度放牧等;气候变化则包括全球性的气温上升、海平面上升,这些都是造成退化的原因(Crosby et al., 2016;Gedan et al., 2009)。此外,生物盐渍化导致盐沼物种组成的连续变化(Nieva et al., 2005;Minchinton et al., 2006)。 
\end{quotation}

\paragraph*{para04~
    \textcolor[RGB]{17, 205, 29}{我国盐沼分布遭到破坏. 引出要监测盐沼. 有文献引证. }}
\begin{quotation}
    \itshape
    在中国海岸带,盐沼分布情况同样经历剧变。1980 至 2010 年间,围垦工程、气候变化等原因导致中国海岸带 59\%盐沼丧失(Tian et al., 2016;Gu et al., 2018)。2013~2016 年莱州湾天然滨海湿地面积共减少 2744.0 hm2(杨帆,2017)。江苏省盐城岸段 2010~2015 年围垦面积超过 25000 hm2,其中 90\%为互花米草为主的盐沼湿地(张濛  和  濮励杰,2017)。同时,部分区域盐沼面积增长迅速,河口入海泥沙淤积、潮滩变化则导致局部区域新生盐沼增长,长江河口九段沙区域近年来植被面积增速超过 500 hm2/a(朱串串,2018)。2011 至 2017 年,黄河口三角洲北岸的互花米草增速达到 528.5 hm2/a(Ren et al., 2019)。中国海岸带盐沼不仅面积数量发生变化,植被结构也快速发生变化,典型如互花米草入侵,造成植被种间组成和空间结构快速转变,中国海岸带本土盐沼植被群落受到破坏,对生态系统结构和服务功能造成影响(Liu et al., 2018)。 

\end{quotation}

\paragraph*{para05~
    \textcolor[RGB]{17, 205, 29}{传统盐沼调查不可取, 要使用遥感影像进行盐沼的调查.}}
\begin{quotation}
    \itshape
    传统的盐沼植被野外调查方法效率低下,价格昂贵,且由于盐沼湿地本身往往很难进入,很难获得样点。遥感影像可覆盖大范围区域,且在一个位置重复多次拍摄,可作为有效的监测方法.
\end{quotation}

\paragraph*{para06~
    \textcolor[RGB]{17, 205, 29}{光学影像常用于盐沼植被分类. 有文献引证.}}
\begin{quotation}
    \itshape
    光学遥感影像数据历史累积量大、数据源选择较多、分类方法成熟多样、分类精度较高,是大尺度范围盐沼植被分类最常用的遥感数据。已有一些学者采用光学影像对中国海岸带盐沼分布情况进行研究,分类精度均在 90\%以上。刘月明(2016)采用 Landsat 影像,提取了 1990、2000、2010 和 2015 年中国海岸带互花米草信息。罗敏(2019)采用 Landsat 影像提取了 2015 年中国海岸带盐沼植被分布信息,并以此计算中国海岸带盐沼生态服务价值。 
\end{quotation}

\paragraph*{para07
    \textcolor[RGB]{17, 205, 29}{时序光学影像可更好提高盐沼植被分类精度. 有文献引证.}}
\begin{quotation}
    \itshape
    使用光学影像进行盐沼植被分类时,采用时序遥感分析法,能够结合海岸带盐沼植被本身的物候特征,提升盐沼植被分类精度。海岸带盐沼湿地中生长植被多为一年生草本植物,在一年中经历完整的发芽期、生长期、枯萎期物候过程,植株形态和叶绿素含量不断发生变化。许多研究显示,盐沼植被在一年中的不同时期的不同特征,可反映在其光谱特征上的差异,基于这一现象可对盐沼植被类型进行分类(Fernandes et al., 2013)。 
\end{quotation}

\paragraph*{para08~
    \textcolor[RGB]{17, 205, 29}{对于高频次分类需求, 由于光学遥感影像本身特点, 云雾遮挡, 及其时序数据获取问题, 光学遥感影像无法单独用于盐沼植被分类. 由此引出雷达影像. 但无文献引证.}}
\begin{quotation}
    \itshape
    但面对大尺度、高频次的海岸带盐沼植被分类需求,光学影像下植被光谱特征易混淆,可用数据获取不稳定的缺陷则尤为明显。盐沼植被光谱特征具有共性,多数时间内  “同物异谱、异物同谱”现象显著,需要特定季相影像才能保证较高区分度。同时,海岸带区域光学影像受云雾遮挡、潮汐淹没影响很大,光学遥感数据的时间分布稀疏且不均匀,全年内可能无法获取合适季相影像。通常,大尺度范围盐沼分类牺牲时效性,使用三年范围内合适影像,以满足分类精度需求。从长期监测的目的来看,部分区域的盐沼数据更新频率在三年以上,对于快速变化的海岸带,盐沼数据的获取仍存在空缺与挑战。
\end{quotation}

\paragraph*{para09~
    \textcolor[RGB]{17, 205, 29}{SAR影像特点, 结合光学影像在盐沼植被分类有很大优势.}}
\begin{quotation}
    \itshape
    合成孔径雷达 SAR(Synthetic Aperture Radar)影像具有数据源稳定,能够反映植被结构信息的特点,与光学影像结合可发挥二者优势,在盐沼分类方面有很大潜力。SAR 具有全天时全天候、能够穿透云雾特点,能获取长时间序列影像数据,可反映盐沼植被完整生长周期内变化信息。此外,由于成像原理不同,SAR影像能够提供光学遥感影像所缺乏的对植被冠层的穿透性,使植被的结构信息也能得到利用。SAR 信号能够穿透植被冠层,与植被叶片、枝杈、土壤、植被下层水体等均能发生相互作用,并将不同的植被结构类型以镜面散射、二次回波散射、体散射等各类散射模式叠加的信号返回到接收器。(Kwoun and Lu, 2009)。
\end{quotation}

\paragraph*{para10~
    \textcolor[RGB]{17, 205, 29}{针对SAR影像用于时序分类的现状, 可能是论文的研究内容. 无文献引证.}}
\begin{quotation}
    \itshape
    相较光学影像而言,SAR 数据出现更晚,目视可读性弱,数据处理计算量大,在盐沼植被方面的应用较少,仍存在研究空缺。盐沼植被物候特征能否同样在时序 SAR 数据上体现;如何选择时序 SAR 数据合适特征用于分类;结合时序光学和 SAR 数据能否满足大尺度盐沼植被分类精度和时效性兼顾的需求等,都是有待解答的问题。 
\end{quotation}

\paragraph*{para11~
    \textcolor[RGB]{17, 205, 29}{研究的数据基础}}
\begin{quotation}
    \itshape
    2014 年升空的 Sentinel-1 卫星获取 C 波段  SAR 数据,具有时空分辨率高,获取成本低廉优点。与其同系列的 Sentinel-2 光学卫星同样具有高时空分辨率的特点,重放周期和空间分辨率均优于 Landsat 影像。二者结合,能够在最大程度上获得稳定的数据,适用于时序分析研究。此外,得益于谷歌地球引擎 GEE(Google Earth Engine)的云计算功能,大量遥感数据的快速计算得以实现。在此基础上,分析时序 SAR 与光学与盐沼物候特征之间的关系,建立一种准确快速获取大范围盐沼植被空间分布与种间组成信息的方法,对于生物多样性保护、湿地生态系统功能提升与海岸带生态环境管理等具有重要意义  。 
    
\end{quotation}

\subsection{国内外研究现状}

\subsubsection{盐沼植被分类采用的遥感数据源}
\paragraph*{para01~
    \textcolor[RGB]{17, 205, 29}{目前用于海岸带盐沼植被监测的遥感数据源介绍. 有文献引证.}}
\begin{quotation}
    \itshape
    目前,已有多种遥感数据源被应用于海岸带盐沼植被监测。多光谱光学影像中,中低空间分辨率影像如 MODIS(Moderate-resolution Imaging Spectroradiometer)和 AVHRR(Advanced Very High Resolution Radiometer)数据,拥有高时间分辨率的优势,但空间分辨率低于植被种间分类需要,往往被用于区域、尺度上的湿地信息提取(Takeuchi et al., 2003;张猛等,2017)。中等空间分辨率卫星影像,如 Landsat5 TM、Landsat7 ETM+、Landsat8 OLI、Spot6/7、HJ 1A/1B 等,时间分辨率相对较低,但空间分辨率能够满足盐沼植被中间分类精度需要,常用于湿地类型提取及变化研究(Davranche et al., 2010;雷璇等,2012)。尤其是 Landsat系列,由于拥有长达 30 年之久的时间序列影像,多被用于长时间尺度上的变化分析(Jia et al., 2015;Sun et al., 2018)。高空间分辨率光学影像,如 QuickBird、IKONOS、GF 卫星等,空间分辨率很高,但单景影像覆盖范围小,价格高,多被用于局部重点区域的植被群落调查(Kumar et al., 2014;Belluco et al., 2006)。由于不同类型的盐沼植被光谱特征往往非常相似,在光学影像中“同物异谱,异物同谱”现象十分显著,使用多光谱影像难以实现精确的种间分类(Zhang et al., 2016)。 
\end{quotation}

\paragraph*{para02~
    \textcolor[RGB]{17, 205, 29}{辅助光学遥感影像分类的数据源SAR, LIDAR, 高光谱等. 有文献引证.}}
\begin{quotation}
    \itshape
    为解决植被种类间易混淆的现象,近年来,随着卫星影像种类的增多和无人机技术的普及,更多种类型的数据源作为辅助数据应用于盐沼植被类型区分,以提升光学影像分类的精度。SAR 数据有全天时、全天候,不受云雾干扰的特点,多被用于作为光学影像的辅助数据提升盐沼植被分类精度,如 C 波段的Radarsat、X 波段的 TerraSAR 等,均已有研究证明其可提升湿地植被制图精度(Mandianpari et al., 2012;Lee et al., 2012)。激光雷达 LiDAR(Light Detection and Ranging)获得的点云数据可将植被高程纳入植被分类过程(Collin et al., 2010)。高光谱影像拥有极高的空间分辨率,能够识别不同植被类型在光谱特征上的差异,也可作为辅助信息在小范围内提升分类准确性(Millard and Richardson, 2013)。 
\end{quotation}

\paragraph*{para03~
    \textcolor[RGB]{17, 205, 29}{以上辅助数据由于其性质无法用于大尺度盐沼植被提取. 无文献引证. }}
\begin{quotation}
    \itshape
    上述的辅助数据能够大大提升传统光学方法的分类精度,但往往数据获取难度大,成本高,且需要极大的计算量,因此通常被用于小区域范围内的高精度盐沼制度,目前暂不适用于大范围盐沼植被提取。 
\end{quotation}


\subsubsection{盐沼植被遥感分类方法进展}

\paragraph*{para01~
    \textcolor[RGB]{17, 205, 29}{光学影像分类使用特征(指数)}}
\begin{quotation}
    \itshape
    光学影像最常用的特征为指数计算,归一化植被指数 NDVI(Normalized difference vegetation index)、大气抗性植被指数 ARVI(Atmospherically resistant vegetation index)、土壤调节植被指数 SAVI(Soil adjusted vegetation index)等均为经典的植被分类指数(Zhang et al., 1997;Kanfman and Tanre, 1992;Huete et al., 1988)。针对盐沼植被受淹没特点,过滤沼泽植被时间序列产品潮汐影响的潮沼泽淹没指数被提出(O’Connell et al., 2017)。
\end{quotation}

\paragraph*{para02~
    \textcolor[RGB]{17, 205, 29}{全极化SAR影像可用于分类的极化特征}}
\begin{quotation}
    \itshape
    对于全极化雷达影像,极化分解方法可分离地物不同散射机制引起的极化特征,从而提取目标地物的主要散射特征。其中  Cloude-Pottier、Freeman-Durden 和Van-Zyl 等为经典极化分解方法(van Beijma et al., 2014)。
\end{quotation}

\paragraph*{para03~
    \textcolor[RGB]{17, 205, 29}{直接跳到监督分类.}}
\begin{quotation}
    \itshape
    简单的监督分类分类方法包括支持向量机 SVM(Support Vector Machine)、最大似然法 ML(Maximum  Likelihood)和决策树等监督分类方法(Sanchez-Hernandez et al., 2007;Hladik et al., 2013)。随着日渐多源且高空间分辨率遥感数据不断增加,针对海量数据的分类方法也逐渐发展。神经网络、深度学习等图像识别方法逐渐被运用于湿地植被分类(Wang et al., 2007)。面向对象的方法通过将信息类似的像元组合为对象,并以对象而非像元作为最小分类单元,以此能够将对象的形状、纹理等空间信息运用于分类过程(Ouyang et al., 2011)。高光谱影像则可通过子空间划分、S-Recorre、Relief F 等降维方法去除无关特征和特征间的冗余性(朱玉玲,2019)。
\end{quotation}

\paragraph*{para04~
    \textcolor[RGB]{17, 205, 29}{植被物候特性可在光学影像上体现, 暗示着时序光学影像分类结果优于单时相分类结果(这一点可以直接写出来). 有文献引证.}}
\begin{quotation}
    \itshape
    在中国海岸带,多个研究已证实部分典型盐沼植被的物候特征在光学影像上的可识别性。Sun 等(2016)采用全年 HJ-1 光学影像进行江苏省盐城区域的盐沼植被制图,基于全年每月 NDVI 时间序列的制图精度比单相分类高 16.0\%;在12 个月的 NDVI 中,仅采用 6 个关键月的 NDVI,总体准确率几乎不下降;且由于互花米草枯萎期较其余本土盐沼植被类型晚,11 月监测互花米草最合适的时期。Gao 和 Zhang(2006)研究发现,长江口区域 4 月下旬芦苇的近红外反射率比互花米草高 10\%,而海三棱藨草反射率很低,无明显红边;到 10 月末,长江口盐沼植被在近红外波段的曲线不发生重叠,彼此间区分度最高。 
\end{quotation}

\paragraph*{para05~
    \textcolor[RGB]{17, 205, 29}{针对(时序)光学影像受云雾影像, 或是自然现象影响, 现阶段他人解决方案. 引入SAR影像, 对SAR的具体应用没有文献引证, 在一小节详细叙述.}}
\begin{quotation}
    \itshape
    但采用时序光学影像进行海岸带盐沼植被分类有其一定的局限。利用植被物候期分类依赖大量高时间分辨率遥感影像,但海岸带区域受云雾影响,全年可用光学影像数量往往无法满足分类需求(Dong et al., 2016;李清泉等,2016)。此外,周期性潮汐会淹没植被前缘,并导致下垫面湿度变化,使得植被外缘无法准确提取且植被光谱混淆难以识别(O’Connell et al., 2017)。解决这一问题的方案包括使用时间分辨更高的新传感器传回的光学影像,如 HJ-1、Sentinel-2(Kaplan and  Avdan,  2017),或将 Landsat 影像与 MODIS 影像融合,同时采用二者的高空间分辨率与高时间分辨率(Hilker et al., 2009)。此外,SAR 数据被认为是光学影像很好的补充数据。 
\end{quotation}

\subsubsection{盐沼植被雷达遥感应用进展}
\paragraph*{para01~
    \textcolor[RGB]{17, 205, 29}{SAR可用于盐沼植被分类, 不适合大范围盐沼制图.}}
\begin{quotation}
    \itshape
    使用 SAR 数据进行盐沼植被分类研究考虑波长、极化方式等因素。考虑波长情况下,大量实验指出草本沼泽湿地适合采用 C 波段或 X 波段,森林沼泽湿地适合采用波长较长的 L 波段(Zhang et al., 2016;Betbeder et al., 2015). 考虑极化方式,多数研究使用 Radarsat-2、TerraSAR-X 等全极化影像,采用极化分解结合多源数据监督分类方法进行,并已证实此类方法有助于提升盐沼分类精度(Mleczko and Mróz, 2018;van Beijma et al., 2014)。但由于全极化 SAR 数据价格高、成像范围小的特点,此类研究多采用小范围单景 SAR 影像,不适用于大范围盐沼制图,且无法利用 SAR 数据时序性强的优势。 

\end{quotation}

\paragraph*{para02~
    \textcolor[RGB]{17, 205, 29}{时序SAR可用在农作物分类和水体表面监测, 因此在盐沼(地物+水)的区域也具有优势.}}
\begin{quotation}
    \itshape
    目前,时序 SAR 数据应用与盐沼植被分类的研究较少。大量时序 SAR 数据与光学数据结合的研究应用于农作物类别监测,并已发现了 SAR 对农作物生长过程物候特征具有可识别性,植被结构的变化在很大程度上导致后向散射强度变化(Veloso et al., 2017;Van Tricht et al., 2018)。此外,时序 SAR 数据还被应用
    于水体表面的监测,是基于 SAR 在水体处的后向散射强度永远保持在极低值(Huang et al., 2018)。由此可见,在以植被和水体为主要地物类型的盐沼区域,时序 SAR 数据具有其独特优势。
\end{quotation}

\subsection{研究内容与目标}

\paragraph*{para01~
    \textcolor[RGB]{17, 205, 29}{研究目标}}
\begin{quotation}
    \itshape
    本文研究目标为,基于不同盐沼植被物候期内,时序光学反射率与雷达后向散射特征,建立一种光学与雷达影像结合,准确快速获取中国海岸带盐沼植被空间分布与种间组成信息的方法,以此确保能够获取高时效的中国海岸带盐沼植被分布状况。 
\end{quotation}

\paragraph*{para02~
    \textcolor[RGB]{17, 205, 29}{内容1: 海岸带典型地物的光学反射特征、时序雷达后向散射特征}}
\begin{quotation}
    \itshape
    通过实地考察、文献查阅、Google 高光谱影像采集中国海岸带盐沼植被样本,结合年度气温、潮汐、降水、地质地貌等数据,使用预处理后的 2018 年全年 Sentinel-1、Sentinel-2 影像,分析典型潮滩地物类型、及其在不同潮汐条件、不同植被盖度条件下光谱反射率、全年雷达后向散射特征,分析其变化规律。
\end{quotation}

\paragraph*{para03~
    \textcolor[RGB]{17, 205, 29}{内容2: 基于时序光学、雷达影像的中国海岸带盐沼分类方法 
    }}
\begin{quotation}
    \itshape
    以典型区域为例,分析随机森林方法、基于专家知识分类方法的优劣;雷达影像与雷达光学影像结合的精度差异,确定全国海岸带盐沼分类所用方法及数据源。通过不同气候带时序 NDVI 与雷达后向散射特征分析,确定全国海岸带盐沼分类所用方法及数据源,构建光学和雷达遥感集成的海岸带盐沼分类方法。 
\end{quotation}

\paragraph*{para04~
    \textcolor[RGB]{17, 205, 29}{中国海岸带盐沼分布情况及重点区域变化情况 }}
\begin{quotation}
    \itshape
    基于 2018 年全年雷达 Sentinel-1、光学 Sentinel-2 影像,利用光学和雷达遥感集成盐沼分类方法,提取 2018 年中国海岸带盐沼分布,分省市区统计计算各区域盐沼植被面积、类型、数量,分析空间分布特点和规律,结合历史数据解释其变化情况及原因。 
\end{quotation}

