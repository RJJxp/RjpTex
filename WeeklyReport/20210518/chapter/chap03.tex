\section{小结}
林蕾的 ``基于循环神经网络模型的遥感影像时间序列分类及变化检测方法研究'' 第一章的研究背景介绍, 首先介绍土地资源重要性, 之后讨论土地覆盖分类变化检测重要性和必要性, 点题 ``时序分类及变化检测'', 同时表明自己的研究对象. 通过文献引证, 卫星影像可用于土地覆盖, 点题 ``遥感影像''. 在分类方面, 目前分类用单影像和时序影像, 通过引证文献, 时序影像优于单影像. 同样地, 在变化检测方面, 时序影像比双时相精度高且能定量分析. 下一段中, 叙述遥感影像几十年积累, 意在论证论文数据基础方面的可行性. 研究背景最后一段叙述使用时序影像做分类和变化检测的必要性, 时序数据特点和时序遥感影像特殊性, 现有数据挖掘算法不能满足需求. 这一部分也是对存在问题的概括总结. 需要学习的是: 作者在重要的论述时(时序优越性, 数据可行性, 可能存在问题等), 都是用其他论文做引述, 这使论文在学术层面更加专业. 这也是看其他论文, 无论大小论文需要做摘抄的地方.

第一章的国内外研究现状. 主要对时序影像分类方法进行小结. 作者将时序影像分类分为 数据源, 分类框架, 分类算法三大类. 单独将数据源作为时序分类的一大类是由于论文题目中并未提到所使用数据源. 作者先将叙述了适合时序分类的遥感影像的特征, 主要为时间间隔. 下一段对论文所使用的MODIS和Landsat两个卫星影像总结, 对其不足之处, 介绍改进方法. 分类框架, 作者引自另一篇时序影像的分类文章.

重点为分类算法. 作者首先将分类算法分为监督分类和非监督分类. 由于非监督分类耗时且精度下降的缺点, 监督分类算法正成为土地覆盖分类的主流算法. 此处有文献引证. 文章之后将非监督算法抛弃, 主要介绍监督分类算法. 其中单时相监督算法多可用在遥感影像时序分类, 有大量文献引证. 作者将单时相分类监督算法分类单分类器和多分类器. 但是单影像分类算法不足: 在单时相影像上表现效果良好的分类算法有时并不适合挖掘时间序列的时空信息. 因此有必要探索新的时序影像分类方法, 由此引出深度学习. 之后一段对深度学习模型在时序分类做了整体的介绍. 

存在的问题. 模型泛化能力, 训练样本不足主要针对深度学习方面, 更偏向计算机. 序列对序列的对应关系, 作者发现时序影像分类结果输出的分类结果是某一地物, 但在有了时序数据, 对应的输出应该是一条能够表示地物状态变化的序列. 论文中的思想上的大创新点. 之后的研究目标与研究内容皆从存在的问题出发, 一一对应. 