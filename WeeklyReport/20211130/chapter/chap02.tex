\section{存在问题}

\begin{frame}{问题1}
    Niu等人使用Pix2Pix对SAR到光学遥感影像进行翻译, 但监督的方法难以获取实地影像对, 而深度学习对数据要求较为严格, 不同地区影像特征不同, 训练出模型可迁移性较差. Liu等人对SAR和光学影像相互翻译的原理进行探究, 验证了使用非监督的方法也可进行SAR和光学影像的翻译. Wang等人提出S-CycleGAN, 结合了Pix2Pix的优点, 但却回到了监督学习的方法, Hwang等人将图像修复的思想引入图像翻译的去噪, 结果较为不错. 
    \\[0.5cm]
    \textbf{这些基于GAN的翻译方法, 其对于部分地物和纹理特征的翻译效果较差, 如何提升翻译的纹理效果和地物的翻译质量, 在后续的超分中至关重要.}
     
 \end{frame}

 \begin{frame}{问题2}
    在进行超分影像对制作时, 使用哨兵二号影像作为低分辨率影像, 高分一号影像作为高分辨率影像. 现今遥感卫星传感器各波段相近, 但并不完全相同, 外加哨兵和高分成像时间不同, 得到同一区域相近时间的色调差异较大. 在现有基于GAN的深度学习超分模型中, 由于其网络结构和损失函数设计, 并没有考虑到训练数据色调不一致的问题, 导致训练不稳定甚至崩溃. 
    \\[0.5cm]
    \textbf{解决这一问题, 对不同源光学遥感影像超分辨率训练稳定性有极大提升.}

 \end{frame}

 \begin{frame}{问题3}
    模拟遥感影像噪声, 生成低分辨率影像
    \\[0.5cm]
    需要进一步看文献
 \end{frame}