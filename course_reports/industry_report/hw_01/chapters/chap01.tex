\section{点云拟合平面研究现状}

点云平面拟合也是点云预处理过程中的关键环节, 是三维建模的基础, 如对建筑物进行三维建模时需要对各种立面(墙面, 地面, 屋顶和天花板等)进行平面拟合. 最小二乘(LS)法是最常用的拟合方法, 其核心是拟合出一个采样值与实际值的偏差平方和最小的平面. 由于该算法在进行合时假设x, y为独立变量且不存在误差, 仅考虑了z值误差, 而实际上由于设备误差和不可控因素的影响, 点云坐标的测量值(X, Y, Z)是一定存在误差的, 所以最小二乘法并不能得到较好的拟合结果, 严格来说它并不适用于平面拟合. 官云兰提出一种基于整体最小二乘法(TLS)的拟合方法, 在一定程度上克服了最小二乘法的误差影响, 但该方法对于误差较大的情况稳健性不强. 在此基础上, 苍桂华等又提出了一种基于加权总体最小二乘法的点云平面拟合方法.

随机采样一致性(RANSAC)算法也是常用的拟合办法, 北京工业大学魏英姿和长安大学曹霆都对随机采样一致算法做过深入的研究, 该算法不同其他类似算法的是, 不针对全体点云而是通过随机采样获取一个最小样本的, 基于这个样本拟合出初始平面模型, 然后将全体样本代入模型判断内点和外点, 并记录内点总数和对应的参数值, 如此重复迭代直到满足结束条件, 即认为内点总数最多的模型即为最接近于理想平面的最佳模型.