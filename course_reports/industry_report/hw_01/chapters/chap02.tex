\section{算法与模型}
平面方程表示为
\begin{equation}
    ax+by+cz+d=0 \label{e0201}
\end{equation}

即:
\begin{equation}
    e^{T}X+d=0 \label{e0202}
\end{equation}

式\eqref{e0202}中, $e^{T}=(a\ b\ c)^{T}$ 为平面的法线方向的单位矢量, 以$X_i=(x_i\ y_i\ h_i)^T$, $i=1,2,\ldots,n$表示观测点坐标.

\subsection{直接拟合}
过$i$点垂直于平面的直线方程为:
\begin{equation}
    \left\{
    \begin{array}{rl}
        x = & x_i + at \\
        y = & y_i + bt \\
        h = & h_i + ct \\
    \end{array}
    \right.
\end{equation}

即:
\begin{equation}
    X = X_i + et \label{e0203}
\end{equation}

带入式\eqref{e0203}, 得到$i$点至平面的距离$t$:
\begin{equation}
    t = -e^TX_i-d = -ax_i-by_i-ch_i-d
\end{equation}

按各观测点至平面的距离平方和最小, 拟合求出$a,b,c,d$, 因此误差方程为
\begin{equation}
    v_i = -ax_i-by_i-ch_i-d
\end{equation}

条件式为:
\begin{equation}
    a^2 + b^2 + c^2 = 1
\end{equation}

即:
\begin{equation}
    a\partial a + b\partial b + c\partial c = 0
\end{equation}

每次迭代均需对$a,b,c$作单位化修正.

\subsection{按 \texorpdfstring{$h=ax+by+c$}{h=ax+by+c} 拟合}
为了避免直线平行于某坐标轴而出现数值问题, $(x_i\ y_i\ h_i)^T, i=1,2,\ldots,n$ 中的三个坐标分量的最大值和最小值之差 $\triangle x, \triangle y, \triangle h$.

在 $\triangle x, \triangle y, \triangle h$ 中, 若 $\triangle x$ 最小, 则按 $x=ay+bh+c$ 拟合. 若 $\triangle y$ 最小, 则按 $y=ah+bx+c$ 拟合. 若 $\triangle h$ 最小, 则按 $h=ax+by+c$ 拟合.

另外, 也可以先将坐标分量变换至$[-1,1]$区间, 拟合后再回代.

\subsection{按 \texorpdfstring{$d=1$}{d=1}  拟合}
平面发现单位向量条件需要线性化来实现, 每次迭代加上改正数后, 不满足单位向量. 可先令$d=1$, 按一下误差方程来拟合:
\begin{equation}
    v_i = a'x+b'y+c'h+1
\end{equation}

求得系数$a',b',c'$后:
\begin{equation}
    e=
    \begin{pmatrix}
        a\\
        b\\
        c\\
    \end{pmatrix}
    /\sqrt{a'^2+b'^2+c'^2},d=\frac{1}{\sqrt{a'^2+b'^2+c'^2}}
\end{equation}

为避免 $(x_i\ y_i\ h_i)^T, i=1,2,\ldots,n$中坐标分量值过大而引起的数值问题, 也可将其正则化后拟合.

\subsection{特征根拟合} \label{evfit}
对所有点累加矩阵:
\begin{equation}
    Q=\sum (X_i-X_m)(X_i-X_m)^T \label{e0210}
\end{equation}

式\eqref{e0210}中, $X_m$为坐标均值. 对$Q$矩阵做雅克比分解(对称正定矩阵的SVD分解):
\begin{equation}
    R^TQR=
    \begin{bmatrix}
        \lambda_1 & & \\
        & \lambda_2 & \\
        & & \lambda_3 \\
    \end{bmatrix}
    \label{e0211}
\end{equation}
上式正交矩阵中对应最小特征值的特征向量为平面的法线方向$e=(a\  b\ c)^T$.

平面方程为:
\begin{equation}
    e^TX+d=0
\end{equation}

式中的 $d=-e^TX_m$.

\subsection{点在平面上的投影}
求得平面方程\eqref{e0201}后, $i$点至平面的距离为:
\begin{equation}
    t = -e^TX_i-d = -ax_i-by_i-ch_i-d
\end{equation}

上式中的残差通常用于表示测得平面的平整度.

$i$点在平面上的投影点$X_pi=(xp_i\ yp_i\ hp_i)^T$为:
\begin{equation}
    X_pi=X_i+et,
    \left\{
    \begin{array}{rl}
        xp_i = & x_i + at \\
        yp_i = & y_i + bt \\
        hp_i = & h_i + ct 
    \end{array}
    \right.
\end{equation}

投影点坐标一定满足平面方程$e^{T}X+d=0$

\subsection{点在平面坐标系中的坐标}
建立平面坐标系$0-xxyyhh$, 其中$xx$和$yy$轴在平面内, $hh$轴与平面的法线方向一致, 各投影点在平面坐标系中的高程$hh=0$. $hh$轴在测量坐标系内的单位矢量为:
\begin{equation}
    e_{hh} = (a\ b\ c)^T
\end{equation}

坐标原点$oo$在测量坐标系$o-xyh$中的定义为:
\begin{equation}
\left\{
    \begin{array}{rl}
        x_{oo} = & \tfrac{\sum_{1}^{n}xp_i}{n} \\
        y_{oo} = & \tfrac{\sum_{1}^{n}yp_i}{n} \\
        h_{oo} = & \tfrac{\sum_{1}^{n}hp_i}{n}
    \end{array}
\right.
\end{equation}

$xx$轴在测量坐标系$o-xyz$中的方向定义为$oo$点至平面与$h$轴的角点, 即$xx$轴与测量坐标$h$轴共面.

若$c=0$, 平面与$h$轴平行, 平面与$h$轴无交点, 则:
\begin{equation}
    e_{xx}=(0\ 0\ 1)^T
\end{equation}

若$c\neq 0$, 平面与$h$轴交点为$(0\ 0\ -\dfrac{d}{c})^T$:
\begin{equation}
    e_{xx} = \frac{v}{||v||}, v=(-x_{oo}\ -y_{oo}\ -\tfrac{d}{c}-h_{oo})^T
\end{equation}

为了保证$xx$轴指向高程正向, 若$e_{xx}[2,0]<0$, 则$e_{xx}=-e_{xx}$. 同时
\begin{equation}
    e_{yy}=e_{xx}\times e_{hh}
\end{equation}

任意点在测量坐标系$o-xyh$中坐标$(x_i\ y_i\ h_i)^T$ 与其在平面坐标系$oo-xxyyhh$中坐标$(xx_i\ yy_i\ hh_i)^T$之间的关系为:
\begin{equation}
    \begin{bmatrix}
        x_i\\
        y_i\\
        h_i 
    \end{bmatrix}
    =
    \begin{bmatrix}
        x_{oo}\\
        y_{oo}\\
        h_{oo}
    \end{bmatrix}
    +R 
    \begin{bmatrix}
        xx_i\\
        yy_i\\
        hh_i
    \end{bmatrix}
    \label{e0221}
\end{equation}

式\eqref{e0221}中:
\begin{equation}
    R=(e_{xx}\ e_{yy}\ e_{hh})= 
    \begin{bmatrix}
        e_{xx1} & e_{yy1} & e_{hh1} \\
        e_{xx2} & e_{yy2} & e_{hh2} \\
        e_{xx3} & e_{yy3} & e_{hh3} \\
    \end{bmatrix}
\end{equation}

由平面坐标系中坐标求测量坐标系中坐标的转换关系为:
\begin{equation}
    \begin{bmatrix}
        xx_i\\
        yy_i\\
        hh_i
    \end{bmatrix}
    =-R^T
    \begin{bmatrix}
        x_{oo}\\
        y_{oo}\\
        h_{oo}
    \end{bmatrix}
    +R^T
    \begin{bmatrix}
        x_i\\
        y_i\\
        h_i 
    \end{bmatrix}
\end{equation}

以上旋转矩阵$R$若直接采用特征根拟合平面时, 雅克比分解可得到的正交矩阵$R$, 则其中$xx$轴相应于点最密集的方向, $hh$轴相应于点最不密集的地方, 即法线方向.