\section{研究背景}
随着国家经济的高速发展和城市规模的日益增大, 城市人口也越来越多. 城市居民便捷出行的需求与城市交通不便的矛盾也日益突出. 为了解决城市交通拥堵问题, 除了增加公交的数量和进一步优化公交线路外, 许多城市已经开始建设地铁和轻轨交通网络, 尤其是地下铁路工程建设, 因其拆迁工程少, 交通运输快捷等优点受到越来越多城市的青睐, 当城市的建设发展和人口规模达到一定的数量时需要进行地铁的规划, 设计与建设, 在地铁建设施工期间, 其周围的建筑物或构筑物以及相关荷载均会对地铁基坑的开挖建设产生影响. 同时由于城市当中建筑物较多, 地下市政管线分布众多, 所以在地铁工程的开工建设期间均需要对工程建设的安全性和结构的稳定性进行长期的工程变形监测, 并根据测量结果采取相应的保障性措施, 确保地铁工程建设的安全. 

随着开挖和降水等施工的进行, 地铁车站基坑和隧道工程建设的持续进行, 其原始自然土壤结构会发生一系列复杂的变化, 与此同时, 由于城市地下, 建构筑物较多, 地铁工程施工周边的建构筑物的土力平衡也发生了难以预测的改变. 地铁建设工程的沿线也伴随其他的工程建设共存的情况, 两者之间相互影响. 据有关方面不完全统计, 受自然地质和人为因素影响近年来地铁工程建设事故频发, 2002 年至 2016 年期间全国共发生于地铁工程建设有关的各类事故共 246 起, 严重威胁着地铁工程的安全建设和安全运营. 因此在地铁工程建设的施工, 运营等各个阶段需对结构主体进行一定的工程变形监测, 并根据数据监测的结果对变形状态进行分析和预警, 确保地铁工程建设, 运营等各个阶段安全, 和保障人民生命财产安全. 

工程变形监测的手段和方法有很多, 常用的方法有角度测量, 距离测量和水准测量等技术手段, 相较于其他变形监测手段, 常规的变形监测手段优点主要如下: 

\begin{enumerate}
    \item 可以得出整体变形状态;
    \item 对不同的观测值可以进行组网检核, 并根据检核的结果进行相应的精度评定;
    \item 常规的变形测量手段布网相对灵活, 同时能根据现场环境和建构筑物主体形态以及相应的精度要求进行测量监测. 
\end{enumerate}

近年来, 随着工程变形监测技术的不断发展, 一些新的技术和手段被不断应用到变形监测当中, 例如高精度 GNSS 变形测量, 近景摄影测量, 无人机倾斜摄影测量, 工程应变测量, 建构筑物准直测量和卫星激光测距, 卫星重力探测技术等. 这些新技术被广泛的应用到需要获取大量和更广阔地区的形变数据监测工程中, 从而使得变形数据的获取变得相对简单, 高效一些. 如何有效的利用监测数据, 分析监测数据, 并能够对变形体将来的物理变形趋势进行准确的预判和预警则变得越来越重要, 因为通过工程变形监测, 变形预报和变形预警, 可以有效的提高预防由此引发的不安全工程事故, 减少甚至采取措施消除不安全工程隐患. 

通过数学模型等技术手段对已有的沉降监测数据进行对比分析, 从而对变形体后期的变形趋势进行量化预测, 这是目前进行工程变形沉降观测的主要目的. 该项工作具有很强的现实指导意义, 对形变体的未来变形作预测, 进一步指导施工方采取相应的工程措施. 