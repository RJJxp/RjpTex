\chapter{毛瑞丰个人实习报告}


\section{实习经历}


\subsubsection{一}
在本次摄影测量与遥感实习中,我将上学期学习的摄影测量知识点进行了实际的巩固与运用。与小组其他成员互帮互助,通力完成了航片判读、调绘到外业像控点的联测、像片同名点的量取,以及内定向元素、相对定向元素、模型坐标和绝对定向参数的计算等内容。基于共同的成果完成了大礼堂、西北一地物坐标的CAD成图。

实习中我们将课本上枯燥的公式与航摄影像相结合运用,使理论不再是空中阁楼,而与实际紧紧地连接起来。我们更加相信:实践是检验真理的唯一标准。没有通过本次实习的锻炼,没有通过程序的编写,摄影测量基础的运用对我们来说将是一个棘手的问题。在上学期,我们完成了前方—后方交会法的编程,让我们对此方法理解较深,这次的实习我们动手完成了相对-绝对定向法,对摄影测量整个过程流程有了更加深刻的理解。

\subsubsection{二}
本次实习,我们发现了许多自身的不足之处:在外业实习中,以房屋角点为测量点时,如果过于追求测量的准确性,将准心对准角点,反而很有可能对到角点之后的地方去。在更换过若干个测站点,我向现实妥协,退而求其次,试着将准心对准房屋角点偏下的地方,最终才把将近五十米的误差消除了。此次问题并未拖延我们过多的外业时间,迄今为止我们一同经过多次的实验,面对突发问题和BUG更加成熟老练。

\subsubsection{三}
第一次接触了Kinect Fusion建模,补充了我们对摄影测量的认识,Kinect的专业令我难以想象这最开始居然是孩童的玩具。从玩具到产品,我看到的是开发者和使用者的努力与协作。Kinect的实验给我们不仅仅是科技的新鲜感,而且还激发了我们探索知识海洋的好奇心。

\section{程序集}

\subsection{内定向}

\begin{lstlisting}[caption=Interior\_Orientation.m文件]
mark=textread(' C:\Users\毛瑞丰\Desktop\摄影测量实验\camera.use');
pixel=[7679 13823;0 13823 ;0 0;7679 0];
y=reshape(mark',[8 1]);
X=zeros(8,6); 
for i=1:4,
    X(2*i-1:2*i,:)=[1,pixel(i,1),pixel(i,2),0,0,0;
                    0,0,0,1,pixel(i,1),pixel(i,2)];
end
N=inv(X'*X);
beta=N*X'*y
csvwrite('C:\Users\毛瑞丰\Desktop\摄影测量实习\毛瑞丰\ Interior_Orientation.csv',beta);
r=size(y,1)-6;
e=y-X*beta;
sigma=sqrt(norm(e)/r);
cc=xlsread('C:\Users\毛瑞丰\Desktop\摄影测量实验\点之记.xlsx',1,'B3:E13');
retVal=zeros(size(cc,1),4);
for i=1:size(cc,1),
    line=[1 cc(i,1) cc(i,2) 0 0 0;
    0 0 0 1 cc(i,1) cc(i,2);
    1 cc(i,3) cc(i,4) 0 0 0;
    0 0 0 1 cc(i,3) cc(i,4)]*beta;
    retVal(i,:)=line';
end
xlswrite('C:\Users\毛瑞丰\Desktop\摄影测量实验\框标点.xlsx',retVal);
\end{lstlisting}

模型:
\begin{equation}
\begin{bmatrix}
x \\ y
\end{bmatrix}
=\begin{bmatrix}
h_1 & h_2 \\
k_1 & k_2 
\end{bmatrix}
\begin{bmatrix}
i \\ j
\end{bmatrix}
+\begin{bmatrix}
h_0 \\ k_0
\end{bmatrix}
\end{equation}

得出的内定向参数如下:
\begin{equation}
\begin{array}{lll}
h_0=-46.08 & h_1=0.012 & h_2=0 \\
k_0=82.944 & k_1=0 & k_2=-0.012
\end{array}
\end{equation}

\subsection{相对定向}

\begin{lstlisting}[caption=Relative\_Orientation.m文件]
syms theta;
syms phi1 kappa1;
syms phi2 omega2 kappa2;
f=120; %mm
x0=0; %principal point shift
y0=0;
Rx=[1 0 0;0 cos(theta) -sin(theta);0 sin(theta) cos(theta)];
Ry=[cos(theta) 0 -sin(theta); 0 1 0; sin(theta) 0 cos(theta)];
Rz=[cos(theta) -sin(theta) 0; sin(theta) cos(theta) 0;0 0 1];
R1=subs(Ry,phi1)*subs(Rz,kappa1);
R2=subs(Ry,phi2)*subs(Rx,omega2)*subs(Rz,kappa2);
mat=xlsread('C:\Users\毛瑞丰\Desktop\摄影测量实验\框标点.xlsx',1,'A1:D11');
mat1=mat(:,1:2);
mat2=mat(:,3:4);
mat1=[mat1,-f*ones(size(mat1,1),1)]';
mat2=[mat2,-f*ones(size(mat2,1),1)]';
mat11=R1*mat1;
mat22=R2*mat2;
r=(mat11(2,:).*mat22(3,:)-mat11(3,:).*mat22(2,:)).';
x=[0,0,0,0,0]';
v=GaussNewton(r,x,2e-7)
csvwrite('C:\Users\毛瑞丰\Desktop\摄影测量实验\毛瑞丰\ Relative_Orientation.csv',v);
\end{lstlisting}


\begin{lstlisting}[caption=Absolute\_Orientation.m文件]
function [retVal]=GaussNewton(f,x,error)
syms phi1 kappa1 phi2 omega2 kappa2;
v=[phi1 kappa1 phi2 omega2 kappa2];
j=jacobian(f,v);
J=eval(subs(j,v,x'));
F=eval(subs(f,v,x'));
k=0;
d=1;
while norm(d)>error,
    d=-inv(J'*J)*J'*F;
    x=x+d;
    J=eval(subs(j,v,x'));
    F=eval(subs(f,v,x'))
    k=k+1
    disp(norm(J'*F));
end
retVal=x;
\end{lstlisting}

得出的相对定向参数如下:
\begin{equation}
\begin{array}{lll}
\phi_1=-0.012455 & & \kappa_1=0.028415 \\
\phi_2=0.013341 & \omega_2=-0.022506 & \kappa_2=0.036366
\end{array}
\end{equation}

\subsection{绝对定向}

\begin{lstlisting}[caption=Absolute\_Orientation.m文件]
allpoints=xlsread('C:\Users\毛瑞丰\Desktop\摄影测量实验\inspect副本.xlsx');
xyz=allpoints(:,6:8);
XYZ=allpoints(:,9:11);
XYZmean=mean(XYZ);
XYZ=XYZ-ones(size(XYZ))*diag(mean(XYZ));
xyz=xyz-ones(size(xyz))*diag(mean(xyz));
syms theta;
Rx=[1 0 0;0 cos(theta) -sin(theta);0 sin(theta) cos(theta)];
Ry=[cos(theta) 0 -sin(theta);0 1 0;sin(theta) 0 cos(theta)];
Rz=[cos(theta) -sin(theta) 0;sin(theta) cos(theta) 0;0 0 1];
syms phi omega kappa lambda dx dy dz;
R=subs(Ry,phi)*subs(Rx,omega)*subs(Rz,kappa);
groundStretch=reshape(XYZ',size(XYZ,1)*size(XYZ,2),1);
rotateModel=lambda*R*xyz';
r=rotateModel+diag([dx dy dz])*ones(size(rotateModel));
r=reshape(r,size(r,1)*size(r,2),1);
r=r-groundStretch;
x=[0 0 0 1 0 0 0]';
v=GaussNewton2(r,x,2e-7)
csvwrite('C:\Users\毛瑞丰\Desktop\摄影测量实验\毛瑞丰\Absolute_Orientation.csv',[v;XYZmean']);
\end{lstlisting}


\begin{lstlisting}[caption=GaussNewton2.m文件]
function [retVal]=GaussNewton2(f,x,error)
syms phi omega kappa lambda dx dy dz;
v=[phi omega kappa lambda dx dy dz];
j=jacobian(f,v);
J=eval(subs(j,v,x'));
F=eval(subs(f,v,x'));
k=0;
while norm(J'*F)>error,
    d=-inv(J'*J)*J'*F;
    x=x+d
    J=eval(subs(j,v,x'));
    F=eval(subs(f,v,x'));
    k=k+1
    disp(norm(J'*F));
end
retVal=x;
\end{lstlisting}

得出的绝对定向参数如下:
\begin{equation}
\begin{array}{lll}
\phi=0.011264 & \omega=0.005638 & \kappa=0.0053766 \\
& \lambda=1.1318 & \\
dx=0 & dy=0 & dz=0  
\end{array}
\end{equation}

\section{调绘图件}

\begin{figure}[htbp]
\centering
\caption{调绘图概览}
\includegraphics[width=0.9\textwidth]{diaohui.jpg}
\end{figure}

\begin{figure}[htbp]
\centering
\caption{调绘图细节}
\includegraphics[width=0.9\textwidth]{diaohui1.jpg}
\end{figure}

\section{同名点}

\begin{table}[htbp]
    \centering
    \begin{tabular}{lrrrr}
        \toprule
        & \multicolumn{2}{c}{63} & \multicolumn{2}{c}{64} \\ \hline
        点号 & \multicolumn{1}{c}{i} & \multicolumn{1}{c}{j} & \multicolumn{1}{c}{i} & \multicolumn{1}{c}{j} \\ \midrule
        15 & 5220 & 13116 & 2015 & 12700 \\
        16 & 7133 & 12458 & 3893 & 12056 \\
        17 & 6480 & 10899 & 3268 & 10534 \\
        18 & 5463 & 9850 & 2249 & 9501 \\
        19 & 5545 & 10612 & 2339 & 10256 \\ \bottomrule
    \end{tabular}
    \end{table}