%!TEX program = xelatex
\documentclass[twoside,color=blue,mathpazo,titlestyle=hang,10pt]{elegantbook}
% \email{elegantlatex2e@gmail.com}
% \title{统计观点下的测量平差}
% \zhtitle{统计观点下的测量平差}
% \zhend{}
% \entitle{\ }
% \enend{}
% \version{2.10}
% \myquote{Victory won\rq t come to us unless we go to it.}
% \logo{logo.pdf}
% \cover{cover.pdf}

% title 
\setcounter{tocdepth}{2}
\numberwithin{equation}{section}


\title{\Huge\emph{2018年摄影测量与遥感}\\实习报告}
\author
{
\Large\textbf{同济大学测绘与地理信息学院} \\
测绘工程\quad 第七组 \\
余周炜\quad 1551126 \\
王雪辰\quad 1551140\\
毛瑞丰\quad 1551175\\
贾\hspace{1em}锐\quad 1551181\\
陈\hspace{1em}晨\quad 1551156
}
\date{2018年3月5日至2018年4月28日}

%green color
%    \definecolor{main1}{RGB}{0,120,2}
%    \definecolor{seco1}{RGB}{230,90,7}
%    \definecolor{thid1}{RGB}{0,160,152}

   \definecolor{main1}{RGB}{0,0,0}
   \definecolor{seco1}{RGB}{0,0,0}
   \definecolor{thid1}{RGB}{0,0,0}
%cyan color
   \definecolor{main2}{RGB}{0,175,152}
   \definecolor{seco2}{RGB}{239,126,30}
   \definecolor{thid2}{RGB}{120,8,13}
%blue color
   \definecolor{main3}{RGB}{100,149,237}
   \definecolor{seco3}{RGB}{180,50,131}
   \definecolor{thid3}{RGB}{7,127,128}

\usepackage{siunitx}
\usepackage[section]{placeins}
\usepackage{makecell}
\usepackage{listings}

\newcommand\mgape[1]{\gape{$\vcenter{\hbox{#1}}$}}

\newfontfamily\conso{Inconsolata}
\definecolor{mygreen}{rgb}{0,0.6,0}
\definecolor{mygray}{rgb}{0.5,0.5,0.5}
\definecolor{mymauve}{rgb}{0.58,0,0.82}
\lstset{backgroundcolor=\color{white},
numbers=left,
numberstyle=\small,
frame=lines,
basicstyle=\conso,
columns=fullflexible,
breaklines=true,                 % automatic line breaking only at whitespace
tabsize=4,
keywordstyle=\color{blue},
commentstyle=\color{mygreen},
stringstyle=\color{mymauve}\ttfamily,
language=Matlab}
% \usepackage[numbered,autolinebreaks,useliterate]{mcode}

% \usepackage{mdwlist}
\usepackage{lipsum}
\usepackage{colortbl}
\usepackage{texnames}
\usepackage{metalogo}
\usepackage{mflogo}
\usepackage{mathtools}
\usepackage{algorithm}
% \usepackage{algorithmicx}
\usepackage{algpseudocode}
\usepackage{longtable}
\usepackage{supertabular}
\usepackage{multirow}
\renewcommand{\algorithmicrequire}{\textbf{Input:}}  % Use Input in the format of Algorithm  
\renewcommand{\algorithmicensure}{\textbf{Output:}} % Use Output in the format of Algorithm 


\RequirePackage{cite}
\RequirePackage[square,numbers]{natbib}
\newlength{\notationgap}
\setlength{\notationgap}{1pc}

\newlength{\figwidth}
\setlength{\figwidth}{26pc}

\begin{document}
\frontmatter

\maketitle

\newcommand{\argmax}{\arg\max}
\newcommand{\argmin}{\arg\min}
\newcommand{\norm}[1]{\left\lVert#1\right\rVert}
% \newcommand{\norm}[2]{\left\lVert#1\right\rVert_{#2}}

\newcommand{\var}{\mathrm{Var}}
\newcommand{\cov}{\mathrm{Cov}}
\newcommand{\tr}{\mathrm{tr}}
\newcommand{\rank}{\mathrm{rank}}
\newcommand{\myt}{^\mathrm{T}}
\newcommand{\mye}{\mathrm{E}}
% \newcommand{\degree}{^\circ}
\newcommand{\ud}{\,\mathrm{d}}
\newcommand{\diag}{\mathrm{diag}}
\newcommand{\ones}{\mathrm{ones}}
\newcommand{\col}{\mathrm{Col}}
\newcommand{\mean}{\mathrm{mean}}
\newcommand{\myvec}{\mathrm{vec}}
% \newcommand{\ln}{\mathrm{ln}}
% \newcommand{\Pr}{\mathrm{P}}
% \newcommand{\myl}{^{-1}}
% \newcommand{\mydet}{\text{det}}
% \newcommand{\Setnature}[1][n]{\{0, 1, \dots, #1 \}}

% Vector
\newcommand{\Va}{\boldsymbol{\mathit{a}}}
\newcommand{\Vb}{\boldsymbol{\mathit{b}}}
\newcommand{\Vc}{\boldsymbol{\mathit{c}}}
\newcommand{\Vd}{\boldsymbol{\mathit{d}}}
\newcommand{\Ve}{\boldsymbol{\mathit{e}}}
\newcommand{\Vf}{\boldsymbol{\mathit{f}}}
\newcommand{\Vg}{\boldsymbol{\mathit{g}}}
\newcommand{\Vh}{\boldsymbol{\mathit{h}}}
\newcommand{\Vi}{\boldsymbol{\mathit{i}}}
\newcommand{\Vj}{\boldsymbol{\mathit{j}}}
\newcommand{\Vk}{\boldsymbol{\mathit{k}}}
\newcommand{\Vl}{\boldsymbol{\mathit{l}}}
\newcommand{\Vm}{\boldsymbol{\mathit{m}}}
\newcommand{\Vn}{\boldsymbol{\mathit{n}}}
\newcommand{\Vo}{\boldsymbol{\mathit{o}}}
\newcommand{\Vp}{\boldsymbol{\mathit{p}}}
\newcommand{\Vq}{\boldsymbol{\mathit{q}}}
\newcommand{\Vr}{\boldsymbol{\mathit{r}}}
\newcommand{\Vs}{\boldsymbol{\mathit{s}}}
\newcommand{\Vt}{\boldsymbol{\mathit{t}}}
\newcommand{\Vu}{\boldsymbol{\mathit{u}}}
\newcommand{\Vv}{\boldsymbol{\mathit{v}}}
\newcommand{\Vw}{\boldsymbol{\mathit{w}}}
\newcommand{\Vx}{\boldsymbol{\mathit{x}}}
\newcommand{\Vy}{\boldsymbol{\mathit{y}}}
\newcommand{\Vz}{\boldsymbol{\mathit{z}}}

% Matrix
\newcommand{\MA}{\boldsymbol{\mathit{A}}}
\newcommand{\MB}{\boldsymbol{\mathit{B}}}
\newcommand{\MC}{\boldsymbol{\mathit{C}}}
\newcommand{\MD}{\boldsymbol{\mathit{D}}}
\newcommand{\ME}{\boldsymbol{\mathit{E}}}
\newcommand{\myMF}{\boldsymbol{\mathit{F}}}
\newcommand{\MG}{\boldsymbol{\mathit{G}}}
\newcommand{\MH}{\boldsymbol{\mathit{H}}}
\newcommand{\MI}{\boldsymbol{\mathit{I}}}
\newcommand{\MJ}{\boldsymbol{\mathit{J}}}
\newcommand{\MK}{\boldsymbol{\mathit{K}}}
\newcommand{\ML}{\boldsymbol{\mathit{L}}}
\newcommand{\MM}{\boldsymbol{\mathit{M}}}
% \newcommand{\MN}{\boldsymbol{\mathit{N}}}
\newcommand{\MO}{\boldsymbol{\mathit{O}}}
% \newcommand{\MP}{\boldsymbol{\mathit{P}}}
\newcommand{\MQ}{\boldsymbol{\mathit{Q}}}
\newcommand{\MR}{\boldsymbol{\mathit{R}}}
\newcommand{\MS}{\boldsymbol{\mathit{S}}}
\newcommand{\MT}{\boldsymbol{\mathit{T}}}
\newcommand{\MU}{\boldsymbol{\mathit{U}}}
\newcommand{\MV}{\boldsymbol{\mathit{V}}}
% \newcommand{\MW}{\boldsymbol{\mathit{W}}}
\newcommand{\MX}{\boldsymbol{\mathit{X}}}
\newcommand{\MY}{\boldsymbol{\mathit{Y}}}
\newcommand{\MZ}{\boldsymbol{\mathit{Z}}}


% Random Scala
\newcommand{\RSa}{\mathrm{a}}
\newcommand{\RSb}{\mathrm{b}}
\newcommand{\RSc}{\mathrm{c}}
\newcommand{\RSd}{\mathrm{d}}
\newcommand{\RSe}{\mathrm{e}}
\newcommand{\RSf}{\mathrm{f}}
\newcommand{\RSg}{\mathrm{g}}
\newcommand{\RSh}{\mathrm{h}}
\newcommand{\RSi}{\mathrm{i}}
\newcommand{\RSj}{\mathrm{j}}
\newcommand{\RSk}{\mathrm{k}}
\newcommand{\RSl}{\mathrm{l}}
\newcommand{\RSm}{\mathrm{m}}
\newcommand{\RSn}{\mathrm{n}}
\newcommand{\RSo}{\mathrm{o}}
\newcommand{\RSp}{\mathrm{p}}
\newcommand{\RSq}{\mathrm{q}}
\newcommand{\RSr}{\mathrm{r}}
\newcommand{\RSs}{\mathrm{s}}
\newcommand{\RSt}{\mathrm{t}}
\newcommand{\RSu}{\mathrm{u}}
\newcommand{\RSv}{\mathrm{v}}
\newcommand{\RSw}{\mathrm{w}}
\newcommand{\RSx}{\mathrm{x}}
\newcommand{\RSy}{\mathrm{y}}
\newcommand{\RSz}{\mathrm{z}}


% Random Vector
\newcommand{\RVa}{\mathbf{a}}
\newcommand{\RVb}{\mathbf{b}}
\newcommand{\RVc}{\mathbf{c}}
\newcommand{\RVd}{\mathbf{d}}
\newcommand{\RVe}{\mathbf{e}}
\newcommand{\RVf}{\mathbf{f}}
\newcommand{\RVg}{\mathbf{g}}
\newcommand{\RVh}{\mathbf{h}}
\newcommand{\RVi}{\mathbf{i}}
\newcommand{\RVj}{\mathbf{j}}
\newcommand{\RVk}{\mathbf{k}}
\newcommand{\RVl}{\mathbf{l}}
\newcommand{\RVm}{\mathbf{m}}
\newcommand{\RVn}{\mathbf{n}}
\newcommand{\RVo}{\mathbf{o}}
\newcommand{\RVp}{\mathbf{p}}
\newcommand{\RVq}{\mathbf{q}}
\newcommand{\RVr}{\mathbf{r}}
\newcommand{\RVs}{\mathbf{s}}
\newcommand{\RVt}{\mathbf{t}}
\newcommand{\RVu}{\mathbf{u}}
\newcommand{\RVv}{\mathbf{v}}
\newcommand{\RVw}{\mathbf{w}}
\newcommand{\RVx}{\mathbf{x}}
\newcommand{\RVy}{\mathbf{y}}
\newcommand{\RVz}{\mathbf{z}}
\newcommand{\Rvep}{\mathbf{\varepsilon}}

% Random Matrix
% will be added later
\newcommand{\RMX}{\boldsymbol{\mathrm{X}}}
\newcommand{\RMA}{\boldsymbol{\mathrm{A}}}
\newcommand{\RMZ}{\boldsymbol{\mathrm{Z}}}

\newcommand{\Valpha}{\boldsymbol{\alpha}}
\newcommand{\Vbeta}{\boldsymbol{\beta}}
\newcommand{\Vtheta}{\boldsymbol{\theta}}
\newcommand{\Vlambda}{\boldsymbol{\lambda}}
\newcommand{\VLambda}{\boldsymbol{\Lambda}}
\newcommand{\Vepsilon}{\boldsymbol{\epsilon}}
\newcommand{\Vvarep}{\boldsymbol{\varepsilon}}
\newcommand{\VOmega}{\boldsymbol{\Omega}}
\newcommand{\Vmu}{\boldsymbol{\mu}}
\newcommand{\VPhi}{\boldsymbol{\Phi}}
\newcommand{\Vsigma}{\boldsymbol{\sigma}}
\newcommand{\VSigma}{\boldsymbol{\Sigma}}
\newcommand{\Vrho}{\boldsymbol{\rho}}
\newcommand{\Vgamma}{\boldsymbol{\gamma}}
\newcommand{\Vomega}{\boldsymbol{\omega}}
\newcommand{\Vpsi}{\boldsymbol{\psi}}
\newcommand{\Vzeta}{\boldsymbol{\zeta}}
\newcommand{\Vone}{\boldsymbol{1}}
\newcommand{\Vzero}{\boldsymbol{0}}
\newcommand{\VDelta}{\boldsymbol{\Delta}}


\newcommand{\CalB}{\mathcal{B}}
\newcommand{\CalC}{\mathcal{C}}
\newcommand{\CalG}{\mathcal{G}}
\newcommand{\CalH}{\mathcal{H}}
\newcommand{\CalL}{\mathcal{L}}
\newcommand{\CalM}{\mathcal{M}}
\newcommand{\CalN}{\mathcal{N}}
\newcommand{\CalO}{\mathcal{O}}
\newcommand{\CalD}{\mathcal{D}}
\newcommand{\CalU}{\mathcal{U}}
\newcommand{\CalF}{\mathcal{F}}
\newcommand{\CalT}{\mathcal{T}}

% Set
\newcommand{\SetA}{\mathbb{A}}
\newcommand{\SetB}{\mathbb{B}}
\newcommand{\SetD}{\mathbb{D}}
\newcommand{\SetE}{\mathbb{E}}
\newcommand{\SetG}{\mathbb{G}}
\newcommand{\SetL}{\mathbb{L}}
\newcommand{\SetN}{\mathbb{N}}
\newcommand{\SetR}{\mathbb{R}}
\newcommand{\SetS}{\mathbb{S}}
\newcommand{\SetT}{\mathbb{T}}
\newcommand{\SetV}{\mathbb{V}}
\newcommand{\SetX}{\mathbb{X}}
\newcommand{\SetY}{\mathbb{Y}}
\tableofcontents

\mainmatter

\part{小组实习报告}
\include{chapter/chap13}
\chapter{实习各项目具体情况报告}

\section{航测外业实习}

航测外业实习包括:像片联测、航片判读调绘。

\subsection{像片联测}

\label{sub:waiye}
利用摄影测量方法进行地形图测绘有几种方法,但不论采用哪种方法都需要测定一些像点的对应的地面坐标,这些测定了地面坐标的像点称为\textbf{像片控制点}。将像片上的点根据大地点及水准点测得其平面位置和高程的过程称为\textbf{像片联测}。

摄影测量与遥感实习的像片联测工作包括像控点坐标的量测与点之记的制作。实习中量测的像控点将用于后面航测内业实习中的绝对定向以及同济航空影像的纠正。选取像片控制点的原则如下:

\begin{enumerate}[label=\alph*.]
    \item 控制点必须在影像上能清晰辨认;
    \item 实地能找到与影像一致的控制点;
    \item 该点易于测量。
\end{enumerate}

像片联测实习在同济大学校本部范围内进行,测量所选坐标系是同济独立坐标系。实际测量过程使用SET530R索佳全站仪进行。该全站仪可以无棱镜测量,无棱镜测程为150米(或采用其他类似类型的全站仪进行)。测量工作利用校园内布置的导线网作为控制施测。像控点量测步骤及方法如下:

\begin{enumerate}
\item \textbf{勘探选点}\\
从像片上选取同济大学校园内若干个(不少于6个)可量测特征点作为像片控制点,要求分布大致均匀,能在像片上清晰判断,实地便于测量。并到实地踏勘确认。

\item \textbf{坐标量测}\\
利用同济校园内布设的导线网和高程网作为控制基础,利用电子全站仪测量所选像控点的坐标$(X,Y,Z)$。测量可采用普通测量中所学的各种方法(如支导线法、前方交会等方法),观测需量测的像片控制点,并将测量结果记录下来$(X,Y,Z)$;对每一像控点测量都应有检核方法,得到该像控点的两次量测结果$(X_1,Y_1,Z_1),(X_2,Y_2,Z_2)$。

\item \textbf{测量精度的要求}\\
为提高测量精度,要求每个待定控制点的两站量测值之差满足:$dX<0.02\si{m},
dY<0.02\si{\metre},dZ<0.1\si{\metre}$的要求。由于校园内建筑物较为密集,互相通视的点很少能达到三个,在测量中还需利用全站仪后视定向之后直接打后视点的坐标来进行检查,该值与所给的已知值的误差,$X(N)$方向为$1\sim 2\si{\milli\metre}$,$Y(E)$方向为$1\sim 2\si{cm}$,$Z$(高程)方向的误差小于$3\si{cm}$。最后取两次观测结果的平均值作为最终像控点的坐标量测成果。

在像控点量测结束后应进行点之记的制作,以利于后续处理人员在航片影像上对所测控制点的位置能确认无误。由于采用的是航片数字影像,因此点之记的制作过程如下:
\begin{enumerate}[label=\Alph*.]
\item 在所给航片影像上判断所测像控点的位置,并将该点的像素坐标记录下来---$(i,j)$;
\item 以该点为中心,在原影像上取一100px$\times$100px的方块,其中心点即所测量的纠正控制点位置,将该图像存为点之记图片文件,文件以该像控点的点号命名;
\item 对该点位位置进行简要的说明;
\item 生成点之记文件。
\end{enumerate}
\end{enumerate}

\subsection{像片判读调绘实习}

像片判读是根据航摄像片上地物影像的特征来识别地物的实质内容。像片判读分野外判读、室内判读和综合判读。在野外实地对像片上影像进行的像片判读即为野外判读。一般将两者结合起来,用野外判读和实验方法先取得标准判读像片,然后在室内以标准判读像片为准绳与其他航摄像片相比照,这样可以节省大量野外工作。像片调绘是在像片判读的基础上实地检查室内判读的正确性,并在影像上进行标注;对影像上没有的地物通过补测与调绘后绘注于影像上。

\noindent 实习中调绘范围:同济大学本部校园、南校区及附近地区;\\
调绘影像的航摄时间:2005年,\\
影像分辨率:约$0.2\si{m/px}$

实习中调绘判读的影像主要是同济校本部,调绘判读的内容包括:
\begin{enumerate}
\item 建筑物的名称、层数、结构(包括砖石、砖木和钢混结构)、楼内分区单位及用途(商用、教学、居住、办公);
\item 道路名称、宽度,其中宽度需用皮尺到实地量测;
\item 绿化(细分为林地和草地);
\item 土地用途(包括教学、办公、商住、住宿四类)。
\end{enumerate}

此外,还将同济校园范围在影像上标注出来。此次实习中调绘的目的主要是确定校园中地物名称等属性,为内业成图提供属性信息。

根据所给航片影像到实地进行判读调绘,\textbf{要求}:走到、看到、问到。

调绘结果利用通用图像处理软件photoshop软件,将实地调查所得资料标注在校园数字影像上。\textbf{要求}:注记文字的位置要能明确表明所指地物;有变化的地物要予以圈出,并标出当前的名称及其属性。用photoshop软件将调查成果注记在图上,生成一幅调绘图像。


\section{航测内业实习}
\label{sec:neiye}
航测内业实习包括:同名点坐标量测(相对定向点、像片控制点、待成图地物点),编程(包括数字内定向、解析相对定向、解析绝对定向、前方交会),部分地物的CAD成图以及航片影像的数字微分处理。

\subsection{同名点坐标量测}
\label{sub:tongming}

同名点坐标量测包括相对定向点、像片控制点、待定地物点的量测,在本实习中,利用photoshop软件打开左右航片影像,人工判断左右片的同名点并将同名点对的行、列号记录下来,生成同名点对文件。

% (如表\ref{tab:23})。

% \begin{table}[htbp]
% \centering
% \begin{tabular}{ccccc}
% \toprule
% ID & \multicolumn{2}{c}{左片(119058)} & \multicolumn{2}{c}{右片(119057)} \\
% \midrule
%  & i & j & i & j \\
% $x_1$ & 7668 & 86 & 4651 & 463 \\
% $x_2$ & 6198 & 195 & 3205 & 562 \\
% $x_3$ & 3683 & 346 & 699 & 702 \\
% $x_4$ & 6914 & 2560 & 3899 & 2909 \\
% \bottomrule
% \end{tabular}
% \caption{左右片坐标}
% \label{tab:23}
% \end{table}

\subsubsection{相对定向点的量测}

像对的相对定向至少需要五个同名点,但是在模拟与解析测图作业中常常利用六点法或者九点法选点。六点法为:围绕两张像片的重叠区域四周上下两边上左右各选一点,中间一条线的两端各选一点,一共六点;九点法为:上下两边和中间一条线的左中右各选一点,一共九点。但是在数字摄影测量中相对定向数目大大超过上面的点数。

实习中每人完成5点的量测,为了避免量测的点集中在一个区域,在量测中可将航片由上到下分为六条带,每条上量测4-6点,总共量测了30-40点,生成相对定向点文件,用于参与后面的相对定向的平差解算。

\subsubsection{像片控制点的像点坐标量测}

利用像控点点之记图像,文字注记以及在所给航片影像上的像素位置,判断该像控点在其他航片影像(左片或右片)上的位置将该点的像素坐标记录下来---$(i,j)$,生成此像控点的坐标文件,用于前方交会计算该像控点的模型坐标。

\subsubsection{待成图地物点的像点坐标量测}

对于待定地物点,采用同样的方法,量测该地物特征点(如房屋角点、道路拐弯点等)在左右航片上的像素位置,并记录下来。注意:在记录这些地物点时,应给其编号、按照一定的顺序记录,并绘出点位连接的草图,以便于后续的CAD成图(类似于普通测量中的数字平板测图记录的要求)。

\subsection{解析摄影测量编程练习}
\label{sub:biancheng}

\begin{figure}[htbp]
\centering
\caption{流程图}
\includegraphics[width=\textwidth]{flow.PNG}
\label{fig:22}
\end{figure}

实习中解析摄影测量编程包括数字内定向,解析相对定向,前方交会,解析绝对定向,最终实现相对定向---绝对定向解析摄影测量的整个过程,主要流程如图\ref{fig:22}所示。编程可采用同学已学的编程工具如VB、matlab、VC等。



\subsubsection{数字内定向}

\begin{enumerate}
\item \textbf{内定向参数求取}\\
方法:利用航摄像片上的四个框标点的理论位置以及四个框标点的像素坐标为依据,通过最小二乘法计算内定向参数。\\
计算式:
\begin{equation}
\begin{bmatrix}
x \\ y
\end{bmatrix}
=\begin{bmatrix}
h_1 & h_2 \\
k_1 & k_2 
\end{bmatrix}
\begin{bmatrix}
i \\ j
\end{bmatrix}
+\begin{bmatrix}
h_0 \\ k_0
\end{bmatrix}
\label{eq:convert}
\end{equation}
参数:$h_0,h_1,h_2,k_0,k_1,k_2$ \\
求解过程:列误差方程$\rightarrow$ 法化 $\rightarrow$ 求参数
\begin{solution}
将式\ref{eq:convert}列为关于参数的形式:
\begin{equation}
\begin{bmatrix}
x \\ y 
\end{bmatrix}
=
\begin{bmatrix}
1 & i & j & 0 & 0 & 0\\
0 & 0 & 0 & 1 & i & j
\end{bmatrix}
\begin{bmatrix}
h_0 \\ h_1 \\ h_2 \\ k_0 \\ k_1 \\ k_2
\end{bmatrix}
+\Vvarep
\label{eq:erroreq}
\end{equation}

将4个点的框标坐标$(x_1,y_1),\cdots,(x_4,y_4)$和像素坐标$(i_1,j_1,\cdots,i_4,j_4)$带入可列出8个方程:
\begin{equation}
\begin{bmatrix}
x_1 \\ y_1 \\ \vdots \\ x_4 \\ y_4 
\end{bmatrix}
=
\begin{bmatrix}
1 & i_1 & j_1 & 0 & 0 & 0\\
0 & 0 & 0 & 1 & i_1 & j_1 \\
\vdots & & \vdots & & & \vdots \\
1 & i_4 & j_4 & 0 & 0 & 0\\
0 & 0 & 0 & 1 & i_4 & j_4 
\end{bmatrix}
\begin{bmatrix}
h_0 \\ h_1 \\ h_2 \\ k_0 \\ k_1 \\ k_2
\end{bmatrix}
+\Vvarep 
\end{equation}
记为:
\begin{equation}
\Vy=\MX\Vbeta+\Vvarep
\end{equation}
参数的最小二乘估计为
\begin{equation}
\hat\Vbeta=(\MX\myt\MX)^{-1}\MX\myt\Vy
\end{equation}
\end{solution}
\item \textbf{将像点扫描坐标转化为框标坐标}\\
用参数的最小二乘估计代替公式\ref{eq:erroreq}中的参数向量可将点P的像素坐标$(i,j)$转换到框标坐标$(x,y)$:
\begin{equation}
\hat\Vy_p=\MX_p\hat\Vbeta
\end{equation}
\end{enumerate}

\subsubsection{单独相对相对定向编程}

\noindent\textbf{目的}:编写单独像对相对定向程序,利用\ref{sub:tongming}中量测的相对定向同名点解算两像片之间的相对定向元素并进一步利用前方交会公式计算模型点坐标,为后续的绝对定向、待定地物点的坐标计算做准备。\\
\textbf{地点}:机房、宿舍。\\
\textbf{使用仪器与设备}:电脑。 \\
\textbf{相对定向原理与步骤}:根据同名射线相交位于同一个平面上,利用共面条件方程,以及量测的同名像点坐标解求五个相对定向元素。

单独像对相对定向是以摄影基线作为像空间辅助坐标系的$X$轴,以左摄影中心$S_1$为原点,左像片主光轴与摄影基线$B$组成的主核面为$XZ$平面,构成右手直角坐标系$S_1-X_1Y_1Z_1$。此时,左、右像片的相对方位元素分别为:
\begin{description}
\item[左片] $Xs_1=0,Ys_1=0,Zs_1=0,\phi_1,\omega_1=0,\kappa_1$
\item[右片] $Xs_2=b,Ys_2=0,Zs_2=0,\phi_2,\omega_2,\kappa_2$
\end{description}
需要求解的相对定向元素为:左片$\phi_1,\kappa_1$,右片$\phi_2,\omega_2,\kappa_2$。

在介绍解算算法之前我们先介绍一下Gauss-Newton法。它的基本思想是使用泰勒级数展开式去近似地代替非线性回归模型,然后通过多次迭代,多次修正回归系数,使回归系数不断逼近非线性回归模型的最佳回归系数,最后使原模型的残差平方和达到最小。\footnote{这个算法的停车条件是驻点条件,或者用改正数条件$||x^{(k)}-x^{(k-1)}||\ge err$}

\begin{algorithm}[htbp]
\caption{Gauss-Newton algorithm}
\begin{algorithmic}[1]
\Require{迭代初值$\Vbeta$,m个函数$\Vr=(r_1,r_2,\cdots,r_m)\myt$,误差限err}
\Ensure{$\Vbeta$的非线性最小二乘估计$\hat\Vbeta$。}
\State $k=0$
\State $\MJ=jacobian(\Vr,\Vbeta),\Vr=\Vr(\Vbeta)$
\While{$||\MJ\myt\Vr||>err $}
\State $\Vd=-(\MJ\myt\MJ)^{-1}\MJ\myt\Vr$
\State $\Vbeta=\Vbeta+\Vd$
\State $\MJ=jacobian(\Vr,\Vbeta),\Vr=\Vr(\Vbeta)$
\State $k=k+1$
\EndWhile
\State $\hat\Vbeta=\Vbeta$
\end{algorithmic}
\end{algorithm}

我们先来解释一下这个算法,给定$m$个关于n个变量$\Vbeta=(\beta_1,\beta_2,\cdots,\beta_n)\myt$的函数$\Vr=(r_1,\cdots,r_m)\myt$(也称为残差),$m\ge n$,Gauss-Newton算法找出了$\Vbeta$的值,使得平方和最小:
\begin{equation}
S(\Vbeta)=\frac{1}{2}\sum_{i=1}^mr_i^2(\Vbeta)=\frac{1}{2}\Vr(\Vbeta)\myt\Vr(\Vbeta)
\end{equation}
即求
\begin{equation}
\hat\Vbeta=\mathop{\arg\min}_{\Vbeta}S(\Vbeta)
\end{equation}
对$S(\Vbeta)$求导有:
\begin{equation}
\Vg=\nabla S(\Vbeta)=S'(\Vbeta)=\MJ(\Vbeta)\myt\Vr(\Vbeta)=0
\end{equation}
即驻点条件。

我们还有
\begin{equation}
\MH=S''(\Vbeta)=\MJ(\Vbeta)\myt\MJ(\Vbeta)+\sum_{i=1}^nr_i(\Vbeta)r''_i(\Vbeta)
\end{equation}

如果我们计算非线性方程$S'(\Vbeta)=\Vzero$的根,而已有近似值$\Vbeta^{(k)}$,采用牛顿法,我们将要计算
\begin{equation}
\Vbeta^{(k+1)}=\Vbeta^{(k)}-[\MH(\Vbeta^{(k)})]^{-1}\Vg(\Vbeta^{(k)})
\end{equation}

由于矩阵$\MJ(\Vbeta)\myt\MJ(\Vbeta)$至少是半正定的,如果忽略$S''(\Vbeta)$的后面一项,例如$\Vr(\Vbeta)$接近线性的情况,我们就得到了Gauss-Newton法。

这个迭代开始于近似值$\Vbeta^{(0)}$这个方法继续进行通过迭代
\begin{equation}
\Vbeta^{(s+1)}=\Vbeta^{(s)}-(\MJ\myt\MJ)^{-1}\MJ\myt\Vr(\Vbeta^{(s)})
\end{equation}
这里,如果$\Vr$和$\Vbeta$是列向量,Jacobian矩阵\footnote{Jacobian矩阵$\MJ$也记为$\frac{\partial \Vr}{\partial \Vbeta}$}的元素定义为
\begin{equation}
(\MJ)_{ij}=\frac{\partial r_i(\Vbeta^{(s)})}{\partial\Vbeta_j}
\end{equation}
如果$m=n$,这个迭代简化为
\begin{equation}
\Vbeta^{(s+1)}=\Vbeta^{(s)}-\MJ^{-1}\Vr(\Vbeta^{(s)})
\end{equation}
这是Newton法在一维情况的直接推广。

在数据拟合中,目标是找出在模型函数$y=f(x,\Vbeta)$的参数$\Vbeta$让它能最好拟合一些数据点$(x_i,y_i)$,函数$r_i$即为残差。
\begin{equation}
r_i(\Vbeta)=y_i-f(x_i,\Vbeta)
\end{equation}

值得注意的是,$(\MJ\myt\MJ)^{-1}\MJ\myt$是$\MJ$的伪逆。

下面讲述相对定向的过程。

首先,我们定义三个旋转矩阵
\begin{equation}
\MR_x(\theta)=\begin{bmatrix}
1 & 0 & 0 \\
0 & \cos\theta & -\sin\theta \\
0 & \sin\theta & \cos\theta
\end{bmatrix},\quad \MR_y(\theta)
\begin{bmatrix}
\cos\theta & 0 & -\sin\theta \\
0 & 1 & 0 \\
\sin\theta & 0 & \cos\theta
\end{bmatrix},\quad \MR_z(\theta)
\begin{bmatrix}
\cos\theta & -\sin\theta & 0 \\
\sin\theta & \cos\theta & 0 \\
0 & 0 & 1
\end{bmatrix}
\label{eq:rotationmatrix}
\end{equation}

由于单独相对相对定向采用以$y$为主轴的转角系统,左片的旋转矩阵和右片的旋转矩阵分别为
\begin{align}
\MR_1 &=\MR_y(\phi_1)\MR_z(\kappa_1) \\
\MR_2 &=\MR_y(\phi_2)\MR_x(\omega_2)\MR_z(\kappa_2)
\end{align}
设$p_1(x_1,y_1)$的同名点为$p_2(x_2,y_2)$,n个同名点横向堆叠形成的矩阵为$img_1,img_2$,将其写成像空间坐标系中的坐标
\begin{align}
img_1=[img_1;-f\cdot \ones(1,n)]\\
img_2=[img_2;-f\cdot \ones(1,n)]
\end{align}

将其变换到像空间辅助坐标系中
\begin{align}
img_1=\MR_1\cdot img_1\\
img_2=\MR_2\cdot img_2\label{eq:img2}
\end{align}

由于摄影基线$\Vb$与两幅构象光线$\Vl_1,\Vl_2$共面
\begin{equation}
\Vb\cdot(\Vl_1\times\Vl_2)=
\begin{vmatrix}
b & 0 & 0 \\
X_1 & Y_1 & Z_1 \\
X_2 & Y_2 & Z_2
\end{vmatrix}=b(Y_1Z_2-Y_2Z_1)=0
\end{equation}

我们采用matlab中的点乘记号$\cdot\times$表示对应元素相乘,点幂记号$\cdot '$表示实数转置$^T$\footnote{在matlab中$'$表示共轭转置$^H$}我们就可以得到残差向量
\begin{equation}
\Vr=(img_1(2,:)\cdot\times img_2(3,:)-img_2(2,:)\cdot\times img_1(3,:))\cdot '
\end{equation}

将五个转角参数$\Vbeta=(\phi_1,\kappa_1,\phi_2,\omega_2,\kappa_2)\myt$的初值置为$\Vzero$,置$err=\num{1e-7}$,即可用Gauss-Newton法迭代求解。

\subsubsection{前方交会建立模型}

运用上一小节中求得的独立像对相对定向参数以及\ref{sub:tongming}中量测的同名点坐标,根据前方交会计算式计算模型坐标,得到像片控制点、待定点的模型坐标。

具体过程如下
\begin{enumerate}
\item 根据内定向参数将\ref{sub:tongming}中量测的同名点的像素坐标转换为像点(框标坐标系)坐标$(x,y)$。
\item 确定摄影基线$b$值。摄影基线$b=l\times(1-p)\times M,B=(b,0,0)$,式中$l$为航片沿航向的像片边长,$p$为航向重叠度,$M$为摄影比例尺分母。

在本实验中,$l=46.08\times 2\si{mm},\; p=0.6,\; M=(7680\times 0.2)\si{m}/(46.08\times 2)\si{mm}$,其中,$0.2$是像元宽度。
\item 利用上一小节中求得的相对定向元素以及确定的$b$值,逐点计算点投影系数以及每对同名点的模型坐标,为检验相对定向的准确性和测量同名像点坐标的准确性,计算各点的上下视差$q$,$q$值的大小与正负可用于精度检查。
\end{enumerate}

以下是具体的计算式。

\begin{enumerate}
\item 与上小节类似地,进行由式\ref{eq:rotationmatrix}到\ref{eq:img2}的过程,将同名像点的坐标转换到像空间辅助坐标系中得$img_1,img_2$。
\item 对其中的每一对同名像点计算点投影系数
\begin{align}
N_1=& \frac{bZ_2}{X_1Z_2-X_2Z_1} \\
N_2=& \frac{bZ_1}{X_1Z_2-X_2Z_1}
\end{align}
\item 计算模型点坐标及上下视差
\begin{align}
x=& N_1X_1 \\
y=& 0.5(N_1Y_1+N_2Y_2) \\
z=& N_1Z_1 \\
q=& N_1Y_1-N_2Y_2
\end{align}
\end{enumerate}

\begin{note}
如果遇到上下视差特别大的情况,应该将该点剔除,重新相对定向。
\end{note}

\subsubsection{模型绝对定向编程}

经过相对定向之后,我们已经得到了以相对定向选定的像空间辅助坐标系为基准的模型,而我们需要的是地面测量坐标。绝对定向就是完成这两个坐标系之间的转化。

利用前方交会所求的像片控制点的模型坐标以及\ref{sub:waiye}中外业测量所测得的像控点地面坐标进行绝对定向,求出模型的7个绝对定向参数---三个平移参数$\Delta X,\Delta Y,\Delta Z$,一个缩放参数$\lambda$,三个旋转参数$\phi,\omega,\kappa$。再利用所求得的绝对定向参数将地物点的模型坐标转换到地面坐标系统。

\paragraph{绝对定向参数的求取}一个相对的两张相片共有十二个外方位元素,像对定向求得了五个元素以后,要恢复像对的绝对位置,还要解求七个绝对定向元素。它需要地面控制点(至少两个平高点和一个高程点)来解求,这种变换,在数学上为一个不同原点的三维空间相似变换,采用的数学模型是三维线性相似变换式:
\begin{equation}
\begin{bmatrix}
X \\ Y \\ Z
\end{bmatrix}=
\lambda\MR\begin{bmatrix}
x \\ y \\ z
\end{bmatrix}+
\begin{bmatrix}
dx \\
dy \\
dz
\end{bmatrix}
\label{eq:similar}
\end{equation}
其中,$(X,Y,Z)\myt$为地面点坐标,$(x,y,z)\myt$为模型点坐标,$\MR=R_y(\phi)R_x(\omega)R_z(\kappa)$。

因为这个模型又是关于绝对定向七参数的非线性模型,故又可以使用Gauss-Newton法来求解。

由于地面测量坐标系是左手系,像空间辅助坐标系是右手系,由解析几何的理论可知,平移旋转是第一类正交变换\footnote{变换矩阵$\MA$的行列式$\det(A)=1$},反射是第二类正交变换\footnote{变换矩阵$\MA$的行列式$\det(A)=-1$},这两类变换是无法相互转换的,即由上述七参数模型不可能直接将地面测量坐标和模型坐标对应起来,所以需要通过反射变换将地面测量坐标$(N,E,U)$\footnote{实验指导书中将地面测量坐标也记作$(X,Y,Z)$,本报告中统一记为$(N,E,U)$}变到地面摄测坐标$(X,Y,Z)$,即
\begin{equation}
\begin{bmatrix}
X \\  Y \\ Z
\end{bmatrix}=
\begin{bmatrix}
0 & 1 & 0 \\
1 & 0 & 0 \\
0 & 0 & 1
\end{bmatrix}
\begin{bmatrix}
N \\ E \\ U
\end{bmatrix}
\end{equation}

直观地理解就是将$N,E$互换位置。如表\ref{tab:coor}
\begin{table}[htbp]
\centering
\caption{坐标转换表(单位:$\si{m}$)}
\label{tab:coor}
\begin{tabular}{|c|c|c|c|c|c|}
\hline
\multicolumn{3}{|c|}{地面测量坐标} & \multicolumn{3}{c|}{地面摄测坐标} \\ \hline
      N  &   E    &   U   &   X    &   Y    &   Z   \\ \hline
   5299.483     &   3065.544    &   3.658   &  3065.544     &  5299.483     & 3.658   \\
\hline
\end{tabular}
\end{table}

解算步骤如下:
\begin{enumerate}
\item 将控制点的模型坐标和地面坐标\footnote{指地面摄测坐标}重心化。设控制点的模型坐标矩阵\footnote{坐标为行向量,按列堆叠,以后不再说明}为$\MM_{n\times 3}$(model),地面坐标矩阵为$\MG_{n\times 3}$(ground)。则重心化步骤如下
\begin{align}
\MM=& \MM-\ones(n,3)\cdot \diag(\mean(\MM)) \\
\MG=& \MG-\ones(n,3)\cdot \diag(\mean(\MG))
\end{align}
\item 将变换公式“拉直”\footnote{下述公式中的$\myvec$表示按列的方向对矩阵拉直}形成Gauss-Newton法的迭代格式
\begin{align*}
\MR=& \MR_y(\phi)\MR_x(\omega)\MR_z(\kappa) \\
\Vg=& \myvec(\MG\myt) \\
\Vr=& \lambda\MR\MM\myt+\diag([dx,dy,dz])\cdot\ones(3,n) \\
\Vr=& \myvec(\Vr)
\end{align*}
对于上述公式可以这样理解,右乘矩阵是对列进行操作,左乘矩阵是对行进行操作,将对矩阵的变换看做是对矩阵每一列的变换。
\item 设定迭代的初值:$\phi=0,\omega=0,\kappa=0,\lambda=1,dx=0,dy=0,dz=0$
\item 用高斯牛顿法迭代求解,并求出残差向量$\Vr$,将残差超限的点剔除。重新相对定向。
\end{enumerate}

\paragraph{模型坐标转换}

对相对定向建立的模型进行缩放、平移、旋转,将模型纳入到地面坐标系,利用上面求得的7个参数以及三维线性相似变换式\ref{eq:similar},将前方交会得到的模型坐标转换到地面坐标。

步骤:
\begin{enumerate}
\item 根据三个旋转参数计算旋转矩阵$\MR=\MR_y(\phi)\MR_x(\omega)\MR_z(\kappa)$,和一个缩放参数$\lambda$,三个平移参数$dx,dy,dz$组成三维线性相似变式。
\item 将模型点坐标带入三维线性相似变换式计算模型点的地面摄测坐标。
\end{enumerate}

\subsection{CAD成图练习}

根据地物点在左右影像上的像素位置(\ref{sub:tongming}量测得到),利用\ref{sub:biancheng}中求得的内定向、相对定向、绝对定向参数,以及内定向、前方交会、三维线性相似变换程序,确定地物点的地面位置。利用AutoCAD程序对要求成图范围内的地物进行成图。

步骤:
\begin{enumerate}
\item 将像对上量取地物特征点的同名像素坐标,通过内定向转换为像框坐标系的坐标。
\item 利用\ref{sub:biancheng}中求得的相对定向元素,进行前方交会计算,得到地物特征点的模型坐标。
\item 利用\ref{sub:biancheng}中求得的绝对定向元素,对求得的地物特征点模型坐标进行三维线性相似变换,得到地面特征点的地面坐标。
\item 根据地物特征点的地面坐标、以及\ref{sub:tongming}中量测时所绘的示意草图,利用CAD生成所要求区域的地形图。
\end{enumerate}

% (如图\ref{fig:dixing})
% \begin{figure}[htbp]
% \centering
% \includegraphics[width=\textwidth]{tu.PNG}
% \caption{CAD地形图示例}
% \label{fig:dixing}
% \end{figure}

% \subsection{航片的数字微分纠正}

% 利用已有的遥感软件(如ENVI等),根据\ref{sub:waiye}中量测的像控点坐标进行同济校园影像的数字微分纠正。

% 步骤:
% \begin{enumerate}
% \item 纠正函数:多项式;
% \item 在待纠正影像和已纠正影像上寻找易于测量的同名像点,并量测出其像点坐标。
% \item 利用软件对所量测出的点进行平差计算,去掉Q值大的点,最后确定21个点作为纠正用的控制点。
% \item 利用软件进行航片的数字微分纠正。
% \end{enumerate}

\section{Kinect软件使用}

课程安排中,有一节课是让我们练习使用Kinect。

Kinect是微软在2010年6月14日对XBOX360体感周边外设正式发布的名字。它是一种3D体感摄影机(开发代号“Project Natal”),同时它导入了即时动态捕捉、影像辨识、麦克风输入、语音辨识、社群互动等功能。玩家可以通过这项技术在游戏中开车、与其他玩家互动、通过互联网与其他Xbox玩家分享图片和信息等。

在本次实习中,我们用Kinect进行了三维建模。首先,将Kinect开启,然后,一个人手持Kinect围绕着被摄景物缓慢走动,最后,将生成的文件保存,并用专门的软件打开并查看效果。

\begin{figure}[htbp]
\centering
\caption{Kinect三维建模}
\includegraphics[width=0.8\textwidth]{3d.PNG}
\label{fig:3d}
\end{figure}

如图\ref{fig:3d}所示,这是我们组三维建模的成果,可以看出,Kinect三维建模的效果还是算不错的。

之后我们将模型的长度和实际的长度进行了对比。先将电脑的宽度用卷尺量出,然后和电脑上点出的长度进行了对比,我记得大概30厘米误差在$3\sim 4\si{cm}$左右,说明其精度并不算高,这也可能与其用途为游戏有关。


\part{小组实习成果}
\chapter{外业成果}

\section{像控点坐标文件}
% Table generated by Excel2LaTeX from sheet 'Sheet1'
\begin{table}[htbp]
    \centering
    \caption{小组像控点坐标(单位:$\si{px},\si{m}$)}
      \begin{tabular}{|l|c|c|c|c|c|c|c|}
      \hline
      ID    & \multicolumn{2}{c|}{63} & \multicolumn{2}{c|}{64} & \multicolumn{3}{c|}{地面坐标}\\
      \hline
            & X     & Y     & X     & Y     & X     & Y     & Z \\
      \hline
      34    & 5297  & 3337  & 2146  & 3075  & 5660.56 & 2771.877 & 6.221 \\
      \hline
      35    & 5707  & 2345  & 2683  & 2057  & 5876.63 & 2889.648 & 16.589 \\
      \hline
      36    & 5094  & 3158  & 1945  & 2793  & 5701.465 & 2719.948 & 4.231 \\
      \hline
      37    & 5338  & 3330  & 2235  & 2968  & 5661.921 & 2791.356 & 9.81 \\
      \hline
      38    & 5596  & 2975  & 2442  & 2612  & 5684.058 & 2792.756 & 7.135 \\
      \hline
      \end{tabular}%
    \label{tab:imgCtrlPnt}%
  \end{table}%

需要说明的是,左片、右片的坐标以像素表示,$X$为列数,$Y$为行数,地面坐标是指在同济独立坐标系下的地面测量坐标(左手系)$\{X,Y,Z\}=\{N,E,U\}$。


\section{测量坐标原始文件}

\begin{table}[htbp]
  \centering
  \caption{具体测量坐标}
    \begin{tabular}{|c|p{6em}|r|r|r|}
\hline    \multicolumn{1}{|r|}{成员} & 测量坐标(\si{m}) & \multicolumn{1}{p{5em}|}{N} & \multicolumn{1}{p{5em}|}{E} & \multicolumn{1}{p{5em}|}{U}\\ \hline
    \multicolumn{1}{|c|}{\multirow{3}[5]{*}{陈晨}} & 第一次 & 5517.385 & 2847.557 & 3.801 \\
\cline{2-5}       & 第二次 & 5517.38 & 2847.577 & 3.821 \\
\cline{2-5}       & 平均值 & 5517.383 & 2847.567 & 3.811 \\
    \hline
    \multicolumn{1}{|c|}{\multirow{3}[6]{*}{贾锐}} & 第一次 & 5684.06 & 2792.756 & 7.105 \\
\cline{2-5}       & 第二次 & 5684.056 & 2792.756 & 7.165 \\
\cline{2-5}       & 平均值 & 5684.058 & 2792.756 & 7.135 \\
    \hline
    \multicolumn{1}{|c|}{\multirow{3}[6]{*}{余周炜}} & 第一次 & 5535.474 & 2861.033 & 13.14 \\
\cline{2-5}       & 第二次 & 5535.474 & 2861.035 & 13.13 \\
\cline{2-5}       & 平均值 & 5535.474 & 2861.034 & 13.14 \\
    \hline
    \multicolumn{1}{|c|}{\multirow{3}[6]{*}{王雪辰}} & 第一次 & 5660.54 & 2771.876 & 6.217 \\
\cline{2-5}       & 第二次 & 5660.58 & 2771.879 & 6.225 \\
\cline{2-5}       & 平均值 & 5660.56 & 2771.877 & 6.221 \\
    \hline
    \multicolumn{1}{|c|}{\multirow{3}[6]{*}{毛瑞丰}} & 第一次 & 5661.92 & 2791.358 & 9.72 \\
\cline{2-5}       & 第二次 & 5661.923 & 2791.354 & 9.9 \\
\cline{2-5}       & 平均值 & 5661.921 & 2791.356 & 9.81 \\
    \hline
    \end{tabular}%
  \label{tab:2}%
\end{table}%

\section{测量方法图示}

本组成员选择的点一般为房屋角点,故多采用无棱镜的方式测量,测得一个点之后,一般用支导线的方式换一站,再测一次进行检查,得到两次量测结果$(X_1,Y_1,Z_1),(X_2,Y_2,Z_2)$,如果两次量测的结果在限差之内,取两次测量的平均值作为最终的结果,否则,进行返测。

大致流程为:架站$\rightarrow$后视定向$\rightarrow$第一次测量$\rightarrow$搬站$\rightarrow$第二次测量(检查)。

具体如下:
\begin{enumerate}
\item \textbf{陈晨的点}:该点位于同济大学大礼堂东南侧的的开水房的东北角点上。此点为房屋角点,故采用无棱镜的方式测量,第一次从控制点235进行观测,后视点为控制点234,后视坐标差见表\ref{tab:1},满足要求。由于这个点的周围控制点较多,所以没有利用支导线进行架站,而是再次利用控制点进行测量。第二次测量选取控制点234作为测站,控制点234-1作为后视定向点,后视坐标差见表\ref{tab:1},满足要求。同时由于测区其他地区缺少控制点,我在西北二寝室楼西侧布设一个点TP1,作为其他组员的控制点。两次测定坐标的坐标见表\ref{tab:2},坐标差满足精度要求。

\begin{figure}[htbp]
\begin{minipage}[c]{0.5\textwidth}
\centering
\includegraphics[width=6cm]{survey1.png}
\end{minipage}%
\begin{minipage}[c]{0.5\textwidth}
\centering
\includegraphics[width=6cm]{survey2.png}
\end{minipage}
\caption{陈晨测量方法图示}
\end{figure}

\item\textbf{贾锐的点}:海洋地质教育部重点实验室西南房角点。此点为房屋角点,故采用无棱镜的方式测量。第一次从控制点234进行观测,后视点为234-1,后视坐标差见表\ref{tab:1},满足要求,由于这个点的周围控制点较多,所以我们没有利用支导线进行架站,而是再次利用控制点进行测量,第二次我们选取234-1作为测站,235作为后视定向点,后视坐标差见表\ref{tab:1},满足要求,两次测定坐标的坐标见表\ref{tab:2},坐标差满足要求。

\begin{figure}[htbp]
\begin{minipage}[c]{0.5\textwidth}
\centering
\includegraphics[width=6cm]{survey3.png}
\end{minipage}%
\begin{minipage}[c]{0.5\textwidth}
\centering
\includegraphics[width=6cm]{survey4.png}
\end{minipage}
\caption{贾锐测量方法图示}
\end{figure}

\item\textbf{余周炜的点}:西北二楼楼顶东侧南角点。此点为房屋角点,故采用无棱镜的方式测量。第一次测量从控制点230进行观测,后视点为TP1,后视坐标差见表\ref{tab:1},满足要求。同时,布设一个支导线点。第二次测量利用支导线进行架站,选取234作为测站,支导线点作为后视定向点,后视坐标差见表\ref{tab:1},满足要求。两次量测的结果见表\ref{tab:2},坐标差满足精度要求。
\begin{figure}[htbp]
\centering
\includegraphics[width=12cm]{survey5.png}
\caption{余周炜测量方法图示}
\end{figure}

\item\textbf{王雪辰的点}:大礼堂南侧从东向西数地4根弯梁底端。由于此处的控制点稀疏,故先把控制点234作为后视点,控制点235作为测站,布设两个支导线点T1,T2。再在T1,T2分两次对像片控制点进行量测,后视点选择控制点235,后视坐标差见表\ref{tab:1},满足要求,两次量测的结果见表\ref{tab:2},坐标差满足精度要求。
\begin{figure}[htbp]
  \centering
  \includegraphics[width=12cm]{survey6.png}
  \caption{王雪辰测量方法图示}
  \end{figure}


\item\textbf{毛瑞丰的点}:大澡堂西面,开水房东面之间的房屋,房脊交线处,位于该房屋的东北边。此点为房屋角点,故采用无棱镜的方式测量。第一次测量从控制点235进行,后视点为234,后视坐标差见表\ref{tab:1},满足要求。第二次测量选取控制点234作为测站,控制点234-1作为后视定向点,后视坐标差见表\ref{tab:1},满足要求。两次量测的结果见表\ref{tab:2},坐标差满足精度要求。

\begin{figure}[htbp]
  \begin{minipage}[c]{0.5\textwidth}
  \centering
  \includegraphics[width=6cm]{survey7.png}
  \end{minipage}%
  \begin{minipage}[c]{0.5\textwidth}
  \centering
  \includegraphics[width=6cm]{survey8.png}
  \end{minipage}
  \caption{毛瑞丰测量方法图示}
  \end{figure}

\end{enumerate}


\begin{table}[htbp]
\centering
\caption{后视坐标差}
\begin{tabular}{|c|r|r|r|}
\hline
\multicolumn{1}{|c|}{后视坐标差(\si{mm})} & \multicolumn{1}{p{2cm}|}{$dx$} & \multicolumn{1}{p{2cm}|}{$dy$} & \multicolumn{1}{p{2cm}|}{$dz$} \\
\hline
\multicolumn{1}{|c|}{\multirow{2}[4]{*}{陈晨}} & 9  & 3  & 1 \\
\cline{2-4}       & 5  & 1  & 20 \\
\hline
\multicolumn{1}{|c|}{\multirow{2}[4]{*}{贾锐}} & 1  & 4  & 0 \\
\cline{2-4}       & 5  & 1  & 20 \\
\hline
\multicolumn{1}{|c|}{\multirow{2}[4]{*}{余周炜}} & 7  & 7  & 10\\
\cline{2-4}       & 9  & 1  & 10 \\
\hline
\multicolumn{1}{|c|}{\multirow{2}[4]{*}{王雪辰}} & 9  & 4  & 0 \\
\cline{2-4}       & 1  & 10 & 10 \\
\hline
\multicolumn{1}{|c|}{\multirow{2}[4]{*}{毛瑞丰}} & 2  & 4  & 10\\
\cline{2-4}       & 3  & 2  & 0 \\
\hline
\end{tabular}%
\label{tab:1}%
\end{table}%

% Table generated by Excel2LaTeX from sheet 'Sheet2'
% Table generated by Excel2LaTeX from sheet 'Sheet2'




本组成员在测量过程中,每人独立量测一个点,增强了独立解决问题的能力。

\section{点之记}

\begin{center} 
\begin{longtable}{cp{5cm}p{5cm}}
\caption{小组像控点说明} \\
\hline
\multicolumn{1}{l}{ID} & 点之记位置说明 & 点之记 \\
\hline
\endhead
\hline
\endfoot
34 & 该点位于同济大学大礼堂东南侧的的开水房的东北角点上 & \mgape{\includegraphics[width=5cm]{image1.png}} \\ 
35 & 西北二楼楼顶东侧南角点 & \mgape{\includegraphics[width=5cm]{image2.png}} \\
36 & 大礼堂南侧从东向西数第4根弯梁底端 & \mgape{\includegraphics[width=5cm]{image3.png}} \\
37 & 大澡堂西面,开水房东面之间的房屋,房脊交线处,位于该房屋的东北边 & \mgape{\includegraphics[width=5cm]{image4.png}} \\
38 & 海洋地质教育部重点实验室西南房角点 & \mgape{\includegraphics[width=5cm]{image5.png}} \\
\end{longtable}%
\end{center}%

  其说明见Excel:点之记7.xlsx
  
\section{校园调绘成果}

我们组认真地完成了校园的调绘,调绘图文件是调绘(1)文件夹的调绘.jpg文件。

根据要求,调绘的范围为整个同济校园的所有建筑、道路、土地用途等,调绘的项目在前一部分已经说明,这里不再列举。

\textbf{完整调绘图见附录,该图选自陈晨的成果,每个人的调绘图见个人报告的截图。}

下面是调绘的详细记录表格。
% Table generated by Excel2LaTeX from sheet 'Sheet1'
\begin{center}

\begin{longtable}{|l|l|r|l|}

\hline 编号 & 名称 & \multicolumn{1}{l|}{层数} & 用途 \\ \hline
\endhead
\hline
\endfoot

C01 & 云通楼(人文学院) & 5  & 教学 办公 \\
C02 & 土木工程学院 & 8  & 教学 办公 \\
C03 & 中德楼(中德学院) & 11 & 教学 办公 \\
C04 & 干训南楼 & 6  & 商用  \\
C05 & 半亩园餐厅 & 2  & 商用  \\
C06 & 干训北楼 & 6  & 商用  \\
C07 & 西南七楼 & 6  & 居住 \\
C08 & 西南八楼 & 6  & 居住 \\
C09 & 西南九楼 & 5  & 居住 \\
C10 & 西南一楼 & 3  & 居住 \\
C11 & 西南三楼 & 6  & 居住 \\
C12 & 西南二楼 & 6  & 居住 \\
C13 & 教育超市 & 2  & 商用  \\
C14 & 信息馆 & 2  & 办公 \\
C15 & 教学实践机房 & 3  & 教学 办公 \\
C16 & 物理馆(物理科学与工程学院) & 8  & 教学 办公 \\
C17 & 浴室 & 2  & 商用  \\
C18 & 经纬楼 & 3  & 办公 \\
C19 & 实验动物中心 & 3  & 教学 办公 \\
C20 & 医学与生命科学实验教学中心 & 3  & 教学 办公 \\
C21 & 宁静楼(数学科学学院) & 3  & 教学 办公 \\
C22 & 保卫处 & $2+3+4$ & 办公 \\
C23 & 机械馆 & 4  & 教学 办公 \\
C24 & 西苑饮食广场 & 3  & 商用  \\
C25 & 健身中心 & 1  & 商用  \\
C26 & 大学生购物中心 & 2  & 商用  \\
    &    &    &  \\
D01 & 大礼堂 & 2  & 商用  \\
D02 & 教育超市 & 2  & 商用  \\
D03 & 西北一楼 & 4  & 居住 \\
D04 & 西北二楼 & 4  & 居住 \\
D05 & 北苑饮食广场 & 2  & 商用  \\
D06 & 西北五楼 & 6  & 居住 \\
D07 & 西北四楼 & 6  & 居住 \\
D08 & 西北三楼 & 6  & 居住 \\
D09 & 岩土楼 & 8  & 教学 办公 \\
D10 & 岩土工程实验中心 & 1  & 教学 办公 \\
D11 & 土木工程试验实践基地 & 3  & 教学 办公 \\
D12 & 震动台 & 0  & 教学 办公 \\
D13 & 结构工程研究所 & 3  & 教学 办公 \\
D14 & 结构试验馆 & 1  & 教学 办公 \\
D15 & 桥梁馆 & 7  & 教学 办公 \\
D16 & 风工程馆 & 3  & 教学 办公 \\
D17 & 风工程实验室 & 2  & 教学 办公 \\
D18 & 致远楼(数学科学学院) & 3  & 教学 办公 \\
D19 & 生态楼 & 5  & 教学 办公 \\
D20 & 城市污染检制国家工程研究中心 & 6  & 教学 办公 \\
D21 & 明净楼(环境科学与工程学院) & 5  & 教学 办公 \\
D22 & 济阳楼 & 3  & 教学 办公 \\
D23 & 博思楼3号楼 & 15 & 居住 \\
D24 & 博思楼4号楼 & 15 & 居住 \\
D25 & 博思楼5号楼 & 15 & 居住 \\
    &    &    &  \\
A01 & 逸夫楼 & 2  & 办公 \\
A02 & 行政南楼 & 5  & 办公 \\
A03 & 行政北楼 & 5  & 办公 \\
A04 & 同创楼(校史馆) & 3  & 教学 \\
A05 & 衷和楼 & 21 & 教学 办公 \\
A06 & 综合服务大厅 & 4  & 办公 \\
A07 & 同文楼(外国语学院) & 3  & 教学 办公 \\
A08 & 汇文楼(外国语学院) & 5  & 教学 办公 \\
A09 & 文远楼(建筑与城市规划学院A楼) & 3  & 教学 办公 \\
A10 & 明成楼(建筑与城市规划学院B楼) & 4  & 教学 办公 \\
A11 & 建筑与城市规划学院C楼 & 8  & 教学 办公 \\
A12 & 建筑与城市规划学院D楼 & 5  & 教学 办公 \\
A13 & 北楼 & 4  & 教学 \\
A14 & 图书馆 & 11 & 教学 \\
A15 & 南楼 & 4  & 教学 \\
A16 & 工程试验馆 & 3  & 教学 办公 \\
A17 & 电子与信息学院实验楼 & 4  & 教学 办公 \\
A18 & 化学馆(化学科学与工程学院) & $6+3$ & 教学 办公 \\
A19 & 化工楼 & 3  & 教学 办公 \\
A20 & 声学馆 & 4  & 教学 办公 \\
A21 & 瑞安楼 & 8  & 教学 办公 \\
A22 & 海洋楼(海洋与地球科学学院) & $6+3$ & 教学 办公 \\
A23 & 运筹楼 & 5  & 教学 办公 \\
A24 & 学三楼 & 6  & 居住 \\
A25 & 学四楼 & 6  & 居住 \\
A26 & 学五楼 & 6  & 居住 \\
A27 & 留学生楼 & 12 & 居住 \\
A28 & 友谊大楼 & 12 & 办公 \\
A29 & 三好坞饮食广场 & 2  & 商用  \\
    &    &    &  \\
B01 & 仁心楼(附属同济医院分院) & 4  & 商用  \\
B02 & 游泳馆 & 1  & 商用 教学 \\
B03 & 医学院 生命科学与技术学院 & $12+5$ & 教学 办公 \\
B04 & 解放楼 & 2  & 教学 办公 \\
B05 & 青年楼 & 2  & 教学 办公 \\
B06 & 测绘馆(测绘与地理信息学院) & 2  & 教学 办公 \\
B07 & 学苑饮食广场 & 3  & 商用  \\
B08 & 乒乓球馆 & 1  & 商用 教学 \\
B09 & 攀岩馆 & 1  & 商用 教学 \\
B10 & 羽毛球馆 & 2  & 商用 教学 \\
B11 & 一二·九大楼 & 1  & 办公 \\
B12 & 博物馆 & 3  & 教学 \\
B13 & 一·二九礼堂 & 1  & 商用  \\
B14 & 体育馆 & 4  & 商用 教学 \\
B16 & 中法中心 & 5  & 教学 办公 \\
B17 & 旭日楼 & 3  & 教学 办公 \\
\end{longtable}%
\end{center}%



\chapter{内业成果}
\section{相对定向点}

见Excel表格:小组同名点.xlsx 和每个组员的个人报告。

每个人的量测的点如下:
\begin{table}[htbp]
  \centering
  \begin{tabular}{p{5cm}c}
    \toprule
    姓名 & 点号 \\
    \midrule
    余周炜 & 1$\sim$ 7 \\
    王雪辰 & 8$\sim$ 14 \\
    毛瑞丰 & 15$\sim$19 \\
    陈晨 & 20$\sim$ 25 \\
    贾锐 & 26$\sim$ 31 \\
    \bottomrule 
  \end{tabular}
\end{table}

\section{待定地物点}

待定地物点主要分为两部分一部分为大礼堂,另一部分为西北一。分别记在大礼堂.xlsx和西北一.xlsx中。

\subsubsection{大礼堂同名点}

\begin{center}
   \tablehead{ \hline 同名点 & \multicolumn{2}{c|}{63} & \multicolumn{2}{c|}{64} \\ \hline
   & x  & y  & x  & y \\ \hline}
   \tabletail{\hline}
      \begin{supertabular}{|l|cc|cc|}
      1  & 4933 & 2849 & 1774 & 2469 \\
      2  & 5024 & 2894 & 1866 & 2521 \\
      3  & 5031 & 2883 & 1884 & 2505 \\
      4  & 5035 & 2885 & 1890 & 2508 \\
      5  & 5029 & 2897 & 1880 & 2520 \\
      6  & 5062 & 2912 & 1904 & 2539 \\
      7  & 5069 & 2902 & 1924 & 2525 \\
      8  & 5073 & 2903 & 1927 & 2526 \\
      9  & 5067 & 2913 & 1913 & 2541 \\
      10 & 5093 & 2927 & 1939 & 2554 \\
      11 & 5099 & 2915 & 1953 & 2542 \\
      12 & 5103 & 2917 & 1957 & 2543 \\
      13 & 5096 & 2927 & 1946 & 2556 \\
      14 & 5123 & 2940 & 1971 & 2569 \\
      15 & 5129 & 2930 & 1984 & 2556 \\
      16 & 5132 & 2932 & 1987 & 2558 \\
      17 & 5128 & 2941 & 1977 & 2572 \\
      18 & 5154 & 2955 & 2002 & 2584 \\
      19 & 5159 & 2944 & 2014 & 2572 \\
      20 & 5163 & 2946 & 2017 & 2573 \\
      21 & 5158 & 2956 & 2006 & 2586 \\
      22 & 5183 & 2569 & 2008 & 2586 \\
      23 & 5183 & 2969 & 2033 & 2599 \\
      24 & 5190 & 2960 & 2046 & 2586 \\
      25 & 5193 & 2961 & 2049 & 2589 \\
      26 & 5188 & 2971 & 2038 & 2602 \\
      27 & 5124 & 2584 & 2063 & 2615 \\
      28 & 5221 & 2973 & 2076 & 2602 \\
      29 & 5223 & 2975 & 2078 & 2605 \\
      30 & 5218 & 2985 & 2070 & 2617 \\
      31 & 5244 & 3000 & 2094 & 2631 \\
      32 & 5250 & 2988 & 2105 & 2618 \\
      33 & 5254 & 2990 & 2110 & 2619 \\
      34 & 5248 & 3001 & 2100 & 2632 \\
      35 & 5273 & 3012 & 2124 & 2645 \\
      36 & 5280 & 3002 & 2134 & 2633 \\
      37 & 5284 & 3004 & 2139 & 2635 \\
      38 & 5261 & 3051 & 2108 & 2680 \\
      39 & 5300 & 3070 & 2145 & 2700 \\
      40 & 5254 & 3120 & 2101 & 2793 \\
      41 & 5215 & 3142 & 2062 & 2774 \\
      42 & 5187 & 3205 & 2040 & 2838 \\
      43 & 5183 & 3202 & 2035 & 2836 \\
      44 & 5194 & 3176 & 2042 & 2807 \\
      45 & 5169 & 3163 & 2018 & 2796 \\
      46 & 5158 & 3191 & 2010 & 2824 \\
      47 & 5154 & 3186 & 2005 & 2822 \\
      48 & 5163 & 3162 & 2011 & 2792 \\
      49 & 5139 & 3150 & 1987 & 2783 \\
      50 & 5127 & 3175 & 1979 & 2809 \\
      51 & 5124 & 3171 & 1974 & 2806 \\
      52 & 5134 & 3148 & 1981 & 2779 \\
      53 & 5170 & 3138 & 1955 & 2768 \\
      54 & 5097 & 3162 & 1949 & 2794 \\
      55 & 5092 & 3161 & 1943 & 2793 \\
      56 & 5103 & 3135 & 1949 & 2763 \\
      57 & 5077 & 3121 & 1925 & 2751 \\
      58 & 5065 & 3149 & 1917 & 2780 \\
      59 & 5062 & 3146 & 1914 & 2779 \\
      60 & 5073 & 3119 & 1921 & 2747 \\
      61 & 5047 & 3107 & 1894 & 2736 \\
      62 & 5036 & 3132 & 1886 & 2762 \\
      63 & 5032 & 3130 & 1883 & 2759 \\
      64 & 5043 & 3104 & 1890 & 2733 \\
      65 & 5016 & 3092 & 1862 & 2720 \\
      66 & 5003 & 3122 & 1855 & 2751 \\
      67 & 5001 & 3121 & 1852 & 2748 \\
      68 & 5013 & 3090 & 1858 & 2717 \\
      69 & 4986 & 3078 & 1832 & 2703 \\
      70 & 4974 & 3107 & 1824 & 2735 \\
      71 & 4969 & 3106 & 1821 & 2734 \\
      72 & 4982 & 3075 & 1826 & 2701 \\
      73 & 4942 & 3041 & 1781 & 2667 \\
      74 & 4858 & 3004 & 1697 & 2626 \\
      \end{supertabular}%
  \end{center}%
  
\subsubsection{西北一同名点}

% Table generated by Excel2LaTeX from sheet 'Sheet1'
\begin{table}[htbp]
    \centering
      \begin{tabular}{|l|cc|cc|}
        \hline
      同名点 & \multicolumn{2}{c|}{63} & \multicolumn{2}{c|}{64} \\
      \hline
         & x  & y  & x  & y \\ \hline
      1  & 4915 & 2645 & 1753 & 2260 \\
      2  & 5173 & 2618 & 2014 & 2236 \\
      3  & 5181 & 2681 & 2020 & 2301 \\
      4  & 5165 & 2682 & 2008 & 2302 \\
      5  & 5169 & 2722 & 2011 & 2345 \\
      6  & 5351 & 2703 & 2195 & 2327 \\
      7  & 5359 & 2767 & 2201 & 2393 \\
      8  & 5343 & 2769 & 2188 & 2394 \\
      9  & 5348 & 2810 & 2191 & 2437 \\
      10 & 5532 & 2789 & 2376 & 2421 \\
      11 & 5538 & 2853 & 2382 & 2483 \\
      12 & 5290 & 2879 & 2129 & 2508 \\
      13 & 5282 & 2777 & 2124 & 2402 \\
      14 & 5111 & 2793 & 1951 & 2417 \\
      15 & 5104 & 2690 & 1945 & 2308 \\
      16 & 4922 & 2707 & 1758 & 2325 \\
      \hline
      \end{tabular}%
  \end{table}%
  

\section{内定向}

以本组其中一个内定向程序neidingxiang.m(余周炜)为例,阐明内定向所使用的文件。

内定向使用了三个文件:camera.use和小组同名点.xlsx,点之记改.xlsx。

camera.use提供了像片四个角点的框标坐标,打开影片的元数据可以得知其四个角点的像素坐标,这样就可以用最小二乘法求出内定向的六个参数,为
\begin{align}
h_0&=-46.08 &h_1&=0.012 &h_2&=0 \\
k_0&=82.944 &k_1&=0 &k_2&=-0.012
\end{align}

点之记改.xlsx是点之记.xlsx文件去掉没有在航片63,64上出现的点后剩下的点,是全班的像点,需要进行内定向以便后续处理。小组同名点.xlsx是小组的像点,同样需要进行内定向。内定向的结果记在全班框标点.xlsx和框标点.xlsx中。

整理如下:
\begin{table}
\begin{tabular}{p{10em}p{8cm}}
\toprule
\textbf{Program} & neidingxiang.m \\
\textbf{Require} & camera.use \\
\textbf{Input} & 小组同名点.xlsx/点之记改.xlsx \\
\textbf{Output} & 框标点.xlsx/全班框标点.xlsx \\
\bottomrule
\end{tabular}
\end{table}
\section{相对定向}

以本组其中一个相对定向程序xiangdui.m(余周炜)为例,阐明相对定向所使用的文件。

相对定向包含两个程序xiangdui.m(脚本文件)和GaussNewton.m(函数文件)。其中xiangdui.m文件包含文件的读写,和有关数据的简单处理,然后将处理出来的数据带入GaussNewton.m中迭代计算。

相对定向用框标点.xlsx求取参数,结果写入xiangdui.csv中。
\begin{table}[htbp]
\begin{tabular}{p{10em}p{8cm}}
\toprule
\textbf{Program} & xiangdui.m,和GaussNewton.m \\
\textbf{Input} & 框标点.xlsx \\
\textbf{Output} & xiangdui.csv \\
\bottomrule
\end{tabular}
\end{table}

得出的结果如下:
\begin{equation}
\begin{array}{lllll}
\phi_1=-0.01245 & \kappa_1=0.02841 & \phi_2=0.01334 & \omega_2=-0.02251 & \kappa_2=0.036366
\end{array}
\end{equation}

\section{前方交会}

以本组其中一个前方交会程序qianfang.m(余周炜)为例,阐述前方交会所使用的文件。

前方交会是将框标点转换到模型坐标的程序,将其列举如下

\begin{table}[htbp]
\begin{tabular}{p{10em}p{8cm}}
\toprule
\textbf{Program} & qianfang.m \\
\textbf{Require} & xiangdui.csv \\
\textbf{Input} & 全班框标点.xlsx \\
\textbf{Output} & 点之记改.xlsx和inspect.xlsx \\
\bottomrule
\end{tabular}
\end{table}

其中inspect.xlsx的作用是检查上下视差,剔除视差较大的点,以便绝对定向的进行。

\section{绝对定向程序}

以本组其中一个绝对定向程序juedui.m(余周炜)为例,阐述绝对定向所使用的文件。

绝对定向包含两个程序juedui.m(脚本文件)和GaussNewton2.m(函数文件)。其中juedui.m文件包含文件的读写,和有关数据的简单处理,然后将处理出来的数据带入GaussNewton2.m中迭代计算\footnote{这里的$\phi,\omega,\kappa$与任务书上有所不同!依然是以$y$为主轴的变换,即$\MR=\MR_y(\phi)\MR_x(\omega),\MR_z(\kappa)$}。

\begin{table}[htbp]
\begin{tabular}{p{10em}p{8cm}}
\toprule
\textbf{Program} & juedui.m和GaussNewton2.m \\
\textbf{Input} & inspect.xlsx \\
\textbf{Output} & juedui.csv \\
\bottomrule
\end{tabular}
\end{table}

得出的结果如下:
\begin{equation}
\begin{array}{lll}
\phi=0.01126 & \omega=0.00564 & \kappa=0.000538 \\
dx=0 & dy=0 & dz=0 \\
\lambda=1.1318 & &
\end{array}
\end{equation}

\section{绝对定向坐标转换}


\subsection{从模型点转到地面点}

以本组其中一个绝对定向坐标转换程序jueduitransform.m(余周炜)为例,阐述阐述绝对定向坐标转换所使用的文件。

绝对定向坐标转换是将模型点转到地面点的程序,说明如下:

\begin{table}[htbp]
  \begin{tabular}{p{10em}p{8cm}}
  \toprule
  \textbf{Program} & juiduitransform.m \\
  \textbf{Require} & juedui.csv \\
  \textbf{Input} & inspect.xlsx \\
  \textbf{Output} & dimianzuobiao.xlsx \\
  \bottomrule
  \end{tabular}
  \end{table}

\subsection{从像素坐标转到地面点}

为了更加方便地转换坐标,特别编写了一个程序transform.m(余周炜),可以直接将像素坐标转换到地面摄测坐标,即将内定向,前方交会,绝对定向坐标转换三合一。

先将该程序说明如下:
\begin{table}[htbp]
\begin{tabular}{p{10em}p{8cm}}
\toprule
\textbf{Program} & transform.m \\
\textbf{Require} & neidingxiang.csv, xiangdui.csv, juedui.csv \\
\textbf{Input} & 点之记 - 副本.xls \\
\textbf{Output} & dimianzuobiao.xlsx \\
\bottomrule
\end{tabular}
\end{table}

\section{校园局部CAD图}

我们小组的负责区域是大礼堂及大礼堂以北的区域,我们选取了其中的两栋标志性建筑物:大礼堂和西北一楼成图。

dwg的文件名为:第七组展绘(1).dwg。
\begin{figure}[htbp]
\centering
\caption{校园局部CAD图(截图)}
\includegraphics[width=0.8\textwidth]{cad.PNG}
\end{figure}

\textbf{完整CAD图见附录}。

\section{像控点的质量评价}

% Table generated by Excel2LaTeX from sheet 'Sheet1'
我们组的王雪辰同学对像控点的质量进行了评价,贾锐对这些结果进行了整理。结果保存在点之记评价.xls中,这里也抄录了一份评价,表中,绿色的点表示选用的点,黄色的点上下视差过大,除去视差较大的点,其余的点舍去的原因是残差较大,即$\varepsilon_X>0.5$或$\varepsilon_Y>0.5$或$\varepsilon_Z>0.5$。在进行相对定向或者绝对定向的时候需慎用这些点。



\subsubsection{点质量检查过程}
\begin{enumerate}
  \item 先用全部45个点进行相对定向,前方交会,求出视差。
  \item 将视差大的点进行剔除:6、10、22、31、34。
  \item 利用剩余的点进行绝对定向,将利用前方交会、绝对定向过程求得的点的地面测量坐标与实地测得的地面控制坐标进行比较,看其残差。
  \item 将所有$\varepsilon_X$、$\varepsilon_Y$、$\varepsilon_Y$大与0.5的点剔除。即8、9、11、13、16、21、23、24、25、26、28、29、30、32、37、38、39、40、41、44、45、46点剔除。
\end{enumerate}

\subsubsection{剔除点的原因}
\begin{enumerate}
  \item 视差过大:同名点没有量测好
  \item XYZ方向残差(绝对定向结果与实测结果的差)过大:可能是实际测量出现了问题。
\end{enumerate}
\subsubsection{质量检查表}
% \begin{center}
%  \begin{longtable}{|p{0.3cm}|p{0.5cm}p{0.5cm}|p{0.5cm}p{0.5cm}|ccc|ccc|}
%   \hline
%   ID & \multicolumn{2}{c|}{63} & \multicolumn{2}{c|}{64} & \multicolumn{3}{c|}{地面坐标} & \multicolumn{3}{c|}{地面摄测坐标} \\ \hline
%       & X  & Y  & X  & Y  & N  & E  & U  & X  & Y  & Z \\ \hline
%   \endhead
%   \hline
%   \endfoot
 % Table generated by Excel2LaTeX from sheet 'Sheet1'
 {\scriptsize
\begin{center}
    \begin{longtable}{rrrrrrrrrrr}
      \hline
    \multicolumn{1}{c}{$X_1$} & \multicolumn{1}{c}{$Y_1$} & \multicolumn{1}{c}{$Z_1$} & \multicolumn{1}{c}{$X$} & \multicolumn{1}{c}{$Y$} & \multicolumn{1}{c}{$Z$} & \multicolumn{1}{c}{$\varepsilon_X$} & \multicolumn{1}{c}{$\varepsilon_Y$} & \multicolumn{1}{c}{$\varepsilon_Z$} & \multicolumn{1}{c}{取舍} & \multicolumn{1}{c}{\cellcolor[rgb]{ 1,  .922,  .612}\textcolor[rgb]{ .612,  .396,  0}{视差}} \\ \hline
    \endhead
    \hline
    \endfoot
    3325.255 & 5618.453 & 5.224245 & 3325.155 & 5618.452 & 4.763 & 0.09994 & 0.000753 & 0.461245 & \cellcolor[rgb]{ 0,  1,  0}1 & 0.297181 \\
    3184.812 & 5654.026 & 2.897074 & 3184.427 & 5653.929 & 3.522 & 0.384661 & 0.09749 & 0.624926 & \cellcolor[rgb]{ 0,  1,  0}1 & 0.377562 \\
    3083.96 & 5658.854 & 3.063883 & 3083.859 & 5658.79 & 3.312 & 0.1012 & 0.064073 & 0.248117 & \cellcolor[rgb]{ 0,  1,  0}1 & -0.18291 \\
    3111.425 & 5520.782 & 3.227235 & 3111.317 & 5520.525 & 3.425 & 0.108157 & 0.25667 & 0.197765 & \cellcolor[rgb]{ 0,  1,  0}1 & 0.32122 \\
    3143.943 & 5678.49 & -738.298 & 3147.206 & 5470.207 & 4.169 & 3.262565 & 208.2828 & 742.467 & 0  & \cellcolor[rgb]{ 1,  .922,  .612}\textcolor[rgb]{ .612,  .396,  0}{-4.07085} \\
    3228.777 & 5480.75 & 2.99556 & 3228.796 & 5480.649 & 3.267 & 0.019044 & 0.100925 & 0.27144 & \cellcolor[rgb]{ 0,  1,  0}1 & 0.071135 \\
    3145.07 & 5249.663 & 3.840062 & 3144.775 & 5250.421 & 3.713 & 0.295457 & 0.758289 & 0.127062 & 0  & 0.297194 \\
    3154.425 & 5235.886 & 3.367915 & 3154.083 & 5236.616 & 3.883 & 0.342282 & 0.730335 & 0.515085 & 0  & 0.745434 \\
    3162.523 & 5295.517 & 6.821206 & 3163.058 & 5284.273 & 5.13 & 0.535353 & 11.24422 & 1.691206 & 0  & \cellcolor[rgb]{ 1,  .922,  .612}\textcolor[rgb]{ .612,  .396,  0}{-20.9876} \\
    3146.828 & 5406.833 & 6.588907 & 3147.106 & 5407.432 & 3.519 & 0.277898 & 0.599082 & 3.069907 & 0  & -0.73315 \\
    3206.447 & 5380.779 & 5.953802 & 3206.535 & 5381.13 & 5.721 & 0.088134 & 0.35065 & 0.232802 & \cellcolor[rgb]{ 0,  1,  0}1 & 0.001018 \\
    3172.05 & 5389.952 & 2.917194 & 3171.487 & 5390.877 & 3.477 & 0.562912 & 0.924515 & 0.559806 & 0  & -0.57639 \\
    3019.958 & 5439.047 & 18.92497 & 3019.786 & 5439.221 & 18.858 & 0.171974 & 0.173586 & 0.066971 & \cellcolor[rgb]{ 0,  1,  0}1 & -0.21386 \\
    2995.393 & 5482.496 & 5.097186 & 2995.51 & 5482.386 & 5.036 & 0.116671 & 0.110457 & 0.061186 & \cellcolor[rgb]{ 0,  1,  0}1 & -0.19903 \\
    2967.536 & 5480.259 & 2.794274 & 2968.319 & 5480.476 & 3.552 & 0.783068 & 0.217483 & 0.757726 & 0  & 0.117919 \\
    2982.813 & 5327.731 & 3.89828 & 2982.782 & 5327.922 & 3.686 & 0.031388 & 0.191383 & 0.21228 & \cellcolor[rgb]{ 0,  1,  0}1 & -0.04636 \\
    2820.283 & 5259.801 & 24.92544 & 2820.262 & 5259.748 & 25.074 & 0.02104 & 0.053471 & 0.148561 & \cellcolor[rgb]{ 0,  1,  0}1 & -0.41929 \\
    2698.771 & 5396.679 & 25.48408 & 2698.826 & 5396.59 & 24.948 & 0.055347 & 0.088873 & 0.536076 & \cellcolor[rgb]{ 0,  1,  0}1 & -0.06351 \\
    2664.823 & 5284.881 & 39.86166 & 2664.958 & 5285.255 & 39.314 & 0.134846 & 0.37388 & 0.547658 & \cellcolor[rgb]{ 0,  1,  0}1 & -0.32063 \\
    2614.094 & 5433.659 & 5.928632 & 2614.312 & 5433.274 & 7.184 & 0.218052 & 0.385223 & 1.255368 & 0  & 0.014302 \\
    2702.725 & 5478.588 & 109.5921 & 2714.137 & 5536.209 & 5.052 & 11.41173 & 57.62053 & 104.5401 & 0  & \cellcolor[rgb]{ 1,  .922,  .612}\textcolor[rgb]{ .612,  .396,  0}{42.99215} \\
    2759.544 & 5602.958 & 16.411 & 2759.629 & 5602.687 & 15.135 & 0.084923 & 0.271276 & 1.275997 & 0  & -0.29363 \\
    2944.734 & 5657.787 & 10.99146 & 2944.622 & 5656.879 & 12.318 & 0.111607 & 0.907834 & 1.326541 & 0  & 0.248677 \\
    3036.207 & 5805.065 & 39.47812 & 3035.991 & 5804.33 & 39.393 & 0.216064 & 0.734684 & 0.085122 & 0  & -0.21186 \\
    2914.508 & 5789.933 & 5.344705 & 2914.961 & 5789.961 & 3.508 & 0.453057 & 0.028015 & 1.836705 & 0  & -0.00953 \\
    3001.68 & 5713.906 & 12.52921 & 3001.643 & 5714.065 & 12.354 & 0.036529 & 0.159326 & 0.17521 & \cellcolor[rgb]{ 0,  1,  0}1 & 0.028133 \\
    2673.634 & 5343.384 & 5.231976 & 2673.824 & 5343.921 & 3.802 & 0.189676 & 0.537285 & 1.429976 & 0  & 0.048064 \\
    2788.98 & 5339.714 & 4.822834 & 2788.519 & 5339.584 & 3.685 & 0.460666 & 0.129667 & 1.137834 & 0  & -0.13145 \\
    2794.132 & 5258.57 & 14.63085 & 2793.894 & 5258.26 & 18.254 & 0.238312 & 0.30972 & 3.623147 & 0  & -0.22608 \\
    2859.356 & 5541.164 & -66.9855 & 2847.567 & 5517.383 & 3.811 & 11.78888 & 23.78114 & 70.79655 & 0  & \cellcolor[rgb]{ 1,  .922,  .612}\textcolor[rgb]{ .612,  .396,  0}{-3.30561} \\
    2860.672 & 5536.317 & 13.63305 & 2861.034 & 5535.474 & 13.14 & 0.361976 & 0.842683 & 0.49305 & 0  & -0.17649 \\
    3111.195 & 5520.64 & 4.058702 & 3111.322 & 5520.511 & 3.312 & 0.126535 & 0.128927 & 0.746702 & \cellcolor[rgb]{ 0,  1,  0}1 & 0.122392 \\
    3168.858 & 5731.799 & 4.020554 & 3168.406 & 5733.826 & 3.861 & 0.451515 & 2.026886 & 0.159554 & 0  & \cellcolor[rgb]{ 1,  .922,  .612}\textcolor[rgb]{ .612,  .396,  0}{4.923757} \\
    3184.318 & 5653.957 & 3.707308 & 3184.577 & 5653.93 & 3.514 & 0.258927 & 0.026923 & 0.193308 & \cellcolor[rgb]{ 0,  1,  0}1 & 0.390958 \\
    2995.643 & 5482.281 & 5.053977 & 2995.632 & 5482.021 & 5.131 & 0.011041 & 0.26043 & 0.077023 & \cellcolor[rgb]{ 0,  1,  0}1 & 0.209386 \\
    2454.64 & 5513.5 & 6.36637 & 2454.777 & 5508.459 & 7.65 & 0.137192 & 5.040551 & 1.28363 & 0  & -0.16594 \\
    2455.294 & 5495.215 & 10.79527 & 2455.552 & 5495.855 & 6.007 & 0.257511 & 0.639703 & 4.788266 & 0  & -0.60617 \\
    2528.315 & 5471.491 & 15.59424 & 2488.758 & 5481.32 & 17.977 & 39.5573 & 9.828765 & 2.382759 & 0  & 0.049632 \\
    2543.713 & 5645.396 & 19.56766 & 2518.955 & 5632.976 & 10.242 & 24.75778 & 12.42 & 9.325656 & 0  & 0.060322 \\
    2444.063 & 5512.932 & 51.35935 & 2483.411 & 5588.006 & 19.59 & 39.34828 & 75.07421 & 31.76935 & 0  & -0.42067 \\
    2770.157 & 5649.058 & 7.636558 & 2771.877 & 5660.56 & 6.221 & 1.720432 & 11.50226 & 1.415558 & 0  & \cellcolor[rgb]{ 1,  .922,  .612}\textcolor[rgb]{ .612,  .396,  0}{20.58445} \\
    2877.692 & 5935.674 & -88.3402 & 2889.648 & 5876.63 & 16.589 & 11.95554 & 59.04416 & 104.9292 & 0  & \cellcolor[rgb]{ 1,  .922,  .612}\textcolor[rgb]{ .612,  .396,  0}{17.76358} \\
    2720.502 & 5702.085 & 6.900078 & 2719.948 & 5701.465 & 4.231 & 0.55366 & 0.619827 & 2.669078 & 0  & 0.897326 \\
    2784.796 & 5676.349 & -28.616 & 2791.356 & 5661.921 & 9.81 & 6.559622 & 14.42761 & 38.42595 & 0  & -0.31841 \\
    2837.509 & 5745.567 & 16.7849 & 2792.756 & 5684.058 & 7.135 & 44.75303 & 61.50872 & 9.649902 & 0  & 0.39082 \\
    \end{longtable}%
\end{center}%
 }




\part{个人实习报告}
\chapter{余周炜个人实习报告}

摄影测量实习是摄影测量的实践课程,本着“理论指导实践”的原则,在上学期学了摄影测量的理论与方法之后,在本学期进行了摄影测量实习,提高了我们对摄影测量知识的理解,加强了我们实际运用的能力。

作为本摄影测量小组的组长,在本次摄影测量的实习过程中,承担了更大的责任,对实习的各个方面感触颇深,值得写在实习报告中,和大家分享交流。

\section{实习经历}

\subsubsection{一}
本次实习的时长将近半个学期,将摄影测量的流程走了一遍。从像片控制测量,到相关测量程序的编写,再到CAD成图和航片的调绘和判读,从外业到内业,熟悉了摄影测量的全过程,增强了团队协作意识和编程能力。

提到像片控制测量,我们组的进程不算快,但我力求让每个人都做一遍。任务的执行是这样的:先让一个人来测量他选定的像控点,如果两次测量的差异较大,即误差超限的情况,则此组员放弃他的测量,让下一个测量,他排到队尾,这样轮流测量,增加了大家控制质量的意识,同时强迫自己学习别人测量得又快又准的人的经验,也节省了时间。

可能是因为结果不太好的缘故,我们组的几个同学在第二天进行了补测,将原来误差较大的点的坐标测得更加准确,体现了我们组一丝不苟的素质和精益求精的精神。

通过外业的实习,我也向组员表明了我的态度:\emph{不以效率为最高目的,而是以质量、以理解和掌握水平为最高的追求目标。}


\subsubsection{二}
接下来就是编写程序,在编写程序刚刚开始的阶段,我们组的王雪辰同学向老师反映了任务书中程序的一处符号错误,但由于是迭代,程序依然收敛,我不知道最后的讨论结果如何,我认为可能会对迭代收敛的快慢有影响。


既然任务书上的程序也有错误抑或是争议,这验证了我上学期向老师提出的观点。
我在上学期已经向老师说明过她的这种推导方式极易发生错误,有时候是别人抄错了,有时候是自己想错了,自己想错了可以避免,抄错了这种情况却难以避免,在很多教材中各种各样的错误依然难以避免,所以理解原理是十分有必要的。我们不应该把这么重要的程序放在人的耐心和细心程度上,这种想法通常是不靠谱和极易发生错误的,而是应该把推导过程用计算机实现,例如用Matlab的符号变量,这样大家在推导公式的同时,也增强了对Matlab软件的理解和运用,符合信息时代的潮流。

近些年来,线性代数与矩阵分析的知识发展迅猛,在熟练地使用矩阵这个工具之后,不仅降低了思维的难度,也将繁重的计算交给计算机来完成,实现了生产力的解放。然而,书中的方法在我看来依然是简单的迭代方法,虽然有矩阵的外形,但没有运用矩阵的实质,如矩阵求导、Jacobian行列式的知识没有运用到,Matlab中强大的符号计算也没有运用到。

所以我没有采用实习任务书上的方法,我也不去验证任务书上的方法迭代效果如何。在上学期的摄影测量编程和本次摄影测量实习的编程过程中,我着重强调并实现了Gauss-Newton法,运用了Matlab符号变量和Matlab对矩阵的一些操作,此中的编程思想详见我编写的航测内业实习报告\ref{sec:neiye}部分和接下来我要贴的程序,我认为我已经讲得比较详细了,这里无需赘述。可以看出运用矩阵和符号变量的程序远比不用这些功能的程序简洁,这样,可以让我们站在一个更高的抽象层面来思考问题。

编写的程序存在不收敛的问题,这种问题一般是同名点找的不好或有错误造成的,即初值的选取问题。对于这种情况,我们可以通过上下视差判断,将上下视差较大的点剔除,此种方法类似于粗差探测。将粗差找出来之后,再进行迭代,则可以使迭代收敛。另外,如果是程序编写的逻辑错误,则需耐心细致地检查。


\subsubsection{三}

接下来就是CAD成图、航片的调绘判读,和一些实验室的参观。

对于调绘和相片判读部分,我强烈建议让这项工作由小组来完成。这项工作没有任何技术含量可言,需要的只是时间和一份难得的耐心,如果给小组分工,在加快进度的同时,也增强团队分工与协作意识,每个人单打独斗,实无必要,这样不仅是进行重复的劳动,也毫无效率可言。我也相信真实情况肯定不会是每个人都去调绘的。搞得上面说一套,下面做一套,也不好。

另外,我们还练习使用了Kinect产品。Kinect是微软的一款体感游戏产品,也可以用作场景的三维建模。练习过程中,我们运用Kinect将桌子和椅子进行了三维建模,通过软件观察了建模的效果。之后,我们用钢尺量了电脑的宽度,和软件上的距离做了比较,发现还是有一些差别的,说明Kinect主要用作体感游戏的,并不适合较为精确的三维建模。


\subsubsection{四}

说到报告的撰写,是我提议用\LaTeX{}撰写的,这样做提高了排版的质量,表达了我们组的\emph{审美追求}。

\LaTeX{}不同于Word这种所见即所得的软件,是一个所想即所得(What You Think Is What You Get, WYTIWTG)的开源排版软件,他的作者是鼎鼎大名的高德纳(Donald Ervin Knuth)先生,他是算法和程序设计技术的先驱者,计算机排版系统\TeX{}和\MF 的发明者,他因这些成就和大量创造性的影响深远的著作而誉满全球。被誉为“人工智能之父”。

本报告的模板是Elegant\LaTeX{} Book,在此感谢其作者ddswhu和LiamHuang0205,其邮箱为\url{elegantlatex@gmail.com}。

\section{实习感想}

本次实习,我们通过实践对书中的理论进行了验证,加深了对摄影测量学基础理论、测量原理的理解和掌握程度,提高了实践技能,同时也增进了同学间的友谊,增加了配合的默契程度,体现了团队精神。在我看来,通过实习获得的解决问题的能力和收获的精神品质,远比实习学到的知识重要的多。

\section{程序集}

\subsection{内定向}

\begin{lstlisting}[caption=nedingxiang.m文件]
mark=textread('C:\Users\wode\Desktop\摄影测量实验\camera.use');
pixel=[7679 13823;0 13823 ;0 0;7679 0];
y=reshape(mark',[8 1]);
X=zeros(8,6); 
for i=1:4,
    X(2*i-1:2*i,:)=[1,pixel(i,1),pixel(i,2),0,0,0;
                    0,0,0,1,pixel(i,1),pixel(i,2)];
end
N=inv(X'*X);
beta=N*X'*y
csvwrite('C:\Users\wode\Desktop\摄影测量实验\余周炜\neidingxiang.csv',beta);
r=size(y,1)-6;
e=y-X*beta;
sigma=sqrt(norm(e)/r);
% cc=xlsread('C:\Users\wode\Desktop\摄影测量实验\点之记改.xlsx',1,'B3:E47');
cc=xlsread('C:\Users\wode\Desktop\摄影测量实验\点之记新.xlsx',1,'B3:E13');
retVal=zeros(size(cc,1),4);
for i=1:size(cc,1),
    line=[1 cc(i,1) cc(i,2) 0 0 0;
    0 0 0 1 cc(i,1) cc(i,2);
    1 cc(i,3) cc(i,4) 0 0 0;
    0 0 0 1 cc(i,3) cc(i,4)]*beta;
    retVal(i,:)=line';
end
% xlswrite('C:\Users\wode\Desktop\摄影测量实验\框标点.xlsx',retVal);
% xlswrite('C:\Users\wode\Desktop\摄影测量实验\全班框标点.xlsx',retVal);
xlswrite('C:\Users\wode\Desktop\摄影测量实验\框标点新.xlsx',retVal);
\end{lstlisting}

模型:
\begin{equation}
\begin{bmatrix}
x \\ y
\end{bmatrix}
=\begin{bmatrix}
h_1 & h_2 \\
k_1 & k_2 
\end{bmatrix}
\begin{bmatrix}
i \\ j
\end{bmatrix}
+\begin{bmatrix}
h_0 \\ k_0
\end{bmatrix}
\end{equation}

得出的内定向参数如下:
\begin{equation}
\begin{array}{lll}
h_0=-46.08 & h_1=0.012 & h_2=0 \\
k_0=82.944 & k_1=0 & k_2=-0.012
\end{array}
\end{equation}

\subsection{相对定向}

\begin{lstlisting}[caption=xiangdui.m文件]
syms theta;
syms phi1 kappa1;
syms phi2 omega2 kappa2;
f=120; %mm
x0=0; %principal point shift
y0=0;
Rx=[1 0 0;0 cos(theta) -sin(theta);0 sin(theta) cos(theta)];
Ry=[cos(theta) 0 -sin(theta); 0 1 0; sin(theta) 0 cos(theta)];
Rz=[cos(theta) -sin(theta) 0; sin(theta) cos(theta) 0;0 0 1];
R1=subs(Ry,phi1)*subs(Rz,kappa1);
R2=subs(Ry,phi2)*subs(Rx,omega2)*subs(Rz,kappa2);
% mat=xlsread('C:\Users\wode\Desktop\摄影测量实验\框标点.xlsx',1,'A1:D34');
mat=xlsread('C:\Users\wode\Desktop\摄影测量实验\框标点新.xlsx',1,'A1:D11');
% mat=xlsread('C:\Users\wode\Desktop\摄影测量实验\全班框标点.xlsx',1,'A1:D31');
mat1=mat(:,1:2);
mat2=mat(:,3:4);
mat1=[mat1,-f*ones(size(mat1,1),1)]';
mat2=[mat2,-f*ones(size(mat2,1),1)]';
mat11=R1*mat1;
mat22=R2*mat2;
r=(mat11(2,:).*mat22(3,:)-mat11(3,:).*mat22(2,:)).';
x=[0,0,0,0,0]';
v=GaussNewton(r,x,2e-7)
csvwrite('C:\Users\wode\Desktop\摄影测量实验\余周炜\xiangdui.csv',v);
\end{lstlisting}


\begin{lstlisting}[caption=GaussNewton.m文件]
function [retVal]=GaussNewton(f,x,error)
syms phi1 kappa1 phi2 omega2 kappa2;
v=[phi1 kappa1 phi2 omega2 kappa2];
j=jacobian(f,v);
J=eval(subs(j,v,x'));
F=eval(subs(f,v,x'));
k=0;
d=1;
while norm(d)>error,
    d=-inv(J'*J)*J'*F;
    x=x+d;
    J=eval(subs(j,v,x'));
    F=eval(subs(f,v,x'))
    k=k+1
    disp(norm(J'*F));
end
retVal=x;
\end{lstlisting}

还有一个和相对定向相关的前方交会程序:
\begin{lstlisting}[caption=qianfang.m]
v=csvread('C:\Users\wode\Desktop\摄影测量实验\余周炜\xiangdui.csv');
% kuangbiao=xlsread('C:\Users\wode\Desktop\摄影测量实验\框标点.xlsx');
% kuangbiao=xlsread('C:\Users\wode\Desktop\摄影测量实验\全班框标点.xlsx');
kuangbiao=xlsread('C:\Users\wode\Desktop\摄影测量实验\框标点副本.xlsx');
kuangbiao=kuangbiao(:,1:4);
syms theta;
f=120;
Rx=[1 0 0;0 cos(theta) -sin(theta);0 sin(theta) cos(theta)];
Ry=[cos(theta) 0 -sin(theta);0 1 0;sin(theta) 0 cos(theta)];
Rz=[cos(theta) -sin(theta) 0;sin(theta) cos(theta) 0;0 0 1];
phi1=v(1);
kappa1=v(2);
phi2=v(3);
omega2=v(4);
kappa2=v(5);
R1=eval(subs(Ry,phi1)*subs(Rz,kappa1));
R2=eval(subs(Ry,phi2)*subs(Rx,omega2)*subs(Rz,kappa2));
image1=[kuangbiao(:,1:2) -f*ones(size(kuangbiao,1),1)]';
image2=[kuangbiao(:,3:4) -f*ones(size(kuangbiao,1),1)]';
a1=R1*image1;
a2=R2*image2;
M=(7680*0.2)/(46.08*2); %0.2是像元宽度。
b=46.08*2*(1-0.6)*M;
B=[b 0 0];
xyzq=zeros(size(a1,2),4);
for i=1:size(a1,2),
    N1=(B(1)*a2(3,i))/(a1(1,i)*a2(3,i)-a2(1,i)*a1(3,i));
    N2=(B(1)*a1(3,i))/(a1(1,i)*a2(3,i)-a2(1,i)*a1(3,i));
    x=N1*a1(1,i);
    t1=N1*a1(2,i)
    t2=N2*a2(2,i)
    y=0.5*(N1*a1(2,i)+N2*a2(2,i));
    z=N1*a1(3,i);
    q=N1*a1(2,i)-N2*a2(2,i);
    xyzq(i,:)=[x y z q]
end
q=xyzq(:,4);
% XYZ=xlsread('C:\Users\wode\Desktop\摄影测量实验\点之记改.xlsx',1,'H3:J47');
XYZ=xlsread('C:\Users\wode\Desktop\摄影测量实验\点之记 - 副本.xls',1,'F3:H13');
xyz=xyzq(:,1:3);
xlswrite('C:\Users\wode\Desktop\摄影测量实验\inspect副本.xlsx',[kuangbiao q xyz XYZ]);
\end{lstlisting}

得出的相对定向参数如下:
\begin{equation}
\begin{array}{lll}
\phi_1=-0.012455 & & \kappa_1=0.028415 \\
\phi_2=0.013341 & \omega_2=-0.022506 & \kappa_2=0.036366
\end{array}
\end{equation}

\subsection{绝对定向}

\begin{lstlisting}[caption=juedui.m文件]
allpoints=xlsread('C:\Users\wode\Desktop\摄影测量实验\inspect副本.xlsx');
xyz=allpoints(:,6:8);
XYZ=allpoints(:,9:11);
XYZmean=mean(XYZ);
XYZ=XYZ-ones(size(XYZ))*diag(mean(XYZ));
xyz=xyz-ones(size(xyz))*diag(mean(xyz));
syms theta;
Rx=[1 0 0;0 cos(theta) -sin(theta);0 sin(theta) cos(theta)];
Ry=[cos(theta) 0 -sin(theta);0 1 0;sin(theta) 0 cos(theta)];
Rz=[cos(theta) -sin(theta) 0;sin(theta) cos(theta) 0;0 0 1];
syms phi omega kappa lambda dx dy dz;
R=subs(Ry,phi)*subs(Rx,omega)*subs(Rz,kappa);
groundStretch=reshape(XYZ',size(XYZ,1)*size(XYZ,2),1);
rotateModel=lambda*R*xyz';
r=rotateModel+diag([dx dy dz])*ones(size(rotateModel));
r=reshape(r,size(r,1)*size(r,2),1);
r=r-groundStretch;
x=[0 0 0 1 0 0 0]';
v=GaussNewton2(r,x,2e-7)
csvwrite('C:\Users\wode\Desktop\摄影测量实验\余周炜\juedui.csv',[v;XYZmean']);
\end{lstlisting}


\begin{lstlisting}[caption=GaussNewton2.m文件]
function [retVal]=GaussNewton2(f,x,error)
syms phi omega kappa lambda dx dy dz;
v=[phi omega kappa lambda dx dy dz];
j=jacobian(f,v);
J=eval(subs(j,v,x'));
F=eval(subs(f,v,x'));
k=0;
while norm(J'*F)>error,
    d=-inv(J'*J)*J'*F;
    x=x+d
    J=eval(subs(j,v,x'));
    F=eval(subs(f,v,x'));
    k=k+1
    disp(norm(J'*F));
end
retVal=x;
\end{lstlisting}

\begin{lstlisting}[caption=jueduitransform.m]
v=csvread('C:\Users\wode\Desktop\摄影测量实验\余周炜\juedui.csv');
allpoints=xlsread('C:\Users\wode\Desktop\摄影测量实验\inspect副本.xlsx');

syms theta;
Rx=[1 0 0;0 cos(theta) -sin(theta);0 sin(theta) cos(theta)];
Ry=[cos(theta) 0 -sin(theta);0 1 0;sin(theta) 0 cos(theta)];
Rz=[cos(theta) -sin(theta) 0;sin(theta) cos(theta) 0;0 0 1];
phi=v(1);
omega=v(2);
kappa=v(3);
lambda=v(4);
d=v(5:7);
R=eval(subs(Ry,phi)*subs(Rx,omega)*subs(Rz,kappa));

xyz=allpoints(:,6:8);
XYZ0=allpoints(:,9:11);
xyz=xyz-ones(size(xyz))*diag(mean(xyz));
XYZ=lambda*R*xyz'+diag(d)*ones(size(xyz'))
XYZ=XYZ+diag(mean(XYZ0))*ones(size(XYZ))
xlswrite('C:\Users\wode\Desktop\摄影测量实验\dimianzuobiao.xlsx',XYZ);
\end{lstlisting}

我还编写了一个程序在参数已知的情形下直接从像素坐标转换到地面摄测坐标,即将内定向、前方交会、绝对定向坐标转换三合一了:
\begin{lstlisting}[caption=transform.m]
v1=csvread('C:\Users\wode\Desktop\摄影测量实验\余周炜\neidingxiang.csv');
v2=csvread('C:\Users\wode\Desktop\摄影测量实验\余周炜\xiangdui.csv');
v3=csvread('C:\Users\wode\Desktop\摄影测量实验\余周炜\juedui.csv');

pair=xlsread('C:\Users\wode\Desktop\摄影测量实验\点之记 - 副本.xls',1,'B3:E76');

syms theta;
f=120;
Rx=[1 0 0;0 cos(theta) -sin(theta);0 sin(theta) cos(theta)];
Ry=[cos(theta) 0 -sin(theta);0 1 0;sin(theta) 0 cos(theta)];
Rz=[cos(theta) -sin(theta) 0;sin(theta) cos(theta) 0;0 0 1];

n=size(pair,1);
%内定向
img1=[pair(:,1:2),ones(n,1)];
img2=[pair(:,3:4),ones(n,1)];
A=[v1(2) v1(3) v1(1);v1(5) v1(6) v1(4);0 0 1];
img1=A*img1';
img2=A*img2';

%前方交会

phi1=v2(1);
kappa1=v2(2);
phi2=v2(3);
omega2=v2(4);
kappa2=v2(5);

R1=eval(subs(Ry,phi1)*subs(Rz,kappa1));
R2=eval(subs(Ry,phi2)*subs(Rx,omega2)*subs(Rz,kappa2));

img1(3,:)=-f*img1(3,:);
img2(3,:)=-f*img2(3,:);

M=(7680*0.2)/(46.08*2); %0.2是像元宽度。
b=46.08*2*(1-0.6)*M;
B=[b 0 0];

a1=R1*img1;
a2=R2*img2;

N1=(B(1).*a2(3,:))./(a1(1,:).*a2(3,:)-a2(1,:).*a1(3,:));
N2=(B(1).*a1(3,:))./(a1(1,:).*a2(3,:)-a2(1,:).*a1(3,:));

x=N1.*a1(1,:);
y=0.5*(N1.*a1(2,:)+N2.*a2(2,:));
z=N1.*a1(3,:);

xyz=[x;y;z]';


%模型坐标转换

phi=v3(1);
omega=v3(2);
kappa=v3(3);
lambda=v3(4);
d=v3(5:7);

meanXYZ=v3(8:10);
meanxyz=v3(11:13);
R=eval(subs(Ry,phi)*subs(Rx,omega)*subs(Rz,kappa));
xyz=xyz-ones(size(xyz))*diag(meanxyz);

XYZ=lambda*R*xyz';
XYZ=lambda*R*xyz'+diag(d)*ones(size(xyz'));
XYZ=XYZ+diag(meanXYZ)*ones(size(XYZ));
xlswrite('C:\Users\wode\Desktop\摄影测量实验\dimianzuobiao.xlsx',XYZ');
\end{lstlisting}

得出的绝对定向参数如下:
\begin{equation}
\begin{array}{lll}
\phi=0.011264 & \omega=0.005638 & \kappa=0.0053766 \\
& \lambda=1.1318 & \\
dx=0 & dy=0 & dz=0  
\end{array}
\end{equation}
值得注意的是,这里的$\MR=\MR_y(\phi)\MR_x(\omega)\MR_z(\kappa)$,与书中的布尔萨模型不同。同时地面坐标和模型坐标均是重心化的结果。故转换前,需减去参与绝对定向的模型的重心坐标,转换后,需加上参与绝对定向的地面点的重心坐标。

参与定向的模型的重心坐标如下:
\begin{equation}
\begin{array}{lll}
\bar x=513.05 & \bar y=611.92 & \bar z=-2113.3
\end{array}   
\end{equation}

参与定向的地面点重心坐标如下:
\begin{equation}
\begin{array}{lll}
\bar X=3057.3 & \bar Y=5490.9 & \bar Z=8.1253
\end{array}
\end{equation}


\section{同名点}


% \section{相对定向点}

% 我们小组量测得到的相对定向点如下表:

% % Table generated by Excel2LaTeX from sheet 'Sheet1'
% \begin{center}
%     \tablehead{   \hline     & \multicolumn{2}{c|}{63} & \multicolumn{2}{c|}{64} \\ \hline
%     点号 & \multicolumn{1}{c}{i} & \multicolumn{1}{c|}{j} & \multicolumn{1}{c}{i} & \multicolumn{1}{c|}{j} \\ \hline}
%     \tabletail{\hline}
%       \begin{supertabular}{|l|rr|rr|}

     
     
    
  
%       \end{supertabular}%
%   \end{center}%

\begin{table}[htbp]
\centering
\begin{tabular}{lrrrr}
    \toprule
    & \multicolumn{2}{c}{63} & \multicolumn{2}{c}{64} \\ \hline
    点号 & \multicolumn{1}{c}{i} & \multicolumn{1}{c}{j} & \multicolumn{1}{c}{i} & \multicolumn{1}{c}{j} \\ \midrule
    1  & 3507 & 1509 & 316 & 1067 \\
    2  & 3699 & 1408 & 466 & 965 \\
    3  & 3791 & 1410 & 556 & 968 \\
    4  & 4442 & 1397 & 1278 & 969 \\
    5  & 4446 & 1457 & 1281 & 1031 \\
    6  & 4821 & 1394 & 1663 & 975 \\
    7  & 5672 & 2044 & 2409 & 1657 \\ \bottomrule
\end{tabular}
\end{table}
  


\section{调绘图件}

\begin{figure}[htbp]
\centering
\caption{调绘图概览}
\includegraphics[width=\textwidth]{diaohui.jpg}
\end{figure}

\begin{figure}[htbp]
\centering
\caption{调绘图细节}
\includegraphics[width=\textwidth]{diaohui1.jpg}
\end{figure}

\chapter{王雪辰个人实习报告}

\section{数字内定向}

\subsection{程序}

\begin{lstlisting}[caption=get\_tmd\_dpj.m]
function [xy_63dpj,xy_64dpj]=get_tmd_dpj(fpoint_dpj)
xy_63dpj=xlsread(fpoint_dpj,'B3:C47')
xy_64dpj=xlsread(fpoint_dpj,'D3:E47')
end
\end{lstlisting}

\begin{lstlisting}[caption=get\_tmd\_dzj.m]
function [xy_63dzj,xy_64dzj]=get_tmd_dzj(fpoint_dzj)
xy_63dzj=xlsread(fpoint_dzj,'B3:C47')
xy_64dzj=xlsread(fpoint_dzj,'D3:E47')
end
\end{lstlisting}

\begin{lstlisting}[caption=get\_tmd\_xby.m]
function [xy_63xby,xy_64xby]=get_tmd_xby(fpoint_xby)
xy_63xby=[xlsread(fpoint_xby,'B3:B18'),xlsread(fpoint_xby,'C3:C18')]
xy_64xby=[xlsread(fpoint_xby,'D3:D18'),xlsread(fpoint_xby,'E3:E18')]
end
\end{lstlisting}

\begin{lstlisting}[caption=get\_tmd\_dlt.m]
function [xy_63dlt,xy_64dlt]=get_tmd_dlt(fpoint_dlt)
xy_63dlt=[xlsread(fpoint_dlt,'B3:B76'),xlsread(fpoint_dlt,'C3:C76')]
xy_64dlt=[xlsread(fpoint_dlt,'D3:D76'),xlsread(fpoint_dlt,'E3:E76')]
end
\end{lstlisting}

\begin{lstlisting}[caption=get\_tmd\_xz.m]
function [xy_63xz,xy_64xz]=get_tmd_xz(fpoint_xz)
xy_63xz=[xlsread(fpoint_xz,'B3:B37'),xlsread(fpoint_xz,'C3:C37')]
xy_64xz=[xlsread(fpoint_xz,'D3:D37'),xlsread(fpoint_xz,'E3:E37')]
end
\end{lstlisting}

\begin{lstlisting}[caption=inner\_orientation.m]
function points_out=inner_orientation(a0,a1,a2,b0,b1,b2,points_in)
para=[a0,b0;a1,b1;a2,b2]
points_out=points_in*para
end
\end{lstlisting}

\begin{lstlisting}[caption=inner\_orientation\_for\_xby.m]
function inner_orientation_for_xby()
file='D:\大三下\摄影测量实习\Photogrammetry\camera.use'
[a0,a1,a2,b0,b1,b2]=inner_orientation_para(file)
fpoint_xby='D:\大三下\摄影测量实习\西北一.xlsx'
[xy_63xby,xy_64xby]=get_tmd_xby(fpoint_xby)
points_in63xby=[ones(length(xy_63xby),1),xy_63xby]
points_in64xby=[ones(length(xy_64xby),1),xy_64xby]
points_out63xby=inner_orientation(a0,a1,a2,b0,b1,b2,points_in63xby)
points_out64xby=inner_orientation(a0,a1,a2,b0,b1,b2,points_in64xby)
points_outxby=[points_out63xby,points_out64xby]
fwrite_xby='D:\大三下\摄影测量实习\程序\数字内定向\西北一内定向点.xlsx'
xlswrite(fwrite_xby,points_outxby)
end
\end{lstlisting}

\begin{lstlisting}[caption=inner\_orientation\_for\_dlt.m]
function inner_orientation_for_dlt()
file='D:\大三下\摄影测量实习\Photogrammetry\camera.use'
[a0,a1,a2,b0,b1,b2]=inner_orientation_para(file)
fpoint_dlt='D:\大三下\摄影测量实习\大礼堂.xlsx'
[xy_63dlt,xy_64dlt]=get_tmd_dlt(fpoint_dlt)
points_in63dlt=[ones(length(xy_63dlt),1),xy_63dlt]
points_in64dlt=[ones(length(xy_64dlt),1),xy_64dlt]
points_out63dlt=inner_orientation(a0,a1,a2,b0,b1,b2,points_in63dlt)
points_out64dlt=inner_orientation(a0,a1,a2,b0,b1,b2,points_in64dlt)
points_outdlt=[points_out63dlt,points_out64dlt]
fwrite_dlt='D:\大三下\摄影测量实习\程序\数字内定向\大礼堂内定向点.xlsx'
xlswrite(fwrite_dlt,points_outdlt)
end
\end{lstlisting}

\begin{lstlisting}[caption=run\_innerorientation.m]
function [a0,a1,a2,b0,b1,b2]=run_innerorientation()
file='D:\大三下\摄影测量实习\Photogrammetry\camera.use'
[a0,a1,a2,b0,b1,b2]=inner_orientation_para(file)
fpoint_xz='D:\大三下\摄影测量实习\程序\数字内定向\小组同名点.xlsx'
fpoint_dzj='D:\大三下\摄影测量实习\程序\点之记 - 副本.xls'
fpoint_dpj='D:\大三下\摄影测量实习\程序\点评价.xls'
[xy_63xz,xy_64xz]=get_tmd_xz(fpoint_xz)
[xy_63dzj,xy_64dzj]=get_tmd_dzj(fpoint_dzj)
[xy_63dpj,xy_64dpj]=get_tmd_dpj(fpoint_dpj)
points_in63xz=[ones(length(xy_63xz),1),xy_63xz]
points_in63dzj=[ones(length(xy_63dzj),1),xy_63dzj]
points_in63dpj=[ones(length(xy_63dpj),1),xy_63dpj]
points_in64xz=[ones(length(xy_64xz),1),xy_64xz]
points_in64dzj=[ones(length(xy_64dzj),1),xy_64dzj]
points_in64dpj=[ones(length(xy_64dpj),1),xy_64dpj]
points_out63xz=inner_orientation(a0,a1,a2,b0,b1,b2,points_in63xz)
points_out64xz=inner_orientation(a0,a1,a2,b0,b1,b2,points_in64xz)
points_out63dzj=inner_orientation(a0,a1,a2,b0,b1,b2,points_in63dzj)
points_out64dzj=inner_orientation(a0,a1,a2,b0,b1,b2,points_in64dzj)
points_out63dpj=inner_orientation(a0,a1,a2,b0,b1,b2,points_in63dpj)
points_out64dpj=inner_orientation(a0,a1,a2,b0,b1,b2,points_in64dpj)
points_outxz=[points_out63xz,points_out64xz]
points_outdzj=[points_out63dzj,points_out64dzj]
points_outdpj=[points_out63dpj,points_out64dpj]
fwrite_xz='D:\大三下\摄影测量实习\程序\数字内定向\小组内定向点.xlsx'
fwrite_dzj='D:\大三下\摄影测量实习\程序\数字内定向\点之记内定向点.xlsx'
fwrite_dpj='D:\大三下\摄影测量实习\程序\数字内定向\点评价内定向点.xlsx'
xlswrite(fwrite_xz,points_outxz)
xlswrite(fwrite_dzj,points_outdzj)
xlswrite(fwrite_dpj,points_outdpj)
end
\end{lstlisting}

\begin{lstlisting}[caption=inner\_orientation\_para.m]
function [a0,a1,a2,b0,b1,b2]=inner_orientation_para(file)
f=fopen(file,'r')
fgets(f)
fgets(f)
xy=[]
for i=1:4
    line=fgets(f)
    ii=str2num(line(3:10))
    jj=str2num(line(15:22))
    insert=[ii,jj]
    xy=[xy;insert]
end
A=[1,7679,13823;1,0,13823;1,0,0;1,7679,0]
para=inv(A'*A)*A'*xy
a0=para(1,1)
a1=para(2,1)
a2=para(3,1)
b0=para(1,2)
b1=para(2,2)
b2=para(3,2)
end   
\end{lstlisting}

\subsection{运行结果}

\begin{figure}[htbp]
\caption{运行结果}
\centering
\includegraphics[width=5cm]{result1.png}
\end{figure}

\section{单独像对相对定向}

\subsection{程序}

\begin{lstlisting}[caption=relative\_orientation.m]
function [d_fai1,d_kapa1,d_omega2,d_fai2,d_kapa2]=relative_orientation(x1y1,x2y2,f,fai1_0,kapa1_0,fai2_0,omega2_0,kapa2_0)
R0_1=R(fai1_0,0,kapa1_0)
R0_2=R(fai2_0,omega2_0,kapa2_0)
A=[]
Q=[]
fai1_1=0
okapa1_1=0
fai2_1=0
omega2_1=0
kapa2_1=0
for i=1:length(x1y1)
    x1=x1y1(i,1)
    y1=x1y1(i,2)
    [X1,Y1,Z1]=xyz_to_XYZ(R0_1,x1,y1,f)
    x2=x2y2(i,1)
    y2=x2y2(i,2)
    [X2,Y2,Z2]=xyz_to_XYZ(R0_2,x2,y2,f)
    ai=Ai(X1,Y1,X2,Y2,Z1,Z2)
    A=[A;ai]
    qi=Qi(f,Y1,Z1,Y2,Z2)
    Q=[Q;qi]
end
adjust=inv(A'*A)*A'*Q
d_fai1=adjust(1)
d_kapa1=adjust(2)
d_fai2=adjust(3)
d_omega2=adjust(4)
d_kapa2=adjust(5)
end
\end{lstlisting}

\begin{lstlisting}[caption=relative\_orientation\_dlt.m]
function relative_orientation_dlt()
file_xz='D:\大三下\摄影测量实习\程序\数字内定向\小组内定向点.xlsx'
file_dlt='D:\大三下\摄影测量实习\程序\数字内定向\大礼堂内定向点.xlsx'
x1y1_xz=xlsread(file_xz,'A:B')
x2y2_xz=xlsread(file_xz,'C:D')
x1y1_dlt=xlsread(file_dlt,'A:B')
x2y2_dlt=xlsread(file_dlt,'C:D')
f=120
[fai1_t,kapa1_t,fai2_t,omega2_t,kapa2_t]=iterate_for_adjustment(x1y1_xz,x2y2_xz,f)
%[fai1_t,kapa1_t,fai2_t,omega2_t,kapa2_t]=iterate_for_adjustment(x1y1_dzj,x2y2_dzj,f)
% fwrite='D:\大三下\摄影测量实习\程序\单独像对相对定向\相对定向参数.xlsx'
% xlswrite(fwrite,[fai1_t,kapa1_t,fai2_t,omega2_t,kapa2_t])
X1Y1_xz=[]
R1=R(fai1_t,0,kapa1_t)
X2Y2_xz=[]
R2=R(fai2_t,omega2_t,kapa2_t)
X1Y1_dlt=[]
%R1=R(fai1_t,0,kapa1_t)
X2Y2_dlt=[]
%R2=R(fai2_t,omega2_t,kapa2_t)
for i=1:length(x1y1_xz)
    x=x1y1_xz(i,1)
    y=x1y1_xz(i,2)
    [X,Y,Z]=xyz_to_XYZ(R1,x,y,f)
    X1Y1_xz=[X1Y1_xz;[X,Y,Z]]
end
for i=1:length(x2y2_xz)
    x=x2y2_xz(i,1)
    y=x2y2_xz(i,2)
    [X,Y,Z]=xyz_to_XYZ(R2,x,y,f)
    X2Y2_xz=[X2Y2_xz;[X,Y,Z]]
end    
for i=1:length(x1y1_dlt)
    x=x1y1_dlt(i,1)
    y=x1y1_dlt(i,2)
    [X,Y,Z]=xyz_to_XYZ(R1,x,y,f)
    X1Y1_dlt=[X1Y1_dlt;[X,Y,Z]]
end
for i=1:length(x2y2_dlt)
    x=x2y2_dlt(i,1)
    y=x2y2_dlt(i,2)
    [X,Y,Z]=xyz_to_XYZ(R2,x,y,f)
    X2Y2_dlt=[X2Y2_dlt;[X,Y,Z]]
end    
XY_xz=[X1Y1_xz,X2Y2_xz]
XY_dlt=[X1Y1_dlt,X2Y2_dlt]
%ffinal_xz='D:\大三下\摄影测量实习\程序\单独像对相对定向\小组相对定向后的坐标.xlsx'
ffinal_dlt='D:\大三下\摄影测量实习\程序\单独像对相对定向\大礼堂相对定向后的坐标.xlsx'
%xlswrite(ffinal_xz,XY_xz)
xlswrite(ffinal_dlt,XY_dlt)
end
\end{lstlisting}

\begin{lstlisting}[caption=relative\_orientation\_xby.m]
function relative_orientation_xby()
file_xz='D:\大三下\摄影测量实习\程序\数字内定向\小组内定向点.xlsx'
file_dlt='D:\大三下\摄影测量实习\程序\数字内定向\西北一内定向点.xlsx'
x1y1_xz=xlsread(file_xz,'A:B')
x2y2_xz=xlsread(file_xz,'C:D')
x1y1_xby=xlsread(file_dlt,'A:B')
x2y2_xby=xlsread(file_dlt,'C:D')
f=120
[fai1_t,kapa1_t,fai2_t,omega2_t,kapa2_t]=iterate_for_adjustment(x1y1_xz,x2y2_xz,f)
%[fai1_t,kapa1_t,fai2_t,omega2_t,kapa2_t]=iterate_for_adjustment(x1y1_dzj,x2y2_dzj,f)
% fwrite='D:\大三下\摄影测量实习\程序\单独像对相对定向\相对定向参数.xlsx'
% xlswrite(fwrite,[fai1_t,kapa1_t,fai2_t,omega2_t,kapa2_t])
X1Y1_xz=[]
R1=R(fai1_t,0,kapa1_t)
X2Y2_xz=[]
R2=R(fai2_t,omega2_t,kapa2_t)
X1Y1_xby=[]
%R1=R(fai1_t,0,kapa1_t)
X2Y2_xby=[]
%R2=R(fai2_t,omega2_t,kapa2_t)
for i=1:length(x1y1_xz)
    x=x1y1_xz(i,1)
    y=x1y1_xz(i,2)
    [X,Y,Z]=xyz_to_XYZ(R1,x,y,f)
    X1Y1_xz=[X1Y1_xz;[X,Y,Z]]
end
for i=1:length(x2y2_xz)
    x=x2y2_xz(i,1)
    y=x2y2_xz(i,2)
    [X,Y,Z]=xyz_to_XYZ(R2,x,y,f)
    X2Y2_xz=[X2Y2_xz;[X,Y,Z]]
end    
for i=1:length(x1y1_xby)
    x=x1y1_xby(i,1)
    y=x1y1_xby(i,2)
    [X,Y,Z]=xyz_to_XYZ(R1,x,y,f)
    X1Y1_xby=[X1Y1_xby;[X,Y,Z]]
end
for i=1:length(x2y2_xby)
    x=x2y2_xby(i,1)
    y=x2y2_xby(i,2)
    [X,Y,Z]=xyz_to_XYZ(R2,x,y,f)
    X2Y2_xby=[X2Y2_xby;[X,Y,Z]]
end    
XY_xz=[X1Y1_xz,X2Y2_xz]
XY_xby=[X1Y1_xby,X2Y2_xby]
%ffinal_xz='D:\大三下\摄影测量实习\程序\单独像对相对定向\小组相对定向后的坐标.xlsx'
ffinal_xby='D:\大三下\摄影测量实习\程序\单独像对相对定向\西北一相对定向后的坐标.xlsx'
%xlswrite(ffinal_xz,XY_xz)
xlswrite(ffinal_xby,XY_xby)
end
\end{lstlisting}

\begin{lstlisting}[caption=relative\_orientation\_procedure.m]
function relative_orientation_procedure()
file_xz='D:\大三下\摄影测量实习\程序\数字内定向\小组内定向点.xlsx'
file_dzj='D:\大三下\摄影测量实习\程序\数字内定向\点之记内定向点.xlsx'
file_dpj='D:\大三下\摄影测量实习\程序\数字内定向\点评价内定向点.xlsx'
x1y1_xz=xlsread(file_xz,'A:B')
x2y2_xz=xlsread(file_xz,'C:D')
x1y1_dzj=xlsread(file_dzj,'A:B')
x2y2_dzj=xlsread(file_dzj,'C:D')
x1y1_dpj=xlsread(file_dpj,'A:B')
x2y2_dpj=xlsread(file_dpj,'C:D')
f=120
[fai1_t,kapa1_t,fai2_t,omega2_t,kapa2_t]=iterate_for_adjustment(x1y1_xz,x2y2_xz,f)
%[fai1_t,kapa1_t,fai2_t,omega2_t,kapa2_t]=iterate_for_adjustment(x1y1_dzj,x2y2_dzj,f)
% fwrite='D:\大三下\摄影测量实习\程序\单独像对相对定向\相对定向参数.xlsx'
% xlswrite(fwrite,[fai1_t,kapa1_t,fai2_t,omega2_t,kapa2_t])
X1Y1_xz=[]
R1=R(fai1_t,0,kapa1_t)
X2Y2_xz=[]
R2=R(fai2_t,omega2_t,kapa2_t)
X1Y1_dzj=[]
%R1=R(fai1_t,0,kapa1_t) 
X2Y2_dzj=[]
%R2=R(fai2_t,omega2_t,kapa2_t)
X1Y1_dpj=[]
X2Y2_dpj=[]
for i=1:length(x1y1_xz)
    x=x1y1_xz(i,1)
    y=x1y1_xz(i,2)
    [X,Y,Z]=xyz_to_XYZ(R1,x,y,f)
    X1Y1_xz=[X1Y1_xz;[X,Y,Z]]
end
for i=1:length(x2y2_xz)
    x=x2y2_xz(i,1)
    y=x2y2_xz(i,2)
    [X,Y,Z]=xyz_to_XYZ(R2,x,y,f)
    X2Y2_xz=[X2Y2_xz;[X,Y,Z]]
end    
for i=1:length(x1y1_dzj)
    x=x1y1_dzj(i,1)
    y=x1y1_dzj(i,2)
    [X,Y,Z]=xyz_to_XYZ(R1,x,y,f)
    X1Y1_dzj=[X1Y1_dzj;[X,Y,Z]]
end
for i=1:length(x2y2_dzj)
    x=x2y2_dzj(i,1)
    y=x2y2_dzj(i,2)
    [X,Y,Z]=xyz_to_XYZ(R2,x,y,f)
    X2Y2_dzj=[X2Y2_dzj;[X,Y,Z]]
end    
for i=1:length(x1y1_dpj)
    x=x1y1_dpj(i,1)
    y=x1y1_dpj(i,2)
    [X,Y,Z]=xyz_to_XYZ(R1,x,y,f)
    X1Y1_dpj=[X1Y1_dpj;[X,Y,Z]]
end
for i=1:length(x2y2_dpj)
    x=x2y2_dpj(i,1)
    y=x2y2_dpj(i,2)
    [X,Y,Z]=xyz_to_XYZ(R2,x,y,f)
    X2Y2_dpj=[X2Y2_dpj;[X,Y,Z]]
end    
XY_xz=[X1Y1_xz,X2Y2_xz]
XY_dzj=[X1Y1_dzj,X2Y2_dzj]
XY_dpj=[X1Y1_dpj,X2Y2_dpj]
ffinal_xz='D:\大三下\摄影测量实习\程序\单独像对相对定向\小组相对定向后的坐标.xlsx'
ffinal_dzj='D:\大三下\摄影测量实习\程序\单独像对相对定向\点之记相对定向后的坐标.xlsx'
ffinal_dpj='D:\大三下\摄影测量实习\程序\单独像对相对定向\点评价相对定向后的坐标.xlsx'
xlswrite(ffinal_xz,XY_xz)
xlswrite(ffinal_dzj,XY_dzj)
xlswrite(ffinal_dpj,XY_dpj)
end
\end{lstlisting}

\begin{lstlisting}[caption=Ai.m]
function A=Ai(X1,Y1,X2,Y2,Z1,Z2)
a_fai1=-1*X1*Y2/Z1
a_kapa1=X1
a_fai2=X2*Y1/Z1
a_omega2=Z1+Y1*Y2/Z1
a_kapa2=-1*X2
A=[a_fai1,a_kapa1,a_fai2,a_omega2,a_kapa2]
end
\end{lstlisting}

\begin{lstlisting}[caption=R.m]
function R=R(fai,omega,kapa)
Rfai=[cos(fai),0,-sin(fai);0,1,0;sin(fai),0,cos(fai)]
Romega=[1,0,0;0,cos(omega),-sin(omega);0,sin(omega),cos(omega)]
Rkapa=[cos(kapa),-sin(kapa),0;sin(kapa),cos(kapa),0;0,0,1]
R=Rfai*Romega*Rkapa
end
\end{lstlisting}

\begin{lstlisting}[caption=Qi.m]
function q=Qi(f,Y1,Z1,Y2,Z2)
q=f*(Y1/Z1-Y2/Z2)
end
\end{lstlisting}

\begin{lstlisting}[caption=iterate\_for\_adjustment.m]
function [fai1_t,kapa1_t,fai2_t,omega2_t,kapa2_t]=iterate_for_adjustment(x1y1,x2y2,f)
fai1=0
kapa1=0
fai2=0
omega2=0
kapa2=0
n=0
[d_fai1,d_kapa1,d_omega2,d_fai2,d_kapa2]=relative_orientation(x1y1,x2y2,f,fai1,kapa1,fai2,omega2,kapa2)
while d_fai1>=0.00003 | d_kapa1>=0.00003 | d_omega2>=0.00003 | d_fai2>=0.00003 | d_kapa2>=0.00003
    n=n+1
    fai1=fai1+d_fai1
    kapa1=kapa1+d_kapa1
    fai2=fai2+d_fai2
    omega2=omega2+d_omega2
    kapa2=kapa2+d_kapa2
    [d_fai1,d_kapa1,d_omega2,d_fai2,d_kapa2]=relative_orientation(x1y1,x2y2,f,fai1,kapa1,fai2,omega2,kapa2)
end
fai1_t=fai1
kapa1_t=kapa1
fai2_t=fai2
omega2_t=omega2
kapa2_t=kapa2
end    
\end{lstlisting}

\subsection{运行结果}

\begin{figure}[htbp]
\caption{运行结果}
\centering
\includegraphics[width=5cm]{result2.png} \\
\includegraphics[width=5cm]{result3.png}
\end{figure}

\section{前方交会}

\subsection{程序}

\begin{lstlisting}[caption=run\_forward\_intersection.m]
function run_forward_intersection()
fwritexz='D:\大三下\摄影测量实习\程序\前方交会求模型坐标\小组模型坐标.xlsx'
fwritedzj='D:\大三下\摄影测量实习\程序\前方交会求模型坐标\点之记模型坐标.xlsx'
fwritedpj='D:\大三下\摄影测量实习\程序\前方交会求模型坐标\点评价模型坐标.xlsx'
M=(7680*0.2)/(46.08*2)
l=46.08*2
p=0.6
B=l*(1-p)*M
freadxz='D:\大三下\摄影测量实习\程序\单独像对相对定向\小组相对定向后的坐标.xlsx'
freaddzj='D:\大三下\摄影测量实习\程序\单独像对相对定向\点之记相对定向后的坐标.xlsx'
freaddpj='D:\大三下\摄影测量实习\程序\单独像对相对定向\点评价相对定向后的坐标.xlsx'
X1_xz=xlsread(freadxz,'A:A')
X1_dzj=xlsread(freaddzj,'A:A')
X1_dpj=xlsread(freaddpj,'A:A')
Y1_xz=xlsread(freadxz,'B:B')
Y1_dzj=xlsread(freaddzj,'B:B')
Y1_dpj=xlsread(freaddpj,'B:B')
Z1_xz=xlsread(freadxz,'C:C')
Z1_dzj=xlsread(freaddzj,'C:C')
Z1_dpj=xlsread(freaddpj,'C:C')
X2_xz=xlsread(freadxz,'D:D')
X2_dzj=xlsread(freaddzj,'D:D')
X2_dpj=xlsread(freaddpj,'D:D')
Y2_xz=xlsread(freadxz,'E:E')
Y2_dzj=xlsread(freaddzj,'E:E')
Y2_dpj=xlsread(freaddpj,'E:E')
Z2_xz=xlsread(freadxz,'F:F')
Z2_dzj=xlsread(freaddzj,'F:F')
Z2_dpj=xlsread(freaddpj,'F:F')
XYZQ_xz=forward_intersection(B,X1_xz,Y1_xz,Z1_xz,X2_xz,Y2_xz,Z2_xz)
XYZQ_dzj=forward_intersection(B,X1_dzj,Y1_dzj,Z1_dzj,X2_dzj,Y2_dzj,Z2_dzj)
XYZQ_dpj=forward_intersection(B,X1_dpj,Y1_dpj,Z1_dpj,X2_dpj,Y2_dpj,Z2_dpj)
xlswrite(fwritexz,XYZQ_xz)
xlswrite(fwritedzj,XYZQ_dzj)
xlswrite(fwritedpj,XYZQ_dpj)
end 
\end{lstlisting}

\begin{lstlisting}[caption=forward\_intersection.m]
function XYZQ=forward_intesection(B,X1,Y1,Z1,X2,Y2,Z2)
%M为比例尺
%l为像片长度
%p为航向重叠度
Bx=B
By=0
Bz=0
Xs1=0
Ys1=0
Zs1=0
Xs2=Bx
Ys2=0
Zs2=0
N1=(Bx.*Z2-Bz.*X2)./(X1.*Z2-Z1.*X2)
N2=(Bx.*Z1-Bz.*X1)./(X1.*Z2-Z1.*X2)
X=Xs1+N1.*X1
Y=0.5*(Ys1+N1.*Y1+N2.*Y2+Ys2)
Z=Zs1+N1.*Z1
Q=N1.*Y1-N2.*Y2-By
XYZQ=[X,Y,Z,Q]
end
\end{lstlisting}

\begin{lstlisting}[caption=forward\_intersection\_xby.m]
function forward_intersection_xby()
fwritexby='D:\大三下\摄影测量实习\程序\前方交会求模型坐标\西北一模型坐标.xlsx'
M=(7680*0.2)/(46.08*2)
l=46.08*2
p=0.6
B=l*(1-p)*M
freaddlt='D:\大三下\摄影测量实习\程序\单独像对相对定向\西北一相对定向后的坐标.xlsx'
X1_xby=xlsread(freaddlt,'A:A')
Y1_xby=xlsread(freaddlt,'B:B')
Z1_xby=xlsread(freaddlt,'C:C')
X2_xby=xlsread(freaddlt,'D:D')
Y2_xby=xlsread(freaddlt,'E:E')
Z2_xby=xlsread(freaddlt,'F:F')
XYZQ_xby=forward_intersection(B,X1_xby,Y1_xby,Z1_xby,X2_xby,Y2_xby,Z2_xby)
xlswrite(fwritexby,XYZQ_xby)
end
\end{lstlisting}

\begin{lstlisting}[caption=forward\_intersection\_dlt.m]
function forward_intersection_dlt()
fwritedlt='D:\大三下\摄影测量实习\程序\前方交会求模型坐标\大礼堂模型坐标.xlsx'
M=(7680*0.2)/(46.08*2)
l=46.08*2
p=0.6
B=l*(1-p)*M
freaddlt='D:\大三下\摄影测量实习\程序\单独像对相对定向\大礼堂相对定向后的坐标.xlsx'
X1_dlt=xlsread(freaddlt,'A:A')
Y1_dlt=xlsread(freaddlt,'B:B')
Z1_dlt=xlsread(freaddlt,'C:C')
X2_dlt=xlsread(freaddlt,'D:D')
Y2_dlt=xlsread(freaddlt,'E:E')
Z2_dlt=xlsread(freaddlt,'F:F')
XYZQ_dlt=forward_intersection(B,X1_dlt,Y1_dlt,Z1_dlt,X2_dlt,Y2_dlt,Z2_dlt)
xlswrite(fwritedlt,XYZQ_dlt)
end
\end{lstlisting}

\subsection{运行结果}

运行结果见Excel文档。

\section{绝对定向(布尔沙模型)}

\subsection{程序}

\begin{lstlisting}[caption=absolute\_orientation\_dlt.m]
function absolute_orientation_dlt()
fmodeldlt='D:\大三下\摄影测量实习\程序\前方交会求模型坐标\大礼堂模型坐标.xlsx'
fmodeldzj='D:\大三下\摄影测量实习\程序\前方交会求模型坐标\点之记模型坐标.xlsx'
XYZ_modeldlt=xlsread(fmodeldlt,'A:C')
XYZ_modeldzj=xlsread(fmodeldzj,'A:C')
fcontrol='D:\大三下\摄影测量实习\程序\点之记 - 副本.xls'
XYZ_control=get_XYZ_control(fcontrol)
[dXt,dYt,dZt,dlamdat,epsi_xt,epsi_yt,epsi_zt]=iterate_absolute_orientation(XYZ_modeldzj,XYZ_control,0,0,0,0,0,0,0)
fwritedlt='D:\大三下\摄影测量实习\程序\绝对定向\大礼堂绝对定向结果.xls'
fwritedzj='D:\大三下\摄影测量实习\程序\绝对定向\点之记绝对定向结果.xls'
XYZ_transtdlt=trans_points(XYZ_modeldlt,dXt,dYt,dZt,dlamdat,epsi_xt,epsi_yt,epsi_zt)
XYZ_transtdzj=trans_points(XYZ_modeldzj,dXt,dYt,dZt,dlamdat,epsi_xt,epsi_yt,epsi_zt)
xlswrite(fwritedlt,XYZ_transtdlt)
%xlswrite(fwritedzj,[XYZ_transtdzj ,XYZ_control])
end
\end{lstlisting}

\begin{lstlisting}[caption=absolute\_orientation\_xby.m]
function absolute_orientation_xby()
fmodelxby='D:\大三下\摄影测量实习\程序\前方交会求模型坐标\西北一模型坐标.xlsx'
fmodeldzj='D:\大三下\摄影测量实习\程序\前方交会求模型坐标\点之记模型坐标.xls'
XYZ_modelxby=xlsread(fmodelxby,'A:C')
XYZ_modeldzj=xlsread(fmodeldzj,'A:C')
fcontrol='D:\大三下\摄影测量实习\程序\点之记 - 副本.xls'
XYZ_control=get_XYZ_control(fcontrol)
[dXt,dYt,dZt,dlamdat,epsi_xt,epsi_yt,epsi_zt]=iterate_absolute_orientation(XYZ_modeldzj,XYZ_control,0,0,0,0,0,0,0)
fwritexby='D:\大三下\摄影测量实习\程序\绝对定向\西北一绝对定向结果.xls'
fwritedzj='D:\大三下\摄影测量实习\程序\绝对定向\点之记绝对定向结果.xls'
XYZ_transtxby=trans_points(XYZ_modelxby,dXt,dYt,dZt,dlamdat,epsi_xt,epsi_yt,epsi_zt)
XYZ_transtdzj=trans_points(XYZ_modeldzj,dXt,dYt,dZt,dlamdat,epsi_xt,epsi_yt,epsi_zt)
xlswrite(fwritexby,XYZ_transtxby)
%xlswrite(fwritedzj,[XYZ_transtdzj ,XYZ_control])
end
\end{lstlisting}

\begin{lstlisting}[caption=Ai\_a.m]
function Aia=Ai_a(xyz_model)
x=xyz_model(1)
y=xyz_model(2)
z=xyz_model(3)
Aia=[1,0,0,x,0,-z,y;0,1,0,y,z,0,-x;0,0,1,z,-y,x,0]
end
\end{lstlisting}

\begin{lstlisting}[caption=Li\_a.m]
function Lia=Li_a(xyz_model,xyz_control,dlamda,Repsi,dX,dY,dZ)
Lia=xyz_control-(1+dlamda)*Repsi*xyz_model-[dX;dY;dZ]
end
\end{lstlisting}

\begin{lstlisting}[caption=R\_epsi.m]
function Repsi=R_epsi(epsi_x,epsi_y,epsi_z)
Repsi_z=[cos(epsi_z),sin(epsi_z),0;-sin(epsi_z),cos(epsi_z),0;0,0,1]
Repsi_y=[cos(epsi_y),0,-sin(epsi_y);0,1,0;sin(epsi_y),0,cos(epsi_y)]
Repsi_x=[1,0,0;0,cos(epsi_x),sin(epsi_x);0,-sin(epsi_x),cos(epsi_x)]
Repsi=Repsi_z*Repsi_y*Repsi_x
end
\end{lstlisting}

\begin{lstlisting}[caption=get\_XYZ\_control.m]
function XYZ_control=get_XYZ_control(fcontrol)
X_control=xlsread(fcontrol,'F3:F47')
Y_control=xlsread(fcontrol,'G3:G47')
Z_control=xlsread(fcontrol,'H3:H47')
XYZ_control=[X_control,Y_control,Z_control]
end
\end{lstlisting}

\begin{lstlisting}[caption=get\_XYZ\_model.m]
function XYZ_model=get_XYZ_model(fmodel)
XYZ_model=xlsread(fmodel,'A:C')
end
\end{lstlisting}

\begin{lstlisting}[caption=run\_absolute\_orientation.m]
function run_absolute_orientation()
fmodelxz='D:\大三下\摄影测量实习\程序\前方交会求模型坐标\小组模型坐标.xlsx'
fmodeldzj='D:\大三下\摄影测量实习\程序\前方交会求模型坐标\点之记模型坐标.xlsx'
fmodeldpj='D:\大三下\摄影测量实习\程序\前方交会求模型坐标\点评价模型坐标.xlsx'
XYZ_modelxz=xlsread(fmodelxz,'A:C')
XYZ_modeldzj=xlsread(fmodeldzj,'A:C')
XYZ_modeldpj=xlsread(fmodeldpj,'A:C')
fcontrol='D:\大三下\摄影测量实习\程序\点之记 - 副本.xls'
fcontrol_dpj='D:\大三下\摄影测量实习\程序\点评价.xls'
XYZ_control=get_XYZ_control(fcontrol)
XYZ_control_dpj=get_XYZ_control(fcontrol_dpj)
[dXt,dYt,dZt,dlamdat,epsi_xt,epsi_yt,epsi_zt]=iterate_absolute_orientation
(XYZ_modeldzj,XYZ_control,0,0,0,0,0,0,0)
fwritexz='D:\大三下\摄影测量实习\程序\绝对定向\小组绝对定向结果.xlsx'
fwritedzj='D:\大三下\摄影测量实习\程序\绝对定向\点之记绝对定向结果.xlsx'
fwritedpj='D:\大三下\摄影测量实习\程序\绝对定向\点评价绝对定向结果.xlsx'
XYZ_transtxz=trans_points(XYZ_modelxz,dXt,dYt,dZt,dlamdat,epsi_xt,epsi_yt,epsi_zt)
XYZ_transtdzj=trans_points(XYZ_modeldzj,dXt,dYt,dZt,dlamdat,epsi_xt,epsi_yt,epsi_zt)
XYZ_transtdpj=trans_points(XYZ_modeldpj,dXt,dYt,dZt,dlamdat,epsi_xt,epsi_yt,epsi_zt)
xlswrite(fwritexz,XYZ_transtxz)
xlswrite(fwritedzj,[XYZ_transtdzj ,XYZ_control])
xlswrite(fwritedpj,[XYZ_transtdpj ,XYZ_control_dpj])
end
\end{lstlisting}

\begin{lstlisting}[caption=iterate\_absolute\_orientation.m]
function [dXt,dYt,dZt,dlamdat,epsi_xt,epsi_yt,epsi_zt]=iterate_absolute_orientation(XYZ_model,XYZ_control,dX0,dY0,dZ0,dlamda0,epsi_x0,epsi_y0,epsi_z0)
dX=dX0
dY=dY0
dZ=dZ0
dlamda=dlamda0
epsi_x=epsi_x0
epsi_y=epsi_y0
epsi_z=epsi_z0
limit=10^(-6)
[d_dX,d_dY,d_dZ,d_dlamda,d_epsi_x,d_epsi_y,d_epsi_z]=absolute_orientation
(XYZ_model,XYZ_control,dlamda,epsi_x,epsi_y,epsi_z,dX,dY,dZ)
while abs(d_dX)>=0.1 | abs(d_dY)>=0.1 | abs(d_dZ)>=0.1 | abs(d_dlamda)>=limit | abs(d_epsi_x)>=limit | abs(d_epsi_y)>=limit | abs(d_epsi_z)>=limit
    dX=dX+d_dX
    dY=dY+d_dY
    dZ=dZ+d_dZ
    dlamda=dlamda+d_dlamda
    epsi_x=epsi_x+d_epsi_x
    epsi_y=epsi_y+d_epsi_y
    epsi_z=epsi_z+d_epsi_z
    [d_dX,d_dY,d_dZ,d_dlamda,d_epsi_x,d_epsi_y,d_epsi_z]=absolute_orientation
    (XYZ_model,XYZ_control,dlamda,epsi_x,epsi_y,epsi_z,dX,dY,dZ)
end
dXt=dX+d_dX
dYt=dY+d_dY
dZt=dZ+d_dZ
dlamdat=dlamda+d_dlamda
epsi_xt=epsi_x+d_epsi_x
epsi_yt=epsi_y+d_epsi_y
epsi_zt=epsi_z+d_epsi_z
end     
\end{lstlisting}

\begin{lstlisting}[caption=absolute\_orientation.m]
function [d_dX,d_dY,d_dZ,d_dlamda,d_epsi_x,d_epsi_y,d_epsi_z]=absolute_orientation
(XYZ_model,XYZ_control,dlamda,epsi_x,epsi_y,epsi_z,dX,dY,dZ)
A=[]
L=[]
Repsi=R_epsi(epsi_x,epsi_y,epsi_z)
for i=1:length(XYZ_model)
    xyz_model=XYZ_model(i,:)
    xyz_model=xyz_model'
    xyz_control=XYZ_control(i,:)
    xyz_control=xyz_control'
    Lia=Li_a(xyz_model,xyz_control,dlamda,Repsi,dX,dY,dZ)
    Aia=Ai_a(xyz_model)
    L=[L;Lia]
    A=[A;Aia]
end
d_para=inv(A'*A)*A'*L
d_dX=d_para(1)
d_dY=d_para(2)
d_dZ=d_para(3)
d_dlamda=d_para(4)
d_epsi_x=d_para(5)
d_epsi_y=d_para(6)
d_epsi_z=d_para(7)
end
\end{lstlisting}

\begin{lstlisting}[caption=trans\_points.m]
function XYZ_transt=trans_points
(XYZ_trans0,dXt,dYt,dZt,dlamdat,epsi_xt,epsi_yt,epsi_zt)
Repsi=R_epsi(epsi_xt,epsi_yt,epsi_zt)
XYZ_transt=[]
for i=1:length(XYZ_trans0)
    xyz_trans=XYZ_trans0(i,:)
    xyz_trans=xyz_trans'
    xyz_transt=[dXt;dYt;dZt]+(1+dlamdat)*Repsi*xyz_trans
    xyz_transt=xyz_transt'
    XYZ_transt=[XYZ_transt;xyz_transt]
end
end
\end{lstlisting}

此外,还有一个lsp展绘的程序:

\begin{lstlisting}[language=Lisp,caption=cadzd.lsp]
(defun c:cadzh()
(setq file (getfiled "Select a file please" "D:/大三下/摄影测量实习/程序/CAD展绘/大礼堂绝对定向结果.txt" "txt" 0))
(setq f (open file "r"))
(setq line (read-line f))
(command "pline")
(while (/= line nil)
    (setq seq1 (vl-string-search "\t" line))
    (setq l1 (substr line (+ 2 seq1)))
    (setq seq2 (vl-string-search "\t" l1))
    (setq X (atof (substr line 1 seq1)))
    (setq Y (atof (substr l1 1 seq2)))
    (setq Z (atof (substr l1 (+ seq2 2))))
    (command (list X Y))
    (setq line (read-line f))
    )
(close  f)
(command "c")
(princ)
)
\end{lstlisting}

\subsection{运行结果}

\begin{figure}[htbp]
\caption{运行结果}
\centering
\includegraphics[width=5cm]{result4.png}
\end{figure}

\section{调绘成果}

调绘成果详见报告中,下面是两张截图。

\begin{figure}[htbp]
\centering
\caption{调绘图概览}
\includegraphics[width=0.7\textwidth]{diaohui.jpg}
\end{figure}

\begin{figure}[htbp]
\centering
\caption{调绘图细节}
\includegraphics[width=0.7\textwidth]{diaohui1.jpg}
\end{figure}

\section{同名点}

\begin{table}[htbp]
    \centering
    \begin{tabular}{lrrrr}
        \toprule
        & \multicolumn{2}{c}{63} & \multicolumn{2}{c}{64} \\ \hline
        点号 & \multicolumn{1}{c}{i} & \multicolumn{1}{c}{j} & \multicolumn{1}{c}{i} & \multicolumn{1}{c}{j} \\ \midrule
        8  & 6592 & 9831 & 3414 & 9487 \\
        9  & 5442 & 9906 & 2229 & 9560 \\
        10 & 5225 & 9518 & 2041 & 9176 \\
        11 & 4613 & 9717 & 1318 & 9368 \\
        12 & 6996 & 9145 & 3812 & 8813 \\
        13 & 6184 & 8914 & 2901 & 8578 \\
        14 & 5799 & 8440 & 2596 & 8110 \\ \bottomrule
    \end{tabular}
    \end{table}

\section{摄影测量实习感悟}

\subsubsection{一}
在这为期半学期的摄影测量实习中,我们既体验了外业测量,又进行了内业数据处理,对摄影测量有了一个完整的了解,也对上学期学习的课本知识有了更加深刻的认识。

\subsubsection{二}
先说外业测量吧。最直观深刻的感受就是,如果不理解绝对定向的原理,连测控制点的时候都会犯错误。之前我们普通测量实习展点,对第三维的信息是没有要求的,所以我们并不太顾及高程测量。这次控制测量刚开始的阶段,我们只一心只想着相片是平面的,只有二维信息,也差点就不把高程测量当回事。这点也着实给我们提了个醒——在课程的学习中,我们可能因为一些原因,把更多的注意力放在了实现的公式上而没有太注意内定向、相对定向、前方交会、绝对定向之间的关联,这就给我们的应用带来了障碍,以至于忽略了像点到模型点的二维到三维转换过程。
另外,控制测量的一大难度在于检查的过程。我们测区的大部分普通测量控制点都集中在同一个区域,如果不大量布设支导线的话,测量的摄影测量控制点都将是共线的。这样,再算上检查设站,测一个点的工作量就比较大了。

\subsubsection{三}
再者说到的就是内业工作了。上学期学的时候,我们就知道书上的公式推导是存在一些问题的,这次实习终于看到了较完整合理的推导。在编程的时候,我的绝对定向程序一开始怎么都不收敛,反复检查了好几次代码,都没发现有什么问题,为此还苦闷了一小段时间。后来和其他同学交流后才发现,我在做内定向的时候,框标的像素坐标系和PS中不对应,因此用算出来的内定向参数进行内定向得到的像平面坐标是不正确的。此外,我还被分配到了小组的CAD展绘工作,小组的另一位同学负责找同名点,我负责定向并勾画。我们选择的大礼堂,一共有70多对同名点,不得不感叹找同名点真的是一个浩大的工程。做完前方交会之后,发现有三个点的上下视差过大,我们便进行了调整,重新找点。我用Lisp写了一个简单的展点程序将绝对定向后的地面摄测坐标展绘出来,发现有两个点偏的有些厉害,其它点的位置都很好,于是,我们再次更正这两对同名点,再定向展绘。由此重复了好几个循环,我们终于做出了一副比较满意的展绘图。CAD展绘的过程让我对整个摄影测量实习有了更完整、具体的认识。

\subsubsection{四}
本次实习中,我们还体验了运用kinect进行场景三维建模。我们小组一共进行了两次模型数据采集,第一次是对一个同学进行采集,第二次是对桌面进行采集。第一次采集的效果很不好,因为没有经验加上太过于着急,采集数据的同学没能将人像采集完整。第二次桌面采集的效果相对而言还是比较不错的,采集后,我们还量测了电脑宽度等信息,与模型进行对比。

\subsubsection{五}
总之,这次实习让我对上学期的摄影测量知识有了更加深刻的理解。实践着实促进了我们对知识的学习。另外,我们摄影测量小组的同学们一起经历了暑期普通测量实习、控制测量,现在配合得也越来越默契了。









\chapter{毛瑞丰个人实习报告}


\section{实习经历}


\subsubsection{一}
在本次摄影测量与遥感实习中,我将上学期学习的摄影测量知识点进行了实际的巩固与运用。与小组其他成员互帮互助,通力完成了航片判读、调绘到外业像控点的联测、像片同名点的量取,以及内定向元素、相对定向元素、模型坐标和绝对定向参数的计算等内容。基于共同的成果完成了大礼堂、西北一地物坐标的CAD成图。

实习中我们将课本上枯燥的公式与航摄影像相结合运用,使理论不再是空中阁楼,而与实际紧紧地连接起来。我们更加相信:实践是检验真理的唯一标准。没有通过本次实习的锻炼,没有通过程序的编写,摄影测量基础的运用对我们来说将是一个棘手的问题。在上学期,我们完成了前方—后方交会法的编程,让我们对此方法理解较深,这次的实习我们动手完成了相对-绝对定向法,对摄影测量整个过程流程有了更加深刻的理解。

\subsubsection{二}
本次实习,我们发现了许多自身的不足之处:在外业实习中,以房屋角点为测量点时,如果过于追求测量的准确性,将准心对准角点,反而很有可能对到角点之后的地方去。在更换过若干个测站点,我向现实妥协,退而求其次,试着将准心对准房屋角点偏下的地方,最终才把将近五十米的误差消除了。此次问题并未拖延我们过多的外业时间,迄今为止我们一同经过多次的实验,面对突发问题和BUG更加成熟老练。

\subsubsection{三}
第一次接触了Kinect Fusion建模,补充了我们对摄影测量的认识,Kinect的专业令我难以想象这最开始居然是孩童的玩具。从玩具到产品,我看到的是开发者和使用者的努力与协作。Kinect的实验给我们不仅仅是科技的新鲜感,而且还激发了我们探索知识海洋的好奇心。

\section{程序集}

\subsection{内定向}

\begin{lstlisting}[caption=Interior\_Orientation.m文件]
mark=textread(' C:\Users\毛瑞丰\Desktop\摄影测量实验\camera.use');
pixel=[7679 13823;0 13823 ;0 0;7679 0];
y=reshape(mark',[8 1]);
X=zeros(8,6); 
for i=1:4,
    X(2*i-1:2*i,:)=[1,pixel(i,1),pixel(i,2),0,0,0;
                    0,0,0,1,pixel(i,1),pixel(i,2)];
end
N=inv(X'*X);
beta=N*X'*y
csvwrite('C:\Users\毛瑞丰\Desktop\摄影测量实习\毛瑞丰\ Interior_Orientation.csv',beta);
r=size(y,1)-6;
e=y-X*beta;
sigma=sqrt(norm(e)/r);
cc=xlsread('C:\Users\毛瑞丰\Desktop\摄影测量实验\点之记.xlsx',1,'B3:E13');
retVal=zeros(size(cc,1),4);
for i=1:size(cc,1),
    line=[1 cc(i,1) cc(i,2) 0 0 0;
    0 0 0 1 cc(i,1) cc(i,2);
    1 cc(i,3) cc(i,4) 0 0 0;
    0 0 0 1 cc(i,3) cc(i,4)]*beta;
    retVal(i,:)=line';
end
xlswrite('C:\Users\毛瑞丰\Desktop\摄影测量实验\框标点.xlsx',retVal);
\end{lstlisting}

模型:
\begin{equation}
\begin{bmatrix}
x \\ y
\end{bmatrix}
=\begin{bmatrix}
h_1 & h_2 \\
k_1 & k_2 
\end{bmatrix}
\begin{bmatrix}
i \\ j
\end{bmatrix}
+\begin{bmatrix}
h_0 \\ k_0
\end{bmatrix}
\end{equation}

得出的内定向参数如下:
\begin{equation}
\begin{array}{lll}
h_0=-46.08 & h_1=0.012 & h_2=0 \\
k_0=82.944 & k_1=0 & k_2=-0.012
\end{array}
\end{equation}

\subsection{相对定向}

\begin{lstlisting}[caption=Relative\_Orientation.m文件]
syms theta;
syms phi1 kappa1;
syms phi2 omega2 kappa2;
f=120; %mm
x0=0; %principal point shift
y0=0;
Rx=[1 0 0;0 cos(theta) -sin(theta);0 sin(theta) cos(theta)];
Ry=[cos(theta) 0 -sin(theta); 0 1 0; sin(theta) 0 cos(theta)];
Rz=[cos(theta) -sin(theta) 0; sin(theta) cos(theta) 0;0 0 1];
R1=subs(Ry,phi1)*subs(Rz,kappa1);
R2=subs(Ry,phi2)*subs(Rx,omega2)*subs(Rz,kappa2);
mat=xlsread('C:\Users\毛瑞丰\Desktop\摄影测量实验\框标点.xlsx',1,'A1:D11');
mat1=mat(:,1:2);
mat2=mat(:,3:4);
mat1=[mat1,-f*ones(size(mat1,1),1)]';
mat2=[mat2,-f*ones(size(mat2,1),1)]';
mat11=R1*mat1;
mat22=R2*mat2;
r=(mat11(2,:).*mat22(3,:)-mat11(3,:).*mat22(2,:)).';
x=[0,0,0,0,0]';
v=GaussNewton(r,x,2e-7)
csvwrite('C:\Users\毛瑞丰\Desktop\摄影测量实验\毛瑞丰\ Relative_Orientation.csv',v);
\end{lstlisting}


\begin{lstlisting}[caption=Absolute\_Orientation.m文件]
function [retVal]=GaussNewton(f,x,error)
syms phi1 kappa1 phi2 omega2 kappa2;
v=[phi1 kappa1 phi2 omega2 kappa2];
j=jacobian(f,v);
J=eval(subs(j,v,x'));
F=eval(subs(f,v,x'));
k=0;
d=1;
while norm(d)>error,
    d=-inv(J'*J)*J'*F;
    x=x+d;
    J=eval(subs(j,v,x'));
    F=eval(subs(f,v,x'))
    k=k+1
    disp(norm(J'*F));
end
retVal=x;
\end{lstlisting}

得出的相对定向参数如下:
\begin{equation}
\begin{array}{lll}
\phi_1=-0.012455 & & \kappa_1=0.028415 \\
\phi_2=0.013341 & \omega_2=-0.022506 & \kappa_2=0.036366
\end{array}
\end{equation}

\subsection{绝对定向}

\begin{lstlisting}[caption=Absolute\_Orientation.m文件]
allpoints=xlsread('C:\Users\毛瑞丰\Desktop\摄影测量实验\inspect副本.xlsx');
xyz=allpoints(:,6:8);
XYZ=allpoints(:,9:11);
XYZmean=mean(XYZ);
XYZ=XYZ-ones(size(XYZ))*diag(mean(XYZ));
xyz=xyz-ones(size(xyz))*diag(mean(xyz));
syms theta;
Rx=[1 0 0;0 cos(theta) -sin(theta);0 sin(theta) cos(theta)];
Ry=[cos(theta) 0 -sin(theta);0 1 0;sin(theta) 0 cos(theta)];
Rz=[cos(theta) -sin(theta) 0;sin(theta) cos(theta) 0;0 0 1];
syms phi omega kappa lambda dx dy dz;
R=subs(Ry,phi)*subs(Rx,omega)*subs(Rz,kappa);
groundStretch=reshape(XYZ',size(XYZ,1)*size(XYZ,2),1);
rotateModel=lambda*R*xyz';
r=rotateModel+diag([dx dy dz])*ones(size(rotateModel));
r=reshape(r,size(r,1)*size(r,2),1);
r=r-groundStretch;
x=[0 0 0 1 0 0 0]';
v=GaussNewton2(r,x,2e-7)
csvwrite('C:\Users\毛瑞丰\Desktop\摄影测量实验\毛瑞丰\Absolute_Orientation.csv',[v;XYZmean']);
\end{lstlisting}


\begin{lstlisting}[caption=GaussNewton2.m文件]
function [retVal]=GaussNewton2(f,x,error)
syms phi omega kappa lambda dx dy dz;
v=[phi omega kappa lambda dx dy dz];
j=jacobian(f,v);
J=eval(subs(j,v,x'));
F=eval(subs(f,v,x'));
k=0;
while norm(J'*F)>error,
    d=-inv(J'*J)*J'*F;
    x=x+d
    J=eval(subs(j,v,x'));
    F=eval(subs(f,v,x'));
    k=k+1
    disp(norm(J'*F));
end
retVal=x;
\end{lstlisting}

得出的绝对定向参数如下:
\begin{equation}
\begin{array}{lll}
\phi=0.011264 & \omega=0.005638 & \kappa=0.0053766 \\
& \lambda=1.1318 & \\
dx=0 & dy=0 & dz=0  
\end{array}
\end{equation}

\section{调绘图件}

\begin{figure}[htbp]
\centering
\caption{调绘图概览}
\includegraphics[width=0.9\textwidth]{diaohui.jpg}
\end{figure}

\begin{figure}[htbp]
\centering
\caption{调绘图细节}
\includegraphics[width=0.9\textwidth]{diaohui1.jpg}
\end{figure}

\section{同名点}

\begin{table}[htbp]
    \centering
    \begin{tabular}{lrrrr}
        \toprule
        & \multicolumn{2}{c}{63} & \multicolumn{2}{c}{64} \\ \hline
        点号 & \multicolumn{1}{c}{i} & \multicolumn{1}{c}{j} & \multicolumn{1}{c}{i} & \multicolumn{1}{c}{j} \\ \midrule
        15 & 5220 & 13116 & 2015 & 12700 \\
        16 & 7133 & 12458 & 3893 & 12056 \\
        17 & 6480 & 10899 & 3268 & 10534 \\
        18 & 5463 & 9850 & 2249 & 9501 \\
        19 & 5545 & 10612 & 2339 & 10256 \\ \bottomrule
    \end{tabular}
    \end{table}
\chapter{贾锐个人实习报告}

\section{实习内容}

\subsection{外业测量} 

\paragraph{基本过程}

在这一阶段,我们小组根据老师所划测区进行了量测工作。体会了从影像上面选取点、之后到实地进行量测的过程。在量测过程中,我们利用支导线布点的方式提高了量测精度。除此之外。我们采用在两个站观测同一点,来进行所量测点的精度检校。

\paragraph{完成方式} 小组一同出工,各自测取各自点。

\subsection{像点坐标量测} 

\paragraph{基本过程}

这部分实习中,我们小组根据我们的测区选择了63、64两幅影像,六个人分工各自量取6对同名点,并且进行点之记。在这个过程中,我熟悉了PhotoShop的操作。

\paragraph{完成方式} 
完成所属部分6对同名点像素坐标选定以及点之记。

\subsection{内定向--相对定向} 
\paragraph{基本过程}在这一实习过程,根据老师所给的相机camera文件,我首先求取了内定向参数。之后利用内定向参数开始编写相对定向程序。首先根据老师所给公式进行编写,之后发现迭代次数很高,在后续课程中,根据与同学商讨、自行修正对于程序进行了完善,对于所用数据源像素点进行了取舍。最终迭代了4次完成了相对定向,求得了相对定向元素。
\paragraph{完成方式}独立完成。
\paragraph{数据源}
内定向结果,小组内经过剔除之后的同名点以及其他小组的一些点,总共11对。

\subsection{Kinect Fusion系统建模} 

\subsubsection{基本原理}
根据实验中的直观感受及课外查阅,我了解了Kinecr Fusion的操作原理。

KinectFusion使用户可以手持Kinect移动,仅用Kinect的深度信息来追踪传感器的3D姿态,实时重建室内场景的详细3D模型。

Kinect运用结构良好的光学技术形成实时的实物场景的离散测量点深度图。这些测量值可以投影变换到一系列离散3D点(或点云)。虽然与别的可商用的深度摄像机所提供的深度数据相比有优势,Kinect的噪声也会引起深度测量值波动,深度图包含没有读到的“空洞”。

系统实时的从移动着的Kinect摄像头中获得深度数据,实时创建一个高质量,几何精确的3D模型。用户手持标准kinect,可在室内任意移动,并在几秒内重建出实物场景的3D模型。系统连续跟踪标定摄像头姿势的六个自由度,把场景的不同新视角融进一个global surface-based展示。一个新的GPU流水线使精确的摄像头跟踪和实时交互速度的表面重建成为可能。随着用户的移动,新摄入的物理场景视角都被融入同一模型中,因此,重建的模型就随着新的测量值的加入而不断完善细节。孔洞慢慢被填平,模型也逐渐趋于完整。

\subsubsection{算法流程}
\begin{enumerate}
\item 深度数据处理,是将传感器原始的深度数据转换成3D点云,得到点云中顶点的3维坐标和法向量; 
\item 相机跟踪:将当前帧3D点云和由现有模型生成的预测的3D点云进行ICP匹配,计算得到当前帧相机的位姿 
\item 点云融合:根据所计算出的当前相机位姿,使用TSDF点云融合算法将当前帧的3D点云融合到现有模型中。 
\item 场景渲染:是使用光线跟踪的方法,根据现有模型和当前相机位姿预测出当前相机观察到的环境点云,一方面用于反馈给用户,另一方面提供给b进行ICP匹配。 
\end{enumerate}

\subsection{前方交会--绝对定向}
\paragraph{基本过程}
利用前方交会的方式求取出像点的模型坐标,算出各点上下视差,剔除视差大的点,选取小点进行实验。利用剩余点的模型坐标以及地面物方坐标,获得物方坐标系与模型坐标系的七个转化参数。完成绝对定向。最终利用11对点进行了3次迭代之后求取了转换参数。
\paragraph{完成方式} 
独立完成程序,在之后剔点时参考了小组最终选择的点。
\paragraph{数据源}
全班提出后剩余的11对点。

\subsection{点位精度的评定}

根据求出的绝对定向坐标,和真实测量出的地面坐标进行比较,求出了每个点XYZ的点位误差。以此也可以反应绝对定向是否成功。\\
我求得的结果为: X的中误差=0.124881, Y的中误差=0.215042, Z的中误差=0.239378,可以看到,精度在接受范围。证明绝对定向成功。
\subsection{调绘} 
\subsubsection{调绘要求}
在调绘之前,我查阅了调绘中需要注意的地方:

\begin{itemize}
\item 地物地貌的调绘要连续进行,避免调绘不连贯和遗漏。
\item 地理名称注记过密时,可适当取舍。
\item 调绘工作应按照国家标准的地形图图式进行,说明性质的注记应采用“简注表”,不得任意命名。
\item 调绘要按照实地情况严格进行,不得伪造、篡改。
\end{itemize}

\subsubsection{调绘过程}
在进行调绘前我先拍摄了同济大学平面图,之后对照平面图对于全学校进行了走访,确定了每幢楼的层高,大致估算了路宽,确定了每幢楼的材质以及用途。

\section{实习中所遇问题及解决}
\subsection{问题及解决}
  \begin{itemize}
\item 问题:实地量测时我的点在不同站间误差大;\\
      解决:之后进行了二次测量。
\item 问题:相对定向迭代次数很多,起初要进行十次以上;   \\
       解决:对于公式进行修正、对于所用同名点进行选取。最终选择35对。
\item 进行前方交会时,发现z坐标算出为-2000,以为算错;  \\
       解决:经过思考以及与我组同学陈晨讨论,明白这是正确的。
\item 问题: 绝对定向无法收敛;\\
      解决: 起初只选择了4对q最小的点进行绝对定向,但是最终迭代1756次不能收敛。 之后意识到需要增加点数,再次重新对于全班的点进行选取,选择了11对点。迭代3次完成。与小组其他成员结果相近。
\item 问题:调绘时候对于地点不能精确记录; \\
      解决:学到了更有条理的利用“草图”、标号代替文字 进行记录的方式。
\end{itemize}

\section{心得体会}

半个学期的时间一晃而过,摄影测量实习也进入了尾声。每一次的课程实习都会让我学到很多。对于书本知识的巩固、对于所学到摄影测量内容的进一步理解、实习过程中同学友谊的增进、实习过程中所遇问题的或独立或团体解决…都让我印象深刻受益匪浅。

特别是,学过摄影测量学后,对摄影测量只是一种理论的粗浅认识,局限于书本公式,而并不知道如何去运用它,也不知道要在什么场合使用。实习使我对于传统摄影测量的基本流程有了清晰的认识和切身的体会。对于利用相对定向-绝对定向求取地面点坐标的方法有了更深的体会。在编写程序时,从一开始的完全找不到头绪、看不懂老师的实验指导书,到后来看懂了指导书却在实际编程中遇到种种问题使得进度一直难以进行。特别是在进行点的剔除的时候,一开始找不到合适的方法,只是在随机剔除,经常不能收敛,以至于自觉是计算出现了问题。使得进度在很长一段时间不能推进。通过这个经历,也锻炼了我发现问题、解决问题、不随意放弃的学习能力。之后在进行前方交会程序的编写时,得到了-2000的Z坐标,又一度认为是程序出错。反复大量排查之后,经过与小组同学的讨论以及课上老师的讲解,意识到这个结果是正确的,可以继续进行。在进行绝对定向程序编写时,也遇到了一开始不能收敛的问题,发现还是自己选择的点有问题,经过对于点的改正后终于实现了结果。

通过上述亲身经历,我意识到点的质量对于我们实际进行摄影测量内业工作的重要程度。从而使得我将在未来的外业测量中各加认真,让我明白了老师们经常说的“控制点很珍贵”的真正含义。

除此之外,老师还为我们安排了关于Kinet的三维建模实习,扩宽了我们的视野,让我们不仅对传统摄影测量方法熟知,对于新兴技术也有了解。出于兴趣,课后还自行查阅了Kinect Fusion算法,拓宽了知识面。

在进行同名点的确认过程中,由于自身感觉人工量点的方式效率很低并且精度真的不高。我对与目前流行的同名点匹配方式进行了查阅,发现利用以下几种方式有利于提高效率、精度:
\begin{itemize}
\item 利用matlab进行配准。matlab中有相关的包。
\item 可以使用opencv中的开源库
\end{itemize}

另外,调绘的过程使我对于学校更加熟悉,来到同济大学已经第三年,这次调绘我才发现我对于学校其实知之甚少,很多建筑、道路完全没有过了解。是这次调绘给了我这个机会。我发现了“同济大学后勤处”所在地,发现了三好坞餐厅。。。在利用Photoshop进行标注的时候,我对于这个软件有了更加深的了解。

在小组的配合上也增进了同学友谊,在相处过程中有摩擦、也有思想的碰撞。私以为这样的经历提高了我的包容性。程序上遇到了问题,可以请教组内同学,知识上有不理解的地方,也有同学帮我解答。大家齐心协作又各自分工,各自高质量独立完成自己的任务。

在老师安排的校友讲座上,使得我对于摄影测量在实际的生产中目前的应用情况以及方法有了清晰的认识。实现了课本--实习--生产的接轨。听了校友讲述的关于深度学习和神经网络的人工智能方面的新技术后,出于好奇我进行了粗浅查阅,经过一段时间的摸索我发现尽管有了新的进行摄影测量的方式,但是新技术仍然存在很多局限。它不能适应多变的、大范围的环境,它需要大量的控制点,它易受到气候影响等。这就使得在例如火星探测这样的特殊场合,不能进行使用。在很特殊的场合,在新技术不能使用的时候,传统摄影测量的方法依旧值得运用。并且在很多新技术的内核中,都有基础知识的充斥,我们作为优秀的大学生,应当做到“知其然,并知其所以然”。踏踏实实,一步一个脚印,把步子走稳走实。

最后,特别感谢老师的认真指导,因为您在班上公开解答了很多我遇到的问题,才使得我的实验可以进行下来。您让我们有了一个很踏实的实习经历,这两个月的时间让我收获颇丰,相信将对我今后的学习生活大有裨益。

\section{程序集}
\subsection{内定向}

\subsubsection{内定向程序}


\begin{lstlisting}[caption=oriention.m]
function orientation()
format long
[a1,a2]=textread('camera.txt','%n%n');
a=[a1 a2];
theoreticalpositions = a(2:2:end,:);%框标的理论位置
PC = a(1:2:end,:);%框标像素坐标
L=theoreticalpositions';
L=L(:);
B=zeros(8,6);
for i=1:2:8
B(i,1)=1;
end
for i=2:2:8
    B(i,4)=1;
end
for i=1:2:8
    B(i,2)=PC(((i+1)/2),1);
end
for i=1:2:7
 B(i,3)=PC(((i+1)/2),2);
end
for i=2:2:8
    B(i,5)=PC((i/2),1);
end
for i=2:2:8
    B(i,6)=PC((i/2),2);
end
%B=B.*(-1);
%BB=B'
X=inv(B'*B)*(B')*L;
V=-B*X+L;
v1=V(1:2:end);
v2=V(2:2:end);
qxx=inv(B'*B);
omega1=sqrt(v1'*v1)/2;
omega2=sqrt(v2'*v2)/2;

fid=fopen('Orientation_R.txt','wt');

[row,col]=size(X);
for i=1:1:row
    for j=1:1:col
        if(j==col)
            fprintf(fid,'%f\n',X(i,j));
        else
            fprintf(fid,'%f\n',X(i,j));
        end
    end
end
fprintf(fid,'%s\t','X precision:');
fprintf(fid,'%f\n',omega1);
fprintf(fid,'%s\t','Y precision:');
fprintf(fid,'%f\n',omega2);
fclose(fid);
\end{lstlisting}

\subsubsection{结果}
\begin{equation}
\begin{array}{lll}
h_0=-46.08 & h_1=0.012 & h_2=0 \\
k_0=82.944 & k_1=0 & k_2=-0.012\\
\end{array}
\end{equation}

X precision:0.000000\\
Y precision:0.000000

\subsection{相对定向}
\subsubsection{程序}

\begin{lstlisting}[caption=XDDX.m]
function xiangduiDX %相对定向
clear;
clc;
[filename,pathname]=uigetfile('相定点标.txt','选择相对定向点文件');
fid1=fopen(strcat(pathname,filename),'rt');
i=find('.'==filename);
net_name=filename(1:i-1);
nt=fscanf(fid1,'%f',1);%点对个数
f1=fscanf(fid1,'%f',1);%左片
f2=fscanf(fid1,'%f',1);%右片
f3=fscanf(fid1,'%f',[4,nt]);%4行nt列
fclose(fid1);
f3=f3';
i1=f3(:,1);
j1=f3(:,2);
i2=f3(:,3);
j2=f3(:,4);
sz=size(j2);
np=sz(1);
[filename,pathname]=uigetfile('ndx结果.txt','选择内定向参数文件');
fid1=fopen(strcat(pathname,filename),'rt');
S=textread(strcat(pathname,filename),'%f',6);
fclose(fid1);
for i=1:6
    A(i)=S(i);
   % A(i)=A(i)+1;
end
%内定向--像素坐标转成像空坐标
x1=A(1)+A(2)*i1+A(3)*j1;
y1=A(4)+A(5)*i1+A(6)*j1;
x2=A(1)+A(2)*i2+A(3)*j2;
y2=A(4)+A(5)*i2+A(6)*j2;
f=120;w1=0;
%确定初始值
fai1=0;k1=0;fai2=0;w2=0;k2=0;
X0=[fai1;k1;fai2;w2;k2];
%lim=0.0003*ones(5,1);
DX=ones(5,1);
z=-f*ones(1,np);
ct=0;%循环次数
while (abs(DX(1))>0.0003 | abs(DX(2))>0.0003 | abs(DX(3))>0.0003 | abs(DX(4))>0.0003 | abs(DX(5))>0.0003)
%计算像空辅助坐标
a11=cos(X0(1))*cos(X0(2))-sin(X0(1))*sin(w1)*sin(X0(2));
a12=-cos(X0(1))*sin(X0(2))-sin(X0(1))*sin(w1)*cos(X0(2));
a13=-sin(X0(1))*cos(w1);
b11=cos(w1)*sin(X0(2));
b12=cos(w1)*cos(X0(2));
b13=-sin(w1);
c11=sin(X0(1))*cos(X0(2))+cos(X0(1))*sin(w1)*sin(X0(2));
c12=-sin(X0(1))*sin(X0(2))+cos(X0(1))*sin(w1)*cos(X0(2));
c13=cos(X0(1))*cos(w1);
R1=[a11 a12 a13;b11 b12 b13;c11 c12 c13];
a21=cos(X0(3))*cos(X0(5))-sin(X0(3))*sin(X0(4))*sin(X0(5));
a22=-cos(X0(3))*sin(X0(5))-sin(X0(3))*sin(X0(4))*cos(X0(5));
a23=-sin(X0(3))*cos(X0(4));
b21=cos(X0(4))*sin(X0(5));
b22=cos(X0(4))*cos(X0(5));
b23=-sin(X0(4));
c21=sin(X0(3))*cos(X0(5))+cos(X0(3))*sin(X0(4))*sin(X0(5));
c22=-sin(X0(3))*sin(X0(5))+cos(X0(3))*sin(X0(4))*cos(X0(5));
c23=cos(X0(3))*cos(X0(4));
R2=[a21 a22 a23;b21 b22 b23;c21 c22 c23];
P1=R1*[x1';y1';z];
P2=R2*[x2';y2';z];
%生成误差方程
B=zeros(np:5);
B(:,1)=(-P1(1,:).*P2(2,:)./P1(3,:))';
B(:,2)=P1(1,:)'
B(:,3)=(P2(1,:).*P1(2,:)./P1(3,:))';
B(:,4)=(P1(3,:)+P1(2,:).*P2(2,:)./P1(3,:))';
%B(:,4)=-(P1(3,:)-P1(2,:).*P2(2,:)./P1(3,:))';
B(:,5)=-P2(1,:)';

L=f*(P1(2,:)./P1(3,:)-P2(2,:)./P2(3,:))';
DX=inv(B'*B)*B'*L;
X=X0+DX;
ct=ct+1;
%判断
%if (DX<lim)
 %   break;
 %end
if ct>10
    msgbox('the cycling times is more than 5','Warning','warn');
    break;
end;
X0=X;
end
fid=fopen(strcat(pathname,net_name,'result_new.txt'),'wt');
c=fprintf(fid,'%s\n','相对定向参数为');
c=fprintf(fid,'%7.6f\n',X);%X=[fai1 k1 fai2 w2 k2]
st=fclose(fid);
return
\end{lstlisting}

\subsubsection{结果}
\begin{equation}
\begin{array}{lll}
\phi_1=-0.012473 & & \kappa_1=0.028427 \\
\phi_2=0.013329 & \omega_2=-0.022502 & \kappa_2=0.036377
\end{array}
\end{equation}

\subsection{前方交会-绝对定向}

\subsubsection{程序}

\begin{lstlisting}[caption=QF\_JDDX.m]
function [retval] = QF ()
%kbpoints=xlsread('F:\octave\kbpoints.xlsx');
[FileName,PathName]=uigetfile('.txt','please choose XSZB'); %像素坐标获得的内定向结果
[x1,y1,x2,y2]=textread([PathName,FileName],'%f %f %f %f');
[filename,pathname]=uigetfile('.txt','please choose XDDX result');
 X=textread(strcat(pathname,filename),'%n','headerline、',1);


fi1=X(2);
wa1=0;
ka1=X(3);
fi2=X(4);
wa2=X(5);
ka2=X(6);

al1=cos(fi1)*cos(ka1)-sin(fi1)*sin(wa1)*sin(ka1);
 al2=-cos(fi1)*sin(ka1)-sin(fi1)*sin(wa1)*cos(ka1);
 al3=-sin(fi1)*cos(wa1);
 bl1=cos(wa1)*sin(ka1);
 bl2=cos(wa1)*cos(ka1);
 bl3=-sin(wa1);
 cl1=sin(fi1)*cos(ka1)+cos(fi1)*sin(wa1)*sin(ka1);
 cl2=-sin(fi1)*sin(ka1)+cos(fi1)*sin(wa1)*cos(ka1);
 cl3=cos(fi1)*cos(wa1);
 ar1=cos(fi2)*cos(ka2)-sin(fi2)*sin(wa2)*sin(ka2);
 ar2=-cos(fi2)*sin(ka2)-sin(fi2)*sin(wa2)*cos(ka2);
 ar3=-sin(fi2)*cos(wa2);
 br1=cos(wa2)*sin(ka2);
 br2=cos(wa2)*cos(ka2);
 br3=-sin(wa2);
 cr1=sin(fi2)*cos(ka2)+cos(fi2)*sin(wa2)*sin(ka2);
 cr2=-sin(fi2)*sin(ka2)+cos(fi2)*sin(wa2)*cos(ka2);
 cr3=cos(fi2)*cos(wa2);
 
R1=[al1 al2 al3;bl1 bl2 bl3;cl1 cl2 cl3];
 R2=[ar1 ar2 ar3;br1 br2 br3;cr1 cr2 cr3];
 p=0.6;
 l=46.08*2;
 M=(7680*0.2)/(46.08*2);
 b=l*(1-0.6)*M;
 B=[b 0 0];
 f=120;
%L_image=[kbpoints(:,1:2),-f*ones(size(kbpoints,1),1)]';
%R_image=[kbpoints(:,3:4),-f*ones(size(kbpoints,1),1)]';
L_image=[[x1,y1],-f*ones(size(x1,1),1)]';
R_image=[[x2,y2],-f*ones(size(x1,1),1)]';
i1=R1*L_image;
i2=R2*R_image;
xyzq=zeros(size(i1,2),4);
for i=1:size(i1,2),
N1=(B(1)*i2(3,i))/(i1(1,i)*i2(3,i)-i2(1,i)*i1(3,i));
N2=(B(1)*i1(3,i))/(i1(1,i)*i2(3,i)-i2(1,i)*i1(3,i));
x=N1*i1(1,i);
y=0.5*(N1*i1(2,i)+N2*i2(2,i));
z=N1*i1(3,i);
q=N1*i1(2,i)-N2*i2(2,i);
xyzq(i,:)=[x y z q]
end
q=xyzq(:,4);
%XYZ=xlsread('F:\octave\points.xlsx',1,'H3:J47');
[FileName,PathName]=uigetfile('.txt','please choose DMZB'); %像素坐标获得的内定向结果
[X,Y,Z]=textread([PathName,FileName],'%f %f %f ');

xyz=xyzq(:,1:3);
%X=XYZ(:,1:1);
%Y=XYZ(:,2:2);
%Z=XYZ(:,3:3);
fid=fopen('F:\octave\QFresult.txt','wt');
i=0;
fi0=0;
m0=0;
ka0=0;
lamda0=1;
detX=0;
detY=0;
detZ=0;  
Xg=0;
Yg=0;
Zg=0;
theta=0;
fi0=0;
om0=0;
ka0=0;
lamda0=1;
detX=0;
detY=0;
detZ=0;  
Xg=0;
Yg=0;
Zg=0;
n=length(X);
for i=1:n
  Xg=Xg+X(i);
  Yg=Yg+Y(i);
  Zg=Zg+Z(i);
end
  Xg=Xg/n;
  Yg=Yg/n;
  Zg=Zg/n;
xg=0;
yg=0;
zg=0;
x=xyzq(:,1)
y=xyzq(:,2)
z=xyzq(:,3:3)
for i=1:n
  xg=xg+x(i);
  yg=yg+y(i);
  zg=zg+z(i);
  end
  xg=xg/n;
  yg=yg/n;
  zg=zg/n;
 zblc=0;
for i=1:n
    Xyh(i)=X(i)-Xg;
    Yyh(i)=Y(i)-Yg;
    Zyh(i)=Z(i)-Zg;
    xyh(i)=x(i)-xg;
    yyh(i)=y(i)-yg;
    zyh(i)=z(i)-zg;                       %重心化坐标
    jl(i)=sqrt(Xyh(i)*Xyh(i)+Yyh(i)*Yyh(i)+Zyh(i)*Zyh(i));
    JL(i)=sqrt(xyh(i)*xyh(i)+yyh(i)*yyh(i)+zyh(i)*zyh(i));
    blc(i)=jl(i)/JL(i);
    zblc=zblc+blc(i);
end
pjblc=zblc/n;     
for i=1:n
    xyhh(i)=pjblc*xyh(i);
    yyhh(i)=pjblc*yyh(i);
    zyhh(i)=pjblc*zyh(i);
end
 dfi0=1;
dom0=1;
dka0=1;
count=0;
while abs(dfi0)>1e-5||abs(dom0)>1e-5||abs(dka0)>1e-5
    count=count+1;
    a10=cos(fi0)*cos(ka0)-sin(fi0)*sin(om0)*sin(ka0);
    a20=-cos(fi0)*sin(ka0)-sin(fi0)*sin(om0)*cos(ka0);
    a30=-sin(fi0)*cos(om0);
    b10=cos(om0)*sin(ka0);
    b20=cos(om0)*cos(ka0);
    b30=-sin(om0);
    c10=sin(fi0)*cos(ka0)+cos(fi0)*sin(om0)*sin(ka0);
    c20=-sin(fi0)*sin(ka0)+cos(fi0)*sin(om0)*cos(ka0);
    c30=cos(fi0)*cos(om0);
    R0=[a10 a20 a30;b10 b20 b30; c10 c20 c30];    
 for i=1:n
    xsjz(3*i-2,1)=1;
    xsjz(3*i-2,2)=0;
    xsjz(3*i-2,3)=0;
    xsjz(3*i-2,4)=xyhh(i);
    xsjz(3*i-2,5)=-zyhh(i);
    xsjz(3*i-2,6)=0;
    xsjz(3*i-2,7)=-yyhh(i);
    xsjz(3*i-1,1)=0;
    xsjz(3*i-1,2)=1;
    xsjz(3*i-1,3)=0;
    xsjz(3*i-1,4)=yyhh(i);
    xsjz(3*i-1,5)=0;
    xsjz(3*i-1,6)=-zyhh(i);
    xsjz(3*i-1,7)=xyhh(i);
    xsjz(3*i,1)=0;
    xsjz(3*i,2)=0;
    xsjz(3*i,3)=1;
    xsjz(3*i,4)=zyhh(i);
    xsjz(3*i,5)=xyhh(i);
    xsjz(3*i,6)=yyhh(i);
    xsjz(3*i,7)=0; 
    ljz(3*i-2)=Xyh(i)-lamda0*[a10 a20 a30]*[xyhh(i);yyhh(i);zyhh(i)]-detX;
    ljz(3*i-1)=Yyh(i)-lamda0*[b10 b20 b30]*[xyhh(i);yyhh(i);zyhh(i)]-detY;
    ljz(3*i)=Zyh(i)-lamda0*[c10 c20 c30]*[xyhh(i);yyhh(i);zyhh(i)]-detZ; 
 end
    dxjz=inv(xsjz'*xsjz)*xsjz'*ljz';
    vjz=-(xsjz*dxjz-ljz');
    detX=detX+dxjz(1);
    detY=detY+dxjz(2);
    detZ=detZ+dxjz(3);
 lamda0=lamda0*(1+dxjz(4));
fi0=fi0+dxjz(5);
om0=om0+dxjz(6);
ka0=ka0+dxjz(7);
dfi0=dxjz(5);
dom0=dxjz(6);
dka0=dxjz(7);
fprintf(fid,'第%d次迭代   %f    %f    %f    %f    %f    %f    %f\r\n\r',count,detX,detY,detZ,fi0,om0,ka0,lamda0);
end
fclose(fid);

%(fid,'%f %f %f %f  \n', xyzq(:,i));
 
end function
\end{lstlisting}

\subsubsection{结果}
\begin{equation}
\begin{array}{lll}
\phi=0.011309 & \omega=0.005634 & \kappa=0.00524 \\
& \lambda=1.00071& \\
dx=0 & dy=0 & dz=0  \\
count=3\\
\end{array}
\end{equation}

\subsection{点误差判断}
\subsubsection{程序}

注:我将点位误差判断与前方交会-绝对定向程序放在了一起。
\begin{lstlisting}
fprintf(fid,'\r\n点名      X                Y             Z             dX         dY         dZ       误差\r\n');
for i=1:n
    XYH(i)=lamda0*[a10 a20 a30]*[xyhh(i);yyhh(i);zyhh(i)]+detX;
    YYH(i)=lamda0*[b10 b20 b30]*[xyhh(i);yyhh(i);zyhh(i)]+detY;
    ZYH(i)=lamda0*[c10 c20 c30]*[xyhh(i);yyhh(i);zyhh(i)]+detZ;%待求点的重心化地面摄影测量坐标
    Xq(i)=XYH(i)+Xg;
    Yq(i)=YYH(i)+Yg;
    Zq(i)=ZYH(i)+Zg;
    xc(i)=Xq(i)-X(i);
    yc(i)=Yq(i)-Y(i);
    zc(i)=Zq(i)-Z(i);
    wuchashi(i)=sqrt(xc(i)*xc(i)+yc(i)*yc(i)+zc(i)*zc(i));
    fprintf(fid,'%d   %f   %f   %f   %f   %f   %f   %f\r\n',i,Xq(i),Yq(i),Zq(i),xc(i),yc(i),zc(i),wuchashi(i));
end
xzwc=sqrt(xc*xc'/(n-1));
yzwc=sqrt(yc*yc'/(n-1));
zzwc=sqrt(zc*zc'/(n-1));
fprintf(fid ,'\r\nX的中误差=%f, Y的中误差=%f, Z的中误差=%f',xzwc,yzwc,zzwc);
fprintf(fid,'\r\n*********************************************分格线***************************************************');
%plot(q);
\end{lstlisting}

\subsubsection{结果}
X的中误差=0.124881, Y的中误差=0.215042, Z的中误差=0.239378

\section{调绘成果}

\begin{figure}[htbp]
\centering
\caption{调绘图概览}
\includegraphics[width=0.9\textwidth]{diaohuijr.jpg}
\end{figure}

\begin{figure}[htbp]
\centering
\caption{调绘图细节}
\includegraphics[width=0.9\textwidth]{diaohui1jr.JPG}
\end{figure}

\section{同名点}


\begin{table}[htbp]
    \centering
    \begin{tabular}{lrrrr}
        \toprule
        & \multicolumn{2}{c}{63} & \multicolumn{2}{c}{64} \\ \hline
        点号 & \multicolumn{1}{c}{i} & \multicolumn{1}{c}{j} & \multicolumn{1}{c}{i} & \multicolumn{1}{c|}{j} \\ \midrule
      20 & 5790 & 2636 & 2630 & 2268 \\
      21 & 5637 & 2737 & 2491 & 2368 \\
      22 & 5669 & 2523 & 2514 & 2150 \\
      23 & 5179 & 2642 & 2014 & 2263 \\
      24 & 5395 & 2448 & 2244 & 2067 \\
      25 & 5475 & 2545 & 2314 & 2169 \\ \bottomrule
    \end{tabular}
    \end{table}
      

\chapter{陈晨个人实习报告}

如何将理论与实际联系,进行知识向技能的改变,这是我一直在思考的问题,如果只能纸上谈兵,知识可能就会失去活力。我们在上次暑假期间进行的测量就是在提高我们的仪器操作水平和增强我们实地进行测量的能力,而不是让我们仅仅掌握零碎的知识点,这次的摄影测量实习我也有同样的感受,实习是一个整体,让我掌握了从书本中无法习的得流畅的知识体系,让我感受深刻。

\section{实习感想}

在这次摄影测量中,我们结合上学期所学习的摄影测量的知识,将摄影测量的工作流程完整的体验了一遍,在外业进行调绘,在内业进行点之记编写、数字成图等工作,着重进行了相对定向和绝对定向的编程。

有了这一次经历,我更深刻地感受到“纸上得来终觉浅,绝知此事要躬行”,不仅是因为一次次调试程序带来的麻烦;而且是因为学习的知识只有在被应用的时才真正知道自己有没有掌握,应用知识才是学习知识的第一驱动力(印象最深的就是编写相对定向程序的过程,下面会详细说明)。


\subsection{相片联测和同名点坐标量测}
首先这次实验从实地测量开始,选点、观测与记录都是我们小组成员共同完成。我们按照任务书上的指导,每个点测了两遍作为校核。与课本中提到的不同,在现实里我们要考虑更多的东西,比如树木的遮挡,建筑是否有变化,如果周围的控制点不够或者控制点较远我们如何解决,也体现了我们对知识的掌握程度。更具体的例子就是,在第一次测量时候,由于我们对无棱镜测量的特点没有充分了解,选取了一个有较多树枝遮挡的建筑角点,这就造成了偏差较大。后来我们及时意识到问题,并更换了点。

接着组长分配任务,按照摄影测量实习要求,我们组的每个人都在影像上找6对同名点,作为相对定向的基础。在找点的时候,要特别小心仔细,要注意时候图片不能缩放的过大,避免找不到点。这是一个基本功,在后面的CAD成图中也涉及到用PS找像素坐标。

\subsection{摄影测量编程}
编程是一个非常重要的环节,当然掌握知识是必要的前提条件。

我觉得这次编程对于我来说是意义重大的。其一是知识的重温,虽然在上学期我学习了相对-绝对定向是将像素坐标转化为物方坐标的一种基本方法,也了解它的基本思想,相对-绝对是先求解左右相片的相对位置,然后再逐步求解其在空间中的位置,但我却不是特别清楚具体操作的流程,我可以很流畅的把流程图背下来,但是这些知识没有内化,在实习中我实现了知识的内化吸收。其二是克服了编程的畏难情绪,面对这样一个大的问题,在一开始我没有分解的意识,我对于把这些流程转化为准确高效的代码很惶恐,但如果不开始就什么都完不成,一步一步,即使很痛苦的Debug也是通往结果的一个过程吧。后来想清楚之后,感觉学习就应该逼迫自己快点经历,有过经历之后,就不再对结果迷茫,结果就不那么不可知。其三是技能的掌握,程序的编写也让我复习了 MATLAB 的使用,这是一项非常综合的锻炼,让我得到很大的提升。

由于在上学期我们就已经进行了内定向的编程,所以我就直接从相对定向开始编写。正如上文提到的,在单独法相对定向时的知识储备足够了,但是有很多心态上的不足,造成效率比较低下。单独法相对定向就是在利用我们找到的同名点的像素坐标求出的框标坐标之后,求出五个相对定向元素。不过照着流程图我也很快完成了第一次相对定向,求出了相对定向元素,与小组成员比较之后基本满足精度要求。在这里我们有一个不能忽略的问题:这些点是否符合要求吗(即这些点在两张相片上真的是同名点吗)?

为了解决这个问题,我在运用求得的独立像对相对定向参数以及量测的同名点坐标,根据前方交会计算式计算模型坐标,得到像片控制点、待定点的模型坐标之前,我们通过前方交会计算这些点的上下视差来检验点的质量,删除上下视差超过限度的点,重新求取相对定向元素。经过前方交会的过程我们发现有一些点中有的上下视差过大,这些点就是我们要剔除的点。当剔除的阈值(上下视差)选择±0.3和±0.5时,发现相差的结果并不大。后来,为了小组内部成员能够相互检查督促,我们选取了相同的点进行相对定向。这个问题是我们在上学期中完全不会遇到的,因为我们默认这些点肯定是正确的,但是实际生产中如何肯定要进行点的质量检测,这就是学习和实习的不同。

在后续的模型点坐标计算和绝对定向参数的求解中,我遇到了很多问题,不过在经过思考和讨论后,问题都得到了合理的解决。问题列表如下:
\begin{itemize}
	\item[-] 绝对定向中Bx的确定\\
	Bx代表摄影基线的长度。相对定向中,bx其实是一个无关紧要的量,因此,我算出来相对定向的值是正确的;然而到了绝对定向,出现问题了,七参数中z异常大!然而找不到问题所在,最后在调试时,发现不对,才改正过来。
	\item[-]参数过多,传值频繁。\\
	在数值传递过程中就会出现变量混乱,变量名污染的情况。为了避免这种状态我将不同功能用不同的函数分开求解,才解决这个问题。
	\item[-]数据录入错误/公式录入错误。\\
	在绝对定向输入系数阵时,我把一个项的系数阵的正负号搞错了,这也是绝对定向不收敛的原因之一。这可能是最低级的错误,但是这次我又犯了。
	\item[-]点的质量。\\
	同上文一样,在进行绝对定向过程中要对控制点的质量进行检查,我们的做法是剔除上下视差Q过大的点,留下最好的11个点作为计算绝对定向元素的点,剩下的较好的点做为检查点。
\end{itemize}

\subsection{校园调绘}

在调绘的过程中,我们一边数楼层一边调查这些楼是什么结构,分了两天才将全校调绘完成。调绘完成后,我才知道同济的校园是这么大,这些都是我之前没有认真体会的。而且我们还发现有着一百多年历史的楼-文远楼,这些都是我不曾注意的。可以说这次调绘增强了我对我们学校的了解,也更深刻地体会到我们学校丰富的文化底蕴。

\subsection{三维建模}
我们尝试对一位坐在椅子上的同学进行扫描建模,相对与椅子的模型,人体建模结果较差,整体较模糊。特别是脸部特征,基本全部没有。分析其原因,可能是人体皮肤对红外线反射较少,另同学配戴眼镜,也会有一定反射。另外镜头在移动的时候可能没有控制好速度,这也是原因之一。通过实验,我们熟悉了软件使用,了解了三维建模基本过程。

\subsection{总结}
这次实习中,我们小组里精诚合作,相互帮助,增强了我们的团结协作能力,培养了同学之间深厚的情谊,让我们在相互学习中进步。

虽然这次做的成果并不是很完美,但通过对实习成果的分析,我了解了摄影测量作业的基本流程,对摄影测量课程有了更深更具体的体会。在这次实习中, 老师不再是单纯的给我们讲解知识, 而是给了我们大量的时间去实践,再针对问题进行解答,很好的锻炼了我们的各种能力,既让我们更加清晰的认识了摄影测量学这门学科, 使课本知识得到了巩固,也让编程能力进行了历练,使我获益良多。

\section{程序及计算结果}

\subsection{内定向}
\textbf{内定向参数:}
\begin{equation}
\begin{array}{lll}
h_0=-46.080000 & h_1=0.012000 & h_2=0.000000 \\
k_0=82.944000 & k_1=-0.000000 & k_2=-0.012000
\end{array}
\end{equation}

\textbf{内定向程序:}
\begin{lstlisting}[caption=orientation.m文件]
function orientation()%内定向
format long
[a1,a2,a3,a4]=textread('camera.txt','%f%f%f%f');%处理过的框标的理论位置
a=[a1 a2 a3 a4];
theoreticalpositions = a(:,1:2);%框标的理论位置
PC = a(:,3:4);%框标像素坐标
%x=a0+a1*x'+a2*y';y=b0+b1*x'+b2*y'
L=theoreticalpositions';
L=L(:);
B=zeros(8,6);
for i=1:2:8
B(i,1)=1;
end
for i=2:2:8
B(i,4)=1;
end
for i=1:2:8
B(i,2)=PC(((i+1)/2),1);
end
for i=1:2:7
B(i,3)=PC(((i+1)/2),2);
end
for i=2:2:8
B(i,5)=PC((i/2),1);
end
for i=2:2:8
B(i,6)=PC((i/2),2);
end
X=inv(B'*B)*(B')*L;
V=-B*X+L;
omega=sqrt(V'*V)/2;
fid=fopen('ndx.txt','wt');
fprintf(fid,'%s\n','Internal fixed parameter(h0,h1,h2,k0,k1,k2) are shown as follows:');
fprintf(fid,'%s\n','precision:');
fprintf(fid,'%f\n',omega);
[row,col]=size(X);
for i=1:1:row
for j=1:1:col
if(j==col)
fprintf(fid,'%f\n',X(i,j));
else
fprintf(fid,'%f\n',X(i,j));
end
end
end
\end{lstlisting}

模型:
\begin{equation}
\begin{bmatrix}
x \\ y
\end{bmatrix}
=\begin{bmatrix}
h_1 & h_2 \\
k_1 & k_2 
\end{bmatrix}
\begin{bmatrix}
i \\ j
\end{bmatrix}
+\begin{bmatrix}
h_0 \\ k_0
\end{bmatrix}
\end{equation}

\subsection{相对定向}
\textbf{相对定向参数:}
\begin{equation}
\begin{array}{lll}
\phi_1=-0.012473 & & \kappa_1=0.028427 \\
\phi_2=0.013329 & \omega_2=-0.022502 & \kappa_2=0.036377
\end{array}
\end{equation}

\textbf{相对定向程序:}
\begin{lstlisting}[caption=ca\_faducialmarks.m文件]
function ca_faducialmarks()%计算框标
[a1, a2, a3, a4]=textread('C:\Users\c\Desktop\新建文件夹\新建文本文档.txt','%f %f %f %f');
num=length(a1);
photo1=[ones(num,1) a1 a2];
photo2=[ones(num,1) a3 a4];
orientation1=textread('ndx.txt','%n','headerlines',3);
orientation2=textread('ndx.txt','%n','headerlines',3);
h1=orientation1(1:1:3,:);k1=orientation1(4:1:6,:);
h2=orientation2(1:1:3,:);k2=orientation2(4:1:6,:);
psx1=photo1*h1;
psy1=photo1*k1;
f=120.000;
psx2=photo2*h2;
psy2=photo2*k2;
fid=fopen('kuangbiao.txt','wt');
[row,~]=size(psx1);
for i=1:1:row
fprintf(fid,'%f %f %f %f\n',psx1(i,1),psy1(i,1),psx2(i,1),psy2(i,1));
end
st=fclose(fid);
return;
\end{lstlisting}

\begin{lstlisting}[caption=ca\_R.m文件]
function R=ca_R(w,p,k)
Rz=[cos(k) -sin(k) 0;sin(k) cos(k) 0;0 0 1];
Ry=[cos(p) 0 -sin(p);0 1 0;sin(p) 0 cos(p)];
Rx=[1 0 0;0 cos(w) -sin(w);0 sin(w) cos(w)];
R=Ry*Rx*Rz;
\end{lstlisting}

\begin{lstlisting}[caption=relative\_oriention.m文件]
function relative_oriention()%相对定向
format long
f=120.000;%mm
[x1,y1,x2,y2]=textread('kuangbiao.txt','%f%f%f%f');
w1=0;
%Number of points of the same name
dotnum= length(x1); 
%initial relative parameters :
p1=0;k1=0;w2=0;  p2=0;  k2=0; 
z=-f*ones(1,dotnum);
X0=[p1;k1;p2;w2;k2];
deltax=ones(5,1);
ct=0;
while (abs(deltax(1))>0.0003 || abs(deltax(2))>0.0003 || abs(deltax(3))>0.0003 || abs(deltax(4))>0.0003 || abs(deltax(5))>0.0003)
%计算像点的像空间辅助坐标X1, Y1, Z1, X2, Y2, Z2
ct=ct+1;
%rotational matrix R
R2 = ca_R(w2,p2,k2);
R1 = ca_R(w1,p1,k1);
P1=R1*[x1';y1';z];
P2=R2*[x2';y2';z];
%Generate error equation
A=zeros(dotnum:5);
A(:,1)=(-P1(1,:).*P2(2,:)./P1(3,:))';
A(:,2)=P1(1,:)';
A(:,3)=(P2(1,:).*P1(2,:)./P1(3,:))';
A(:,4)=(P1(3,:)+P1(2,:).*P2(2,:)./P1(3,:))';
%A(:,4)=-(P1(3,:)-P1(2,:).*P2(2,:)./P1(3,:))';
A(:,5)=-P2(1,:)';
L=f*(P1(2,:)./P1(3,:)-P2(2,:)./P2(3,:))';
deltax=inv(A'*A)*A'*L;
p11=p1+deltax(1,1);
k11=k1+deltax(2,1);
p21=p2+deltax(3,1);
w21=w2+deltax(4,1);
k21=k2+deltax(5,1);
% p1;k1;p2;w2;k2
if ct>6
msgbox('Maybe you are wrong.Please check it.','Hello:)','warn');
break;
end;
p1=p11;
k1=k11;
p2=p21;
w2=w21;
k2=k21;
V= - (A * deltax);
sigma = sqrt(V' * V);
exx = sigma * inv(A' * A);
variances = diag(exx);
deviations = sqrt(variances);
end
fid=fopen('ro_conclusion.txt','wt');
c=fprintf(fid,'%s\n','The relative orientation parameter are');
c=fprintf(fid,'%f\n',p1);
c=fprintf(fid,'%f\n',k1);
c=fprintf(fid,'%f\n',p2);
c=fprintf(fid,'%f\n',w2);
c=fprintf(fid,'%f\n',k2);
st=fclose(fid);
return;
\end{lstlisting}

\subsection{绝对定向}
\textbf{绝对定向参数:}\\
$\phi=0.011271124539547$  \\
$\omega=0.005631855652775$\\  
$\kappa=0.000538186734723$ \\
$\lambda=1.000708892036554$ \\
dx=0,\quad dy=0,\quad dz=0  

 \textbf{地面点的重心化作标:} 
 5490.944272727273\quad 3057.336272727273\quad 008.125272727273

\textbf{绝对定向程序:}
\begin{lstlisting}[caption=absolute\_orientation.m文件]
function absolute_orientation()
[xp,yp,zp,Q]=textread('3.txt','%f%f%f%f');
[N,E,H]=textread('2.txt','%f%f%f');
Xs=0;
Ys=0;
Zs=3;
p=0;
w=0;
k=0;
lamada=1;
M=7680*0.2/46.08/2;%M=1/((46.0800*2)/(7680*0.2));
pointnum=length(xp);
P=[xp,yp,zp];
TP=[N,E,H];%地面摄测坐标
PG=sum(P)/pointnum;
TG=sum(TP)/pointnum;
sumscale=0;
for i=1:1:pointnum   
Pg(i,1)=P(i,1)-PG(1);
Pg(i,2)=P(i,2)-PG(2);
Pg(i,3)=P(i,3)-PG(3);%(1);
Tg(i,1)=TP(i,1)-TG(1);
Tg(i,2)=TP(i,2)-TG(2);
Tg(i,3)=TP(i,3)-TG(3);
JL(i)=sqrt(Pg(i,1)*Pg(i,1)+Pg(i,2)*Pg(i,2)+Pg(i,3)*Pg(i,3));
jl(i)=sqrt(Tg(i,1)*Tg(i,1)+Tg(i,2)*Tg(i,2)+Tg(i,3)*Tg(i,3));
scale(i)=jl(i)/JL(i);
sumscale=sumscale+scale(i);
end
meansacle=sumscale/pointnum;
for i=1:1:pointnum   

Pg(i,1)=Pg(i,1)*meansacle;
Pg(i,2)=Pg(i,2)*meansacle;
Pg(i,3)=Pg(i,3)*meansacle;
end
ct=0;dw=1;dp=1;dk=1;
while(abs(dw)>1e-5|| abs(dp)>1e-5||abs(dk)>1e-5)
R=ca_R(w, p, k);  %求解余弦函数  
for i=1:1:pointnum %组成法方程系数阵%求常数项
A(3*i-2,:)=[1, 0, 0, Pg(i,1), -Pg(i,3), 0, -Pg(i,2)];
A(3*i-1,:)=[0, 1 ,0 ,Pg(i,2),0, -Pg(i,3) , Pg(i,1)];
A(3*i,:)=[0 ,0 ,1 ,Pg(i,3), Pg(i,1) ,Pg(i,2), 0];
l(i,:)=Tg(i,:)-lamada*Pg(i,:)*R'-[Xs Ys Zs];
L(3*i-2)=l(i,1);
L(3*i-1)=l(i,2);
L(3*i)=l(i,3);
end
X=inv((A')*A)*(A')*L'; %求解改正数X
p=p+X(5,1);
w=w+X(6,1);
k=k+X(7,1);
lamada=lamada*(X(4,1)+1);
Xs=Xs+X(1,1);
Ys=Ys+X(2,1);
Zs=Zs+X(3,1);
dp=X(5,1);
dw=X(6,1);
dk=X(7,1);
V=A*X-L;
Qx=inv((A')*A);
m=sqrt(V'*V/(3*pointnum-7));
mx=m*sqrt(Qx(1,1));
my=m*sqrt(Qx(2,2));
mz=m*sqrt(Qx(3,3));
mr=m*sqrt(Qx(4,4));
mq=m*sqrt(Qx(5,5));
mw=m*sqrt(Qx(5,5));
mk=m*sqrt(Qx(7,7));
Accuracy=[m,mx,my,mz,mr,mq,mw,mk];
jP=[Xs,Ys,Zs,p,w,k,lamada];
ct=ct+1;
end
\end{lstlisting}

以下程序用于检查绝对定向元素:
\begin{lstlisting}[caption=inspect.m文件]
for i=1:n
XYZ(i)=lamada*Pg(i,:)*R'+[X(1,1) X(2,1) X(3,1)];%PG为待求点的重心化地面摄影测量坐标
XYZQ(i)=XYZ(i)+PG;
xc(i)=XYZQ(i,1)-XYZ(i,1);
yc(i)=XYZQ(i,2)-XYZ(i,2);
zc(i)=XYZQ(i,3)-XYZ(i,3);
wuchashi(i)=sqrt(xc(i)*xc(i)+yc(i)*yc(i)+zc(i)*zc(i));
end
xzwc=sqrt(xc*xc'/(n-1));
yzwc=sqrt(yc*yc'/(n-1));
zzwc=sqrt(zc*zc'/(n-1));
\end{lstlisting}

\section{调绘成果}
%这个待会才能完成

\begin{figure}[htbp]
\centering
\caption{调绘图概览}
\includegraphics[width=\textwidth]{diaohuicc.jpg}
\end{figure}

\begin{figure}[htbp]
\centering
\caption{调绘图细节}
\includegraphics[width=\textwidth]{diaohui1cc.JPG}
\end{figure}


\section{同名点}

\begin{table}[htbp]
	\centering
	\begin{tabular}{lrrrr}
		\toprule
		& \multicolumn{2}{c}{63} & \multicolumn{2}{c}{64} \\ \hline
		点号 & \multicolumn{1}{c}{i} & \multicolumn{1}{c}{j} & \multicolumn{1}{c}{i} & \multicolumn{1}{c}{j} \\ \midrule
	    26 & 4933 & 2848 & 1773 & 2469 \\
      27 & 5085 & 2922 & 1929 & 2547 \\
      28 & 5252 & 3003 & 2097 & 2633 \\
      29 & 5541 & 3095 & 2375 & 2732 \\
      30 & 5215 & 3141 & 2061 & 2772 \\
      31 & 5670 & 3169 & 2501 & 2809 \\ \bottomrule
	\end{tabular}
	\end{table}
% \include{chapter/part3}
% \chapter{个人实习报告模板}

每个人的个人实习报告是一章(chapter),章下面可分为节(section),小节(subsection),小小节(subsubsection),段(paragraph)等,下面分别呈现。

LateX不同于WYSIWYG(所见即所得,What You See Is What You Get)文字系统,是一种WYTIWYG(所想即所得)文字系统。

\section{第一节}

这里是第一节,每段之间空一行。\LaTeX{}程序会每段开头会自动缩进。

\subsection{第一小节}

这里是第一小节,常用的两个环境是enumrate和itemize,下面有示例。
\begin{enumerate}
\item 项目一。
\item 项目二。
\item 项目三。
\end{enumerate}

\begin{itemize}
\item 项目一。
\item 项目二。
\item 项目三。
\end{itemize}

\subsubsection{第一小小节}

这里是第一小小节

\paragraph{段标题} 如果需要,可以加段标题。

\section{第二节}

常用的强调说明有:\\
加粗:\textbf{粗体,boldface}\\
倾斜:\emph{强调,emphasize}\\
其余的事项一般不用手动调。

大家写好后不用编译(单独一章也无法编译,还需要加一些头),直接给我编译。如果需要使用奇技淫巧还是先编译一遍看看吧。

如果大家需要编译,建议下载最新版的\TeX{}Live,千万不要下载C\TeX。



% \bibliographystyle{plainnat}
% \bibliography{reference}
% \addcontentsline{toc}{chapter}{参考文献}

\end{document}
