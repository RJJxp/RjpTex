\documentclass[12pt, a4paper, UTF8]{article}
\usepackage{ctex}

\begin{document}
\title{\Huge 测绘科学与技术进展报告}
\author{学号 \quad 1931991\\
        姓名 \quad 任家平}
\date{2019-12-18}
\maketitle
\thispagestyle{empty}

\newpage
\pagenumbering{Roman}
\tableofcontents

\newpage
\pagenumbering{arabic}
对课程所讲内容进行总结.

\section{张祖勋院士讲座}
张祖勋院士给我做了关于云控制的摄影测量的报告, 其内容如下:
\subsection{大数据时代下的摄影测量}
大数据时代是当前摄影测量面临的新的时代背景, 它的到来给作为地理空间信息提取主要手段的摄影测量的发展带来了新的机遇. 摄影测量大数据包含两类数据: 新获取的影像数据(包括可能同时获取的定位, 定姿, LiDAR)和已有的, 处理过的地理空间信息数据. 一方面, 随着各种新型航空传感器的不断涌现, 低空无人机的广泛应用和其他消费级摄影设备的普及, 影像获取方式越来越简单, 快捷, 由此带来了航空, 车载影像爆发式的增长. 相对于传统航测获取的影像, 这些影像具有如下特点: 影像数据量大; 获取周期短, 时效性高; 规范性弱, 缺乏严格的航线规划; 大多为非量测型相机拍摄, 相对于量测型相机而言, 影像几何质量不高. 另一方面, 经过测绘部门几十年的测绘生产实践, 我国已经积累了大量的地理空间信息数据, 包括数字正射影像图, 数字高程模型, 数字线划地图等, 这些已有地理空间信息产品数据可作为控制信息.另外必须强调的是, 激光探测与测量数据与已知定向参数的影像都是重要的地理空间信息控制数据. 

摄影测量大数据的智能化处理是现阶段摄影测量迫切需要解决的问题. 通过空中三角测量, 获取航空影像精确内外方位元素, 是摄影测量处理的首要任务.摄影测量大数据的空中三角测量处理主要包含两方面的问题: 如何对新获取的影像进行快速, 高效, 自动, 智能的摄影测量处理以获取准确的地理空间信息; 如何将已有地理空间信息数据作为控制信息, 有效地用于新获取影像的摄影测量处理过程, 以提高处理效率和精度. 对于前者, 得益于 SIFT 等特征匹配技术, 基于词汇树的图像检索技术“运动恢复结构” 技术和针对非量测相机物镜畸变的自检校平差技术的发展, 摄影测量和计算机视觉领域在空中三角测量方面取得了长足的进步, 可实现低空无人机正视与倾斜影像的全自动空中三角测量处理, 但它们在实际应用中仍然存在诸多问题, 如对于某些正常数据无法处理出正确结果, 处理流程难以满足摄影测量实际处理要求等.

\subsection{``云控制''摄影测量的背景}
传统摄影测量的控制信息为外业采集的控制点, 这些点一般位于地面上明显标志点, 如圆点, 交点和角点等, 在此称之为``点控制''摄影测量. 为保证空三加密和立体测图的精度, ``点控制''摄影测量的控制点分布需要遵守严格的作业规范, 一般要求测区周边布设平高控制点, 测区内部按一定倍数的基线长度布设多排高程控制点. 对于传统量测型相机而言, 由于影像像幅大, 测区影像数量少, 需要控制点的数量有限; 然而, 现阶段广泛使用的无人机影像, 其像幅小, 几何畸变大和质量不佳等特点, 决定了其空三加密需要大量的控制点.尽管GNSS技术降低了外业控制点采集的难度, 然而, 测量外业控制点仍然十分困难与耗时, 造成测量外业控制点的时间远超过影像获取的时间. 显然, 外业控制点量测仍然是制约航测空三加密效率的关键因素, 导致``点控制''摄影测量已难以适应大数据时代影像数据的特点和其对处理效率和高度自动化的要求. 

``云控制''摄影测量将改变摄影测量处理的模式. 传统摄影测量处理是从影像数据到地理空间信息的单向过程, 即生产的地理空间信息作为产品归档后不会返回至摄影测量处理过程; ``云控制''摄影测量处理则是一个``闭合''的回路, 生产的地理空间信息产品将作为控制信息用于以后的摄影测量处理过程, 从而实现地理空间信息``从摄影测量生产中来, 又回到摄影测量生产中去''. 另外, 具有地理信息的 LiDAR 数据也可以作为``云控制''的一种控制信息.

\subsection{``云控制''摄影测量方法与应用}
``云控制''摄影测量的实质是控制方式的改变,相对于``点控制''摄影测量, 它使用带有精确地理信息的几何参考数据替代外业控制点作为控制. 根据所使用的几何参考数据的不同, ``云控制''摄影测量具有3种形式: 基于影像的云控制; 基于矢量的云控制; 基于LiDAR点云的云控制.

基于影像的云控制主要包括基于DOM与DEM的云控制, 基于已知定向参数影像的云控制; 基于矢量的云控制是根据已有的矢量数据具有地物信息(道路, 河流等线状地物), 将其作为``云控制''数据. 但是基于矢量的云控制存在两个难点: 首先, 由于线特征控制与影像的像点观测没有严格的对应关系, 因此, 基于线特征控制条件的影像定位方法需要对传统的基于控制点的定位方法进行扩充; 其次, 由于影像和矢量是两种完全不同类型的数据, 二者之间的自动配准也是一个难点. 前者可以通过广义点摄影测量理论解决; 而后者, 学者们研究了很多影像与矢量自动配准的方法. 

\subsection{总结}
大数据的简单算法比小数据的复杂算法更有效, 这是大数据时代``云控制''摄影测量的核心. ``云控制''摄影测量充分利用带有地理空间信息的数据替代外业控制点, 实现摄影测量影像大数据的高效, 自动与智能化处理, 具有较大的现实意义和应用前景. ``云控制''摄影测量的两个支撑条件中数据为主, 技术为辅, 而当前的地理空间信息数据存在以下问题: 已有测绘地理空间信息成果的使用问题, 测绘生产的大量地理空间信息没有作为控制信息有效地用于新影像的摄影测量处理之中, 其价值基本没有得到发挥; 影像数据的管理与存储问题, 在现实生产中, ``重成果、轻影像'', 生产结束后DOM、DEM等成果作为 ``线上''数据保存, 而对于处理的影像及其内定向参数作为``线下''数据保存. 为实现``云控制''摄影测量, 应拓展地理空间信息数据的使用, 加强已知定向参数的影像数据的存储与管理.

\section{王家耀院士讲座}
王家耀院士给我们做了一堂关于新时代的地图学的汇报. 其内容如下:
\subsection{地图学介绍}
地图学可以称得上是一门古老而又不断注入新活力的科学.如果从已经发现的古巴比伦地图算起, 迄今已有4500余年的历史, 期间经历了古代地图学的萌芽与发展(约公元14世纪前), 近代地图测绘与传统地图学的形成(约公元15世纪至20世纪中叶), 地图学的技术革命与信息时代的 地图学(约20世纪中叶以来) 3个时期. 地图学从一开始就是人类活动在一定空间和时间认知世界和改造世界的产物, 地图构建地理世界而非复制地理世界, 地图学的使命是研究构建``地理世界''的理论, 方法, 技术体系和地图表达复杂地理世界的空间结构和空间关系及其时空变化规律, 成为``改变世界的十大地理思想''之一. 这就决定了地图学的科学, 技术和工程的基本属性, 以及地图的科学价值, 社会价值, 法理价值, 文化价值和军事价值. 

人类活动本质上就是一种时空行为, 世界上的任何事物和现象(包括自然和社会人文)的发生, 发展和演变, 都是在一定的时间和空间进行的, 地图学作为表达人类活动与地理环境在时间和空间上的复杂关系的一种独特研究视角, 越来越凸显出其与空间和时间的不可分割性. 任何地图都要标注制作和出版的时间(基于某种时间参考系)及采用的空间参考系(大地坐标系统和高程系统), 地图学要关注空间基准和时间基准的理论和方法; 任何地图都依赖于其制图资料(数据), 都要标注资料的时间, 空间参考系统和出处, 地图学要关注这些资料的可靠性和可用性并研究多种不同制图资料的处理方法; 地图是用来认知, 管理和治理世界的, 能进行时空分析, 地图学要研究时间序列的预测分析和空间分布规律分析的模型和算法. 无论何时期的地图学都如此, 只不过时空精准度及内容的深度和广度不同而已.

当今, 随着天空地海一体的传感器网技术, 人工智能技术、``互联网+''技术和智能计算技术等的快速发展, 全球信息化迈入了``大数据时代'', 地球表层(自然, 社会经济及人文)的几何特征, 物理特征和属性特征, 都成为可被感知, 记录, 存储和分析的数据, 大数据带来的信息风暴正在变革人们的生活, 工作和思维, 必将开启一次重大的时代转型, 即思维变革, 商业变革和管理变革. 大数据之所以能成为一个时代, 在很大程度上是因为这是一个由社会各界广泛参与, 学科之间交叉融合的社会活动, 这场变革必然影响到各行各业和各个学科, 机遇也必将蕴含于各行各业和各个学科. 在大数据成为地图信息源的“数据密集型”科学范式新时代, 地图学又站在了新的历史起点上. 此时此刻, 地图学界理所当然地要思考: 大数据时代的到来将给地图学带来什么变化? 地图学如何迎接大数据时代的挑战和机遇? 

\subsection{时空大数据时代给地图学带来的新变化}

\paragraph*{地图学的时空观和方法论}

时空大数据成为地图学的大规模海量数据源, 使得地图学对时空框架下运动变化的事物和现象的描述和表达变得更加科学实用. 地图学的时空观必将推动地图学理论的新发展, 地图哲学———哲学视野下的地图学作为最高层次的地图学理论的概括, 时空综合认知, 时空信息模型, 时空信息传输, 时空信息本体,时空信息语言学等, 都将成为地图学理论新的研究领域; 互联网, 物联网, 云计算技术的发展, 地图制图的系统论方法, 协同论方法, 最优化方法等将成为地图学的方法论; 新兴计算技术, 人类自然智能与人工智能深度融合技术, 数据挖掘与知识发现技术, 时空大数据可视化技术, 网络/网格/云服务技术, 以及语义网, 语义网格和语义网络技术等, 都将成为时空大数据时代地图学发展的重要支撑技术; 地图的实时动态性, 主题的针对性和多样化, 内容的复合性, 表现形式的个性化, 应用的泛在化等, 必将成为可能. 

\paragraph*{地图学的第一位任务是多源(元)异构时空大数据的融合}

多源(元)异构时空大数据融合, 是时空大数据时代给地图学带来的新问题. 地图学再也无须为数据源发愁了, 但时空大数据的多源异构特征也给地图学数据源的处理增加了新的复杂性和困难. 这主要表现在来自国内外不同部门, 不同行业的时空大数据往往具有多类型, 多分辨率(影像), 多时态, 多尺度, 多参考系, 多语义等特点, 客观上造成集成应用的时空大数据不一致, 不连续的问题十分突出, 给地图制图增加了难度, 无法快速为国家重大工程和信息化条件下的联合作战提供全球一致, 陆海一体, 无缝连续的时空大数据服务. 因此, 如何科学描述, 表达和揭示不同类型, 不同尺度, 不同时间, 不同语义和不同参考系统的时空大数据的复杂关系及其相互转换规律, 从根本上解决多源异构时空大数据的融合, 已成为计算机数字地图制图环境下地图学亟待解决的科学技术问题.

\paragraph*{地图学科范式的变化}

以时空大数据密集型计算为特征的地图学科学范式, 使地图学完全有必要也有可能把着重点放在时空大数据的智能化深加工方面来. 这主要表现在时空大数据多尺度自动变换, 时空大数据分析挖掘与知识发现及时空大数据可视化的理论, 方法与技术等方面. 毫无疑问, 从技术层面讲, 20世纪最具里程碑意义的事件是计算机数字地图制图技术取代了传统手工地图制图技术, 其中多尺度时空大数据自动综合是核心, 至今仍是国际上该领域最具挑战性和创造性的研究热点.

\paragraph*{时空认知与时空地理信息传输模式的变化}

随着全球卫星导航定位(GNSS), 天空地海一体化对地观测(RS), 地理信息系统(GIS), 机器学习与人工智能, ``互联网+''等新兴信息技术的发展, 人类对自己赖以生存的时空环境的认识正在由地图空间认知向以现实地理世界为对象, 由``感知的地理世界''``重构的地理世界''和``认知的地理世界''构成 的全过程, 多模式时空综合认知转移. 时空大数据时代的时空感知与认知模式是开放的, 动态的, 多模式的, 综合的, 整个感知认知全过程中融入了前述各种新兴信息技术, 特别是深度学习和深度增强学习, 人类自然智能与计算机人工智能深度融合等技术.

\subsection{总结}
时空大数据时代的到来,给地图学带来了新 的挑战和机遇,地图学必将又一次站在新的起点 上向更高水平发展. 无论是古代地图学、近代地图学或是现代地 图学,地图学的时空观和方法论都是地图学的最 根本的问题,只不过是时空大数据时代我们认识 到了这个问题的更加重要性.以哲学视野从整体 上研究地图学、地图演化论、地图文化及其时空特 性等,地图学的时空大数据思维必将推动理论、技 术方法和服务模式的变革.

由传统制图时代制图资料编整, 到时空计算机数字化地图时代的制图数据处理, 再到如今时空大数据时代的多源异构时空大数据 融合,反映了地图学数据(信息)源由单源到多源、 由少到多、由简单到复杂的趋势,相应地也驱动了 制图数据源处理的理论、方法和技术的不断发展.

时空大数据时代的到来,使地图学的科学范 式由计算和模拟范式(第三范式)中分离出来而进 入当前的数据密集型计算范式(第四范式),这是 一种以时空大数据计算为特征的地图学科学范 式.这里的“时空大数据计算”,除前述多源异构 时空大数据融合外,主要是时空大数据多尺度自 动变换、时空大数据分析挖掘与知识发现,以及时 空大数据可视化等的理论、方法和技术,最终实现 时空大数据价值的最大化. 

地图空间认知与地图信息传输是现代地图学的基础理论, 对地图科技工作的观念转变和更新起了重要作用. 然而, 当时空大数据时代到来的时候, 由于天空地海一体的智能传感器网技术,移动互联网技术, 新兴计算技术,人工智能技术等的快速发展, 在人类认知自己赖以生存的现实地理世界的科学活动``三要素''(主体要素———科学家, 客体要素———科学活动的对象, 工具要素———科学活动的手段)中, 工具要素处于越来越重要的地位, 作用越来越大, 开放, 动态, 多模式, 综合的时空感知认知和时空信息传输新模式, 必将成为时空大数据时代地图学理论的基础研究任务. 

\end{document}