\documentclass[a4paper, UTF8,  12pt]{article}
\usepackage{graphicx}
\usepackage{ctex}

\graphicspath{{./pic/}}
\pagestyle{plain}

\begin{document}

\title{\Huge 毕业实习报告}
\author{\large 学号: 1551127\\
        姓名: 任家平 \\
        同济大学测绘与地理信息学院}
\date{2019-03-15}
\maketitle

\begin{abstract}
     本篇文章大体分为两大部分: \lq 外\rq, \lq 内\rq . \lq 外\rq 分为之前的实习, 大学课程与毕业实习三个模块, 将毕业实习的内容与之前的两者作对比; \lq 内\rq 分为自我与未来, 是自己在实习过程中对社会, 世界的思考, 更是对未来的一个规划与展望.
\end{abstract}
\thispagestyle{empty}
\newpage

\tableofcontents
\thispagestyle{empty}
\newpage

\pagenumbering{arabic}
\section{前言之废话连篇}
说是毕业实习报告, 但实质只是一篇苦力感想而已. 大多数人的实习都是水水过, 干活也仅仅是重复性劳作. 但我这回的实习截然不同, 是前所未有的究极体验. 

可能是写 markdown 习惯了, 对排版愈发敏感, 现在开始用英文符号加空格代替中文符号, 这样写出的文档会显得比较紧凑但又不拥挤. 上个学期结束时把 \LaTeX\ 教程看了一遍, 这回想拿毕业实习报告练练手. 开始是有些担忧, 写的太欢脱表现对老师的不尊重, 但想到老师只会把收上来的报告撇在电子秤, 称重, 评分, 然后扔掉. 这让我释怀, 所有担忧一扫而空, 更让我对 \LaTeX\ 跃跃欲试. 

之前很多报告也是用 \LaTeX\ 所写, 那时只熟悉某些基本语法, 对排版的整体框架一无所知, 看完书以后, 条条框框的排版规则变得清晰起来, 正苦恼疏于练习而遗忘, 毕业实习报便告来了. 送上门的肥肉哪有不吃的道理.

这篇报告也好, 感想也罢, 完全可以糊弄糊弄, 用毫无感情的官话堆砌而成, 但没必要. 消耗了时间, 浪费了精力, 到头来一看报告内容, 反倒把自己恶心地早饭吐出来. 不如借此机会, 静下心来, 驻足反思. 顺便总结自己的大学生活. 

在此斗胆搏各位一笑.

\newpage
\section{外面的世界很精彩}
课堂上总是被老师们吹, 测绘学科发展前景广阔, 就业形势十分火爆, GIS, 大数据, 交叉学科等等高端名字一个接着一个, 天上的牛都在飞. 初入大学, 不禁疑惑起来. 前景广阔, 也只有站在象牙塔顶尖的人才能看的高看得远, 就业火爆, 那就业 \lq 好\rq\ 吗? 这个 \lq 好\rq\ 的标准, 究竟是普遍能找到工作, 还是能轻易找到好工作? 将信将疑, 突然一惊, \lq 实践是检验真理的唯一标准\rq\ , 那为何不去找点行业相关实习, 用自己的双眼一探究竟?

\subsection{之前的苦力实习}

大一暑假开始的第一份实习. 我在杨光老师推荐的上海南康科技有限公司干活. 公司位于繁华的人民广场附近, 北京西路的地产大厦的高层. 距今已有三年, 但有些细节仍难以忘记. 2016年8月5日开始, 2016年8月25日结束. 一共21天. 前十天在公司用cad改地籍图, 后10天在闵行区辛庄镇政府处理资料, 最后一天回公司摸鱼. 和大多实习一样, 开始的新鲜感被重复性劳作消耗殆尽, 渐渐开始怀疑人生. 

干活当然不白干, 工资是要拿的, 信息收集也不可少. 和同事打成一片, 小领导有一个是我们学院的老学长, 也有几个是刚刚毕业的学长. 和他们讨论薪资问题, 不问不知道, 一问吓一跳. 好歹也算是985毕业, 市中心工作, 但测绘行业的小公司, 刚毕业在上海居然只有3k, 现在也不过4k. 心里默默一算, 自己实习的日薪都要比他们刚进来高. 如果工作轻松, 那这么点钱也可以勉强接受. 然而现实是残酷的, 说起来自己也有点受虐狂, 本着体验生活的态度, 和他们一个作息. 早上9点上班, 偶尔正常6点下班, 多数都要加班到8点. 周六放假, 可以早上十点到, 下午5点走. 由于地处市中心附近, 想吃顿正常的午饭, 最便宜20, 每天地铁往返10块, 如此算计下来, 一个月其实剩不了多少, 好在公司比较良心, 会报销午饭钱和交通费用, 前提是有发票. 尽管如此, 他们还比我这个大学生多了房租, 在低酬金下, 其生活压力可想而知.

第二份实习实在大二2017的暑假. 被井冈山大学的烈日灼伤, 又被校内地图累的半死不活后, 对专业的方向越来越迷茫, 不信邪, 又出去找实习, 看看真实情况是否如想象中那么糟糕. 托家里关系, 在陕西省测绘与地理信息局找到了一份实习. 陕西省测绘地理信息局坐落在西安友谊东路与文艺路交口处, 从我高中母校一直向东行驶, 便可到达这里. 干的活相比于南康, 技术含量是高一些, 但归根结底还是重复性劳作的苦力活, 无非从修改地籍图变成三维建模画房子. 

同事们也是刚毕业, 比我早来了半个月的正式员工. 他们的学历让人战栗, 令人震惊. 本科一个武汉大学, 一个中国地质大学还有长安大学博士, 各种硕士碾压, 让一个准大三的实习生瑟瑟发抖. 他们来自天南海北, 青海, 新疆, 西宁, 四川, 可以和本地西安人平分秋色. 

或许是测绘行业的特点. 外面的私企什么的的加班也就是偶尔一两次, 然而在测绘局我看到的是体制性加班. 一星期七天六天班, 周一到周四, 晚上也要加班加到晚上九点半. 早上八点一刻就要早早上班, 而且重点是并不是我们这些年轻的新兵蛋子要加班, 搞得好像只有年轻人受到剥削, 隔壁几个科室到处都是中年妇女, 她们也是每周六和我们一起 \lq 加班\rq . 加班加班, 内业加班, 这时我突然有点后悔为什么没有去跑外业, 不过毕竟是空调房里, 总比在烈日下烤熟好些. 

工资少的可怜, 在局里认识了一个武大航测毕业的的硕士, 工作三年, 他亲口说月薪5500, 不知道是不是真假. 问一起工作的同事, 他们总是以玩笑的口吻, 别问我们工资, 问了你以后肯定都不想来. 虽然有住房补贴, 但貌似可以忽略, 有宿舍楼, 但是以后有了女朋友要结婚, 总不能把人家往宿舍楼里领吧, 只能买房, 买房工资一个月半平米都不够, 想想也是蛋疼. 

在测绘局里面还认识了一位校友, 她在行政工作. 我又和她在薪资待遇方面进行了深切的交谈. 首先是工作时间, 机关比生产上班晚四十五分钟, 下午下班早, 晚上绝对不会加班, 平时工作也没什么事情, 可以用单位的外网逛逛淘宝购购物, 追追剧都是不错的. 生产则是, 上班早, 下班晚, 晚上有体制性的加班, 像我实习所在的五院, 每晚九点半标配, 隔壁科室每晚十点. 可谓是相当辛苦, 而且除了上升空间能稍微大一些, 工资待遇几乎相同. 

空悲切. 年轻人都想走行政不是没有原因的.

\subsection{天翻地覆的实习}

这次校外实习, 是本科的最后一次实习, 在老张的新公司里帮忙干干杂活, 更重要的是从基础开始, 学技术, 积累工程经验. 名义上毕业实习是从2月26日开始, 但早在2018年11月就已经开始为公司的项目干活, 而这实习不会因为毕业实习结束而结束.

老张的公司是研发无人驾驶的科技公司, 比传统测绘行业的公司不知道高到哪里去了. 以前总觉得重复性劳作的工作会让人疲惫不堪, 现在看来当时太天真. 从调试传感器开始, 惯导, 相机, 雷达, 超声波和红外线, 样样对于第一次搞的人来说, 都是麻烦的要死. 技术上没多难, 一点细节没有搞清楚, 结果就出不来. 尤其第一次上手, 像个无头苍蝇一般乱撞, 找不到方向. 怎么接线, 传感器驱动在哪里, 怎么调用传感器驱动, 如果对数据有特殊要求, 该怎么处理, 每一步都慢慢啃官方的文档, 然而世界上从来不存在完备的说明文档, 还需要发挥自己的聪明才智, 临场发挥随机应变.

接下来就是调试开发板, 从 x3399 到 rk3288, 再到 rk3399. 唯一担心的就是系统烧制失败, 无法进入烧制模式. 每一步都要斟酌再三, 如履薄冰, 如临深渊. 软件库在学长的帮助下, 勉勉强强构建完成, 如释重负. 这些工作做完后, 自己的长进非常大. 

在这边干活也快四个月, 辛苦, 疲累, 与以往慢慢悠悠写程序截然不同. 尤其是下半学期开始, 必须拿出一部分精力去捣鼓毕设, 压力突增. 学了些什么东西呢? 嵌入式系统的烧制, 库的编译与安装, ros系统框架的了解, 传感器的调试及对硬件的熟悉. 之前的实习有学了什么东西呢? 南康熟悉了cad? 测绘局学会了三维软件建模? 谁不会? 培训一下, 初中生干的也不比我们差. 那种实习完完全全在浪费青春. 

翻天覆地, 天翻地覆? 为何说它 \lq 天翻地覆\rq ? 

从古老的70年代坐着时光穿梭机来到21世纪, 逃出地牢, 重见光明. 

\subsection{大学四年反思}

本科生活转眼就要结束, 比较幸运, 稀里糊涂保了研, 也因此才有机会接触智驾领域. 回顾自己的四年大学: 
\begin{quotation}
	脱离高考深渊, 心如脱缰野马, 易放难收

	家事琐事与斩不断的过去

	垃圾课程太多, 缺少独立思考空间

\end{quotation}

私以为高考可怕之处在于思维的固化, 十几年反复刷题, 一味崇尚成绩, 忽略实践和能力的培养. 单单有优秀的成绩和漂亮的履历就沾沾自喜, 这对于长远发展没有益处. 刚进大学, 自己过于相信 \lq 书中自有黄金屋, 书中自有颜如玉\rq\ 把书本捧上天, 毕竟对付高考的秘籍全在参考书中. 现在突然意识到, 大学完全不一样. 对于绩点, 功利一些, 考前突击完全可以应付. 平时空闲, 多看看书, 深钻自己感兴趣的东西, 多注重学习能力的培养才是大学的根本.

结合自己实习所感, 将来想混日子, 去测绘局等一些行政单位工作, 活少钱多责任小. 但人不能和咸鱼一样没有梦想, 要想干出点事情来, 仅凭屎一样的学院课程是在做白日梦. 大多数老师讲课跟个死人一样, 面瘫着读PPT, 如此这般, 大家只能应付. 测绘法, 测量学, 工程力学, 地质等等无用课程, 统统翘掉才是最佳选择. 能上的课程除了数学之外, 还有王穗辉老师的平差, 李博峰老师的gps, 叶勤老师的摄影测量学, 伍吉仓的大地测量学, 认真听听, 课后钻研一下, 还是很锻炼人的思维. 大概自己不了解自己的需求, 没有方向, 总是有一种 \lq 要做个好孩子, 翘课就不是好孩子\rq\ 的典型学生思维束缚着我, 便没了翘课的决心, 结果就是屈服于权威. 

凤凰涅槃, 破茧重生.

垃圾规则都统统去死吧. 突然想到三傻大闹宝莱坞的一个桥段, 不愿陷害朋友而跳楼的拉加在面试时, 坐在轮椅上, 有这样一段台词:
\begin{quotation}
    断了两条腿, 才让我真正站起来, 好不容易获得了这种态度, 我不会改变的. 你留着你们这份工作吧, 我会保留我的态度.        
\end{quotation}

大学四年, 我有三件事情非常骄傲. 

一是选了很多自己喜欢的选修课, 从工笔画, 佛学, 货币金融, 生物, 国际政治到医学理论课, 实验课, 结识了众多优秀的老师, 每次上课犹如置身天堂. 每个学期在选修课上花费的精力比学院专业课要多得多. 尽管因为没有机会使用, 知识全都成功地还给了老师们, 还是由衷感谢他们给予我如此美好的人生回忆. 

选修课最多的时候大二下, 几乎每个晚上都在上课, 精力不足, 白天两次觉: 中午午休, 晚饭小憩. 印象深刻的是清明假前一天要上杨士忠老师的比较政治制度, 课在晚上, 那天实在太累了, 吃完晚饭, 闭眼, 再睁眼看表就发现课已经开始上了, 昏昏沉沉抖了一个机灵, 立马就清醒, 跳下床来, 急急忙忙穿着拖鞋就跑过去, 迟到十分钟. 杨老师正在讲课, 课间点名发现很多人没来, 他一副恍然大悟的样子, \lq\lq 啊, 真对不起大家, 忘了明天是假期, 要有点人情味, 那今天这次点名不算 \rq\rq\ 教室里充满了欢呼, 杨老爷爷真是太可爱了. 

陈实老师的运动解剖学和中枢神经系统解剖学两门课, 明明是周三下午和晚上的课, 但它却让我过上了极为充实的周末, 甚至20岁生日的那个周末也在看课程的课件, 但不好意思, 现在全忘了哈哈哈.

二是选的两门英语课. 大二可以自由选择英语课内容, 很庆幸自己选了韦文晧老师的英美社会与文化和裘立勤奶奶的报刊选读. 两位老师的精彩授课不拘泥于形式讲无聊的文章或落俗的搭配, 而着重关注语言背后的文化背景和思维模式, 和新东方英语课很像, 但远超新东方. 

英美社会文化从历史, 政治, 经济, 法律等多方面介绍美国和英国, 大大开拓了眼界, 报刊选读则是联系单词对搭配后的语言思维进行深度解析. 备战托福雅思这类对词汇量要求很高的考试像是语言的外家功夫, 那么这两门课毫无疑问是对内功的极大提高, 领悟了课程的精髓, 便打通了任督二脉. 冥冥之中参悟了学习语言的方法, 这才是最终要的.

三是坚持游泳. 体育上了两年秦海权老师的游泳课, 课余时间也会经常去游泳池练习, 发现自己爱上了这项运动. 从不会游泳, 到现在每次都是两公里练习, . 同时也在游泳池里结识了一帮14分钟一千米的怪物. 虽然比起大多数泳客, 我可以所向披靡, 但是对于8年, 10年的业余顶尖, 差的不是一点半点. 总之慢慢练习, 缩小差距.

本科居然要结束了, 学习还要继续. 如果能重来, 我真的会翘掉很多无聊的课程(然后就准备考研了). 现在拥有的永远是坠吼的.

\newpage
\section{尾声的内心戏}
\subsection{我就是我}

现在感觉良好, 然而世事无常, 可知福兮是祸, 祸兮是福也. 人生就是一艘漂浮在茫茫大海中的小舟, 身处风浪之中, 暂时的平静后隐藏的不知道是什么惊涛骇浪. 四个月的工作仍让我心有余悸, 我不知道自己是否适合技术
\subsection{不远的将来}

已经对研究生生活不抱有任何幻想, 接受了现实. 没有老师会认真讲课, 不仅如此, 反而还会给学生布置一些用处不大的presentation, 让学生代替老师讲课, 无耻至极. 这种课不翘留着过年吗?


\end{document}