\section{技术发展}

建构筑物的变形是指其大小, 位置以及形状等物理特性在受到自身重力或外部作用力的作用下, 其物理特性在时间域或空间域上发生的变化. 变形监测也称之为变形观测和变形测量, 其主要工作原理为在进行变形监测的构筑物上按照国家相关规范, 设置一定数量的能反应变形体结构特征的监测点, 通过采用相应的测量仪器按照技术设计的测量技术对监测点定期进行周期性的测量工作, 并对测量结果进行分析解算, 进而得到监测点相对于第一观测周期的形变情况, 通过对各个监测点的解读分析, 进而判断出整个变形体所发生的形变. 一般来说建构筑物的变形分为自身变形和刚体位移两种情形, 其中自身变形的主要形式有: 拉伸, 错位, 弯曲和扭转四种; 刚体位移变形的主要形式有: 变形体的水平移动, 变形体的垂直移动, 变形体的旋转和倾斜四种. 按照变形体的变形原因和作用力作用形式又可将其分为静态变形监测和动态变形监测. 其中静态变形监测指的是对变形体在一段时间内进行周期性的变形测量作业, 从而进一步获得静态变形. 动态变形监测主要是对变形体由外力引起的变形进行实时动态的监测, 从而对其形变情况进行分析预警, 达到保障工程安全的作用. 

相应的工程变形监测理论方法主要研究三个方面的工作:
\begin{enumerate}
    \item 如何高效, 精确的获取建构筑物的变形数据信息; 
    \item 如何正确的对所获得的变形数据进行科学的分析; 
    \item 如何准确的对变形体未来的物理变形情况进行预测.
\end{enumerate}

以上三个方面的工作环环相扣, 其中变形数据的采集工作是进行变形监测理论研究的基础性工作, 而建立科学的数学模型对变形数据进行分析则是进行变形监测理论研究的基本手段, 前面两项工作的目的均是为了准确预测变形体的未来变形情况, 进一步揭示和研究工程变形的规律与动态特性, 这对工程建设的安全进行具有重要的意义. 

\subsection{变形监测种类}
进入二十一世纪以来, 随着变形监测手段和硬件水平的提升, 尤其是以计算机为代表的变形数据处理技术提升飞速, 取得了较大的进展. 按照变形的监测范围, 变形监测一般分为三种, 分别是: 全球性变形研究, 区域性变形研究和工程局部性变形研究. 其中全球性变形研究的主要内容是监测全球板块运动, 地极移动, 地球自转速率变化, 地潮运动等地球物理现象, 其应用的方法有甚长基线干涉测量, 卫星激光测距卫星重力勘探等技术手段, 目前应用较为广泛且比较成熟. 区域性变形研究的主要内容是小区域内地壳形变监测, 城市地面沉降等工作, 目前应用的主要技术手段是GNSS 变形监测, InSAR 变形监测技术等. 

工程变形监测主要是在工程建设的各个阶段对建设工程的主体结构进行监测, 尤其是建筑物结构主体的三维变形和自然地形的滑坡以及地面沉降的监测工作. 其主要应用的技术手段是利用全站仪, GNSS, 水准仪的常规测量仪器, 选取变形监测点的位置进行变形数据采集, 目的是保证建构筑物的安全施工建设和运营维护. 这几年变形监测技术又出现了新的发展, 主要如下:

\begin{enumerate}
    \item 合成孔径雷达干涉测量技术发展迅猛, 现在可以对地表的变形监测达到厘米甚至毫米级, 较好的完成城市地面沉降, 地震监测等工作任务. 
    \item 传统的常规地面测量技术日趋完善, 随着全站仪的普及, 尤其是全自动跟踪全站仪, 也称测量机器人技术的应用, 为工程变形监测和地下建构筑物的变形监测提供了有效的解决途径, 在接通外部电源后, 可以长时间, 全天候, 全方位的进行变形监测数据的采集, 再将其通过无线通信或数据线与计算机系统处理系统相连接, 便可以做到对变形体的实时监测和变形数据处理, 变形监测工作功效极大提高. 
    \item 摄影测量技术近年来在变形监测中应用较为广泛, 该技术起步较早, 但受制于观测距离和精度的影响, 局限性较为突出, 仅仅适用于大型构筑物, 古代建筑以及山体的滑坡等变形监测. 近年来随着无人机技术的发展, 摄影测量被更广泛的使用在区域变形监测中, 与此同时三维激光扫描技术获取的测量数据大, 信息全, 属性数据多等特点已逐渐成为变形监测工作中一项重要技术. 
    \item 为了满足抗磁, 抗震, 抗雷击等特殊工程条件下变形监测工作的需要, 综合光, 机, 电等工业技术, 越来越多的专用变形监测仪器开始采用自动化, 信息化的数据采集手段, 促进了行业的发展和升级.
    \item GNSS 测量技术是利用空间后方距离交会原理, 在三维空间坐标系统内对待测点进行坐标测量, 进而计算待测点坐标, 实现空间三维定位的目的, 其作为定位测量新技术将逐渐在工程变形监测工作方面取代传统的光学和电子测量技术. 尤其是进入 21 世纪以来, 随着我国北斗定位导航系统的建成, 该系统的使用使得空间定位技术引起了革命性的变化. 与此同时, 由于 GNSS 技术可以同时测得待测点的三维坐标, 已经逐步实现了从陆地到海洋到外层空间, 从静态测量到动态实时测量, 精度也已从原来的米级逐步提升到厘米, 毫米甚至亚毫米级别, 极大地扩展了应用领域.
\end{enumerate}

\subsection{变形数据处理方法}
在进行工程变形测量数据的处理过程中要进行以下几方面的工作. 20 世纪 70 年代德国测量学家佩尔策提出了平均间隙法对参考点的稳定性进行分析评判. 其主要方法是使用两个变形监测值之差计算单位权方差与两个周期变形监测值所计算的综合单位权方差构成 F 检验, 对变形参考点的稳定性进行判断评估. 基本原理是: 假设变形监测工程当中所有的基准点在两个变形观测周期内都未发生变化, 便可以将这两个观测周期的变形测量当作对该变形监测网的两次连续观测, 基于这两次观测值平差后获得的两组基准点可以作为双观测值使用, 并由此进一步计算单位权方差估计值. 随后发展起来的单点位移分量法, 方差分析和位移矢量分析等方法对参考点的稳定性进行深入研究分析. 

20 世纪 60 年代初期, 迈塞尔从经典测量平差的观点出发, 将其满秩的系数推广到奇异阵, 进一步深入研究和推广使用秩亏自由网平差法, 解决了监测数据在数据处理时局部出现的非满秩问题. 几年后的 1964 年, 高德曼和蔡勒两人共同发现了将满秩权逆阵 Q 扩展到奇异阵的数据处理方法, 并在此基础之上进一步提出具有奇异权逆阵的最小二乘平差方法. 随后的 1971 年劳将对变形数据分析的各种情况进行了综合分析, 得出了广义高斯-马尔柯夫模型, 又称之为“最小二乘统一理论”. 五年后的 1976 年赫尔默特变换法又很好的解决了在进行变形数据处理时产生的随机模型问题. 我国大地测量学家, 误差理论专家周江文的“拟稳平差理论”以及“抗差估计理论”今年也被广泛应用到工程变形监测和预警的工作中. 

根据工程实际建设的需要确定了变形监测的施工方案, 采取正确的方法对变形监测点进行外业数据采集, 并对变形监测的数据信息进行分析, 处理, 预测和判断则是变形监测的核心工作. 对于变形体变形预测的理论研究今年也取得了很大的进步, 发展较快. 相较于传统的时间序列分析法, 多元回归分析方法, 滤波技术和频谱分析方法, 新发展起来的非线性时间序列方法应用领域广泛, 同时可以对多个变量因素进行自主学习和非线性分析预测的灰色系统理论与神经网络模型等方法在计算机技术的应用和普及之下发展迅猛. 与此同时, 随着今年来各类异性建筑和需要进行工程变形监测的建设项目越来越多, 如何更科学, 更准确的进行变形监测预警成为了诸多研究学者研究的课题, 针对数据处理和变形体的变形预测做了大量的研究工作. 香港理工大学的陈永棋教授从变形监测数据的采集, 到多学科融合进行数据分析, 得到变形的体系解释, 再到如何利用模型科学的对变形体的变形趋势进行预测, 成体系的进行研究和应用, 促进了变形工作的发展. 

\subsection{沉降预测理论方法}
\subsubsection*{回归分析法}
多元线性回归模型是变形监测预报中使用最频繁的方法之一. 在现实的工作之中, 建筑物的形变是复杂的, 是由多种因素共同作用而产生的影响, 比如沉降, 不单单与建筑物的重量有关系, 而且还和基础的处理方法, 岩土的力学特性地表水的漏渗, 地下水的补充与排泄等原因紧密相关. 在建筑物变形的因素较多的模型中, 通过分析建筑物变形与变形因子的相关关系, 在该类模型中主要对建筑物自身的主体结构重量的变化和建筑物后期运营的荷载变化建立相应的归化方程, 建立起来的关系为线性关系, 在掌握建筑物荷载数据信息的情况下该类模型预测精度相对较高, 在分析预测时忽略除荷载以外的其他变量因子, 最终的预测值与实际测量值误差较大. 

\subsubsection*{时间序列分析模型}
时间序列分析模型引用较为广泛, 其主要内容是采用一定的测量方法, 对变形体的监测点以时间为周期进行一定时间的变形监测, 且每次按照观测周期的时间顺序进行数据的变形分析, 并以此为依据进一步对该变形体的未来变形趋势做出预测, 也属于变形预测的线性计算, 对多因素, 非线性变形预测的效果有待提升.

\subsubsection*{灰色系统模型}
我们国家的华中理工大学教授邓聚龙提出了灰色模型理论. 它是用来处理信息不完善系统的数学方法. 它并不是和经典的方法一样将系统的行为看作是随机变化的过程, 用概率统计方法, 而是看作与时间有关的灰色过程, 将没有规律的原始数据整理生成有规律的数列之后在进行研究. 灰色系统理论其优点是所需原始数据量不必很多, 而且对短期预测更加有效. 但是, 当预测期较长时, 精度又会变差. 

\subsubsection*{卡尔曼滤波模型}
在进行变形监测数据的分析, 处理和预测工作中, 通过对前期观测数据进行处理分析, 进而对变形体的后期变形趋势进行预测是十分重要的一种手段. 卡尔曼滤波模型对于降低模型分析当中的异常监测数据影响有着理想的效果, 其主要原理是在进行对比分析后该模型可以自动剔除噪声数据, 比较适合于工程变形监测中变形体获取监测数据量大, 多种因素影响的变形监测工程, 同时在工程建设的动态变形监测数据的分析, 处理和预测方面有着十分广泛的应用. 

\subsubsection*{人工神经网络模型}
近年来人工网络模型不仅仅在工程变形监测数据分析领域的应用技术蓬勃发展, 同时受到计算机硬件技术的不断提升和互联网云计算技术的飞速发展, 人工神经网络模型已经被广泛的使用到各行业. 在工程变形监测的数据处理方面, 人工神经网络模型拥有诸多优势, 该模型可以对变形体发生变形的多个因素数据进行分析, 且网络模型内的各个层级之间有着严密的逻辑关系和数据信息递进关系. 统一网络层级内的网络神经单元相互独立工作, 单个神经元可以对变形监测数据信息进行处理, 然后将处理结果传递到下一个层级的神经元进行处理分析, 实现了多个数据独立的并行处理模式, 大大的提高了监测数据模型的处理效率, 同时最终实现单个层级的输出结果. 在隐含层的数据处理和分析过程中整个模型具有自我学习和训练的能力, 不需要确切数学表达式的情况下对监测数据进行非线性计算. 正是以上优点使得人工神经网络模型在工程变形监测领域应用越来越广泛, 优势愈加明显. 

\subsubsection*{频谱分析}
频谱分析的主要功能是通过间隔时间相等的单位对时间段上变形体的变形数据信息进行 分析, 处理, 从而对其变形趋势做出量化预测. 此类模型较适用于观测时间周期相等的变形监测工作中. 但在实际的工程建设中变形监测的数据采集均不可避免的受到现场工作环境和施工作业的影响, 无法严格按照等时间对监测点进行周期性的监测, 只有在长期稳定且数据实时传输, 自动储存和实现数据自动处理的前提下使用该模型对变形的变形预测效果好. 采用傅里叶变换的信号虽然有了频率特征, 但是也丢失了其时间特征, 频谱分析中小波变换可以将时间序列数据表示成时间和频率的组合, 能反映出数据的时域特征和频域特征, 很大程度上限制了它的实用性. 

近年来随着人工智能 AI 技术, 测量机器人, 人工神经网络技术, 信息化无线互联网技术和大数据应用分析技术的蓬勃发展, 变形监测工作也逐步实现了向数据采集自动化, 数据传输无线化, 数据处理预警实时化的特点转变. 工程数量越来越大, 工程建设结构日益复杂, 构筑物荷载量越来越高, 所以工程变形监测技术在工程建设中的作用越来越明显, 随之而来的现象是变形数据采集能力强, 但数据的分析处理能力不足, 尤其是对工程安全事故和地质自热按灾害的变形预警工作还有待提高, 因此在对某个工程进行变形监测数据处理与分析的时候不能单纯的一刀切, 片面的选择单一的模型进行预测. 很可能需要多个模型组合使用, 具体的模型选用要根据工程建设的实际环境和相关要求进行选取, 只有这样才能利用工程变形监测手段预防工程事故的发生, 保障人民生命财产安全. 


