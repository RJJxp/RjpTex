\section{研究意义}
工程变形监测工作是测绘一级学科下列的一个二级学科, 其主要的工作内容是使用先进的高精度测量仪器对变形体实施变形监测, 按照时间的顺序对所有的变形监测点采集到的数据信息进行统计, 分析和研究, 对分析和预测变形体变形数据的数学模型进行反复验证和修改, 最终准确建立数学模型对变形体的变形趋势进行预测和预警, 从而保障工程建设安全, 有序, 高效的进行. 

目前变形监测技术可分为全球性变形研究, 区域性变形研究和工程变形研究. 其中全球性变形研究主要是对地球自身发生的变形进行分析判断, 例如对影响地球自身变形的自转速度, 地壳运动和海洋潮汐等因素进行可量化考核的因素进行测量研究. 区域变形监测则是指利用卫星或干涉测量技术对某一区域或者城市进行地形变化与地面沉降进行监测. 工程变形监测应用较为广泛, 主要在工程建设的勘测设计, 施工建设和运营管理过程中, 对建构筑物的主体变形情况进行监测分析, 确保工程建设的安全进行. 目前, 随着工程建设的大型化, 主体设计的异形化以及钢结构的预制构件房屋越来越多, 工程变形监测在施工过程中的地位和作用越来越凸显. 

当工程变形超过国家规范中规定的阈值时, 就需要进一步加大观测频率, 并组织多方力量进行分析解决, 因为建构筑物的主体变形异常往往是工程事故爆发的征兆. 多次严重的工程事故均清晰的表明, 重大安全工程建设事故的发生往往是以下情形: 
\begin{enumerate}
    \item 变形监测不到位, 即开始的《工程变形监测实施方案》中监测点位设计不科学, 不合理或者监测点布设过少, 无法客观, 科学, 全面的反映变形体主体变形情况. 
    \item 主体变形数据采集不及时, 变形监测周期过长, 有的工程事故中存在多个观测周期未进行变形监测数据, 导致变形监测出现空档, 错过了最佳变形监测分析预警时间, 导致工程事故的发生.
    \item 传统的变形监测和预测手段往往是重数据采集而轻视数据分析, 对采集到的变形监测数据分析不够合理, 处理不够科学, 预测不够准确, 导致主体异常变形的关键结果没有被判断评估出来, 最终导致了由工程变形监测不到位引发的不安全工程建设事故的发生. 
\end{enumerate}

总之, 工程变形监测工作相较于传统的大地测量, 地形测绘, 地图制图的工作起步比较晚, 从 1970 年以后才逐渐被人们所熟知其在工程建设中的作用. 但在大型工程, 异形建筑和地下工程的建设中扮演着越来越重要的角色, 尤其是近年来各个城市地铁工程建设规模和力度的不断加大, 使得变形监测与人们的生活联系日益紧密. 同时, 工程变形监测工作的开展不仅为研究和解释建构筑物主体变形提供技术支撑, 也成为预防自然灾害的有效手段. 通过对易发生地址灾害的地区进行动态变形监测, 数据分析和预警评估, 可以有效地预测自然灾害发生的时间和规模, 提前做好预警措施, 为人员和物资的安全撤离争取宝贵的时间保障人民生命财产安全.

对不同工程建设应根据实际情况和现场环境采用不同的变形监测方法, 在对地铁开挖基坑的变形监测点进行变形监测工作时, 受制于现场工作环境, 现场地形条件和仪器设备, 测量员自身等多方因素的影响会造成采集到的变形监测数据缺失, 失真或不完整的情况出现. 例如在大型水坝的变形监测过程中, 由于水坝本身是曲面异形结构, 结构复杂, 观测点位较多, 且大坝多建立在山区自然观测条件相对较差, 往往造成进行水平和垂直位移监测的监测点在进行布设之初就不够科学合理, 最终导致变形监测信息不全面, 从而降低坝体的变形监测结果与预测. 又例如对在建筑物主体进行沉降观测时, 受制于施工组织管理或现场施工条件的影响, 往往造成部分沉降观测周期无法正常开展变形监测作业, 同时受制于自然天气的原因最终导致变形数据不完整, 无法按照时间序列形成有效的数据支撑. 显而易见, 在实际的变形监测工作中数据的缺失或者不完整往往是不可避免的, 在此基础上进行变形体的变形分析和主体未来变形趋势的预测其科学性和准确性都会出现较大的偏差. 因此如何利用现有条件对确实的数据进行处理, 使得其精度和准确性都达到国家规范要求, 确保变形监测数据处理和预测的准确性对变形监测工作具有着重要意义. 