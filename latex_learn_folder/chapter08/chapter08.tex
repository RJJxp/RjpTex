\documentclass[a4paper, UTF8]{article}

\usepackage{enumerate}  % enumerate numbering package
\usepackage{array}  % these three is for table style
\usepackage{longtable}
\usepackage{multirow}
% below is for math 
\usepackage{amsmath}

\newcommand{\rjpvector}[2]{#1_1,#1_2,\dots,#1_#2}
% \pagestyle{plain}

\begin{document}

\title{\Huge chapter08 \LaTeX\ Learning}
\author{\Large rjp \\
            Tongji Jiading Campus}
\date{2019-01-22}
\maketitle

\begin{abstract}
\large
    this is chapter08 learning of \LaTeX and i find it interesting.
    
    keep going!!

    Don't stop till you get enough
\end{abstract}
\thispagestyle{empty}
\newpage

\pagenumbering{Roman}
\tableofcontents
\addtocontents{toc}{\protect\vspace{10pt}}

\newpage
\pagenumbering{arabic}
\section{\Large The Basics}
\quad\ The equation representing a straight line in the Cartesian plane is of the form $ax+by+c=0$, where $a$, $b$, $c$ are constants.

The equation representing a straight line in the Cartesian plane is of the form $ax+by+c=0$, where $a, b, c$ are constants

% way 01
The equation representing a straight line in the Cartesian plane is of the form 
$$
ax+by+c=0
$$
where $a$, $b$, $c$ are constants.

% way 02
The equation representing a straight line in the Cartesian plane is of the form 
\[
ax+by+c=0
\]
where $a$, $b$, $c$ are constants.

% way 03
The equation representing a straight line in the Cartesian plane is of the form 
\begin{displaymath}
ax+by+c=0
\end{displaymath}
where $a$, $b$, $c$ are constants.

\subsection{\large Superscripts and subscripts}
In the seventeenth century, Fermat conjectured that if $n>2$, then there are no intergers $x$, $y$, $z$ for which
$$
x^n+y^n=z^n.
$$
This is proved in 1994 by Andrew Wiles.

$$
x^{mn} \qquad (x^n)^m \qquad x^mn
$$

The sequence $(X_n)$ defined by
$$
x_1=1,\qquad x_2=1, \qquad x_n=x_{n-1}+x_{n-2}\: (n>2)
$$

\newpage
\subsection{\large Roots}
Which is greater $\sqrt[4]{5}$ or $\sqrt[5]{4}$ ?

\section{\Large custom commands}
Here is the $rjpvector$ by rjp:
$$
\rjpvector{x}{n}
$$

\begin{equation}
    a=1
    \numberwithin{equation}{section}
\end{equation}

\section{\Large more on maths}
\subsection{\large single equation}
The equation representing a straight line in the cartesian plane is of the form
\begin{equation*}
    ax+by+c=0
\end{equation*}
where $a$, $b$, $c$ are constants.

The equation representing a straight line in the cartesian plane is of the form
\begin{equation}
    ax+by+c=0
\end{equation}
where $a$, $b$, $c$ are constants.

Thus for all real numbers $x$ we have
\begin{equation*}
    x\le|x|\quad\text{and}\quad x\ge|x|
\end{equation*}

and so
\begin{equation*}
    x\le|x|\quad\text{for all $x$ in $R$}    
\end{equation*}

\begin{multline*}
    (a+b+c+d+e)^2=a^2+b^2+c^2+d^2+e^2\\
    +2ab+2ac+2ad+2ae+2bc+2bd+2be+2cd+2ce+2de
\end{multline*}

i am really happy, yesterday i went gym and work out. rjp is a handsome boy.

\begin{multline*}
    (a+b+c+d+e)^2=a^2+b^2+c^2+d^2+e^2\\
        +2ab+2ac+2ad+2ae\\
        +2bc+2be+2bd\\
        +2cd+2ce\\
        +2de
\end{multline*}


\begin{equation*}
    \begin{split}
        (a+b+c+d+e)^2 & =a^2+b^2+c^2+d^2+e^2\\
        &\quad+2ab+2ac+2ad+2ae\\
        &\quad+2bc+2be+2bd\\
        &\quad+2cd+2ce\\
        &\quad+2de
    \end{split}
\end{equation*}

\begin{equation*}
    \begin{split}
        (a+b)^2 & =(a+b)(a+b)\\
                & =a^2+ab+ba+b^2\\
                & = a^2+2ab+b^2
    \end{split}
\end{equation*}

\begin{equation}
    a = 1
    b=2    
\end{equation}

\subsection{\large group of equations}
\begin{gather*}
    (a,b)+(c,d)=(a+c,b+d)\\
    (a,b)(c,d)=(ac,bd)
\end{gather*}

\begin{gather}
    (a,b)+(c,d)=(a+c,b+d)\\
    (a,b)(c,d)=(ac,bd)
\end{gather}

Thus $x$, $y$, $z$ satisfy the equations
\begin{align*}
    x+y-z & =1\\
    x-y+z & =1
\end{align*}

Thus $x$, $y$, $z$ satisfy the equations
\begin{align*}
    x+y-z & =1\\
    x-y+z & =1\\
    \intertext{\quad\ and by hypothesis}
    x+y+z & =1
\end{align*}

Compare the following sets of equations
\begin{align*}
    \cos^2x+\sin^2x & =1    & \cosh^2x-\sinh^2x & =1\\
    \cos^2x-\sin^2x & =\cos2x   & \cosh^2x+\sinh^2x &= cosh2x
\end{align*}

Compare the following sets fo equations
\begin{equation*}
    \begin{aligned}
        \cos^2x+\sin^2x & =1\\
        \cos^2x-\sin^2x & =\cos2x
    \end{aligned}
    \qquad\text{and}\qquad
    \begin{aligned}
        \cosh^2x-\sinh^2x & =1\\
        \cosh^2x+\sinh^2x & \cosh2x
    \end{aligned}
\end{equation*}

\begin{equation*}
    |x|=
    \begin{cases}
        x & \text{if $x\ge 0$}\\
        -x & \text{if $x\le 0$}
    \end{cases}
\end{equation*}
\newpage

\subsection{\large numbered equation}
The equation represents a straight line in the plane
\begin{equation}
    ax+by+c=0
    \numberwithin{equation}{section}
\end{equation} 

where $a$, $b$, $c$ are constants.

The equation represents a straight line in the plane
\begin{equation}
    ax+by+c=0
    \tag{rjp}
\end{equation} 

where $a$, $b$, $c$ are constants.

The equation represents a straight line in the plane
\begin{equation}
    ax+by+c=0
    \tag*{rjp}
\end{equation} 

where $a$, $b$, $c$ are constants.


Thus $x$, $y$, $z$ satisfy the equations
\begin{align*}
    x+y-z & =1\notag\\
    x-y+z & =1\notag\\
    \intertext{and by hypothesis}
    x+y+z & =1\tag{rjpH}
\end{align*}

Thus $x$, $y$, $z$ satisfy the equations
\begin{align*}
    x+y-z & =1\\
    x-y+z & =1\\
    \intertext{and by hypothesis}
    x+y+z & =1\tag{rjpH}
\end{align*}

\begin{equation}
    \begin{split}
        (a+b)^2 & =(a+b)(a+b)\\
                & =a^2+ab+ba+b^2\\
                & = a^2+2ab+b^2
    \end{split}
\end{equation}

\section{mathematics miscellany}
% 数学上的混合体
\subsection{matrices}
\quad\ The system of equations
\begin{align*}
    x+y-z & =1\\
    x-y+z & =1\\
    x+y+z & =1
\end{align*}

can be written in matraix terms as 
\begin{equation*}
    \begin{pmatrix}
        1 & 1 & -1\\
        1 & -1 & 1\\
        1 & 1 & 1
    \end{pmatrix}
    \begin{pmatrix}
        x\\
        y\\
        z
    \end{pmatrix}
    =
    \begin{pmatrix}
        1\\
        1\\
        1
    \end{pmatrix}
\end{equation*}

Here, the pmatraix
$\begin{pmatrix}
1 & 1 & -1\\
1 & -1 & 1\\
1 & 1 & 1
\end{pmatrix}$
is invertible

Here, the matraix
$\begin{matrix}
1 & 1 & -1\\
1 & -1 & 1\\
1 & 1 & 1
\end{matrix}$
is invertible

Here, the bmatraix
$\begin{bmatrix}
1 & 1 & -1\\
1 & -1 & 1\\
1 & 1 & 1
\end{bmatrix}$
is invertible

sometimes mathematics write matrices within parentheses as in 
$
\begin{pmatrix}
    a & b\\
    c & d
\end{pmatrix}
$
while others prefer quare brackets as in 
$
\begin{bmatrix}
    a & b\\
    c & d
\end{bmatrix}
$

the determination of 
$
\begin{vmatrix}
    a & b\\
    c & d
\end{vmatrix}
$
is defined by 
\begin{equation*}
    \begin{vmatrix}
        a & b\\
        c & d
    \end{vmatrix}
\end{equation*}

the determination of 
$
\begin{vmatrix}
    a & b\\
    c & d
\end{vmatrix}
$
is defined by 
$$
\begin{vmatrix}
    a & b\\
    c & d
\end{vmatrix}
$$

a general matrix of $m\times n$ is like
\begin{equation*}
    \begin{bmatrix}
        a_{11} & a_{12} & \dots & a_{1n}\\
        a_{21} & a_{22} & \dots & a_{2n}\\
        \hdotsfor{4}\\
        a_{m1} & a_{m2} & \dots & a_{mn}
    \end{bmatrix}
\end{equation*}

\subsection{dots}
consider a finite sequence $X_1,X_2,\dots$, its sum $X_1+X_2+\dots$ and product $X_1X_2\dots$.

consider a finite sequence $X_1,X_2,\dots$, its sum $X_1+X_2+\dots$ and product $X_1X_2\dots$.

consider a finite sequence $X_1,X_2,\dotsc$, its sum $X_1+X_2+\dotsb$ and product $X_1X_2\dotsm$.

\subsection{delimiters}
% 定界符
since 
$
\left|
\begin{smallmatrix}
    a & h & g\\
    h & b & f\\
    g & f & c
\end{smallmatrix}
\right|
=0
$,
the matrix
$
\left(
\begin{smallmatrix}
    a & h & g\\
    h & b & f\\
    g & f & c
\end{smallmatrix}   
\right)
$
is not invertible.

\begin{equation*}
    \left.
    \begin{aligned}
        u_x & =v_y\\
        u_y & = -v_x
    \end{aligned}
    \right\}
    \quad\text{Cauchy-Riemann Equations}
\end{equation*}

\begin{equation*}
    (x+y)^2-(x-y)^2=\left((x+y)+(x-y)\right)\left((x+y)-(x-y)\right)
\end{equation*}

\begin{equation*}
    (x+y)^2-(x-y)^2=\{(x+y)+(x-y)\}\{(x+y)-(x-y)\}
\end{equation*}

\begin{equation*}
    (x+y)^2-(x-y)^2=\bigl((x+y)+(x-y)\bigr)\bigl((x+y)-(x-y)\bigr)
\end{equation*}

For $n$-tuples of comlex numbers $(x_1,x_2,\dotsc,x_n)$ and $(y_1,y_2,\dotsc,y_n)$
\begin{equation*}
    \left(
    \sum_{k=1}^{n}|x_{k}y_{k}|
    \right)^2
    \le
    \left(
    \sum_{k=1}^{n}|x_{k}|    
    \right)
    \left(
    \sum_{k=1}^{n}|y_{k}|    
    \right)
\end{equation*}

For $n$-tuples of comlex numbers $(x_1,x_2,\dotsc,x_n)$ and $(y_1,y_2,\dotsc,y_n)$
\begin{equation*}
    \biggl(
    \sum_{k=1}^{n}|x_{k}y_{k}|
    \biggr)^2
    \le
    \biggl(
    \sum_{k=1}^{n}|x_{k}|    
    \biggr)
    \biggl(
    \sum_{k=1}^{n}|y_{k}|    
    \biggr)
\end{equation*}

\subsection{putting one over another}
From the binomial theorem, it easily follows that if n is an even number, then
\begin{equation*}
    1-\binom{1}{n}\frac{1}{2}+\binom{2}{n}\frac{1}{2^2}-\dotsb-\binom{n-1}{n}\frac{1}{2^{n-1}}=0
\end{equation*}

Since $(X_{n})$ converges to $0$, there exists a positive integer $p$ such that
\begin{equation*}
    |x_{n}|<\frac{1}{2}\quad\text{for all $n\ge p$}
\end{equation*}


\newcommand{\rjpda}[2]{\genfrac{\{}{\}}{0pt}{}{#1}{#2}}
\newcommand{\rjpfang}[2]{\genfrac{[}{]}{0pt}{}{#1}{#2}}
The Christoffel symbol $\genfrac{\{}{\}}{0pt}{}{ij}{k}$ of the second kind is related to the Christoffel symbol $\genfrac{[}{]}{0pt}{}{ij}{k}$ of the first kind by the equation
\begin{equation*}
    \rjpda{ij}{k}=g^{k1}\rjpfang{ij}{k}+g^{k2}\rjpfang{ij}{k}
\end{equation*}

\begin{equation*}
    \frac{4}{\pi}= 1+\cfrac{1^2}
                    {2+\cfrac{3^2}
                    {2+\cfrac{5^2}{2+\dotsb}}}
\end{equation*}

Thus we can see that
\begin{equation*}
    0\xrightarrow{} A\xrightarrow{f} B\xrightarrow{g} C\xrightarrow{} 0
\end{equation*}

is a short exact sequence
\newpage

Thus we can see that
\begin{equation*}
    0\xrightarrow{} A\xrightarrow[\text{moni}]{f} B\xrightarrow[\text{gbi}]{g} C\xrightarrow{} 0
\end{equation*}

is a short exact sequence

Thus we can see that
\begin{equation*}
    0\xrightarrow{} A\xrightarrow[\text{moni}]{f} B\xrightarrow[\text{gbi}]{g} C\xrightarrow{} 0
\end{equation*}

is a short exact sequence

Euler not noly proved that the series $\sum_{n=1}^{\infty}\frac{1}{n^2}$ converges, but also that
\begin{equation*}
    \sum_{n=1}^{\infty}\frac{1}{n^2}=\frac{\pi^{2}}{6}
\end{equation*}

Euler not noly proved that the series $\displaystyle\sum_{n=1}^{\infty}\frac{1}{n^2}$ converges, but also that
\begin{equation*}
    \sum_{n=1}^{\infty}\frac{1}{n^2}=\frac{\pi^{2}}{6}
\end{equation*}

Euler not noly proved that the series $\sum\limits_{n=1}^{\infty} \frac{1}{n^2}$ converges, but also that
\begin{equation*}
    \sum_{n=1}^{\infty}\frac{1}{n^2}=\frac{\pi^{2}}{6}
\end{equation*}

Euler not noly proved that the series $\sum\limits_{n=1}^{\infty} \frac{1}{n^2}$ converges, but also that
\begin{equation*}
    \sum\nolimits_{n=1}^{\infty}\frac{1}{n^2}=\frac{\pi^{2}}{6}
\end{equation*}

Thus
$\lim\limits_{x\to\infty}\int_{0}^{x}\frac{\sin{x}}{x}\mathrm{d}x=\frac{\pi}{2}$
and so by definition
\begin{equation*}
    \int_{0}^{\infty}\frac{\sin{x}}{x}\mathrm{d}x=\frac{\pi}{2}
\end{equation*}

\begin{equation*}
    p_{k}(x)=\prod_{\substack{i=1\\i\ne k}}^n \left(\frac{x-t_i}{t_k-t_i} \right)
\end{equation*}

\subsection{the many faces of math}

% \begin{center}
inputs: Two pointclouds: $A=\{a_{i}\}$, $B=\{b_{i}\}$ \\
outputs: An initial transformation $T_0$
% \end{center}

\end{document} 