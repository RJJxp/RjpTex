\documentclass[a4paper, UTF8]{article}
% enumerate
\usepackage{enumerate}
% geometry
\usepackage{geometry}
% math
\usepackage{array}
\usepackage{amsmath}
\usepackage{amsfonts}
\usepackage{amssymb}
% theorem
\usepackage{amsthm}

% command part
\newtheorem{thmm}{theorem}[section]
\newtheorem{thmm01}{theorem}[section]
\newtheorem{cor}[thmm]{corollary}

\newtheorem{thm}{theorem}[section]
% setup 01
\theoremstyle{definition}
\newtheorem{dfn}{definition}[section]
% setup 02
\theoremstyle{remark}
\newtheorem{note}{note}[section]
% setup 03
\theoremstyle{plain}
\newtheorem{lem}[thm]{lemma}
% setup the page size
\setlength{\textwidth}{15cm}


\begin{document}
% title
\title{\Huge chapter09 Learning}
\author{rjp at Siping campus of Tongji university}
\date{2019-01-30}
\maketitle
% abstract
\begin{abstract}
    forget to push the .tex to github 

    so I have to rewrite the chapter09

    luckily i did not write too mucn on t440

    keep Learning
\end{abstract}
\thispagestyle{empty}   % set the foot and head of the page empty
\newpage
% make the content
\pagenumbering{Roman}   % all the content page number style is upper case roman letter
\tableofcontents
\newpage

\pagenumbering{arabic}  
\section{\Large Theorems in \LaTeX}
\begin{thmm}
    The sum of the angles of a triangle is $180^\circ$.
\end{thmm}

\begin{cor}
    The sum of the angles of a quadrilateral is $360^\circ$.
\end{cor}

rjp is a handsome boy and everybody should love him. And what about you? i love myself. i do not want to go home. i wanna stay here studying and having fun.

\begin{thmm01}
    The sum of the angles of a triangle is $180^\circ$.
\end{thmm01}

\begin{thmm}
    [0,1] is a compact subsect of $\mathbb{R}$.    
\end{thmm}

\begin{thmm}
    $[0,1]$ is a compact subsect of $\mathbb{R}$.    
\end{thmm}
\newpage

\section{\Large designer Theorems -- the amsthm package}
\subsection{\large ready made styles}
\begin{dfn}
    a triangle is the figure formed by joining each pair of three non collinear points by line seqments.
\end{dfn}

\begin{note}
    a triangle has three angles.
\end{note}

\begin{thm}
    the sum of the angles of a triangle is $180^\circ$.
\end{thm}

\begin{lem}
    the sum of any two sides of a triangles is greater than or equal to the third
\end{lem}

\subsection{custom made theorem}
\newtheoremstyle{rjpstyle}{}{}{\slshape}{}{\scshape}{.}{ }{}
\theoremstyle{rjpstyle}
\newtheorem{rrr}{theorem}[section]
\theoremstyle{plain}
\begin{rrr}
    the sum of angles of a triangle is $180^\circ$.
\end{rrr}

\newtheoremstyle{nonum1}{}{}{\itshape}{}{\bfseries}{.}{ }{#1 (\mdseries #3)}
\newtheoremstyle{nonum2}{}{}{\itshape}{}{\bfseries}{.}{ }{}
\theoremstyle{nonum1}
\newtheorem{cau1}{Cauchy's Theorem}

\begin{cau1}[Third version]
    If $G$ is a simply connected open subset of $\mathbb{c}$, then for every closed rectifiable curve $\gamma$ in $G$, we have
    \begin{equation*}
        \int_{\gamma}f=0
    \end{equation*}
\end{cau1}

\theoremstyle{nonum2}
\newtheorem{cau2}{Cauchy's Theorem}
\begin{cau2}[Third version]
    If $G$ is a simply connected open subset of $\mathbb{c}$, then for every closed rectifiable curve $\gamma$ in $G$, we have
    \begin{equation*}
        \int_{\gamma}f=0
    \end{equation*}
\end{cau2}

\theoremstyle{nonum1}
\newtheorem{Riemann}{Riemann Mapping THeorem} 
\begin{Riemann}
    Every open simply connected proper subset of $\mathbb{C}$ is analytically homeomorphic to the open unit disk in $\mathbb{C}$.
\end{Riemann}

\newtheoremstyle{n1}{}{}{\itshape}{}{\bfseries}{.}{ }{\thmname{#1}\thmnote{ (\bfseries #3)}}
\theoremstyle{n1}
\newtheorem{ccc}{Cauchy's Theorem}
\begin{ccc}[Third version]
    If $G$ is a simply connected open subset of $\mathbb{c}$, then for every closed rectifiable curve $\gamma$ in $G$, we have
    \begin{equation*}
        \int_{\gamma}f=0
    \end{equation*}
\end{ccc}

\newtheorem{Riemann123}{Riemann Mapping Theorem}
\begin{Riemann123}
    Every open simply connected proper subset of $\mathbb{C}$ is analytically homeomorphic to the open unit disk in $\mathbb{C}$.
\end{Riemann123}
\newpage

How does a theorem's style is determinated?

Command {\itshape \textbackslash theoremstyle} 's scope will not end until a new {\itshape \textbackslash theoremstyle} shows up. All the style of these theorems defined by {\itshape \textbackslash newtherom} command in the scope of corresponding {\itshape \textbackslash theoremstyle} command.

So assume there are 2 {\itshape \textbackslash theoremstyle} commands {\rmfamily s1, s2} and theorem {\rmfamily s1\_t1, s1\_t2} defined in {\rmfamily s1}, theorem {\rmfamily s2\_t1} defined in {\rmfamily s2}. When you wirte {\rmfamily s1\_t2} after {\rmfamily s2}, {\rmfamily s1\_t2} is still the type of {\rmfamily s1}, though there is already a command setting a new theoremstyle.

\subsection{That's more}
\swapnumbers 
\theoremstyle{plain} 
\newtheorem{numfirstthm}{Theorem}[section]
\begin{numfirstthm}[Euclid]
    The sum of the angles in a triangle is $180ˆ\circ$
\end{numfirstthm}

\begin{proof} 
    Let $\{p_1,p_2,\dotsc p_k\}$ be a finite set of primes. Define $n=p_1p_2\dotsm p_k+1$. Then either $n$ itself is a prime or has a prime factor. Now $n$ is neither equal to nor is divisible by any of the primes $p_1,p_2,\dotsc p_k$ so that in either case, we get a prime different from $p_1,p_2,\dotsc p_k$. Thus no finite set of primes can include all the primes.
\end{proof}

\begin{proof}[\textsc{Proof\,(Euclid)}:]
    Let $\{p_1,p_2,\dotsc p_k\}$ be a finite set of primes. Define $n=p_1p_2\dotsm p_k+1$. Then either $n$ itself is a prime or has a prime factor. Now $n$ is neither equal to nor is divisible by any of the primes $p_1,p_2,\dotsc p_k$ so that in either case, we get a prime different from $p_1,p_2,\dotsc p_k$. Thus no finite set of primes can include all the primes.
\end{proof}

\theoremstyle{plain}
\newtheorem{testrjp}{Theorem}[section]
\begin{testrjp}
    The square of the sum of two numbers is equal to the sum of their quares and twice their product.
\end{testrjp}

\begin{proof}
    This follows easily from the equation
    \begin{equation*}
        (x+y)^2=x^2+y^2+2xy
    \end{equation*}
\end{proof}

\begin{testrjp}
    The square of the sum of two numbers is equal to the sum of their quares and twice their product.
\end{testrjp}
\begin{proof}
    This follows easily from the equation
    \begin{equation*}
        (x+y)^2=x^2+y^2+2xy\tag*{\qed}
    \end{equation*}
    \renewcommand{\qed}{}
\end{proof}

\begin{testrjp}
    The square of the sum of two numbers is equal to the sum of their quares and twice their product.
\end{testrjp}
\begin{proof}
    This follows easily from the equation
    \begin{equation*}
        (x+y)^2=x^2+y^2+2xy\qed
    \end{equation*}
    \renewcommand{\qed}{}
\end{proof}
\newpage

\section{\Large housekeeping}

\end{document}