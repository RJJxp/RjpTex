\chapter{外业成果}

\section{像控点坐标文件}
% Table generated by Excel2LaTeX from sheet 'Sheet1'
\begin{table}[htbp]
    \centering
    \caption{小组像控点坐标(单位:$\si{px},\si{m}$)}
      \begin{tabular}{|l|c|c|c|c|c|c|c|}
      \hline
      ID    & \multicolumn{2}{c|}{63} & \multicolumn{2}{c|}{64} & \multicolumn{3}{c|}{地面坐标}\\
      \hline
            & X     & Y     & X     & Y     & X     & Y     & Z \\
      \hline
      34    & 5297  & 3337  & 2146  & 3075  & 5660.56 & 2771.877 & 6.221 \\
      \hline
      35    & 5707  & 2345  & 2683  & 2057  & 5876.63 & 2889.648 & 16.589 \\
      \hline
      36    & 5094  & 3158  & 1945  & 2793  & 5701.465 & 2719.948 & 4.231 \\
      \hline
      37    & 5338  & 3330  & 2235  & 2968  & 5661.921 & 2791.356 & 9.81 \\
      \hline
      38    & 5596  & 2975  & 2442  & 2612  & 5684.058 & 2792.756 & 7.135 \\
      \hline
      \end{tabular}%
    \label{tab:imgCtrlPnt}%
  \end{table}%

需要说明的是,左片、右片的坐标以像素表示,$X$为列数,$Y$为行数,地面坐标是指在同济独立坐标系下的地面测量坐标(左手系)$\{X,Y,Z\}=\{N,E,U\}$。


\section{测量坐标原始文件}

\begin{table}[htbp]
  \centering
  \caption{具体测量坐标}
    \begin{tabular}{|c|p{6em}|r|r|r|}
\hline    \multicolumn{1}{|r|}{成员} & 测量坐标(\si{m}) & \multicolumn{1}{p{5em}|}{N} & \multicolumn{1}{p{5em}|}{E} & \multicolumn{1}{p{5em}|}{U}\\ \hline
    \multicolumn{1}{|c|}{\multirow{3}[5]{*}{陈晨}} & 第一次 & 5517.385 & 2847.557 & 3.801 \\
\cline{2-5}       & 第二次 & 5517.38 & 2847.577 & 3.821 \\
\cline{2-5}       & 平均值 & 5517.383 & 2847.567 & 3.811 \\
    \hline
    \multicolumn{1}{|c|}{\multirow{3}[6]{*}{贾锐}} & 第一次 & 5684.06 & 2792.756 & 7.105 \\
\cline{2-5}       & 第二次 & 5684.056 & 2792.756 & 7.165 \\
\cline{2-5}       & 平均值 & 5684.058 & 2792.756 & 7.135 \\
    \hline
    \multicolumn{1}{|c|}{\multirow{3}[6]{*}{余周炜}} & 第一次 & 5535.474 & 2861.033 & 13.14 \\
\cline{2-5}       & 第二次 & 5535.474 & 2861.035 & 13.13 \\
\cline{2-5}       & 平均值 & 5535.474 & 2861.034 & 13.14 \\
    \hline
    \multicolumn{1}{|c|}{\multirow{3}[6]{*}{王雪辰}} & 第一次 & 5660.54 & 2771.876 & 6.217 \\
\cline{2-5}       & 第二次 & 5660.58 & 2771.879 & 6.225 \\
\cline{2-5}       & 平均值 & 5660.56 & 2771.877 & 6.221 \\
    \hline
    \multicolumn{1}{|c|}{\multirow{3}[6]{*}{毛瑞丰}} & 第一次 & 5661.92 & 2791.358 & 9.72 \\
\cline{2-5}       & 第二次 & 5661.923 & 2791.354 & 9.9 \\
\cline{2-5}       & 平均值 & 5661.921 & 2791.356 & 9.81 \\
    \hline
    \end{tabular}%
  \label{tab:2}%
\end{table}%

\section{测量方法图示}

本组成员选择的点一般为房屋角点,故多采用无棱镜的方式测量,测得一个点之后,一般用支导线的方式换一站,再测一次进行检查,得到两次量测结果$(X_1,Y_1,Z_1),(X_2,Y_2,Z_2)$,如果两次量测的结果在限差之内,取两次测量的平均值作为最终的结果,否则,进行返测。

大致流程为:架站$\rightarrow$后视定向$\rightarrow$第一次测量$\rightarrow$搬站$\rightarrow$第二次测量(检查)。

具体如下:
\begin{enumerate}
\item \textbf{陈晨的点}:该点位于同济大学大礼堂东南侧的的开水房的东北角点上。此点为房屋角点,故采用无棱镜的方式测量,第一次从控制点235进行观测,后视点为控制点234,后视坐标差见表\ref{tab:1},满足要求。由于这个点的周围控制点较多,所以没有利用支导线进行架站,而是再次利用控制点进行测量。第二次测量选取控制点234作为测站,控制点234-1作为后视定向点,后视坐标差见表\ref{tab:1},满足要求。同时由于测区其他地区缺少控制点,我在西北二寝室楼西侧布设一个点TP1,作为其他组员的控制点。两次测定坐标的坐标见表\ref{tab:2},坐标差满足精度要求。

\begin{figure}[htbp]
\begin{minipage}[c]{0.5\textwidth}
\centering
\includegraphics[width=6cm]{survey1.png}
\end{minipage}%
\begin{minipage}[c]{0.5\textwidth}
\centering
\includegraphics[width=6cm]{survey2.png}
\end{minipage}
\caption{陈晨测量方法图示}
\end{figure}

\item\textbf{贾锐的点}:海洋地质教育部重点实验室西南房角点。此点为房屋角点,故采用无棱镜的方式测量。第一次从控制点234进行观测,后视点为234-1,后视坐标差见表\ref{tab:1},满足要求,由于这个点的周围控制点较多,所以我们没有利用支导线进行架站,而是再次利用控制点进行测量,第二次我们选取234-1作为测站,235作为后视定向点,后视坐标差见表\ref{tab:1},满足要求,两次测定坐标的坐标见表\ref{tab:2},坐标差满足要求。

\begin{figure}[htbp]
\begin{minipage}[c]{0.5\textwidth}
\centering
\includegraphics[width=6cm]{survey3.png}
\end{minipage}%
\begin{minipage}[c]{0.5\textwidth}
\centering
\includegraphics[width=6cm]{survey4.png}
\end{minipage}
\caption{贾锐测量方法图示}
\end{figure}

\item\textbf{余周炜的点}:西北二楼楼顶东侧南角点。此点为房屋角点,故采用无棱镜的方式测量。第一次测量从控制点230进行观测,后视点为TP1,后视坐标差见表\ref{tab:1},满足要求。同时,布设一个支导线点。第二次测量利用支导线进行架站,选取234作为测站,支导线点作为后视定向点,后视坐标差见表\ref{tab:1},满足要求。两次量测的结果见表\ref{tab:2},坐标差满足精度要求。
\begin{figure}[htbp]
\centering
\includegraphics[width=12cm]{survey5.png}
\caption{余周炜测量方法图示}
\end{figure}

\item\textbf{王雪辰的点}:大礼堂南侧从东向西数地4根弯梁底端。由于此处的控制点稀疏,故先把控制点234作为后视点,控制点235作为测站,布设两个支导线点T1,T2。再在T1,T2分两次对像片控制点进行量测,后视点选择控制点235,后视坐标差见表\ref{tab:1},满足要求,两次量测的结果见表\ref{tab:2},坐标差满足精度要求。
\begin{figure}[htbp]
  \centering
  \includegraphics[width=12cm]{survey6.png}
  \caption{王雪辰测量方法图示}
  \end{figure}


\item\textbf{毛瑞丰的点}:大澡堂西面,开水房东面之间的房屋,房脊交线处,位于该房屋的东北边。此点为房屋角点,故采用无棱镜的方式测量。第一次测量从控制点235进行,后视点为234,后视坐标差见表\ref{tab:1},满足要求。第二次测量选取控制点234作为测站,控制点234-1作为后视定向点,后视坐标差见表\ref{tab:1},满足要求。两次量测的结果见表\ref{tab:2},坐标差满足精度要求。

\begin{figure}[htbp]
  \begin{minipage}[c]{0.5\textwidth}
  \centering
  \includegraphics[width=6cm]{survey7.png}
  \end{minipage}%
  \begin{minipage}[c]{0.5\textwidth}
  \centering
  \includegraphics[width=6cm]{survey8.png}
  \end{minipage}
  \caption{毛瑞丰测量方法图示}
  \end{figure}

\end{enumerate}


\begin{table}[htbp]
\centering
\caption{后视坐标差}
\begin{tabular}{|c|r|r|r|}
\hline
\multicolumn{1}{|c|}{后视坐标差(\si{mm})} & \multicolumn{1}{p{2cm}|}{$dx$} & \multicolumn{1}{p{2cm}|}{$dy$} & \multicolumn{1}{p{2cm}|}{$dz$} \\
\hline
\multicolumn{1}{|c|}{\multirow{2}[4]{*}{陈晨}} & 9  & 3  & 1 \\
\cline{2-4}       & 5  & 1  & 20 \\
\hline
\multicolumn{1}{|c|}{\multirow{2}[4]{*}{贾锐}} & 1  & 4  & 0 \\
\cline{2-4}       & 5  & 1  & 20 \\
\hline
\multicolumn{1}{|c|}{\multirow{2}[4]{*}{余周炜}} & 7  & 7  & 10\\
\cline{2-4}       & 9  & 1  & 10 \\
\hline
\multicolumn{1}{|c|}{\multirow{2}[4]{*}{王雪辰}} & 9  & 4  & 0 \\
\cline{2-4}       & 1  & 10 & 10 \\
\hline
\multicolumn{1}{|c|}{\multirow{2}[4]{*}{毛瑞丰}} & 2  & 4  & 10\\
\cline{2-4}       & 3  & 2  & 0 \\
\hline
\end{tabular}%
\label{tab:1}%
\end{table}%

% Table generated by Excel2LaTeX from sheet 'Sheet2'
% Table generated by Excel2LaTeX from sheet 'Sheet2'




本组成员在测量过程中,每人独立量测一个点,增强了独立解决问题的能力。

\section{点之记}

\begin{center} 
\begin{longtable}{cp{5cm}p{5cm}}
\caption{小组像控点说明} \\
\hline
\multicolumn{1}{l}{ID} & 点之记位置说明 & 点之记 \\
\hline
\endhead
\hline
\endfoot
34 & 该点位于同济大学大礼堂东南侧的的开水房的东北角点上 & \mgape{\includegraphics[width=5cm]{image1.png}} \\ 
35 & 西北二楼楼顶东侧南角点 & \mgape{\includegraphics[width=5cm]{image2.png}} \\
36 & 大礼堂南侧从东向西数第4根弯梁底端 & \mgape{\includegraphics[width=5cm]{image3.png}} \\
37 & 大澡堂西面,开水房东面之间的房屋,房脊交线处,位于该房屋的东北边 & \mgape{\includegraphics[width=5cm]{image4.png}} \\
38 & 海洋地质教育部重点实验室西南房角点 & \mgape{\includegraphics[width=5cm]{image5.png}} \\
\end{longtable}%
\end{center}%

  其说明见Excel:点之记7.xlsx
  
\section{校园调绘成果}

我们组认真地完成了校园的调绘,调绘图文件是调绘(1)文件夹的调绘.jpg文件。

根据要求,调绘的范围为整个同济校园的所有建筑、道路、土地用途等,调绘的项目在前一部分已经说明,这里不再列举。

\textbf{完整调绘图见附录,该图选自陈晨的成果,每个人的调绘图见个人报告的截图。}

下面是调绘的详细记录表格。
% Table generated by Excel2LaTeX from sheet 'Sheet1'
\begin{center}

\begin{longtable}{|l|l|r|l|}

\hline 编号 & 名称 & \multicolumn{1}{l|}{层数} & 用途 \\ \hline
\endhead
\hline
\endfoot

C01 & 云通楼(人文学院) & 5  & 教学 办公 \\
C02 & 土木工程学院 & 8  & 教学 办公 \\
C03 & 中德楼(中德学院) & 11 & 教学 办公 \\
C04 & 干训南楼 & 6  & 商用  \\
C05 & 半亩园餐厅 & 2  & 商用  \\
C06 & 干训北楼 & 6  & 商用  \\
C07 & 西南七楼 & 6  & 居住 \\
C08 & 西南八楼 & 6  & 居住 \\
C09 & 西南九楼 & 5  & 居住 \\
C10 & 西南一楼 & 3  & 居住 \\
C11 & 西南三楼 & 6  & 居住 \\
C12 & 西南二楼 & 6  & 居住 \\
C13 & 教育超市 & 2  & 商用  \\
C14 & 信息馆 & 2  & 办公 \\
C15 & 教学实践机房 & 3  & 教学 办公 \\
C16 & 物理馆(物理科学与工程学院) & 8  & 教学 办公 \\
C17 & 浴室 & 2  & 商用  \\
C18 & 经纬楼 & 3  & 办公 \\
C19 & 实验动物中心 & 3  & 教学 办公 \\
C20 & 医学与生命科学实验教学中心 & 3  & 教学 办公 \\
C21 & 宁静楼(数学科学学院) & 3  & 教学 办公 \\
C22 & 保卫处 & $2+3+4$ & 办公 \\
C23 & 机械馆 & 4  & 教学 办公 \\
C24 & 西苑饮食广场 & 3  & 商用  \\
C25 & 健身中心 & 1  & 商用  \\
C26 & 大学生购物中心 & 2  & 商用  \\
    &    &    &  \\
D01 & 大礼堂 & 2  & 商用  \\
D02 & 教育超市 & 2  & 商用  \\
D03 & 西北一楼 & 4  & 居住 \\
D04 & 西北二楼 & 4  & 居住 \\
D05 & 北苑饮食广场 & 2  & 商用  \\
D06 & 西北五楼 & 6  & 居住 \\
D07 & 西北四楼 & 6  & 居住 \\
D08 & 西北三楼 & 6  & 居住 \\
D09 & 岩土楼 & 8  & 教学 办公 \\
D10 & 岩土工程实验中心 & 1  & 教学 办公 \\
D11 & 土木工程试验实践基地 & 3  & 教学 办公 \\
D12 & 震动台 & 0  & 教学 办公 \\
D13 & 结构工程研究所 & 3  & 教学 办公 \\
D14 & 结构试验馆 & 1  & 教学 办公 \\
D15 & 桥梁馆 & 7  & 教学 办公 \\
D16 & 风工程馆 & 3  & 教学 办公 \\
D17 & 风工程实验室 & 2  & 教学 办公 \\
D18 & 致远楼(数学科学学院) & 3  & 教学 办公 \\
D19 & 生态楼 & 5  & 教学 办公 \\
D20 & 城市污染检制国家工程研究中心 & 6  & 教学 办公 \\
D21 & 明净楼(环境科学与工程学院) & 5  & 教学 办公 \\
D22 & 济阳楼 & 3  & 教学 办公 \\
D23 & 博思楼3号楼 & 15 & 居住 \\
D24 & 博思楼4号楼 & 15 & 居住 \\
D25 & 博思楼5号楼 & 15 & 居住 \\
    &    &    &  \\
A01 & 逸夫楼 & 2  & 办公 \\
A02 & 行政南楼 & 5  & 办公 \\
A03 & 行政北楼 & 5  & 办公 \\
A04 & 同创楼(校史馆) & 3  & 教学 \\
A05 & 衷和楼 & 21 & 教学 办公 \\
A06 & 综合服务大厅 & 4  & 办公 \\
A07 & 同文楼(外国语学院) & 3  & 教学 办公 \\
A08 & 汇文楼(外国语学院) & 5  & 教学 办公 \\
A09 & 文远楼(建筑与城市规划学院A楼) & 3  & 教学 办公 \\
A10 & 明成楼(建筑与城市规划学院B楼) & 4  & 教学 办公 \\
A11 & 建筑与城市规划学院C楼 & 8  & 教学 办公 \\
A12 & 建筑与城市规划学院D楼 & 5  & 教学 办公 \\
A13 & 北楼 & 4  & 教学 \\
A14 & 图书馆 & 11 & 教学 \\
A15 & 南楼 & 4  & 教学 \\
A16 & 工程试验馆 & 3  & 教学 办公 \\
A17 & 电子与信息学院实验楼 & 4  & 教学 办公 \\
A18 & 化学馆(化学科学与工程学院) & $6+3$ & 教学 办公 \\
A19 & 化工楼 & 3  & 教学 办公 \\
A20 & 声学馆 & 4  & 教学 办公 \\
A21 & 瑞安楼 & 8  & 教学 办公 \\
A22 & 海洋楼(海洋与地球科学学院) & $6+3$ & 教学 办公 \\
A23 & 运筹楼 & 5  & 教学 办公 \\
A24 & 学三楼 & 6  & 居住 \\
A25 & 学四楼 & 6  & 居住 \\
A26 & 学五楼 & 6  & 居住 \\
A27 & 留学生楼 & 12 & 居住 \\
A28 & 友谊大楼 & 12 & 办公 \\
A29 & 三好坞饮食广场 & 2  & 商用  \\
    &    &    &  \\
B01 & 仁心楼(附属同济医院分院) & 4  & 商用  \\
B02 & 游泳馆 & 1  & 商用 教学 \\
B03 & 医学院 生命科学与技术学院 & $12+5$ & 教学 办公 \\
B04 & 解放楼 & 2  & 教学 办公 \\
B05 & 青年楼 & 2  & 教学 办公 \\
B06 & 测绘馆(测绘与地理信息学院) & 2  & 教学 办公 \\
B07 & 学苑饮食广场 & 3  & 商用  \\
B08 & 乒乓球馆 & 1  & 商用 教学 \\
B09 & 攀岩馆 & 1  & 商用 教学 \\
B10 & 羽毛球馆 & 2  & 商用 教学 \\
B11 & 一二·九大楼 & 1  & 办公 \\
B12 & 博物馆 & 3  & 教学 \\
B13 & 一·二九礼堂 & 1  & 商用  \\
B14 & 体育馆 & 4  & 商用 教学 \\
B16 & 中法中心 & 5  & 教学 办公 \\
B17 & 旭日楼 & 3  & 教学 办公 \\
\end{longtable}%
\end{center}%



\chapter{内业成果}
\section{相对定向点}

见Excel表格:小组同名点.xlsx 和每个组员的个人报告。

每个人的量测的点如下:
\begin{table}[htbp]
  \centering
  \begin{tabular}{p{5cm}c}
    \toprule
    姓名 & 点号 \\
    \midrule
    余周炜 & 1$\sim$ 7 \\
    王雪辰 & 8$\sim$ 14 \\
    毛瑞丰 & 15$\sim$19 \\
    陈晨 & 20$\sim$ 25 \\
    贾锐 & 26$\sim$ 31 \\
    \bottomrule 
  \end{tabular}
\end{table}

\section{待定地物点}

待定地物点主要分为两部分一部分为大礼堂,另一部分为西北一。分别记在大礼堂.xlsx和西北一.xlsx中。

\subsubsection{大礼堂同名点}

\begin{center}
   \tablehead{ \hline 同名点 & \multicolumn{2}{c|}{63} & \multicolumn{2}{c|}{64} \\ \hline
   & x  & y  & x  & y \\ \hline}
   \tabletail{\hline}
      \begin{supertabular}{|l|cc|cc|}
      1  & 4933 & 2849 & 1774 & 2469 \\
      2  & 5024 & 2894 & 1866 & 2521 \\
      3  & 5031 & 2883 & 1884 & 2505 \\
      4  & 5035 & 2885 & 1890 & 2508 \\
      5  & 5029 & 2897 & 1880 & 2520 \\
      6  & 5062 & 2912 & 1904 & 2539 \\
      7  & 5069 & 2902 & 1924 & 2525 \\
      8  & 5073 & 2903 & 1927 & 2526 \\
      9  & 5067 & 2913 & 1913 & 2541 \\
      10 & 5093 & 2927 & 1939 & 2554 \\
      11 & 5099 & 2915 & 1953 & 2542 \\
      12 & 5103 & 2917 & 1957 & 2543 \\
      13 & 5096 & 2927 & 1946 & 2556 \\
      14 & 5123 & 2940 & 1971 & 2569 \\
      15 & 5129 & 2930 & 1984 & 2556 \\
      16 & 5132 & 2932 & 1987 & 2558 \\
      17 & 5128 & 2941 & 1977 & 2572 \\
      18 & 5154 & 2955 & 2002 & 2584 \\
      19 & 5159 & 2944 & 2014 & 2572 \\
      20 & 5163 & 2946 & 2017 & 2573 \\
      21 & 5158 & 2956 & 2006 & 2586 \\
      22 & 5183 & 2569 & 2008 & 2586 \\
      23 & 5183 & 2969 & 2033 & 2599 \\
      24 & 5190 & 2960 & 2046 & 2586 \\
      25 & 5193 & 2961 & 2049 & 2589 \\
      26 & 5188 & 2971 & 2038 & 2602 \\
      27 & 5124 & 2584 & 2063 & 2615 \\
      28 & 5221 & 2973 & 2076 & 2602 \\
      29 & 5223 & 2975 & 2078 & 2605 \\
      30 & 5218 & 2985 & 2070 & 2617 \\
      31 & 5244 & 3000 & 2094 & 2631 \\
      32 & 5250 & 2988 & 2105 & 2618 \\
      33 & 5254 & 2990 & 2110 & 2619 \\
      34 & 5248 & 3001 & 2100 & 2632 \\
      35 & 5273 & 3012 & 2124 & 2645 \\
      36 & 5280 & 3002 & 2134 & 2633 \\
      37 & 5284 & 3004 & 2139 & 2635 \\
      38 & 5261 & 3051 & 2108 & 2680 \\
      39 & 5300 & 3070 & 2145 & 2700 \\
      40 & 5254 & 3120 & 2101 & 2793 \\
      41 & 5215 & 3142 & 2062 & 2774 \\
      42 & 5187 & 3205 & 2040 & 2838 \\
      43 & 5183 & 3202 & 2035 & 2836 \\
      44 & 5194 & 3176 & 2042 & 2807 \\
      45 & 5169 & 3163 & 2018 & 2796 \\
      46 & 5158 & 3191 & 2010 & 2824 \\
      47 & 5154 & 3186 & 2005 & 2822 \\
      48 & 5163 & 3162 & 2011 & 2792 \\
      49 & 5139 & 3150 & 1987 & 2783 \\
      50 & 5127 & 3175 & 1979 & 2809 \\
      51 & 5124 & 3171 & 1974 & 2806 \\
      52 & 5134 & 3148 & 1981 & 2779 \\
      53 & 5170 & 3138 & 1955 & 2768 \\
      54 & 5097 & 3162 & 1949 & 2794 \\
      55 & 5092 & 3161 & 1943 & 2793 \\
      56 & 5103 & 3135 & 1949 & 2763 \\
      57 & 5077 & 3121 & 1925 & 2751 \\
      58 & 5065 & 3149 & 1917 & 2780 \\
      59 & 5062 & 3146 & 1914 & 2779 \\
      60 & 5073 & 3119 & 1921 & 2747 \\
      61 & 5047 & 3107 & 1894 & 2736 \\
      62 & 5036 & 3132 & 1886 & 2762 \\
      63 & 5032 & 3130 & 1883 & 2759 \\
      64 & 5043 & 3104 & 1890 & 2733 \\
      65 & 5016 & 3092 & 1862 & 2720 \\
      66 & 5003 & 3122 & 1855 & 2751 \\
      67 & 5001 & 3121 & 1852 & 2748 \\
      68 & 5013 & 3090 & 1858 & 2717 \\
      69 & 4986 & 3078 & 1832 & 2703 \\
      70 & 4974 & 3107 & 1824 & 2735 \\
      71 & 4969 & 3106 & 1821 & 2734 \\
      72 & 4982 & 3075 & 1826 & 2701 \\
      73 & 4942 & 3041 & 1781 & 2667 \\
      74 & 4858 & 3004 & 1697 & 2626 \\
      \end{supertabular}%
  \end{center}%
  
\subsubsection{西北一同名点}

% Table generated by Excel2LaTeX from sheet 'Sheet1'
\begin{table}[htbp]
    \centering
      \begin{tabular}{|l|cc|cc|}
        \hline
      同名点 & \multicolumn{2}{c|}{63} & \multicolumn{2}{c|}{64} \\
      \hline
         & x  & y  & x  & y \\ \hline
      1  & 4915 & 2645 & 1753 & 2260 \\
      2  & 5173 & 2618 & 2014 & 2236 \\
      3  & 5181 & 2681 & 2020 & 2301 \\
      4  & 5165 & 2682 & 2008 & 2302 \\
      5  & 5169 & 2722 & 2011 & 2345 \\
      6  & 5351 & 2703 & 2195 & 2327 \\
      7  & 5359 & 2767 & 2201 & 2393 \\
      8  & 5343 & 2769 & 2188 & 2394 \\
      9  & 5348 & 2810 & 2191 & 2437 \\
      10 & 5532 & 2789 & 2376 & 2421 \\
      11 & 5538 & 2853 & 2382 & 2483 \\
      12 & 5290 & 2879 & 2129 & 2508 \\
      13 & 5282 & 2777 & 2124 & 2402 \\
      14 & 5111 & 2793 & 1951 & 2417 \\
      15 & 5104 & 2690 & 1945 & 2308 \\
      16 & 4922 & 2707 & 1758 & 2325 \\
      \hline
      \end{tabular}%
  \end{table}%
  

\section{内定向}

以本组其中一个内定向程序neidingxiang.m(余周炜)为例,阐明内定向所使用的文件。

内定向使用了三个文件:camera.use和小组同名点.xlsx,点之记改.xlsx。

camera.use提供了像片四个角点的框标坐标,打开影片的元数据可以得知其四个角点的像素坐标,这样就可以用最小二乘法求出内定向的六个参数,为
\begin{align}
h_0&=-46.08 &h_1&=0.012 &h_2&=0 \\
k_0&=82.944 &k_1&=0 &k_2&=-0.012
\end{align}

点之记改.xlsx是点之记.xlsx文件去掉没有在航片63,64上出现的点后剩下的点,是全班的像点,需要进行内定向以便后续处理。小组同名点.xlsx是小组的像点,同样需要进行内定向。内定向的结果记在全班框标点.xlsx和框标点.xlsx中。

整理如下:
\begin{table}
\begin{tabular}{p{10em}p{8cm}}
\toprule
\textbf{Program} & neidingxiang.m \\
\textbf{Require} & camera.use \\
\textbf{Input} & 小组同名点.xlsx/点之记改.xlsx \\
\textbf{Output} & 框标点.xlsx/全班框标点.xlsx \\
\bottomrule
\end{tabular}
\end{table}
\section{相对定向}

以本组其中一个相对定向程序xiangdui.m(余周炜)为例,阐明相对定向所使用的文件。

相对定向包含两个程序xiangdui.m(脚本文件)和GaussNewton.m(函数文件)。其中xiangdui.m文件包含文件的读写,和有关数据的简单处理,然后将处理出来的数据带入GaussNewton.m中迭代计算。

相对定向用框标点.xlsx求取参数,结果写入xiangdui.csv中。
\begin{table}[htbp]
\begin{tabular}{p{10em}p{8cm}}
\toprule
\textbf{Program} & xiangdui.m,和GaussNewton.m \\
\textbf{Input} & 框标点.xlsx \\
\textbf{Output} & xiangdui.csv \\
\bottomrule
\end{tabular}
\end{table}

得出的结果如下:
\begin{equation}
\begin{array}{lllll}
\phi_1=-0.01245 & \kappa_1=0.02841 & \phi_2=0.01334 & \omega_2=-0.02251 & \kappa_2=0.036366
\end{array}
\end{equation}

\section{前方交会}

以本组其中一个前方交会程序qianfang.m(余周炜)为例,阐述前方交会所使用的文件。

前方交会是将框标点转换到模型坐标的程序,将其列举如下

\begin{table}[htbp]
\begin{tabular}{p{10em}p{8cm}}
\toprule
\textbf{Program} & qianfang.m \\
\textbf{Require} & xiangdui.csv \\
\textbf{Input} & 全班框标点.xlsx \\
\textbf{Output} & 点之记改.xlsx和inspect.xlsx \\
\bottomrule
\end{tabular}
\end{table}

其中inspect.xlsx的作用是检查上下视差,剔除视差较大的点,以便绝对定向的进行。

\section{绝对定向程序}

以本组其中一个绝对定向程序juedui.m(余周炜)为例,阐述绝对定向所使用的文件。

绝对定向包含两个程序juedui.m(脚本文件)和GaussNewton2.m(函数文件)。其中juedui.m文件包含文件的读写,和有关数据的简单处理,然后将处理出来的数据带入GaussNewton2.m中迭代计算\footnote{这里的$\phi,\omega,\kappa$与任务书上有所不同!依然是以$y$为主轴的变换,即$\MR=\MR_y(\phi)\MR_x(\omega),\MR_z(\kappa)$}。

\begin{table}[htbp]
\begin{tabular}{p{10em}p{8cm}}
\toprule
\textbf{Program} & juedui.m和GaussNewton2.m \\
\textbf{Input} & inspect.xlsx \\
\textbf{Output} & juedui.csv \\
\bottomrule
\end{tabular}
\end{table}

得出的结果如下:
\begin{equation}
\begin{array}{lll}
\phi=0.01126 & \omega=0.00564 & \kappa=0.000538 \\
dx=0 & dy=0 & dz=0 \\
\lambda=1.1318 & &
\end{array}
\end{equation}

\section{绝对定向坐标转换}


\subsection{从模型点转到地面点}

以本组其中一个绝对定向坐标转换程序jueduitransform.m(余周炜)为例,阐述阐述绝对定向坐标转换所使用的文件。

绝对定向坐标转换是将模型点转到地面点的程序,说明如下:

\begin{table}[htbp]
  \begin{tabular}{p{10em}p{8cm}}
  \toprule
  \textbf{Program} & juiduitransform.m \\
  \textbf{Require} & juedui.csv \\
  \textbf{Input} & inspect.xlsx \\
  \textbf{Output} & dimianzuobiao.xlsx \\
  \bottomrule
  \end{tabular}
  \end{table}

\subsection{从像素坐标转到地面点}

为了更加方便地转换坐标,特别编写了一个程序transform.m(余周炜),可以直接将像素坐标转换到地面摄测坐标,即将内定向,前方交会,绝对定向坐标转换三合一。

先将该程序说明如下:
\begin{table}[htbp]
\begin{tabular}{p{10em}p{8cm}}
\toprule
\textbf{Program} & transform.m \\
\textbf{Require} & neidingxiang.csv, xiangdui.csv, juedui.csv \\
\textbf{Input} & 点之记 - 副本.xls \\
\textbf{Output} & dimianzuobiao.xlsx \\
\bottomrule
\end{tabular}
\end{table}

\section{校园局部CAD图}

我们小组的负责区域是大礼堂及大礼堂以北的区域,我们选取了其中的两栋标志性建筑物:大礼堂和西北一楼成图。

dwg的文件名为:第七组展绘(1).dwg。
\begin{figure}[htbp]
\centering
\caption{校园局部CAD图(截图)}
\includegraphics[width=0.8\textwidth]{cad.PNG}
\end{figure}

\textbf{完整CAD图见附录}。

\section{像控点的质量评价}

% Table generated by Excel2LaTeX from sheet 'Sheet1'
我们组的王雪辰同学对像控点的质量进行了评价,贾锐对这些结果进行了整理。结果保存在点之记评价.xls中,这里也抄录了一份评价,表中,绿色的点表示选用的点,黄色的点上下视差过大,除去视差较大的点,其余的点舍去的原因是残差较大,即$\varepsilon_X>0.5$或$\varepsilon_Y>0.5$或$\varepsilon_Z>0.5$。在进行相对定向或者绝对定向的时候需慎用这些点。



\subsubsection{点质量检查过程}
\begin{enumerate}
  \item 先用全部45个点进行相对定向,前方交会,求出视差。
  \item 将视差大的点进行剔除:6、10、22、31、34。
  \item 利用剩余的点进行绝对定向,将利用前方交会、绝对定向过程求得的点的地面测量坐标与实地测得的地面控制坐标进行比较,看其残差。
  \item 将所有$\varepsilon_X$、$\varepsilon_Y$、$\varepsilon_Y$大与0.5的点剔除。即8、9、11、13、16、21、23、24、25、26、28、29、30、32、37、38、39、40、41、44、45、46点剔除。
\end{enumerate}

\subsubsection{剔除点的原因}
\begin{enumerate}
  \item 视差过大:同名点没有量测好
  \item XYZ方向残差(绝对定向结果与实测结果的差)过大:可能是实际测量出现了问题。
\end{enumerate}
\subsubsection{质量检查表}
% \begin{center}
%  \begin{longtable}{|p{0.3cm}|p{0.5cm}p{0.5cm}|p{0.5cm}p{0.5cm}|ccc|ccc|}
%   \hline
%   ID & \multicolumn{2}{c|}{63} & \multicolumn{2}{c|}{64} & \multicolumn{3}{c|}{地面坐标} & \multicolumn{3}{c|}{地面摄测坐标} \\ \hline
%       & X  & Y  & X  & Y  & N  & E  & U  & X  & Y  & Z \\ \hline
%   \endhead
%   \hline
%   \endfoot
 % Table generated by Excel2LaTeX from sheet 'Sheet1'
 {\scriptsize
\begin{center}
    \begin{longtable}{rrrrrrrrrrr}
      \hline
    \multicolumn{1}{c}{$X_1$} & \multicolumn{1}{c}{$Y_1$} & \multicolumn{1}{c}{$Z_1$} & \multicolumn{1}{c}{$X$} & \multicolumn{1}{c}{$Y$} & \multicolumn{1}{c}{$Z$} & \multicolumn{1}{c}{$\varepsilon_X$} & \multicolumn{1}{c}{$\varepsilon_Y$} & \multicolumn{1}{c}{$\varepsilon_Z$} & \multicolumn{1}{c}{取舍} & \multicolumn{1}{c}{\cellcolor[rgb]{ 1,  .922,  .612}\textcolor[rgb]{ .612,  .396,  0}{视差}} \\ \hline
    \endhead
    \hline
    \endfoot
    3325.255 & 5618.453 & 5.224245 & 3325.155 & 5618.452 & 4.763 & 0.09994 & 0.000753 & 0.461245 & \cellcolor[rgb]{ 0,  1,  0}1 & 0.297181 \\
    3184.812 & 5654.026 & 2.897074 & 3184.427 & 5653.929 & 3.522 & 0.384661 & 0.09749 & 0.624926 & \cellcolor[rgb]{ 0,  1,  0}1 & 0.377562 \\
    3083.96 & 5658.854 & 3.063883 & 3083.859 & 5658.79 & 3.312 & 0.1012 & 0.064073 & 0.248117 & \cellcolor[rgb]{ 0,  1,  0}1 & -0.18291 \\
    3111.425 & 5520.782 & 3.227235 & 3111.317 & 5520.525 & 3.425 & 0.108157 & 0.25667 & 0.197765 & \cellcolor[rgb]{ 0,  1,  0}1 & 0.32122 \\
    3143.943 & 5678.49 & -738.298 & 3147.206 & 5470.207 & 4.169 & 3.262565 & 208.2828 & 742.467 & 0  & \cellcolor[rgb]{ 1,  .922,  .612}\textcolor[rgb]{ .612,  .396,  0}{-4.07085} \\
    3228.777 & 5480.75 & 2.99556 & 3228.796 & 5480.649 & 3.267 & 0.019044 & 0.100925 & 0.27144 & \cellcolor[rgb]{ 0,  1,  0}1 & 0.071135 \\
    3145.07 & 5249.663 & 3.840062 & 3144.775 & 5250.421 & 3.713 & 0.295457 & 0.758289 & 0.127062 & 0  & 0.297194 \\
    3154.425 & 5235.886 & 3.367915 & 3154.083 & 5236.616 & 3.883 & 0.342282 & 0.730335 & 0.515085 & 0  & 0.745434 \\
    3162.523 & 5295.517 & 6.821206 & 3163.058 & 5284.273 & 5.13 & 0.535353 & 11.24422 & 1.691206 & 0  & \cellcolor[rgb]{ 1,  .922,  .612}\textcolor[rgb]{ .612,  .396,  0}{-20.9876} \\
    3146.828 & 5406.833 & 6.588907 & 3147.106 & 5407.432 & 3.519 & 0.277898 & 0.599082 & 3.069907 & 0  & -0.73315 \\
    3206.447 & 5380.779 & 5.953802 & 3206.535 & 5381.13 & 5.721 & 0.088134 & 0.35065 & 0.232802 & \cellcolor[rgb]{ 0,  1,  0}1 & 0.001018 \\
    3172.05 & 5389.952 & 2.917194 & 3171.487 & 5390.877 & 3.477 & 0.562912 & 0.924515 & 0.559806 & 0  & -0.57639 \\
    3019.958 & 5439.047 & 18.92497 & 3019.786 & 5439.221 & 18.858 & 0.171974 & 0.173586 & 0.066971 & \cellcolor[rgb]{ 0,  1,  0}1 & -0.21386 \\
    2995.393 & 5482.496 & 5.097186 & 2995.51 & 5482.386 & 5.036 & 0.116671 & 0.110457 & 0.061186 & \cellcolor[rgb]{ 0,  1,  0}1 & -0.19903 \\
    2967.536 & 5480.259 & 2.794274 & 2968.319 & 5480.476 & 3.552 & 0.783068 & 0.217483 & 0.757726 & 0  & 0.117919 \\
    2982.813 & 5327.731 & 3.89828 & 2982.782 & 5327.922 & 3.686 & 0.031388 & 0.191383 & 0.21228 & \cellcolor[rgb]{ 0,  1,  0}1 & -0.04636 \\
    2820.283 & 5259.801 & 24.92544 & 2820.262 & 5259.748 & 25.074 & 0.02104 & 0.053471 & 0.148561 & \cellcolor[rgb]{ 0,  1,  0}1 & -0.41929 \\
    2698.771 & 5396.679 & 25.48408 & 2698.826 & 5396.59 & 24.948 & 0.055347 & 0.088873 & 0.536076 & \cellcolor[rgb]{ 0,  1,  0}1 & -0.06351 \\
    2664.823 & 5284.881 & 39.86166 & 2664.958 & 5285.255 & 39.314 & 0.134846 & 0.37388 & 0.547658 & \cellcolor[rgb]{ 0,  1,  0}1 & -0.32063 \\
    2614.094 & 5433.659 & 5.928632 & 2614.312 & 5433.274 & 7.184 & 0.218052 & 0.385223 & 1.255368 & 0  & 0.014302 \\
    2702.725 & 5478.588 & 109.5921 & 2714.137 & 5536.209 & 5.052 & 11.41173 & 57.62053 & 104.5401 & 0  & \cellcolor[rgb]{ 1,  .922,  .612}\textcolor[rgb]{ .612,  .396,  0}{42.99215} \\
    2759.544 & 5602.958 & 16.411 & 2759.629 & 5602.687 & 15.135 & 0.084923 & 0.271276 & 1.275997 & 0  & -0.29363 \\
    2944.734 & 5657.787 & 10.99146 & 2944.622 & 5656.879 & 12.318 & 0.111607 & 0.907834 & 1.326541 & 0  & 0.248677 \\
    3036.207 & 5805.065 & 39.47812 & 3035.991 & 5804.33 & 39.393 & 0.216064 & 0.734684 & 0.085122 & 0  & -0.21186 \\
    2914.508 & 5789.933 & 5.344705 & 2914.961 & 5789.961 & 3.508 & 0.453057 & 0.028015 & 1.836705 & 0  & -0.00953 \\
    3001.68 & 5713.906 & 12.52921 & 3001.643 & 5714.065 & 12.354 & 0.036529 & 0.159326 & 0.17521 & \cellcolor[rgb]{ 0,  1,  0}1 & 0.028133 \\
    2673.634 & 5343.384 & 5.231976 & 2673.824 & 5343.921 & 3.802 & 0.189676 & 0.537285 & 1.429976 & 0  & 0.048064 \\
    2788.98 & 5339.714 & 4.822834 & 2788.519 & 5339.584 & 3.685 & 0.460666 & 0.129667 & 1.137834 & 0  & -0.13145 \\
    2794.132 & 5258.57 & 14.63085 & 2793.894 & 5258.26 & 18.254 & 0.238312 & 0.30972 & 3.623147 & 0  & -0.22608 \\
    2859.356 & 5541.164 & -66.9855 & 2847.567 & 5517.383 & 3.811 & 11.78888 & 23.78114 & 70.79655 & 0  & \cellcolor[rgb]{ 1,  .922,  .612}\textcolor[rgb]{ .612,  .396,  0}{-3.30561} \\
    2860.672 & 5536.317 & 13.63305 & 2861.034 & 5535.474 & 13.14 & 0.361976 & 0.842683 & 0.49305 & 0  & -0.17649 \\
    3111.195 & 5520.64 & 4.058702 & 3111.322 & 5520.511 & 3.312 & 0.126535 & 0.128927 & 0.746702 & \cellcolor[rgb]{ 0,  1,  0}1 & 0.122392 \\
    3168.858 & 5731.799 & 4.020554 & 3168.406 & 5733.826 & 3.861 & 0.451515 & 2.026886 & 0.159554 & 0  & \cellcolor[rgb]{ 1,  .922,  .612}\textcolor[rgb]{ .612,  .396,  0}{4.923757} \\
    3184.318 & 5653.957 & 3.707308 & 3184.577 & 5653.93 & 3.514 & 0.258927 & 0.026923 & 0.193308 & \cellcolor[rgb]{ 0,  1,  0}1 & 0.390958 \\
    2995.643 & 5482.281 & 5.053977 & 2995.632 & 5482.021 & 5.131 & 0.011041 & 0.26043 & 0.077023 & \cellcolor[rgb]{ 0,  1,  0}1 & 0.209386 \\
    2454.64 & 5513.5 & 6.36637 & 2454.777 & 5508.459 & 7.65 & 0.137192 & 5.040551 & 1.28363 & 0  & -0.16594 \\
    2455.294 & 5495.215 & 10.79527 & 2455.552 & 5495.855 & 6.007 & 0.257511 & 0.639703 & 4.788266 & 0  & -0.60617 \\
    2528.315 & 5471.491 & 15.59424 & 2488.758 & 5481.32 & 17.977 & 39.5573 & 9.828765 & 2.382759 & 0  & 0.049632 \\
    2543.713 & 5645.396 & 19.56766 & 2518.955 & 5632.976 & 10.242 & 24.75778 & 12.42 & 9.325656 & 0  & 0.060322 \\
    2444.063 & 5512.932 & 51.35935 & 2483.411 & 5588.006 & 19.59 & 39.34828 & 75.07421 & 31.76935 & 0  & -0.42067 \\
    2770.157 & 5649.058 & 7.636558 & 2771.877 & 5660.56 & 6.221 & 1.720432 & 11.50226 & 1.415558 & 0  & \cellcolor[rgb]{ 1,  .922,  .612}\textcolor[rgb]{ .612,  .396,  0}{20.58445} \\
    2877.692 & 5935.674 & -88.3402 & 2889.648 & 5876.63 & 16.589 & 11.95554 & 59.04416 & 104.9292 & 0  & \cellcolor[rgb]{ 1,  .922,  .612}\textcolor[rgb]{ .612,  .396,  0}{17.76358} \\
    2720.502 & 5702.085 & 6.900078 & 2719.948 & 5701.465 & 4.231 & 0.55366 & 0.619827 & 2.669078 & 0  & 0.897326 \\
    2784.796 & 5676.349 & -28.616 & 2791.356 & 5661.921 & 9.81 & 6.559622 & 14.42761 & 38.42595 & 0  & -0.31841 \\
    2837.509 & 5745.567 & 16.7849 & 2792.756 & 5684.058 & 7.135 & 44.75303 & 61.50872 & 9.649902 & 0  & 0.39082 \\
    \end{longtable}%
\end{center}%
 }



